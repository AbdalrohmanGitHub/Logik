\section{Der Cut-Operator}
Wir haben ein Prädikat als \emph{funktional} definiert,
wenn wir die einzelnen Argumente klar in Eingabe- und Ausgabe-Argumente
aufteilen können.  Wir nennen ein Prädikat \emph{deterministisch} wenn es funktional ist
und wenn außerdem zu jeder Eingabe höchstens eine Ausgabe berechnet
wird.  Diese zweite Forderung ist durchaus nicht immer erfüllt.  Betrachten wir die ersten
beiden Fakten zur Definition des Prädikats \texttt{mix/3}:
\begin{Verbatim}[ frame         = lines, 
                  framesep      = 0.3cm, 
                  labelposition = bottomline,
                  numbers       = left,
                  numbersep     = -0.2cm,
                  xleftmargin   = 0.8cm,
                  xrightmargin  = 0.8cm
                ]
    mix( [], Xs, Xs ).    
    mix( Xs, [], Xs ).
\end{Verbatim}
Für die Anfrage ``\texttt{mix([], [], L)}'' können beide Fakten verwendet werden.
Das Ergebnis ist zwar immer dasselbe, nämlich \texttt{L = []}, es wird aber zweimal ausgegeben:
\begin{Verbatim}[ frame         = lines, 
                  framesep      = 0.3cm, 
                  labelposition = bottomline,
                  numbers       = left,
                  numbersep     = -0.2cm,
                  xleftmargin   = 0.8cm,
                  xrightmargin  = 0.8cm
                ]
    ?- mix([],[],L).
    
    L = [] ;
    
    L = [] 
\end{Verbatim}
Dies kann zu Ineffizienz führen.  Aus diesem Grunde gibt es in \textsl{Prolog} den
Cut-Operator ``\texttt{!}''.  Mit diesem Operator ist es möglich, redundante Lösungen aus
dem Suchraum heraus zu schneiden.  Schreiben wir die ersten beiden Klauseln der
Implementierung von \texttt{mix/3} in der Form 
\begin{Verbatim}[ frame         = lines, 
                  framesep      = 0.3cm, 
                  labelposition = bottomline,
                  numbers       = left,
                  numbersep     = -0.2cm,
                  xleftmargin   = 0.8cm,
                  xrightmargin  = 0.8cm
                ]
    mix( [], Xs, Xs ) :- !.
    mix( Xs, [], Xs ) :- !.
\end{Verbatim}
so wird auf die Anfrage  ``\texttt{mix([], [], L)}'' die Lösung \texttt{L = []} nur noch
einmal generiert.  Ist allgemein eine Regel der Form \\[0.1cm]
\hspace*{1.3cm} $P \;\texttt{:-}\; Q_1, \cdots, Q_m, \texttt{!}, R_1, \cdots, R_k$ \\[0.1cm]
gegeben, und gibt es weiter eine Anfrage $A$, so dass $A$ und $P$  unifizierbar sind, so
wird die Anfrage $A$ zunächst zu der Anfrage \\[0.1cm]
\hspace*{1.3cm} $Q_1\mu, \cdots, Q_m\mu, \texttt{!}, R_1\mu, \cdots, R_k\mu$ \\[0.1cm]
reduziert.  Außerdem wird ein Auswahl-Punkt gesetzt, wenn es noch weitere Klauseln gibt,
deren Kopf mit $A$ unifiziert werden könnt.
Bei der weiteren Abarbeitung dieser Anfrage gilt folgendes:
\begin{enumerate}
\item Falls bereits die Abarbeitung einer Anfrage der Form \\[0.1cm]
      \hspace*{1.3cm} 
      $Q_i\sigma, \cdots, Q_m\sigma, \texttt{!}, R_1\sigma, \cdots, R_k\sigma$ \\[0.1cm]
      für ein $i\in\{1,\cdots,m\}$ scheitert, so wird der Cut nicht erreicht und hat keine
      Wirkung.
\item Eine Anfrage der Form \\[0.1cm]
      \hspace*{1.3cm} 
      $\texttt{!}, R_1\sigma, \cdots, R_k\sigma$ \\[0.1cm]
      wird reduziert zu \\[0.1cm]
      \hspace*{1.3cm} 
      $R_1\sigma, \cdots, R_k\sigma$. \\[0.1cm]
      Dabei werden alle Auswahl-Punkte, die bei der Beantwortung der Teilanfragen
      $Q_1, \cdots, Q_m$ gesetzt worden sind, gelöscht.  Außerdem wird ein eventuell bei
      der Reduktion der Anfrage $A$ auf die Anfrage \\[0.1cm]
      \hspace*{1.3cm} $Q_1\mu, \cdots, Q_m\mu, \texttt{!}, R_1\mu, \cdots, R_k\mu$
      \\[0.1cm]
      gesetzter Auswahl-Punkte gelöscht.
\item Sollte später die Beantwortung der Anfrage \\[0.1cm]
      \hspace*{1.3cm} $R_1\sigma, \cdots, R_k\sigma$. \\[0.1cm]
      scheitern, so scheitert auch die Beantwortung der Anfrage $A$.
\end{enumerate}
Zur Veranschaulichung betrachten wir ein Beispiel.
\begin{Verbatim}[ frame         = lines, 
                  framesep      = 0.3cm, 
                  labelposition = bottomline,
                  numbers       = left,
                  numbersep     = -0.2cm,
                  xleftmargin   = 0.8cm,
                  xrightmargin  = 0.8cm
                ]
    q(Z) :- p(Z).
    q(1).

    p(X) :- a(X), b(X), !, c(X,Y), d(Y).    
    p(3).
    
    a(1).    a(2).    a(3).
    
    b(2).    b(3).
    
    c(2,2).  c(2,4).
    
    d(3).
\end{Verbatim}
Wir verfolgen die Beantwortung der Anfrage \ \texttt{q(U)}. 
\begin{enumerate}
\item Zunächst wird versucht \texttt{q(U)} mit dem Kopf der ersten Klausel 
      des Prädikats \texttt{q/1}  zu unifizieren.  Dabei wird \texttt{Z} mit
      \texttt{U} instantiert und die Anfrage wird reduziert zu \\[0.1cm]
      \hspace*{1.3cm} \texttt{p(U)}. \\[0.1cm]
      Da es noch eine weiter Klausel für das Prädikat \texttt{q/1} gibt, die zur
      Beantwortung der Anfrage q(U) in Frage kommt, setzen wir Auswahl-Punkt Nr.~1.
\item Jetzt wird versucht \texttt{p(U)} mit \texttt{p(X)} zu unifizieren.  Dabei
      wird die Variable \texttt{X} an \texttt{U} gebunden und die ursprüngliche Anfrage
      wird reduziert zu der Anfrage \\[0.1cm]
      \hspace*{1.3cm}  
      \texttt{a(U), b(U), !, c(U,Y), d(Y)}. \\[0.1cm]
      Außerdem wird an dieser Stelle Auswahl-Punkt Nr.~2 gesetzt, denn die zweite Klausel
      des Prädikats \texttt{p/1} kann ja ebenfalls mit der ursprünglichen Anfrage unifiziert
      werden. 
\item Um die Teilanfrage \texttt{a(U)} zu beantworten, wird \texttt{a(U)} mit
      \texttt{a(1)} unifiziert.  Dabei wird \texttt{U} mit 1 instantiiert und die Anfrage wird reduziert zu \\[0.1cm]
      \hspace*{1.3cm} 
      \texttt{b(1), !, c(1,Y), d(Y)}. \\[0.1cm]
      Da es für das Prädikat \texttt{a/1} noch weitere Klauseln gibt, wird Auswahl-Punkt Nr.~3
      gesetzt.
\item Jetzt wird versucht, die Anfrage \\[0.1cm]
      \hspace*{1.3cm}    \texttt{b(1), !, c(1,Y), d(Y)}. \\[0.1cm]
      zu lösen.  Dieser Versuch scheitert jedoch, da sich die für das Prädikat \texttt{b/1}
      vorliegenden Fakten nicht mit \texttt{b(1)} unifizieren lassen.
\item Also springen wir zurück zum letzten Auswahl-Punkt (das ist Auswahl-Punkt Nr.~3)
      und machen die Instantiierung 
      $\texttt{U} \mapsto 1$ rückgängig.  Wir haben jetzt also wieder das Ziel \\[0.1cm]
      \hspace*{1.3cm} \texttt{a(U), b(U), !, c(U,Y), d(Y)}. 
\item Diesmal wählen wir das Fakt \texttt{a(2)} um es mit \texttt{a(U)} zu unifizieren.  Dabei wird \texttt{U} mit
      2 instantiiert und wir haben die  Anfrage \\[0.1cm]
      \hspace*{1.3cm} \texttt{b(2), !, c(2,Y), d(Y)}. \\[0.1cm]
      Da es noch eine weiter Klausel für das Prädikat \texttt{a/1} gibt, setzen wir 
      Auswahl-Punkt Nr.~4 an dieser Stelle.
\item Jetzt unifizieren wir  die Teilanfrage \texttt{b(2)} mit der ersten Klausel für das Prädikat
      \texttt{b/1}.  Die verbleibende Anfrage ist \\[0.1cm]
      \hspace*{1.3cm} \texttt{!, c(2,Y), d(Y)}. 
\item Diese Anfrage wird reduziert zu \\[0.1cm]
      \hspace*{1.3cm}  \texttt{c(2,Y), d(Y)}. \\[0.1cm]
      Außerdem werden bei diesem Schritt die Auswahl-Punkte Nr.~2 und Nr.~4 gelöscht.
\item Um diese Anfrage zu beantworten, unifizieren wir \texttt{c(2,Y)} mit dem Kopf der
      ersten Klausel für das Prädikat \texttt{c/2}, also mit \texttt{c(2,2)}.
      Dabei erhalten wir die Instantiierung
      $\mathtt{Y} \mapsto 2$.  Die Anfrage ist damit reduziert zu \\[0.1cm]
      \hspace*{1.3cm}  \texttt{d(2)}. \\[0.1cm]
      Außerdem setzen wir an dieser Stelle Auswahl-Punkt Nr.~5, denn das Prädikat
      \texttt{c/2} hat ja noch eine weitere Klausel, die in Frage kommt.
\item Die Anfrage ``\texttt{d(2)}'' scheitert.  Also springen wir zurück zum Auswahl-Punkt
      Nr.~5 und machen die Instantiierung $\mathtt{Y} \mapsto 2$ rückgängig.  Wir haben
      also wieder die Anfrage \\[0.1cm]
      \hspace*{1.3cm}  \texttt{c(2,Y), d(Y)}. 
\item Zur Beantwortung dieser Anfrage nehmen wir nun die zweite Klausel der
      Implementierung von \texttt{c/2} und erhalten die Instantiierung 
      $\mathtt{Y} \mapsto 4$.  Die verbleibende Anfrage lautet dann \\[0.1cm]
      \hspace*{1.3cm} \texttt{d(4)}.
\item Da sich \texttt{d(4)} und \texttt{d(3)} nicht unifizieren lassen,
      scheitert diese Anfrage.  Wir springen jetzt zurück zum Auswahl-Punkt Nr.~1 und
      machen die Instantiierung $\mathtt{U} \mapsto \mathtt{Z}$ rückgängig.  Die Anfrage
      lautet also wieder \\[0.1cm]
      \hspace*{1.3cm} \texttt{p(U)}.
\item Wählen wir nun die zweite Klausel der Implementierung von \texttt{q/1},
      so müssen wir \texttt{q(U)} und \texttt{q(1)} unifizieren.  Diese Unifikation ist
      erfolgreich und wir erhalten die Instantiierung $\mathtt{U} \mapsto 1$, die die Anfrage beantwortet.
\end{enumerate}

\subsection{Verbesserung der Effizienz von \textsl{Prolog}-Programmen durch den Cut-Operator}
In der Praxis wird der Cut-Operator eingesetzt, um überflüssige Auswahl-Punkte zu
entfernen und dadurch die Effizienz eines Programms zu steigern.  Als Beispiel
betrachten wir eine Implementierung des Algorithmus 
``\emph{Sortieren durch Vertauschen}'' (engl.~\emph{bubble sort}).
Wir spezifizieren diesen Algorithmus zunächst durch bedingte Gleichungen.  Dabei 
benutzen wir die Funktion \\[0.1cm]
\hspace*{1.3cm} 
$\texttt{append}: \textsl{List}(\textsl{Number}) \times \textsl{List}(\textsl{Number}) \rightarrow \textsl{List}(\textsl{Number})$
\\[0.1cm]
Der Aufruf $\texttt{append}(l_1, l_2)$ liefert eine Liste, die aus allen Elementen von
$l_1$ gefolgt von den Elementen aus $l_2$ besteht.  In dem \textsl{SWI-Prolog}-System ist
ein entsprechendes Prädikat \texttt{append/3} implementiert.  Die Implementierung dieses
Prädikats deckt sich mit der Implementierung des Prädikats \texttt{concat/3}, die wir in
einem früheren Abschnitt vorgestellt hatten.  

Außerdem benutzen wir noch das Prädikat \\[0.1cm]
\hspace*{1.3cm} 
$\texttt{ordered}: \textsl{List}(\textsl{Number}) \rightarrow \mathbb{B}$, \\[0.1cm]
das überprüft, ob eine Liste geordnet ist.
Die bedingten Gleichungen
zur Spezifikation der Funktion \\[0.1cm]
\hspace*{1.3cm} 
$\texttt{bubble\_sort}: \textsl{List}(\textsl{Number}) \rightarrow \textsl{List}(\textsl{Number})$
\\[0.1cm]
lauten nun:
\begin{enumerate}
\item $\mathtt{append}(l_1, [x,y|l_2], l) \wedge x > y \rightarrow \mathtt{bubble\_sort}(l) = \mathtt{bubble\_sort}(\mathtt{append}(l_1, [y,x|l_2]))$

      Wenn die Liste $l$ in zwei Teile $l_1$ und $[x,y|l_2]$ zerlegt werden kann
      und wenn weiter $x>y$ ist, dann vertauschen wir die Elemente $x$ und $y$
      und sortieren die so entstandene Liste rekursiv.
\item $\mathtt{ordered}(l) \rightarrow \mathtt{bubble\_sort}(l) = l$

      Wenn die Liste $l$ bereits sortiert ist, dann kann die Funktion
      $\mathtt{bubble\_sort}$ diese Liste unverändert als Ergebnis zurück geben.
\end{enumerate}
Die Gleichungen um das Prädikat \texttt{ordered} zu spezifizieren lauten:
\begin{enumerate}
\item $\mathtt{ordered}([]) = \mathtt{true}$

      Die leere Liste ist offensichtlich sortiert.
\item $\mathtt{ordered}([x]) = \mathtt{true}$

      Eine Liste, die nur aus einem Element besteht, ist ebenfalls sortiert.
\item $x \leq y \rightarrow \mathtt{ordered}([x,y|r]) = \mathtt{ordered}([y|r])$.

      Eine Liste der Form $[x,y|r]$ ist sortiert, wenn $x \leq y$ ist und
      wenn außerdem die Liste $[y|r]$ sortiert ist.
\end{enumerate}

\begin{figure}[!h]
  \centering
\begin{Verbatim}[ frame         = lines, 
                  framesep      = 0.3cm, 
                  labelposition = bottomline,
                  numbers       = left,
                  numbersep     = -0.2cm,
                  xleftmargin   = 0.8cm,
                  xrightmargin  = 0.8cm
                ]
    bubble_sort( L, Sorted ) :-
        append( L1, [ X, Y | L2 ], L ),
        X > Y,
        append( L1, [ Y, X | L2 ], Cs ),
        bubble_sort( Cs, Sorted ).
    
    bubble_sort( Sorted, Sorted ) :-
        is_ordered( Sorted ).
    
    
    is_ordered( [] ).
    
    is_ordered( [ _ ] ).
    
    is_ordered( [ X, Y | Ys ] ) :-
        X < Y,
        is_ordered( [ Y | Ys ] ).
\end{Verbatim}
\vspace*{-0.3cm}
  \caption{Der Bubble-Sort Algorithmus}
  \label{fig:bubble_sort}
\end{figure}

Abbildung \ref{fig:bubble_sort} zeigt die Implementierung des Bubble-Sort Algorithmus in
\textsl{Prolog}.  In Zeile 2 wird die als Eingabe gegebene Liste \texttt{L} in die beiden
Liste \texttt{L1} und \texttt{[X, Y | L2]} zerlegt.  Da es im Allgemeinen mehrere
Möglichkeiten gibt,  eine Liste in zwei Teillisten zu zerlegen, wird hierbei ein
Auswahl-Punkt gesetzt.  Anschließend wird geprüft, ob \texttt{Y} kleiner als \texttt{X}
ist.  Wenn dies der Fall ist, wird mit \texttt{append/3} die neue Liste 
\\[0.1cm]
\hspace*{1.3cm} $\mathtt{append(L_1, [Y,X|L_2])}$ \\[0.1cm]
gebildet und diese Liste wird rekuriv sortiert.  Wenn es nicht möglich ist,
die Liste \texttt{L} so in zwei Listen \texttt{L1} und \texttt{[X, Y | L2]} zu zerlegen,
dass \texttt{Y} kleiner als \texttt{X} ist, dann muss die Liste \texttt{L} schon sortiert
sein.  In diesem Fall greift die zweite Klausel, die allerdings noch überprüfen muss, ob
\texttt{L} tatsächlich sortiert ist, denn sonst könnte beim Backtracking eine falsche
Lösung berechnet werden.

Das Problem bei dem obigen Programm ist die Effizienz.  Aufgrund der vielen Möglichkeiten
eine Liste zu zerlegen, wird beim Backtracking immer wieder dieselbe Lösung generiert. 
Beispielsweise liefert die Anfrage \\[0.1cm]
\hspace*{1.3cm} \texttt{bubble\_sort( [ 4, 3, 2, 1 ], L ), write(L), nl, fail.} \\[0.1cm]
16 mal dieselbe Lösung.  Abbildung \ref{fig:bubble_sort_cut} zeigt eine Implementierung,
bei der nur eine Lösung berechnet wird.  Dies wird durch den Cut-Operator in Zeile 4
erreicht.   Ist einmal eine Zerlegung der Liste \texttt{L} der Form
\\[0.2cm]
\hspace*{1.3cm}
$\mathtt{L} = \texttt{L}_1 +[ \texttt{X}, \mathtt{Y} | \mathtt{L}_2]$
\\[0.2cm] 
gefunden, bei der \texttt{Y} kleiner als \texttt{X} ist, so bringt es
nichts mehr, nach anderen Zerlegungen zu suchen, denn die ursprünglich gegebene Liste
\texttt{L} lässt sich ja auf jeden Fall dadurch sortieren, dass rekursiv die Liste 
\\[0.2cm]
\hspace*{1.3cm}
 $\mathtt{L}_1 + [\mathtt{Y},\mathtt{X}|\mathtt{L}_2]$ 
\\[0.1cm]
sortiert wird.  Dann kann auch der Aufruf der Prädikats \texttt{ordered/1} im Rumpf der zweiten
Klausel des Prädikats \texttt/2{bubble\_sort} entfallen, denn diese wird beim Backtracking
ja nur dann erreicht, wenn es keine Zerlegung der Liste \texttt{L} in  \texttt{L1} und 
\texttt{[X, Y | L2]} gibt, bei der \texttt{Y} kleiner als \texttt{X} ist.  Dann muss aber
die Liste \texttt{L} schon sortiert sein.

\begin{figure}[!h]
  \centering
\begin{Verbatim}[ frame         = lines, 
                  framesep      = 0.3cm, 
                  labelposition = bottomline,
                  numbers       = left,
                  numbersep     = -0.2cm,
                  xleftmargin   = 0.8cm,
                  xrightmargin  = 0.8cm
                ]
    bubble_sort( List, Sorted ) :-
        append( L1, [ X, Y | L2 ], List ),
        X > Y,
        !,
        append( L1, [ Y, X | L2 ], Cs ),
        bubble_sort( Cs, Sorted ).
    
    bubble_sort( Sorted, Sorted ).
\end{Verbatim}
\vspace*{-0.3cm}
  \caption{Effiziente Implementierung des  Bubble-Sort Algorithmus}
  \label{fig:bubble_sort_cut}
\end{figure}

Wenn wir bei der Entwicklung eines \textsl{Prolog}-Programms von bedingten Gleichungen
ausgehen, dann gibt es ein einfaches Verfahren, um das entstandene
\textsl{Prolog}-Programm durch die Einführung von Cut-Operator effizienter zu machen:
Der Cut-Operator sollte nach den Tests, die vor dem Junktor ``$\rightarrow$'' stehen,
gesetzt werden.  Wir erläutern dies durch ein Beispiel:  Unten sind noch einmal die
Gleichungen zur Spezifikation des Algorithmus ``\emph{Sortieren durch Mischen}'' wiedergegeben.
\begin{enumerate}
\item $\mathtt{odd}([]) = []$.
\item $\mathtt{odd}([h|t]) = [h|\mathtt{even}(t)]$.
\item $\mathtt{even}([]) = []$.
\item $\mathtt{even}([h|t]) = \mathtt{odd}(t)$.
\item $\mathtt{merge}([], l) = l$.
\item $\mathtt{merge}(l, []) = l$.
\item $x \leq y \rightarrow \mathtt{merge}([x|s], [y|t]) = [x|\mathtt{merge}(s, [y|t])]$.
\item $x  >   y \rightarrow \mathtt{merge}([x|s], [y|t]) = [y|\mathtt{merge}([x|s], t)]$.
\item $\mathtt{sort}([]) = []$.
\item $\mathtt{sort}([x]) = [x]$.
\item $\mathtt{sort}([x,y|t]) = \mathtt{merge}( \mathtt{sort}(\mathtt{odd}([x,y|t])), \mathtt{sort}(\mathtt{even}([x,y|t])))$.
\end{enumerate}
Das \textsl{Prolog}-Programm mit Cut-Operatoren sieht dann so aus wie in Abbildung
\ref{fig:merge-sort-cut} gezeigt.  Nur die Gleichungen 7.~und 8.~haben Bedingungen, bei
allen anderen Gleichungen gibt es keine Bedingungen.  Bei den Gleichungen 7.~und 8.~wird
der Cut-Operator daher nach dem Test der Bedingungen gesetzt, bei allen anderen Klauseln
wird der Cut-Operator dann am Anfang des Rumpfes gesetzt.

\begin{figure}[!h]
  \centering
\begin{Verbatim}[ frame         = lines, 
                  framesep      = 0.3cm, 
                  labelposition = bottomline,
                  numbers       = left,
                  numbersep     = -0.2cm,
                  xleftmargin   = 0.8cm,
                  xrightmargin  = 0.8cm
                ]
    % odd( +List(Number), -List(Number) ).    
    odd( [], [] ) :- !.    
    odd( [ X | Xs ], [ X | L ] ) :-
        !,
        even( Xs, L ).

    % even( +List(Number), -List(Number) ).
    even( [], [] ) :- !.
    even( [ _X | Xs ], L ) :-
        !,
        odd( Xs, L ).
    
    % merge( +List(Number), +List(Number), -List(Number) ).
    mix( [], Xs, Xs ) :- !.    
    mix( Xs, [], Xs ) :- !.
    mix( [ X | Xs ], [ Y | Ys ], [ X | Rest ] ) :-
        X =< Y,
        !,
        mix( Xs, [ Y | Ys ], Rest ).
    mix( [ X | Xs ], [ Y | Ys ], [ Y | Rest ] ) :-
        X > Y,
        !,
        mix( [ X | Xs ], Ys, Rest ).
    
    % merge_sort( +List(Number), -List(Number) ).
    merge_sort( [], [] ) :- !.
    merge_sort( [ X ], [ X] ) :- !.
    merge_sort( [ X, Y | Rest ], Sorted ) :-
        !,
        odd(  [ X, Y | Rest ], Odd  ),
        even( [ X, Y | Rest ], Even ),
        merge_sort( Odd,  Odd_Sorted  ),
        merge_sort( Even, Even_Sorted ),
        mix( Odd_Sorted, Even_Sorted, Sorted ).
\end{Verbatim}
\vspace*{-0.3cm}
  \caption{Sortieren durch Mischen mit Cut-Operatoren.}
  \label{fig:merge-sort-cut}
\end{figure}

Analysieren wir das obige Programm genauer, so stellen wir fest, dass viele der
Cut-Operatoren im Grunde überflüssig sind.  Beispielsweise kann immer nur eine der beiden
Klauseln, die das Prädikat \texttt{odd/2} implementieren, greifen, denn entweder ist die
eingegebene Liste leer oder nicht.  Also sind die Cut-Operatoren in den Zeilen 2 und 4
redundant.  Andererseits stören sie auch nicht, so dass es für die Praxis das einfachste
sein dürfte Cut-Operatoren stur nach dem oben angegebenen Rezept zu setzen.
\pagebreak

\vspace*{\fill}

\section{Literaturhinweise}
Für eine umfangreiche und dem Thema angemessene Darstellung der Sprache \textsl{Prolog} 
fehlt in der einführenden Vorlesung leider die Zeit.  Daher wird dieses Thema in einer
späteren Vorlesung auch wieder aufgegriffen.  Den Lesern, die ihre Kenntnisse jetzt schon
vertiefen wollen, möchte ich die folgende Hinweise auf die Literatur geben:
\begin{enumerate}
\item \emph{The Art of Prolog} von Leon Sterling und Ehud Shapiro \cite{sterling:94}.
      Dieses Werk ist ein ausgezeichnete Lehrbuch, das auch für den Anfänger gut lesbar ist.
\item \emph{Prolog Programming for Artificial Intelligence} von Ivan Bratko
      \cite{bratko:90}.  Neben der Sprache \textsl{Prolog} führt dieses Buch auch in die
      künstliche Intelligenz ein.
\item \emph{Foundations of Logic Programnming}   von J.~W.~Lloyd \cite{lloyd:87}
      beschreibt die theoretischen Grundlagen der Sprache \textsl{Prolog}.
\item \emph{Prolog: The Standard} von Pierre Deransart, Abdel Ali Ed-Dbali und Laurent
      Cervoni \cite{deransart:96} gibt den ISO-Standard für die Sprache
      \textsl{Prolog} wieder.
\item \emph{SWI-Prolog 6.2.6 Reference Manual} von Jan Wielemaker \cite{wielemaker:2013}
      beschreibt das SWI-Prolog-System.  Dieses Dokument ist im Internet unter der Adresse 
      \\[0.1cm]
      \hspace*{1.3cm}      
      \href{http://www.swi-prolog.org/download/stable/doc/SWI-Prolog-6.2.6.pdf}{\texttt{http://www.swi-prolog.org/download/stable/doc/SWI-Prolog-6.2.6.pdf}}
      \\[0.1cm]
      verfügbar.
\end{enumerate}

%%% Local Variables: 
%%% mode: latex
%%% TeX-master: "logik"
%%% End: 
