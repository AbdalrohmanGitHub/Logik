\section{Probleme der deklarativen Semantik}
Der der \textsl{Prolog-System} zugrunde liegende Algorithmus sucht systematisch alle
Antworten auf eine Anfrage.  Dass hei\3t aber nicht,
dass er auch alle Antworten  findet!   Das folgende Programm zeigt das Problem:
\begin{verbatim}
    loop :- loop.
    loop.
\end{verbatim}
Wird dieses Programm geladen und wird anschlie\3end versucht, das Ziel \texttt{loop} zu
l\"{o}sen, so liefert der Prolog-Interpreter die Meldung
\begin{verbatim}
    ERROR: Out of local stack
\end{verbatim}

In manchen F\"{a}llen kann  das Problem der Nicht-Terminierung 
 durch eine Umordnung der Programm-Klauseln umgangen werden.
H\"{a}tten wir das obige Programm in  der Form
\begin{verbatim}
  loop.
  loop :- loop.
\end{verbatim}
eingegeben, so w\"{a}re das Problem der Endlos-Schleife nicht aufgetreten.  Leider gibt es bei
komplexeren Programmen oft keine Anordnung der Klauseln, die das Problem der
Endlos-Schleifen l\"{o}st.  

Die handels\"{u}blichen Prolog-Interpreter haben noch ein weiteres Problem: Die Unifikation,
das hei\3t der Algorithmus zum L\"{o}sen syntaktischer Gleichungen, ist (aus Effizienzgr\"{u}nden)
nicht korrekt implementiert.  Wir erinnern noch einmal an die Regeln zur Reduktion
syntaktischer Gleichungen.  Die erste dieser Regeln lautete:
\begin{enumerate}
\item Falls $y\in\mathcal{V}$ eine Variable ist, {\em die nicht in dem Term $t$ auftritt}, so
      k\"{o}nnen wir die folgende Reduktion durchf\"{u}hren: \\[0.1cm]
      \hspace*{2.1cm} $\Big\langle E \cup \big\{ y \doteq t \big\}, \sigma \Big\rangle \quad\leadsto\quad \Big\langle E[y \mapsto t], \sigma\big[ y \mapsto t \big] \Big\rangle$.
\end{enumerate}
In den handels\"{u}blichen Prolog-Interpretern ist statt dieser Regel die folgende Regel
implementiert:
\begin{enumerate}
\item[$1'$.] Falls $y\in\mathcal{V}$ eine Variable ist, so
      k\"{o}nnen wir die folgende Reduktion durchf\"{u}hren: \\[0.1cm]
      \hspace*{2.1cm} $\Big\langle E \cup \big\{ y \doteq t \big\}, \sigma \Big\rangle \quad\leadsto\quad \Big\langle E[y \mapsto t], \sigma\big[ y \mapsto t \big] \Big\rangle$.
\end{enumerate}
Was hier fehlt ist der Test {\em die nicht in dem Term $t$ auftritt}.  Dieser Test
wird als \emph{Occur-Check} bezeichnet.  Mit der modifizierten Regel k\"{o}nnen wir eine
Reduktion der Form
\\[0.1cm]
\hspace*{1.3cm} $\langle \{ x \doteq f(x) \}, [] \rangle \;\leadsto\; \langle \emptyset, [ x \mapsto f(x) ] \rangle$ \\[0.1cm]
durchf\"{u}hren.  Dies liefert aber ein falsches Ergebnis, denn offensichtlich ist die Substitution 
$[x \mapsto f(x)]$ keine L\"{o}sung der syntaktischen Gleichung $x \doteq f(x)$.  
Ein  Programm $\mathcal{P}$, wo dieser Fall tats\"{a}chlich auftritt, ist durch die beiden folgenden Klausel gegeben: 
\begin{verbatim}
      bug :- equal(X, s(X)).
      equal(X,X).
\end{verbatim}
Betrachten wir das Ziel \texttt{bug}, so liefert Prolog hier die Antwort ``\texttt{yes}'', obwohl \\[0.1cm]
\hspace*{1.3cm} $\mathcal{P} \not\models \mathtt{bug}$ \\[0.1cm]
gilt.
Bei Prolog bleibt es dem Programmierer \"{u}berlassen sicherzustellen, dass dergleichen nicht 
passiert.

%%% Local Variables: 
%%% mode: latex
%%% TeX-master: "informatik-script.tex"
%%% End: 
