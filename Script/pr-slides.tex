\documentclass{slides}
\usepackage{german}
\usepackage[latin1]{inputenc}
\usepackage{epsfig}
\usepackage{fancyvrb}
\usepackage{epsfig}
\usepackage{amssymb}

\pagestyle{empty}
\setlength{\textwidth}{20cm}
\setlength{\textheight}{24cm}
\setlength{\topmargin}{0cm}
\setlength{\headheight}{0cm}
\setlength{\headsep}{0cm}
\setlength{\topskip}{0.2cm}
\setlength{\oddsidemargin}{0.5cm}
\setlength{\evensidemargin}{0.5cm}

\newcommand{\myrule}{\rule{20cm}{1mm}\\ }

\def\pair(#1,#2){\langle #1, #2 \rangle}
\newcommand{\is}{\;|\;}
\newcommand{\schluss}[2]{\frac{\displaystyle\quad \rule[-8pt]{0pt}{18pt}#1 \quad}{\displaystyle\quad \rule{0pt}{14pt}#2 \quad}}
\newcommand{\vschlus}[1]{{\displaystyle\rule[-6pt]{0pt}{12pt} \atop \rule{0pt}{10pt}#1}}
\newcommand{\verum}{\top}
\newcommand{\falsum}{\bot}
\newcommand{\var}{\textsl{Var}}
\newcommand{\struct}{\mathcal{S}}
\newcommand{\FV}{\textsl{FV}}
\newcommand{\BV}{\textsl{BV}}
\newcommand{\gentzen}{\vdash_\mathcal{G}}
\newcommand{\komplement}[1]{\overline{#1}}

\newcommand{\circneg}{\mbox{$\bigcirc\hspace*{-0.54cm}\neg$}}
\newcommand{\circwedge}{\mbox{$\bigcirc\hspace*{-0.52cm}\wedge$}}
\newcommand{\circvee}{\mbox{$\bigcirc\hspace*{-0.52cm}\vee$}}
\newcommand{\circright}{\mbox{$\bigcirc\hspace*{-0.79cm}\rightarrow$}}
\newcommand{\circleftright}{\mbox{$\bigcirc\hspace*{-0.79cm}\leftrightarrow$}}

\newcounter{mypage}

\begin{document}
\begin{center}
Pr\"{a}dikatenlogik
\end{center}
\vspace*{0.5cm}

\footnotesize
Neue Begriffe:
\begin{enumerate}
\item \emph{Terme}: Bezeichnungen f\"{u}r Objekte

       Beispiel: $f(x, 1)$

       Terme sind aufgebaut aus
\item \emph{Variablen} und \emph{Funktions-Zeichen}: \\

      Beispiel:\\[0.1cm]
      \hspace*{1.3cm} 
      $x$: Variable; \\
      \hspace*{1.3cm} 
      $1$, $f$: Funktions-Zeichen
\item \emph{Spezifikation der Stelligkeit}: gew\"{a}hrleistet, \\
      dass Terme sinnvoll sind

      Beispiel: $\mathtt{arity}(f) = 2$, \quad $\mathtt{arity}(1) = 0$.
\item \emph{Pr\"{a}dikats-Zeichen}: atomare Formeln

      Beispiel: $\textsl{even}\bigg(f(1,x)\bigg)$ \\[0.1cm]
      \hspace*{1.3cm} $\textsl{arity}(even) = 1$
\item \emph{Quantoren}: Aussagen \"{u}ber Mengen 

      Beispiele: \\[0.1cm]
\hspace*{1.3cm} $\forall x \in \mathbb{N}:\textsl{even}\bigg(f(1,x)\bigg)$\\[0.1cm]
\hspace*{1.3cm} $\exists x \in \mathbb{N}:\textsl{even}\bigg(f(1,x)\bigg)$\\[0.1cm]
\end{enumerate}


\vspace*{\fill}
\tiny \addtocounter{mypage}{1} 
\myrule
Pr\"{a}dikatenlogik -- Syntax \hspace*{\fill} Seite \arabic{mypage}


%%%%%%%%%%%%%%%%%%%%%%%%%%%%%%%%%%%%%%%%%%%%%%%%%%%%%%%%%%%%%%%%%%%%%%%%

\begin{slide}{}
\normalsize
\begin{center}
Signatur
\end{center}
\vspace{0.5cm}

\footnotesize
\textbf{Definition}: Eine Signatur $\Sigma$ ist ein $4$-Tupel  \\[0.3cm]
\hspace*{1.3cm} $\Sigma = \langle \mathcal{V}, \mathcal{F}, \mathcal{P},
\mathtt{arity}\rangle$.  \quad
Dabei gilt:
\begin{enumerate}
\item $\mathcal{V}$:  Menge der \emph{Variablen}
\item $\mathcal{F}$:  Menge der \emph{Funktions-Zeichen}
\item $\mathcal{P}$:  Menge der \emph{Pr\"{a}dikats-Zeichen}
\item $\textsl{arity}: \mathcal{F} \cup \mathcal{P} \rightarrow \mathbb{N}$ \\[0.1cm]
      ordnet Funktions und Pr\"{a}dikats-Zeichen Stelligkeit zu.
\item Die Mengen $\mathcal{V}$, $\mathcal{F}$ und $\mathcal{P}$ sind paarweise disjunkt: 
      \begin{enumerate}
      \item $\mathcal{V} \cap \mathcal{F} = \emptyset$
      \item $\mathcal{V} \cap \mathcal{P} = \emptyset$
      \item $\mathcal{F} \cap \mathcal{P} = \emptyset$.
      \end{enumerate}
\end{enumerate}

Geg.: Signatur $\Sigma = \langle \mathcal{V}, \mathcal{F}, \mathcal{P}, \textsl{arity}\rangle$ \\[0.3cm]
\hspace*{1.3cm} 
Induktive Definition der $\Sigma$-Terme $\mathcal{T}_\Sigma$:
\begin{enumerate}
\item $x \in \mathcal{T}_\Sigma$ \quad f\"{u}r alle $x \in \mathcal{V}$.
\item Falls $f \in \mathcal{F}$ mit $\mathtt{arity}(f) = n$  und $t_1,\cdots,t_n \in\mathcal{T}_\Sigma$, dann  \\[0.1cm]
      \hspace*{1.3cm} $f(t_1,\cdots,t_n) \in \mathcal{T}_\Sigma$.
 
      Schreibweise: Falls $\textsl{arity}(f) = 0$, dann \\[0.1cm]
      \hspace*{1.3cm} $c$ \quad statt \quad $c()$.
\end{enumerate}
\begin{center}
{\normalsize
Terme $\approx$ Namen f\"{u}r Objekte }
\end{center}

\vspace*{\fill}
\tiny \addtocounter{mypage}{1}
\myrule
Pr\"{a}dikatenlogik -- Syntax  \hspace*{\fill} Seite \arabic{mypage}
\end{slide}

%%%%%%%%%%%%%%%%%%%%%%%%%%%%%%%%%%%%%%%%%%%%%%%%%%%%%%%%%%%%%%%%%%%%%%%%

\begin{slide}{}
\normalsize
\begin{center}
Beispiel
\end{center}
\vspace{0.5cm}

\footnotesize
$\Sigma_\mathrm{arith} = \langle \mathcal{V}, \mathcal{F}, \mathcal{P},\textsl{arity}\rangle$ \quad mit
\begin{enumerate}
\item $\mathcal{V} = \{ x, y, z \}$
\item $\mathcal{F} = \{ 0, 1, \mathtt{+}, \mathtt{-}, \mathtt{*} \}$
\item $\mathcal{P} = \{\mathtt{=}, \leq\}$ 
\item $\textsl{arity} = \bigl\{ \pair(0,0), \pair(1,0), \pair(\mathtt{+},2),\pair(\mathtt{-},2), \pair(\mathtt{*},2) \bigr\}$
\end{enumerate}

Beispiele f\"{u}r  $\Sigma_{\mathrm{arith}}$-Terme:
\begin{enumerate}
\item $x, y, z \in \mathcal{T}_{\Sigma_{\mathrm{arith}}}$
\item $0, 1 \in \mathcal{T}_{\Sigma_{\mathrm{arith}}}$
\item $\mathtt{+}(0,x) \in \mathcal{T}_{\Sigma_{\mathrm{arith}}}$
\item $\mathtt{*}(\mathtt{+}(0,x),1) \in \mathcal{T}_{\Sigma_{\mathrm{arith}}}$
\end{enumerate}

\vspace*{\fill}
\tiny \addtocounter{mypage}{1}
\myrule
Pr\"{a}dikatenlogik -- Syntax  \hspace*{\fill} Seite \arabic{mypage}
\end{slide}


%%%%%%%%%%%%%%%%%%%%%%%%%%%%%%%%%%%%%%%%%%%%%%%%%%%%%%%%%%%%%%%%%%%%%%%%

\begin{slide}{}
\normalsize
\begin{center}
Atomare Formeln
\end{center}
\vspace{0.5cm}

\footnotesize
\textbf{Definition} der atomaren Formeln \\[0.1cm]
\textbf{Geg.}: Signatur $\Sigma = \langle \mathcal{V}, \mathcal{F}, \mathcal{P},
\textsl{arity}\rangle$. \quad Sei
\begin{enumerate}
\item  $p \in \mathcal{P}$ \quad mit \quad $\textsl{arity}(p) = n$,
\item  $t_i \in \mathcal{T}_{\Sigma}$ \quad f\"{u}r alle $i\in\{1,\cdots,n\}$. 
\end{enumerate}
Dann ist \\[0.3cm]
\hspace*{1.3cm}  $p(t_1, \cdots, t_n) \in \mathcal{A}_\Sigma$ 

\begin{center}
{\normalsize
  $\mathcal{A}_\Sigma$  $\approx$ Menge der atomaren $\Sigma$-Formeln}
\end{center}

\textbf{Beispiele}:
\begin{enumerate}
\item $\mathtt{=}(\mathtt{*}(\mathtt{+}(0,x),1),0)$ 
\item $\leq(y,\mathtt{*}(\mathtt{+}(0,x),1))$ 
\end{enumerate}

Infix-Notation zur Klammer-Ersparnis:
\begin{enumerate}
\item $(0+x)*1 = 0$
\item $y \leq (0+x)*1$ 
\end{enumerate}
Notwendig dazu: Festlegung von 
\begin{enumerate}
\item Bindungs-St\"{a}rke
\item Assoziativit\"{a}t
\end{enumerate}

\vspace*{\fill}
\tiny \addtocounter{mypage}{1}
\myrule
Pr\"{a}dikatenlogik -- Syntax  \hspace*{\fill} Seite \arabic{mypage}
\end{slide}

%%%%%%%%%%%%%%%%%%%%%%%%%%%%%%%%%%%%%%%%%%%%%%%%%%%%%%%%%%%%%%%%%%%%%%%%

\begin{slide}{}
\normalsize
\begin{center}
Gebundene und freie Variablen
\end{center}
\vspace{0.5cm}

\footnotesize
\hspace*{1.3cm} $\int\limits_{0}^{x} y * t\, d t = \frac{1}{2} x^2 * y$ \hspace*{1.3cm} in Analysis g\"{u}ltig
\begin{enumerate}
\item $x$, $y$: freie Variablen
\item $t$ gebundene Variable
\end{enumerate}

Unterschiede zwischen gebundenen und freien Variablen
\begin{enumerate}
\item Freie Variablen: beliebige Werte einsetzbar, $x \mapsto 2$: \\[0.3cm]
\hspace*{1.3cm}  $\int\limits_{0}^{2} y * t\, d t = \frac{1}{2} 2^2 * y$ 
\item Gebundene Variablen: umbenennbar \\[0.1cm]
\hspace*{1.3cm} $\int\limits_{0}^{x} y * u\, d u = \frac{1}{2} x^2 * y$ 
\item Vorsicht beim Umbenennen: \\[0.3cm]
\hspace*{1.3cm}  $\int\limits_{0}^{x} y * y\, d y = \frac{1}{2} x^2 * y$. \\[0.3cm]
      Durch Umbenennen k\"{o}nnen \underline{falsche} Formeln entstehen, wenn freie Variablen
      \emph{eingefangen} werden!  
\item Freie Variablen: Terme i.a.~nicht einsetzbar: $y \mapsto t^2$ \\[0.3cm]
\hspace*{1.3cm} $\int\limits_{0}^{x} t^2 * t\, d t = \frac{1}{2} x^2 * t^2$ \\[0.1cm]
\end{enumerate}


\vspace*{\fill}
\tiny \addtocounter{mypage}{1}
\myrule
Pr\"{a}dikatenlogik -- Syntax  \hspace*{\fill} Seite \arabic{mypage}
\end{slide}

%%%%%%%%%%%%%%%%%%%%%%%%%%%%%%%%%%%%%%%%%%%%%%%%%%%%%%%%%%%%%%%%%%%%%%%%

\begin{slide}{}
\normalsize
\begin{center}
$\Sigma$-Formeln
\end{center}
\vspace{0.5cm}

\footnotesize
\textbf{Definition}: 
F\"{u}r  $t \in \mathcal{T}_\Sigma$ definiere $\var(t)$ induktiv:
\begin{enumerate}
\item $\var(x) := \{ x \}$ \quad f\"{u}r $x \in \mathcal{V}$.
\item $\var\Bigg( f(t_1,\cdots,t_n) \Bigg) := \var(t_1) \cup \cdots \cup \var(t_n)$ 
\end{enumerate}

\textbf{Definiere} induktiv:
\begin{itemize}
\item $\mathbb{F}_\Sigma$: Menge der $\Sigma$-Formeln.
\item $\FV(f)$: Menge der \emph{freien Variablen} in $f$.
\item $\BV(f)$:            Menge der \emph{gebundenen Variablen} in $f$.
\end{itemize}
\begin{enumerate}
\item  $\falsum \in \mathbb{F}_\Sigma$, $\verum \in \mathbb{F}_\Sigma$  \\[0.1cm]
       $\FV(\falsum) := \FV(\verum) := \BV(\falsum) := \BV(\verum) := \emptyset$
\item $p(t_1,\cdots,t_n) \in \mathbb{F}_\Sigma$ falls $p(t_1,\cdots,t_n)$   atomare $\Sigma$-Formel  \\[0.3cm]
      $\FV\Bigg(p(t_1,\cdots,t_n) \Bigg) := \var(t_1) \cup \cdots \cup \var(t_n)$ \\
      $\BV\Bigg(p(t_1,\cdots,t_n) \Bigg) := \emptyset$
\item $\neg f \in \mathbb{F}_\Sigma$ falls $f \in \mathbb{F}_\Sigma$ \\[0.3cm]
      $\FV\bigl( \neg f \bigr) := \FV(f)$ \quad und \quad $\BV\bigl( \neg f \bigr) := \BV(f)$
\end{enumerate}


\vspace*{\fill}
\tiny \addtocounter{mypage}{1}
\myrule
Pr\"{a}dikatenlogik -- Syntax  \hspace*{\fill} Seite \arabic{mypage}
\end{slide}

%%%%%%%%%%%%%%%%%%%%%%%%%%%%%%%%%%%%%%%%%%%%%%%%%%%%%%%%%%%%%%%%%%%%%%%%

\begin{slide}{}
\normalsize
\begin{center}
$\Sigma$-Formeln (Fortsetzung)
\end{center}
\vspace{0.5cm}

\footnotesize
\begin{enumerate}
\item[4.] Sei $f, g \in \mathbb{F}_\Sigma$ und gelte \\[0.3cm]
      \hspace*{1.3cm} $\FV(f) \cap \BV(g) = \emptyset$ \quad und \quad
                      $\FV(g) \cap \BV(f) = \emptyset$. \\[0.3cm]
      Dann:
      \begin{enumerate}
      \item $(f \wedge g) \in \mathbb{F}_\Sigma$,
      \item $(f \vee g) \in \mathbb{F}_\Sigma$,
      \item $(f \rightarrow g) \in \mathbb{F}_\Sigma$,
      \item $(f \leftrightarrow g) \in \mathbb{F}_\Sigma$.
      \end{enumerate}
      Weiter gilt f\"{u}r alle $\odot \in \{ \wedge, \vee, \rightarrow, \leftrightarrow \}$:
      \begin{enumerate}
      \item $\FV\bigl( f \odot g \bigr) := \FV(f) \cup \FV(g)$.
      \item $\BV\bigl( f \odot g \bigr) := \BV(f) \cup \BV(g)$.
      \end{enumerate}
\item[5.] Sei  $x \in {\cal V}$  und $f \in \mathbb{F}_\Sigma$ mit $x \not\in \BV(f)$. \\[0.3cm]
      Dann gilt:
      \begin{enumerate}
      \item $(\forall x : f) \in \mathbb{F}_\Sigma$.
      \item $(\exists x : f) \in \mathbb{F}_\Sigma$.
      \end{enumerate}
      Weiter gilt f\"{u}r alle $\mathcal{Q} \in \{ \forall, \exists \}$:
      \begin{enumerate}
      \item $\FV\Bigg( (\mathcal{Q} x : f) \Bigg) := \FV(f) \backslash \{x\}$.
      \item $\BV\Bigg( (\mathcal{Q} x : f) \Bigg) := \BV(f) \cup \{x\}$.  
      \end{enumerate}
\end{enumerate}


\vspace*{\fill}
\tiny \addtocounter{mypage}{1}
\myrule
Pr\"{a}dikatenlogik -- Syntax  \hspace*{\fill} Seite \arabic{mypage}
\end{slide}

%%%%%%%%%%%%%%%%%%%%%%%%%%%%%%%%%%%%%%%%%%%%%%%%%%%%%%%%%%%%%%%%%%%%%%%%

\begin{slide}{}
\normalsize
\begin{center}
Klammer-Ersparnis
\end{center}
\vspace{0.5cm}

\footnotesize
\begin{enumerate}
\item Selbe Regeln wie in Aussagenlogik. Bindungsst\"{a}rke: 
  1: ``$\neg$'', \ 2: ``$\wedge$'' und ``$\vee$'', \ 3: ``$\rightarrow$'', \ 4: ``$\leftrightarrow$''.
\item \"{a}u\3ere Klammern werden weggelassen.
\item Zusammenfassung von gleichen Quantoren: \\[0.3cm]
      \hspace*{1.3cm} $\forall x, y: p(x, y)$  
      statt \\[0.3cm]
      \hspace*{1.3cm} $\forall x: \Bigg( \forall y: p(x,y) \Bigg)$. 
\item Wirkungs-Bereich von Quantoren maximal: \\[0.3cm]
      \hspace*{1.3cm} $\forall x\in\tau: f \wedge g$ \quad statt \quad $\forall x\in\tau: (f \wedge g)$
\item 0-stellige Funktions-Zeichen werden ohne Klammer geschrieben: \\[0.3cm]
      \hspace*{1.3cm} $1$ \quad statt \quad $1()$.
\item Zweistellige Funktions-Zeichen in Infix-Notation,
      wenn Bindungsst\"{a}rke klar ist: \\[0.3cm]
      \hspace*{1.3cm} $x + y * z$ \quad statt \quad $+(x, *(y,z))$.
\item Zweistellige Pr\"{a}dikats-Zeichen in Infix-Notation: \\[0.3cm]
      \hspace*{1.3cm} $x < y$ \quad statt \quad $<(x, y)$.
\end{enumerate}

\vspace*{\fill}
\tiny \addtocounter{mypage}{1}
\myrule
Pr\"{a}dikatenlogik -- Syntax  \hspace*{\fill} Seite \arabic{mypage}
\end{slide}

%%%%%%%%%%%%%%%%%%%%%%%%%%%%%%%%%%%%%%%%%%%%%%%%%%%%%%%%%%%%%%%%%%%%%%%%

\begin{slide}{}
\normalsize
\begin{center}
Beispiele f\"{u}r $\Sigma$-Formeln (Gruppen-Theorie)
\end{center}
\vspace{0.5cm}

\footnotesize
\begin{enumerate}
\item $\Sigma_G = \langle \mathcal{V}, \mathcal{F}, \mathcal{P},\textsl{arity}\rangle$ mit
\item $\mathcal{V} := \{x_i | i \in \mathbb{N} \}$
\item $\mathcal{F} := \{ 1, \mathtt{*} \}$
\item $\mathcal{P} := \{ \mathtt{=} \}$
\item $\textsl{arity} = \bigl\{ \pair(1,0), \pair(\mathtt{*},2), \pair(\mathtt{=},2)\bigr\}$
\item Axiome der Gruppen-Theorie:
  \begin{enumerate}
  \item $1 * x_1 = x_1$
  \item $(x_1 * x_2) * x_3 = x_1 * (x_2 * x_3)$
  \item $\forall x_1: \exists x_2: x_2 * x_1 = 1$
  \end{enumerate}
\item Theoreme der Gruppen-Theorie:
  \begin{enumerate}
  \item $x_1 * 1= x_1$
  \item $\forall x_1: \exists x_2: x_1 * x_2 = 1$
  \end{enumerate}
\item Nicht-Theoreme der Gruppen-Theorie:
  \begin{enumerate}
  \item $x_1 * x_2 = x_2 * x_1$
  \item $\forall x_1: \exists x_2: x_2 * x_2 = x_1$
  \end{enumerate}
\end{enumerate}

\vspace*{\fill}
\tiny \addtocounter{mypage}{1}
\myrule
Pr\"{a}dikatenlogik -- Syntax  \hspace*{\fill} Seite \arabic{mypage}
\end{slide}

%%%%%%%%%%%%%%%%%%%%%%%%%%%%%%%%%%%%%%%%%%%%%%%%%%%%%%%%%%%%%%%%%%%%%%%%

\begin{slide}{}
\normalsize
\begin{center}
Pr\"{a}dikatenlogik -- Semantik
\end{center}
\vspace*{0.5cm}

\footnotesize
\textbf{Def.}:   $\Sigma = \langle \mathcal{V}, \mathcal{F}, \mathcal{P},\textsl{arity}\rangle$ sei Signatur. 

\emph{$\Sigma$-Struktur} $\struct$ ist Paar 
 $\langle \mathcal{U}, \mathcal{J} \rangle$ mit:
\begin{enumerate}
    \item $\mathcal{U}$ ist eine nicht-leere Menge. (\emph{Universum})
    \item $\mathcal{J}$ ist eine Abbildung mit:
    \begin{enumerate}
    \item F\"{u}r alle $f \in \mathcal{F}$ mit $\textsl{arity}(f) = m$ gilt \\[0.3cm]
          \hspace*{1.3cm}
          $f^\mathcal{J}\colon \mathcal{U} \times \cdots \times \mathcal{U} \rightarrow \mathcal{U}$ 
    \item F\"{u}r alle $p \in \mathcal{P}$ mit $\textsl{arity}(p) = n$ gilt \\[0.3cm]
          \hspace*{1.3cm} 
          $p^\mathcal{J}\colon \mathcal{U} \times \cdots \times \mathcal{U} \rightarrow \mathbb{B}$
    \item Falls  $=\,\in\mathcal{P}$, dann \\[0.1cm]
          \hspace*{1.3cm}  $=^\mathcal{J}(u,v) = \mathtt{true}$ \quad g.d.w. \quad $u = v$. \\[0.1cm]
    \end{enumerate}
    $\mathcal{J}$ ist \emph{Interpretation} der Funktions- und \\[0.1cm]
    \hspace*{0.6cm} Pr\"{a}dikats-Zeichen
\end{enumerate}

\textbf{Beispiel}: Sei $\Sigma_G$ Signatur zur Gruppen-Theorie

Definiere $\Sigma_G$ Struktur $\mathcal{Z} = \langle \{a, b\}, \mathcal{J}\rangle$ mit:
\begin{enumerate}
\item $1^\mathcal{J} := a$ 
\item $*^\mathcal{J} := \Bigl\{ \bigl\langle\pair(a,a), a\bigr\rangle,
                                   \bigl\langle\pair(a,b), b\bigr\rangle,
                                   \bigl\langle\pair(b,a), b\bigr\rangle,
                                   \bigl\langle\pair(b,b), a\bigr\rangle \Bigr\}$
\item $=^\mathcal{J}$ ist Identit\"{a}t: \\[0.1cm]
       $=^\mathcal{J} \;:=\; \Bigl\{ \bigl\langle\pair(a,a), \mathtt{true}\bigr\rangle,
                                 \bigl\langle\pair(a,b), \mathtt{false}\bigr\rangle,$\\[0.1cm]
\hspace*{2.3cm} $                \bigl\langle\pair(b,a), \mathtt{false}\bigr\rangle,
                                 \bigl\langle\pair(b,b), \mathtt{true}\bigr\rangle \Bigr\}$
\end{enumerate}



\vspace*{\fill}
\tiny \addtocounter{mypage}{1} 
\myrule
Pr\"{a}dikatenlogik -- Semantik \hspace*{\fill} Seite \arabic{mypage}
\end{slide}

%%%%%%%%%%%%%%%%%%%%%%%%%%%%%%%%%%%%%%%%%%%%%%%%%%%%%%%%%%%%%%%%%%%%%%%%

\begin{slide}{}
\normalsize
\begin{center}
Variablen-Interpretation
\end{center}
\vspace{0.5cm}

\footnotesize
\textbf{Def.}: (\emph{Variablen-Interpretation}) \quad Gegeben seien:
\begin{enumerate}
\item Signatur \quad
      $\Sigma = \langle \mathcal{V}, \mathcal{F}, \mathcal{P}, \textsl{arity} \rangle$ 
\item $\Sigma$-Struktur \quad $\struct = \langle \mathcal{U}, \mathcal{J} \rangle$ 
\end{enumerate}
Dann hei\3t eine Abbildung \\[0.3cm]
\hspace*{1.3cm} $\mathcal{I}: \mathcal{V} \rightarrow \mathcal{U}$ \\[0.3cm]
$\struct$-\emph{Variablen-Interpretation}.

\begin{enumerate}
\item $\Sigma = \langle \mathcal{V}, \mathcal{F}, \mathcal{P}, \textsl{arity} \rangle$ 
      sei Signatur
\item $\Sigma$-Struktur $\struct = \langle \mathcal{U}, \mathcal{J} \rangle$ 
\item $\mathcal{I}$ sei $\struct$-Variablen-Interpretation,

\item $x$ sei Variable 
\item $c \in \mathcal{U}$.
\end{enumerate}
Dann definiere Variablen-Interpretation $\mathcal{I}[x/c]$ durch \\[0.1cm]
\hspace*{1.3cm} 
    $\mathcal{I}[x/c](y) := \left\{
    \begin{array}{ll}
    c               & \mbox{falls}\; y = x;  \\
    \mathcal{I}(y)  & \mbox{sonst}.          \\
    \end{array}
    \right.$ 

\vspace*{\fill}
\tiny \addtocounter{mypage}{1}
\myrule
Pr\"{a}dikatenlogik -- Semantik  \hspace*{\fill} Seite \arabic{mypage}
\end{slide}

%%%%%%%%%%%%%%%%%%%%%%%%%%%%%%%%%%%%%%%%%%%%%%%%%%%%%%%%%%%%%%%%%%%%%%%%

\begin{slide}{}
\normalsize
\begin{center}
Semantik der Terme
\end{center}
\vspace{0.5cm}

\footnotesize

\textbf{Gegeben}: 
\begin{enumerate}
\item Signatur $\Sigma = \langle \mathcal{V}, \mathcal{F}, \mathcal{P}, \textsl{arity} \rangle$,
\item $\Sigma$-Struktur $\struct = \langle \mathcal{U}, \mathcal{J} \rangle$,
\item $\struct$-Variablen-Interpretation $\mathcal{I}: \mathcal{V} \rightarrow \mathcal{U}$.
\end{enumerate}

\textbf{Definition}: Interpretation eines Terms $t$ in Struktur $\mathcal{S}$ unter
Variablen-Interpretation $\mathcal{I}$:  $\struct(\mathcal{I}, t)$
\begin{enumerate}
\item Fall: $t$ Variable: \\[0.3cm]
      \hspace*{1.3cm} $\struct(\mathcal{I}, x) := \mathcal{I}(x)$
\item Fall: $t = f(t_1,\cdots,t_n)$  \\[0.3cm]
      \hspace*{1.3cm} $\struct\Bigg(\mathcal{I}, f(t_1,\cdots,t_n)\Bigg) := 
                       f^\mathcal{J}\Bigg( \struct(\mathcal{I}, t_1), \cdots, \struct(\mathcal{I}, t_n) \Bigg)$.
\end{enumerate}

\textbf{Beispiel}: (Gruppen-Theorie) 

Sei $\mathcal{I} = \{ x_1 \mapsto a, \; x_2 \mapsto b, \;, x_3 \mapsto a, \cdots \}$ 
\begin{enumerate}
\item $\mathcal{Z}\Bigg(\mathcal{I}, x_1 * 1\Bigg) = *^\mathcal{J}\Bigg(\mathcal{I}(x_1), 1^\mathcal{J}\Bigg) = *^\mathcal{J}(a, a) = a$
\item $\mathcal{Z}\Bigg(\mathcal{I}, x_1 * (x_2 * x_3)\Bigg) = *^\mathcal{J}(a, *^\mathcal{J}(b,a)) = *^\mathcal{J}(a, b) = b$
\end{enumerate}

\vspace*{\fill}
\tiny \addtocounter{mypage}{1}
\myrule
Pr\"{a}dikatenlogik -- Semantik  \hspace*{\fill} Seite \arabic{mypage}
\end{slide}

%%%%%%%%%%%%%%%%%%%%%%%%%%%%%%%%%%%%%%%%%%%%%%%%%%%%%%%%%%%%%%%%%%%%%%%%

\begin{slide}{}
\normalsize
\begin{center}
Semantik der atomaren $\Sigma$-Formeln
\end{center}
\vspace{0.5cm}

\footnotesize
\textbf{Gegeben}: 
\begin{enumerate}
\item Signatur $\Sigma = \langle \mathcal{V}, \mathcal{F}, \mathcal{P}, \textsl{arity} \rangle$,
\item $\Sigma$-Struktur $\struct = \langle \mathcal{U}, \mathcal{J} \rangle$,
\item $\struct$-Variablen-Interpretation $\mathcal{I}: \mathcal{V} \rightarrow \mathcal{U}$.
\item atomare Formel $p(t_1, \cdots, t_n)$
\end{enumerate}
\textbf{Definiere}: \\[0.1cm]
\hspace*{1.3cm} $\struct\Bigg(\mathcal{I}, p(t_1,\cdots,t_n)\Bigg) := 
                 p^\mathcal{J}\Bigg( \struct(\mathcal{I}, t_1), \cdots, \struct(\mathcal{I}, t_n) \Bigg)$

\textbf{Beispiel}: (Gruppen-Theorie) 
Sei $\mathcal{I} = \{ x_1 \mapsto a, \; x_2 \mapsto b, \;, x_3 \mapsto a, \cdots \}$ \\[0.3cm]

\hspace*{2.0cm} 
 $
 \begin{array}[c]{cl}
  &  \mathcal{Z}\Bigg( \mathcal{I}, x_2 * x_1 = 1 \Bigg) \\[0.3cm]
= &  =^\mathcal{J}\Bigg(\mathcal{Z}(\mathcal{I}, x_2 * x_1),\;\mathcal{Z}(\mathcal{I},1)\Bigg) \\[0.3cm]
= &  =^\mathcal{J}\Bigg(*^\mathcal{J}(\mathcal{I}(x_2),\, \mathcal{I}(x_1)),\;\mathcal{Z}(\mathcal{I},1)\Bigg) \\[0.3cm]
= &  =^\mathcal{J}\Bigg(*^\mathcal{J}(b, a),\;1^\mathcal{J}\Bigg) \\[0.3cm]
= &  =^\mathcal{J}( b,\; a) \\[0.3cm]
= &  \mathtt{false} 
 \end{array}
$

\vspace*{\fill}
\tiny \addtocounter{mypage}{1}
\myrule
Pr\"{a}dikatenlogik -- Semantik  \hspace*{\fill} Seite \arabic{mypage}
\end{slide}

%%%%%%%%%%%%%%%%%%%%%%%%%%%%%%%%%%%%%%%%%%%%%%%%%%%%%%%%%%%%%%%%%%%%%%%%

\begin{slide}{}
\normalsize
\begin{center}
Semantik der Junktoren
\end{center}
\vspace{0.5cm}

\footnotesize
\textbf{Gegeben}: Funktionen (wie in AL)
\begin{enumerate}
\item $\circneg: \mathbb{B} \rightarrow \mathbb{B}$,
\item $\circvee: \mathbb{B} \times \mathbb{B} \rightarrow \mathbb{B}$,
\item $\circwedge: \mathbb{B} \times \mathbb{B} \rightarrow \mathbb{B}$,
\item $\circright: \mathbb{B} \times \mathbb{B} \rightarrow \mathbb{B}$,
\item $\circleftright: \mathbb{B} \times \mathbb{B} \rightarrow \mathbb{B}$.
\end{enumerate}

\textbf{Definition}: Gegeben
\begin{enumerate}
\item Signatur $\Sigma = \langle \mathcal{V}, \mathcal{F}, \mathcal{P}, \textsl{arity} \rangle$,
\item $\Sigma$-Struktur $\struct = \langle \mathcal{U}, \mathcal{J} \rangle$,
\item $\struct$-Variablen-Interpretation $\mathcal{I}: \mathcal{V} \rightarrow \mathcal{U}$.
\end{enumerate}
\begin{enumerate}
\item $\struct(\mathcal{I},\verum) := \mathtt{true}$ und $\struct(\mathcal{I},\falsum) := \mathtt{false}$.
\item $\struct(\mathcal{I}, \neg f) \;:=\; \circneg\Bigg(\struct(\mathcal{I}, f)\Bigg)$.
\item $\struct(\mathcal{I}, f \wedge g) \;:=\; \circwedge\Bigg(\struct(\mathcal{I}, f), \struct(\mathcal{I}, g)\Bigg)$.
\item $\struct(\mathcal{I}, f \vee g) \;:=\; \circvee\Bigg(\struct(\mathcal{I}, f), \struct(\mathcal{I}, g)\Bigg)$.
\item $\struct(\mathcal{I}, f \rightarrow g) \;:=\; \circright\!\Bigg(\struct(\mathcal{I}, f), \struct(\mathcal{I}, g)\Bigg)$.
\item $\struct(\mathcal{I}, f \leftrightarrow g) \;:=\; \circleftright(\struct(\mathcal{I}, f), \struct(\mathcal{I}, g)\Bigg)$.
\end{enumerate}


\vspace*{\fill}
\tiny \addtocounter{mypage}{1}
\myrule
Pr\"{a}dikatenlogik -- Semantik  \hspace*{\fill} Seite \arabic{mypage}
\end{slide}

%%%%%%%%%%%%%%%%%%%%%%%%%%%%%%%%%%%%%%%%%%%%%%%%%%%%%%%%%%%%%%%%%%%%%%%%

\begin{slide}{}
\normalsize
\begin{center}
Semantik der Quantoren
\end{center}
\vspace{0.5cm}

\footnotesize
\hspace*{1.3cm} 
    $\mathcal{I}[x/c](y) := \left\{
    \begin{array}{ll}
    c               & \mbox{falls}\; y = x;  \\
    \mathcal{I}(y)  & \mbox{sonst}.          \\
    \end{array}
    \right.$

 $\struct\bigl(\mathcal{I}, \forall x \!:\! f\bigr) \;:=\; $ \\[0.3cm]
 \hspace*{1.3cm} $\left\{
       \begin{array}{ll}
       \mathtt{true}  & \mbox{falls}\; \struct(\mathcal{I}[x/c], f) = \mathtt{true}\quad \mbox{f\"{u}r alle}\; c\in \mathcal{U}\;\mbox{gilt}; \\
       \mathtt{false} % \mbox{sonst}.
       \end{array}
       \right.$
\vspace{0.5cm}

 $\struct\bigl(\mathcal{I}, \exists x\!:\! f\bigr) \;:=\; $ \\[0.3cm]
 \hspace*{1.3cm} $\left\{
       \begin{array}{ll}
       \mathtt{true}  & \mbox{falls}\; \struct(\mathcal{I}[x/c], f) = \mathtt{true}\quad \mbox{f\"{u}r ein}\; c\in \mathcal{U}\;\mbox{gilt}; \\
       \mathtt{false} % \mbox{sonst}.
       \end{array}
       \right.$ 
\vspace{0.5cm}

Falls $\FV(f) = \emptyset$, dann $\mathcal{S}(\mathcal{I}_1, f) = \mathcal{S}(\mathcal{I}_2, f)$ 
f\"{u}r alle $\mathcal{I}_1$, $\mathcal{I}_2$.

Schreibweise: Falls $\FV(f) = \emptyset$ \\[0.1cm]
\hspace*{1.3cm} $\mathcal{S}(f) := \mathcal{S}(\mathcal{I}, f)$
\vspace{0.5cm}

\textbf{Beispiele}: 
\begin{enumerate}
\item $\mathcal{Z}( \forall x_1: \exists x_2: x_1 * x_2 = 1) = \mathtt{true}$
\item $\mathcal{Z}( \forall x_1, x_2: x_1 * x_2 = x_2 * x_1) = \mathtt{true}$
\item $\mathcal{Z}( \forall x_1: \exists x_2: x_2 * x_2 = x_1) = \mathtt{false}$
\end{enumerate}

\vspace*{\fill}
\tiny \addtocounter{mypage}{1}
\myrule
Pr\"{a}dikatenlogik -- Semantik  \hspace*{\fill} Seite \arabic{mypage}
\end{slide}

%%%%%%%%%%%%%%%%%%%%%%%%%%%%%%%%%%%%%%%%%%%%%%%%%%%%%%%%%%%%%%%%%%%%%%%%

\begin{slide}{}
\normalsize

\begin{center}
Allgemeing\"{u}ltig
\end{center}
\vspace{0.5cm}

\footnotesize
\textbf{Gegeben}: $f$ sei $\Sigma$-Formel. Falls f\"{u}r jede
\begin{enumerate}
\item $\struct$: \quad  $\Sigma$-Struktur und jede
\item  $\mathcal{I}$: \quad $\struct$-Variablen-Interpretation
\end{enumerate}
\hspace*{1.3cm} $\struct(\mathcal{I}, f) = \mathtt{true}$ \\[0.3cm]
gilt, dann ist  $f$  \emph{allgemeing\"{u}ltig}.  

\textbf{Schreibweise}:
\hspace*{1.3cm} $\models f$. 

\textbf{Beispiele}: (Gruppen-Theorie) Seien
\begin{enumerate}
  \item $g_1 \quad := \quad \forall x_1: 1 * x_1 = x_1$
  \item $g_2 \quad := \quad \forall x_2, x_3, x_4: (x_2 * x_3) * x_4 = x_2 * (x_3 * x_4)$
  \item $g_3 \quad := \quad \forall x_5: \exists x_6: x_5 * x_6 = 1$
  \item $g \quad := \quad g_1 \wedge g_2 \wedge g_3$.
\end{enumerate}
Dann gilt:
\begin{enumerate}
\item $\models g \rightarrow \forall x_{1}: x_{1} * 1 = x_{1}$
\item $\models g \rightarrow \forall x_{1}: \exists x_{2}: x_{1} * x_{2} = 1$
\item $\not\models g \rightarrow \forall x_{1}, x_{2}: x_{1} * x_{2} = x_{2} * x_{1}$
\end{enumerate}

\vspace*{\fill}
\tiny \addtocounter{mypage}{1}
\myrule
Pr\"{a}dikatenlogik -- Semantik  \hspace*{\fill} Seite \arabic{mypage}
\end{slide}

%%%%%%%%%%%%%%%%%%%%%%%%%%%%%%%%%%%%%%%%%%%%%%%%%%%%%%%%%%%%%%%%%%%%%%%%

\begin{slide}{}
\normalsize
\begin{center}
Modell
\end{center}
\vspace{0.5cm}

\footnotesize
\textbf{Gegeben}: $f \in \mathbb{F}_\Sigma$ \\[0.3cm]
\hspace*{1.3cm} $f$ \emph{geschlossen} \quad g.d.w. \quad $\FV(f) = \emptyset$

\textbf{Beobachtung}: F\"{u}r geschlossene Formel h\"{a}ngt Wahrheitswert nicht von
Variablen-Interpretation $\mathcal{I}$ ab: \\[0.1cm]
\hspace*{1.3cm} $\mathcal{S}(\mathcal{I}_1, f) = \mathcal{S}(\mathcal{I}_2, f)$.

Schreibweise: $\mathcal{S}(f) = \mathcal{S}(\mathcal{I}, f)$
\vspace{0.5cm}

\textbf{Gegeben}: 
\begin{enumerate}
\item $f \in \mathbb{F}_\Sigma$ mit $\FV(f) = \emptyset$
\item $\struct$ sei $\Sigma$-Struktur 
\end{enumerate}
\quad $\struct$ \emph{Modell} von $f$ \quad g.d.w. \quad $\struct(f) = \mathtt{true}$
\vspace{0.3cm}

Schreibweise: \quad $\struct \models f$ \quad falls \quad $\struct(f) = \mathtt{true}$
\vspace{0.5cm}

\textbf{Gegeben}:  
\begin{enumerate}
\item $M \subseteq \mathbb{F}_\Sigma$
\item $\FV(f) = \emptyset$ \quad f\"{u}r alle $f \in M$
\end{enumerate}
$M$ \emph{erf\"{u}llbar} \quad g.d.w. \quad
es existiert $\Sigma$-Struktur $\mathcal{S}$ mit \\[0.3cm]
\hspace*{2.3cm} $\Sigma \models f$ \quad f\"{u}r alle $f \in M$.




\vspace*{\fill}
\tiny \addtocounter{mypage}{1}
\myrule
Pr\"{a}dikatenlogik -- Semantik  \hspace*{\fill} Seite \arabic{mypage}
\end{slide}

%%%%%%%%%%%%%%%%%%%%%%%%%%%%%%%%%%%%%%%%%%%%%%%%%%%%%%%%%%%%%%%%%%%%%%%%

\begin{slide}{}

\footnotesize
\textbf{Gegeben}:  
\begin{enumerate}
\item $M \subseteq \mathbb{F}_\Sigma$
\item $\FV(f) = \emptyset$ \quad f\"{u}r alle $f \in M$
\end{enumerate}
$M$ \emph{unerf\"{u}llbar} \quad g.d.w. \\
es existiert keine $\Sigma$-Struktur $\mathcal{S}$ mit \\[0.3cm]
\hspace*{2.3cm} $\mathcal{S} \models f$ \quad f\"{u}r alle $f \in M$.

\textbf{Schreibweise}: \\[0.1cm]
\hspace*{1.3cm} $M \models \falsum$ \quad g.d.w. $M$ unerf\"{u}llbar
\vspace{0.5cm}

\textbf{Satz}: Sei $f$ geschlossene Formel. Dann gilt \\[0.3cm]
\hspace*{1.3cm} 
$\models f$ \quad g.d.w. \quad $\{ \neg f \} \models \falsum$


\normalsize
\begin{center}
\"{a}quivalenzen  
\end{center}
\vspace{0.5cm}
\footnotesize

\textbf{Gegeben}:  $f, g \in \mathbb{F}_\Sigma$
$f$ und $g$ \emph{\"{a}quivalent} \quad g.d.w. \\[0.3cm]
\hspace*{3.3cm}  $\models f \leftrightarrow g$

Beispiele: DeMorgan'sche Gesetze f\"{u}r Quantoren
\begin{enumerate}
\item $\models \neg\big(\forall x\colon F\big) \leftrightarrow \big(\exists x\colon \neg F\big)$
\item $\models \neg\big(\exists x\colon F\big) \leftrightarrow \big(\forall x\colon \neg F\big)$
\end{enumerate}


\vspace*{\fill}
\tiny \addtocounter{mypage}{1}
\myrule
Pr\"{a}dikatenlogik -- Semantik  \hspace*{\fill} Seite \arabic{mypage}
\end{slide}

%%%%%%%%%%%%%%%%%%%%%%%%%%%%%%%%%%%%%%%%%%%%%%%%%%%%%%%%%%%%%%%%%%%%%%%%

\begin{slide}{}
\normalsize
\begin{center}
Pr\"{a}nexe Normalform
\end{center}
\vspace{0.5cm}

\footnotesize
Es gelten die folgenden \"{a}quivalenzen:
  \begin{enumerate}
  \item $\models \big(\forall x\colon f\big) \wedge \big(\forall x\colon g\big) \leftrightarrow \big(\forall x\colon f \wedge g\big)$
  \item $\models \big(\exists x\colon f\big) \vee \big(\exists x\colon g\big) \leftrightarrow \big(\exists x\colon f \vee g\big)$
  \item $\models \big(\forall x\colon \forall y\colon f \big) \leftrightarrow \big(\forall y\colon  \forall x\colon f \big)$
  \item $\models \big(\exists x\colon \exists y\colon f \big) \leftrightarrow \big(\exists y\colon  \exists x\colon f \big)$
  \item Sei $x \in \mathcal{V}$ mit $x \not\in \FV(f) \cup \BV(f)$. Dann gilt:
    \begin{enumerate}
    \item $\models  \big(\forall x\colon f) \leftrightarrow f$ 
    \item $\models  \big(\exists x\colon f) \leftrightarrow f$.
    \end{enumerate}
  \item Sei $x \in \mathcal{V}$ mit $x \not\in \FV(g) \cup \BV(g)$. Dann gilt:
    \begin{enumerate}
    \item $\models \big(\forall x\colon f) \vee g \leftrightarrow \big(\forall x\colon f \vee g\big)$
    \item $\models g \vee \big(\forall x\colon f) \leftrightarrow \big(\forall x\colon g \vee f\big)$
    \item $\models \big(\exists x\colon f) \wedge g \leftrightarrow \big(\exists x\colon f \wedge g\big)$
    \item $\models g \wedge \big(\exists x\colon f) \leftrightarrow \big(\exists x\colon g \wedge f\big)$
    \item $\models f \leftrightarrow f[x/y]$
    \end{enumerate}
  \end{enumerate}

\textbf{Definition}: 
\begin{enumerate}
\item  $f$  pr\"{a}dikatenlogische Formel
\item $x$, $y$ Variablen, $y \not\in \FV(f) \cup BV(f)$
\end{enumerate} 
\qquad  $f[x/y]$: ersetze in $f$ Variable  $x$ durch $y$ 


\vspace*{\fill}
\tiny \addtocounter{mypage}{1}
\myrule
Pr\"{a}dikatenlogik   \hspace*{\fill} Seite \arabic{mypage}
\end{slide}

%%%%%%%%%%%%%%%%%%%%%%%%%%%%%%%%%%%%%%%%%%%%%%%%%%%%%%%%%%%%%%%%%%%%%%%%

\begin{slide}{}
\normalsize
\begin{center}
Skolemisierung
\end{center}
\vspace{0.5cm}

\footnotesize
\textbf{Gegeben}:
\begin{enumerate}
\item $\Sigma = \langle \mathcal{V}, \mathcal{F}, \mathcal{P}, \textsl{arity} \rangle$ Signatur
\item $f = \forall x_1, \cdots, x_n \colon \exists y \colon g(x_1, \cdots, x_n, y)$ \quad
      mit $\FV(f) = \emptyset$    
\end{enumerate}
Schritte zur Skolemisierung
\begin{enumerate}
\item W\"{a}hle neues $n$-stelliges Funktions-Zeichen $s$: \\[0.3cm]
      \hspace*{1.3cm} $s \not\in \mathcal{F}$.
\item Bilde neue Signatur \\[0.3cm]
      \hspace*{1.3cm} 
      $\Sigma' := \Bigl\langle \mathcal{V}, \mathcal{F} \cup \{s\}, \mathcal{P}, \textsl{arity} \cup \bigl\{\pair(s,n)\bigr\} \Bigr\rangle$, 
\item Bilde Skolemisierung von $f$ \\[0.3cm]
      $f' = \textsl{skolem}(f) = \forall x_1 \colon \cdots \forall x_n \colon g\bigl(x_1, \cdots, x_n, s(x_1,\cdots,x_n)\bigr)$\\[0.1cm]
\end{enumerate}


\vspace*{\fill}
\tiny \addtocounter{mypage}{1}
\myrule
Pr\"{a}dikatenlogik   \hspace*{\fill} Seite \arabic{mypage}
\end{slide}

%%%%%%%%%%%%%%%%%%%%%%%%%%%%%%%%%%%%%%%%%%%%%%%%%%%%%%%%%%%%%%%%%%%%%%%%
\begin{slide}{}
\normalsize
\begin{center}
Skolemisierung
\end{center}
\vspace{0.5cm}

\textbf{Gegeben}: $f,g \in \mathbb{F}_\Sigma$ \quad mit $\FV(f) = \FV(g) = \emptyset$

$f$ und $g$ erf\"{u}llbarkeits-\"{a}quivalent ($f \approx_e g$) \quad g.d.w.
\begin{enumerate}
\item $f$ und $g$ beide erf\"{u}llbar oder
\item $f$ und $g$ beide unerf\"{u}llbar
\end{enumerate}

\textbf{Satz}: $f \approx_e \textsl{skolem}(f)$

\textbf{Beispiel}: $\forall x: \exists y: p(x,y) \;\approx_e\; \forall x: p(x,s(x))$

\vspace*{\fill}
\tiny \addtocounter{mypage}{1}
\myrule
Pr\"{a}dikatenlogik   \hspace*{\fill} Seite \arabic{mypage}
\end{slide}

%%%%%%%%%%%%%%%%%%%%%%%%%%%%%%%%%%%%%%%%%%%%%%%%%%%%%%%%%%%%%%%%%%%%%%%%

%%%%%%%%%%%%%%%%%%%%%%%%%%%%%%%%%%%%%%%%%%%%%%%%%%%%%%%%%%%%%%%%%%%%%%%%

\begin{slide}{}
\normalsize
\begin{center}
\"{u}berf\"{u}hrung in Klausel-Normalform
\end{center}
\vspace{0.5cm}

\footnotesize
\textbf{Gegeben}: pr\"{a}dikatenlogische Formel $f$

\textbf{Gesucht}: pr\"{a}dikatenlogischen Klauseln $\{k_1,\cdots,k_n\}$ mit \\[0.1cm]
\hspace*{1.3cm} $f \approx_e k_1 \wedge \cdots \wedge k_n$
\begin{enumerate}
\item Transformation von $f$ in pr\"{a}nexe Normalform 
      \\[0.3cm]
      \hspace*{1.3cm}
      $f \leftrightarrow Q_1 x_1: \cdots Q_m x_m: g$ \quad mit
      \begin{enumerate}
      \item $Q_i \in \{\forall, \exists\}$ f\"{u}r $i=1,\cdots,m$,
      \item $g$ ist Quantoren-frei.
      \end{enumerate}
\item Beseitigung der Existenz-Quantoren durch Skolemisierung

       $f \approx_e \forall x_1, \cdots, x_l: g$
\item Weglassen der Quantoren: Variablen implizit allquantifiziert
\item Berechnung der konjuktiven Normalform der Matrix

      \hspace*{1.3cm} $g \leftrightarrow k_1 \wedge \cdots \wedge k_r$

      $f \approx_e \forall(k_1) \wedge \cdots \wedge \forall(k_r)$
\end{enumerate}
Allabschluss: $\forall(h) := \forall x_1, \cdots, x_n:h$ mit $\{x_1,\cdots,x_n\} = \FV(h)$


\vspace*{\fill}
\tiny \addtocounter{mypage}{1}
\myrule
Pr\"{a}dikatenlogik   \hspace*{\fill} Seite \arabic{mypage}
\end{slide}

%%%%%%%%%%%%%%%%%%%%%%%%%%%%%%%%%%%%%%%%%%%%%%%%%%%%%%%%%%%%%%%%%%%%%%%%

\begin{slide}{}
\normalsize
\begin{center}
Martelli--Montanari--Regeln zur Transformation
 syntaktischer Gleichungssysteme 
\end{center}
\vspace{0.5cm}

\footnotesize
\textbf{Betrachte} Paare $\langle E, \sigma \rangle$ mit
\begin{enumerate}
\item $E$ ist syntaktisches Gleichungssystem,
\item $\sigma$ ist Substitution.
\end{enumerate}

Verfahren zur Transformation solcher Paare:
\begin{enumerate}
\item $\langle E \cup \big\{ y \doteq t \big\},\; \sigma \rangle \;\leadsto\; \langle E[y \mapsto t],\; \sigma\big[ y \mapsto t \big]  \rangle$

      falls $y \not\in \textsl{var}(t)$.
\item $\langle E \cup \big\{ y \doteq t \big\}, \sigma \rangle \quad\leadsto\quad \Omega $

      falls $y \in \textsl{var}(t)$ und $t \not= y$.
\item $\langle E \cup \big\{ t \doteq y \big\}, \sigma \rangle \quad\leadsto\quad \langle E \cup \big\{ y \doteq t \big\}, \sigma \rangle$

      Falls $y \in \mathcal{V}$ und $t \not\in \mathcal{V}$.
\item $\langle E \cup \big\{ x \doteq x \big\}, \sigma \rangle \quad\leadsto\quad \langle E, \sigma \rangle$.
\item \hspace*{1.25cm} $\langle E \cup \big\{ f(s_1,\cdots,s_n) \doteq f(t_1,\cdots,t_n) \big\}, \sigma \rangle$ \\[0.2cm]
       $\leadsto\quad \langle E \cup \big\{ s_1 \doteq t_1, \cdots, s_n \doteq t_n\}, \sigma \rangle$.
\item $\langle E \cup \big\{ f(s_1,\cdots,s_m) \doteq g(t_1,\cdots,t_n) \big\}, \sigma \rangle \;\leadsto\; \Omega$

      falls $f \not= g$.
\end{enumerate}



\vspace*{\fill}
\tiny \addtocounter{mypage}{1}
\myrule
Unifikation  \hspace*{\fill} Seite \arabic{mypage}
\end{slide}

%%%%%%%%%%%%%%%%%%%%%%%%%%%%%%%%%%%%%%%%%%%%%%%%%%%%%%%%%%%%%%%%%%%%%%%%

\begin{slide}{}
\normalsize
\begin{center}
Robinson-Kalk\"{u}l: Resolution + Faktorisierung
\end{center}
\vspace{0.5cm}

\footnotesize
Resolutions-Regel: 
\begin{enumerate}
\item $k_1$, $k_2$:  pr\"{a}dikatenlogische Klauseln,
\item $p(s_1,\cdots,s_n)$, $p(t_1,\cdots,t_n)$: atomare Formeln,
\item syntaktische Gleichung $p(s_1,\cdots,s_n)  \doteq p(t_1,\cdots,t_n)$ l\"{o}sbar, 
\item $\mu = \textsl{mgu}\bigl(p(s_1,\cdots,s_n), p(t_1,\cdots,t_n)\bigr)$.
\end{enumerate}
\hspace*{1.3cm} $\schluss{k_1 \cup\{ p(s_1,\cdots,s_n)\} \quad\quad \{\neg p(t_1,\cdots,t_n)\} \cup k_2}{k_1\mu \cup k_2\mu}$ 

\vspace*{\fill}
\tiny \addtocounter{mypage}{1}
\myrule
Robinson-Kalk\"{u}l \hspace*{\fill} Seite \arabic{mypage}
\end{slide}

%%%%%%%%%%%%%%%%%%%%%%%%%%%%%%%%%%%%%%%%%%%%%%%%%%%%%%%%%%%%%%%%%%%%%%%%

\begin{slide}{}
\normalsize
\begin{center}
Robinson-Kalk\"{u}l: Resolution + Faktorisierung
\end{center}
\vspace{0.5cm}

\footnotesize
Faktorisierungs-Regel:
\begin{enumerate}
\item $k$ ist  eine pr\"{a}dikatenlogische Klausel,
\item $p(s_1,\cdots,s_n)$ und $p(t_1,\cdots,t_n)$ sind atomare Formeln,
\item die syntaktische Gleichung $p(s_1,\cdots,s_n)  \doteq p(t_1,\cdots,t_n)$ ist l\"{o}sbar, 
\item $\mu = \textsl{mgu}\bigl(p(s_1,\cdots,s_n), p(t_1,\cdots,t_n)\bigr)$.
\end{enumerate}
\hspace*{1.8cm}
$\schluss{k \cup \bigl\{p(s_1,\cdots,s_n),\, p(t_1,\cdots,t_n)\bigl\}}{k\mu \cup \bigl\{p(s_1,\cdots,s_n)\mu\bigr\} }$ 
\quad und 

\hspace*{1.8cm}
$\schluss{k \cup \bigl\{ \neg p(s_1,\cdots,s_n),\, \neg p(t_1,\cdots,t_n)\bigl\}}{k\mu \cup \bigl\{\neg p(s_1,\cdots,s_n)\mu\bigr\} }$ 



\vspace*{\fill}
\tiny \addtocounter{mypage}{1}
\myrule
Robinson-Kalk\"{u}l \hspace*{\fill} Seite \arabic{mypage}
\end{slide}

%%%%%%%%%%%%%%%%%%%%%%%%%%%%%%%%%%%%%%%%%%%%%%%%%%%%%%%%%%%%%%%%%%%%%%%%

\begin{slide}{}
\normalsize
\begin{center}
Verfahren zur Pr\"{u}fung von $\models f$ 
\end{center}
\vspace{0.5cm}

\footnotesize
\textbf{Betrachte} 
\begin{enumerate}
\item $\models f$  \quad g.d.w \quad $\{\neg f\} \models \falsum$
\item Bilde Skolem-Normalform von $\neg f$: \\[0.3cm]
      \hspace*{1.3cm} $\neg f \approx_e \forall x_1, \cdots, x_m \colon g$.
\item Bringe Matrix $g$ in konjunktive Normalform: \\[0.3cm]
      \hspace*{1.3cm} 
      $g \leftrightarrow k_1 \wedge \cdots \wedge k_n$.

      Also \\[0.3cm]
      \hspace*{1.3cm} $\neg f \approx_e k_1 \wedge \cdots \wedge k_n$
\item Dann gilt: 
      \\[0.3cm]
      \hspace*{1.3cm}
       $\models f$ \quad g.d.w. \quad $\{\neg f\} \models \falsum$ \quad g.d.w. \quad $\{k_1,\cdots,k_n\} \models \falsum$.
\item Korrektheits-Satz + Widerlegungs-Vollst\"{a}ndigkeit:
      \\[0.3cm]
      \hspace*{1.3cm} $\{k_1,\cdots,k_n\} \models \falsum$ \quad g.d.w. \quad $\{k_1,\cdots,k_n\} \vdash \falsum$. 
\end{enumerate}

\vspace*{\fill}
\tiny \addtocounter{mypage}{1}
\myrule
Robinson-Kalk\"{u}l  \hspace*{\fill} Seite \arabic{mypage}
\end{slide}

%%%%%%%%%%%%%%%%%%%%%%%%%%%%%%%%%%%%%%%%%%%%%%%%%%%%%%%%%%%%%%%%%%%%%%%%

\begin{slide}{}
\normalsize
\begin{center}
Rote Drachen sind gl\"{u}cklich 
\end{center}
\vspace{0.5cm}

\footnotesize
Zoologische Axiome:
\begin{enumerate}
\item Jeder Drache ist gl\"{u}cklich, wenn alle seine Kinder fliegen k\"{o}nnen.
\item Rote Drachen k\"{o}nnen fliegen.
\item Die Kinder eines roten Drachens sind immer rot.
\end{enumerate}
\textbf{Beh.:} Alle roten Drachen sind gl\"{u}cklich.

\textbf{Formalisierung}:
\begin{enumerate}
\item $f_1 := \forall x: \big(\forall y: \textsl{kind}(y,x) \rightarrow \textsl{fliegt}(y)\big) \rightarrow \textsl{gl\"{u}cklich}(x)$
\item $f_2 := \forall x: \textsl{rot}(x) \rightarrow \textsl{fliegt}(x)$
\item $f_3 := \forall x: \textsl{rot}(x) \rightarrow \forall y: \textsl{kind}(y,x) \rightarrow \textsl{rot}(y)$
\item $f_4 := \forall x: \textsl{rot}(x) \rightarrow \textsl{gl\"{u}cklich}(x)$
\end{enumerate}
\textbf{zu zeigen:} \\[0.1cm]
\hspace*{1.3cm} $f := f_1 \wedge f_2 \wedge f_3 \rightarrow f_4$ \\[0.1cm]


\vspace*{\fill}
\tiny \addtocounter{mypage}{1}
\myrule
Robinson-Kalk\"{u}l \hspace*{\fill} Seite \arabic{mypage}
\end{slide}

%%%%%%%%%%%%%%%%%%%%%%%%%%%%%%%%%%%%%%%%%%%%%%%%%%%%%%%%%%%%%%%%%%%%%%%%

\begin{slide}{}
\normalsize
\begin{center}
\textsl{Prolog} -- Ein Beispiel 
\end{center}
\vspace{0.5cm}

\footnotesize
\begin{Verbatim}[ frame         = lines, 
                  framesep      = 0.3cm, 
                  labelposition = bottomline,
                  numbers       = left,
                  numbersep     = -0.2cm,
                  xleftmargin   = 0.8cm,
                  xrightmargin  = 0.8cm
                ]
    gallier(asterix).
    gallier(obelix).

    kaiser(caesar).
    roemer(caesar).

    stark(X) :- gallier(X).

    maechtig(X) :- stark(X).
    maechtig(X) :- kaiser(X), roemer(X).

    spinnt(X) :- roemer(X).
\end{Verbatim}

Interpretation:
\begin{enumerate}
\item Asterix ist ein Gallier.
\item Obelix ist ein Gallier.
\item C\"{a}sar ist ein Kaiser.
\item C\"{a}sar ist ein R\"{o}mer.
\item Alle Gallier sind stark.
\item Wer stark ist, ist m\"{a}chtig.
\item Wer Kaiser und R\"{o}mer ist, der ist m\"{a}chtig.
\item Wer R\"{o}mer ist, spinnt. 
\end{enumerate}



\vspace*{\fill}
\tiny \addtocounter{mypage}{1}
\myrule
Prolog  \hspace*{\fill} Seite \arabic{mypage}
\end{slide}


%%%%%%%%%%%%%%%%%%%%%%%%%%%%%%%%%%%%%%%%%%%%%%%%%%%%%%%%%%%%%%%%%%%%%%%%

\begin{slide}{}
\normalsize
\begin{center}
Prolog --- Syntax
\end{center}
\vspace{0.5cm}

\footnotesize
\begin{enumerate}
\item \textbf{Variable}: String, der 
      \begin{enumerate}
      \item mit gro\3en Buchstaben oder Unterstrich ``\texttt{\_}'' beginnt,
      \item nur aus Buchstaben, Ziffern, und ``\texttt{\_}'' besteht.
      \end{enumerate}
      \textbf{Beispiele}: 
      \\[0.3cm]
      \hspace*{1.3cm} \texttt{X}, \texttt{ABC\_32}, \texttt{\_U}, \texttt{Hugo},
      \texttt{\_1}, \texttt{\_}.
      
      anonyme Variable: \texttt{\_}
\item \textbf{Funktions-Zeichen, Pr\"{a}dikats-Zeichen}: String, der
      \begin{enumerate}
      \item mit kleinen Buchstaben beginnt und 
      \item nur aus Buchstaben, Ziffern und Unterstrich ``\texttt{\_}'' besteht, oder
      \item in einfachen Hochkomma eingeschlossen ist.
      \end{enumerate}
      \textbf{Beispiele}:
      \\[0.3cm]
      \hspace*{1.3cm}      
      \texttt{asterix}, \texttt{a1}, \texttt{i\_love\_prolog}, \texttt{x}, '\texttt{George W. Bush}'.
\end{enumerate}

%%%%%%%%%%%%%%%%%%%%%%%%%%%%%%%%%%%%%%%%%%%%%%%%%%%%%%%%%%%%%%%%%%%%%%%%%%%%%%%%

\vspace*{\fill}
\tiny \addtocounter{mypage}{1}
\myrule
Prolog  \hspace*{\fill} Seite \arabic{mypage}
\end{slide}

\begin{slide}{}
\normalsize
\begin{center}
Prolog --- Syntax-Erweiterungen
\end{center}
\vspace{0.5cm}

\footnotesize
\begin{enumerate}
\item Zus\"{a}tzliche \textbf{Funktions-Zeichen}: \\[0.3cm]
      \hspace*{1.3cm} 
      ``\texttt{+}'', ``\texttt{-}'', ``\texttt{*}'', ``\texttt{/}'', ``\texttt{.}'' 

      k\"{o}nnen als Infix-Operatoren benutzt werden.
\item Zus\"{a}tzliche \textbf{Pr\"{a}dikats-Zeichen}: \\[0.3cm]
      \hspace*{1.3cm} 
      ``\texttt{<}'', ``\texttt{>}'', ``\texttt{=}'', ``\texttt{=<}'', ``\texttt{>=}, 
      ``\texttt{$\backslash$=}'', ``\texttt{==}'', ``\texttt{$\backslash$==}''.

      Infix-Schreibweise zul\"{a}ssig
      
      \begin{tabbing}
      \hspace*{1.3cm} \= ``\texttt{==}'', \= ``\texttt{$\backslash$==}'': \= Vergleichs-Operatoren \\[0.3cm]
                      \> ``\texttt{=}'',  \> ``\texttt{$\backslash$=}'':  \> Unifikations-Operatoren        
      \end{tabbing}

\item Zus\"{a}tzliche \textbf{Funktions-Zeichen}: \\[0.3cm] 
      \hspace*{1.3cm} ganze Zahlen, Flie\3komma-Zahlen

      \textbf{Beispiele}: 
      \\[0.3cm]
      \hspace*{1.3cm}
      \texttt{12}, \texttt{-3}, \texttt{2.5}, \texttt{2.3e-5}.
\end{enumerate}



\vspace*{\fill}
\tiny \addtocounter{mypage}{1}
\myrule
Prolog  \hspace*{\fill} Seite \arabic{mypage}
\end{slide}

%%%%%%%%%%%%%%%%%%%%%%%%%%%%%%%%%%%%%%%%%%%%%%%%%%%%%%%%%%%%%%%%%%%%%%%%

\begin{slide}{}
\normalsize
\begin{center}
Prolog -- Der Algorithmus
\end{center}

\footnotesize
\textbf{Gegeben}
\begin{enumerate}
\item Anfrage \hspace*{3.5cm} $G = Q_1, \cdots, Q_n$
\item \textsl{Prolog}-Programm \quad $P$
\end{enumerate}

\textbf{Gesucht}: Instanz $G\sigma$ mit: \quad $G\sigma$ folgt aus $P$.
\begin{enumerate}
\item Suche (der Reihe nach) alle Regeln  \\[0.3cm]      
      \hspace*{1.3cm} $A \;\mathtt{:-}\; B_1,\cdots,B_m.$ \\[0.3cm]
      so dass Unifikator $\mu = \mathtt{mgu(Q_1,A)}$ existiert.
\item Gibt es mehrere solche Regeln, dann:
      \begin{enumerate}
      \item W\"{a}hle erste Regel in Programm-Reihenfolge
      \item Setze Auswahl-Punkt (\emph{Choice-Point})
      \end{enumerate}
\item Setze $G := G\mu$ 
\item Rekursiv: beantworte Anfrage \\[0.3cm]
      \hspace*{1.3cm} $B_1\mu, \cdots, B_m\mu, Q_2\mu, \cdots, Q_n\mu$ \\[0.3cm]
      Falls $m + n = 0$: Gebe $G\mu$ zur\"{u}ck.
\item Falls 4. erfolglos:
      \begin{enumerate}
      \item Gehe zur\"{u}ck zum letzten Auswahl-Punkt.
      \item Mache alle vom Auswahl-Punkt ab gemachten Substitutionen $G := G\mu$ r\"{u}ckg\"{a}ngig.
      \end{enumerate}
\end{enumerate}



\vspace*{\fill}
\tiny \addtocounter{mypage}{1}
\myrule
Prolog  \hspace*{\fill} Seite \arabic{mypage}
\end{slide}


%%%%%%%%%%%%%%%%%%%%%%%%%%%%%%%%%%%%%%%%%%%%%%%%%%%%%%%%%%%%%%%%%%%%%%%%

\begin{slide}{}
\normalsize
\begin{center}
Ein R\"{a}tsel
\end{center}
\vspace{0.5cm}

\footnotesize
\begin{enumerate}
\item Drei Freunde belegen den ersten, zweiten und dritten \\
      Platz bei einem Programmier-Wettbewerb.
\item Jeder der drei hat genau einen Vornamen, genau ein \\
      Auto und hat sein Programm in genau einer \\
      Programmier-Sprache geschrieben.
\item Michael programmiert in \textsl{Setl} und war besser als \\
      der Audi-Fahrer.
\item Julia, die einen Ford Mustang f\"{a}hrt, war besser \\
      als der Java-Programmierer.
\item Das Prolog-Programm war am besten.
\item Wer f\"{a}hrt Toyota?
\item In welcher Sprache programmiert Thomas?
\end{enumerate}



\vspace*{\fill}
\tiny \addtocounter{mypage}{1}
\myrule
Prolog  \hspace*{\fill} Seite \arabic{mypage}
\end{slide}

%%%%%%%%%%%%%%%%%%%%%%%%%%%%%%%%%%%%%%%%%%%%%%%%%%%%%%%%%%%%%%%%%%%%%%%%

\begin{slide}{}
\normalsize
\begin{center}
Negation in \textsl{Prolog}
\end{center}
\vspace{0.5cm}

\footnotesize
Beantwortung der Anfrage \quad \texttt{\symbol{92}+} $A$:
\begin{enumerate}
\item Versuche Beantwortung der Anfrage $A$.
\item Falls Beantwortung der Anfrage $A$ scheitert: 

      Beantwortung der Anfrage \texttt{\symbol{92}+} $A$ erfolgreich.  

      Keine Variablen-Instantiierung.
\item Falls Beantwortung der Anfrage $A$ erfolgreich mit Subst. $\mu$:

      Beantwortung der Anfrage \texttt{\symbol{92}+} $A$ scheitert.

      Variablen-Instantiierung $\mu$ wird r\"{u}ckg\"{a}ngig gemacht.
\end{enumerate}
\framebox{\framebox{
Beantwortung von \texttt{\symbol{92}+} $A$ instantiiert
\textbf{niemals} \rule[-2pt]{14pt}{0pt}Variable! }}

Deswegen:

\framebox{\framebox{
Bei Anfrage \texttt{\symbol{92}+} $A$ soll $A$
\textbf{keine} \rule[-2pt]{14pt}{0pt}Variablen enthalten!}}



\vspace*{\fill}
\tiny \addtocounter{mypage}{1}
\myrule
Prolog  \hspace*{\fill} Seite \arabic{mypage}
\end{slide}

%%%%%%%%%%%%%%%%%%%%%%%%%%%%%%%%%%%%%%%%%%%%%%%%%%%%%%%%%%%%%%%%%%%%%%%%

\begin{slide}{}
\normalsize
\begin{center}
Negation und Cut
\end{center}
\vspace{0.5cm}

\footnotesize
Probleme bei der Negation
\begin{Verbatim}[ frame         = lines, 
                  framesep      = 0.3cm, 
                  labelposition = bottomline,
                  numbers       = left,
                  numbersep     = -0.2cm,
                  xleftmargin   = 0.8cm,
                  xrightmargin  = 0.8cm
                ]
    gallier(miraculix).
    
    roemer(caesar).
    
    smart1(X) :- \+ roemer(X), gallier(X).
    
    smart2(X) :- gallier(X), \+ roemer(X).
\end{Verbatim}
\vspace{0.5cm}

Funktion des Cut-Operators
\begin{Verbatim}[ frame         = lines, 
                  framesep      = 0.3cm, 
                  labelposition = bottomline,
                  numbers       = left,
                  numbersep     = -0.2cm,
                  xleftmargin   = 0.8cm,
                  xrightmargin  = 0.8cm
                ]
    q(Z) :- p(Z).
    q(1).

    p(X) :- a(X), b(X), !, c(X,Y), d(Y).    
    p(3).
    
    a(1).    a(2).    a(3).
    
    b(2).    b(3).
    
    c(2,2).  c(2,4).
    
    d(3).
\end{Verbatim}



\vspace*{\fill}
\tiny \addtocounter{mypage}{1}
\myrule
Prolog  \hspace*{\fill} Seite \arabic{mypage}
\end{slide}

%%%%%%%%%%%%%%%%%%%%%%%%%%%%%%%%%%%%%%%%%%%%%%%%%%%%%%%%%%%%%%%%%%%%%%%%

\begin{slide}{}
\normalsize
\begin{center}
Abarbeitung des Cut-Operators
\end{center}
\vspace{0.5cm}

\footnotesize
\textbf{Gegeben}: 
\begin{enumerate}
\item Klausel $P \;\texttt{:-}\; Q_1, \cdots, Q_m, \texttt{!}, R_1, \cdots, R_k$ 
\item Anfrage $A$ mit $A\mu = P\mu$
\end{enumerate}
\textbf{Abarbeitung}
\begin{enumerate}
\item neue Anfrage:  $Q_1\mu, \cdots, Q_m\mu, \texttt{!}, R_1\mu, \cdots, R_k\mu$ 

      Auswahl-Punkt falls weitere Klauseln f\"{u}r $A$ vorhanden
\item Falls sp\"{a}ter Abarbeitung von Anfrage  \\[0.3cm]
      \hspace*{1.3cm} 
      $Q_i\sigma, \cdots, Q_m\sigma, \texttt{!}, R_1\sigma, \cdots, R_k\sigma$ \\[0.3cm]
      f\"{u}r  $i\in\{1,\cdots,m\}$ scheitert:

       Cut wird nicht erreicht, hat keine Wirkung.
\item Reduziere Anfrage $\texttt{!}, R_1\sigma, \cdots, R_k\sigma$ \\[0.3cm]
      zu neuer Anfrage $R_1\sigma, \cdots, R_k\sigma$. \\[0.3cm]
      L\"{o}sche  Auswahl-Punkte, die bei Beantwortung der Teilanfragen \\[0.1cm]
      \hspace*{1.3cm} $Q_i\sigma, \cdots, Q_m\sigma, \texttt{!}, R_1\sigma, \cdots, R_k\sigma$ \\[0.3cm]
      gesetzt wurden.

      L\"{o}sche Auswahl-Punkt, der evtl.~im 1.~Schritt gesetzt wurde
\item Falls  Beantwortung der Anfrage \\[0.3cm]
      \hspace*{1.3cm} $R_1\sigma, \cdots, R_k\sigma$. \\[0.3cm]
      scheitert, so scheitert auch Beantwortung von  $A$.
\end{enumerate}

\vspace*{\fill}
\tiny \addtocounter{mypage}{1}
\myrule
Prolog  \hspace*{\fill} Seite \arabic{mypage}
\end{slide}

%%%%%%%%%%%%%%%%%%%%%%%%%%%%%%%%%%%%%%%%%%%%%%%%%%%%%%%%%%%%%%%%%%%%%%%%


\begin{slide}{}
\normalsize
\begin{center}
Sortieren durch Mischen
\end{center}
\vspace{0.5cm}

\footnotesize
\begin{enumerate}
\item $\mathtt{odd}([]) = []$.
\item $\mathtt{odd}([h|t]) = [h|\mathtt{even}(t)]$.
\item $\mathtt{even}([]) = []$.
\item $\mathtt{even}([h|t]) = \mathtt{odd}(t)$.
\item $\mathtt{merge}([], l) = l$.
\item $\mathtt{merge}(l, []) = l$.
\item $x \leq y \rightarrow \mathtt{merge}([x|s], [y|t]) = [x|\mathtt{merge}(s, [y|t])]$.
\item $x  >   y \rightarrow \mathtt{merge}([x|s], [y|t]) = [y|\mathtt{merge}([x|s], t)]$.
\item $\mathtt{sort}([]) = []$.
\item $\mathtt{sort}([x]) = [x]$.
\item $\mathtt{sort}([x,y|t]) = \mathtt{merge}( \mathtt{sort}(\mathtt{odd}([x,y|t])), \mathtt{sort}(\mathtt{even}([x,y|t])))$.
\end{enumerate}

\vspace*{\fill}
\tiny \addtocounter{mypage}{1}
\myrule
Prolog  \hspace*{\fill} Seite \arabic{mypage}
\end{slide}


\end{document}

%%% Local Variables: 
%%% mode: latex
%%% TeX-master: t
%%% End: 
