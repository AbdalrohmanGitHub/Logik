\section{Der Kompaktheits-Satz der Aussagen-Logik}
Das Ziel dieses Abschnittes ist der Beweis des Kompaktheits-Satzes der Aussagen-Logik.  
Dieser Satz liegt tiefer als alle bisher behandelten S\"{a}tze.  Wir ben\"{o}tigen den Kompaktheits-Satz
sp\"{a}ter, um die Widerlegungs-Vollst\"{a}ndigkeit der Pr\"{a}dikaten-Logik beweisen zu k\"{o}nnen.
Wir folgen bei unserer Darstellung des Kompaktheits-Satzes dem Artikel von Jon Barwise
\cite{barwise:1991a} aus dem Handbuch der mathematischen Logik \cite{barwise:1991}.

\begin{Definition}[endlich erf\"{u}llbar, maximal endlich erf\"{u}llbar]
  {\em Eine Menge $M$ von aussagenlogischen Formeln hei\3t \emph{endlich erf\"{u}llbar} (abgek\"{u}rzt: e.e.) genau
    dann, wenn jede endliche Teilmenge von $M$ erf\"{u}llbar ist:
    \\[0.2cm]
    \hspace*{1.3cm} 
    $M$ e.e. \quad g.d.w. \quad 
    $\forall E \subseteq M: \textsl{card}(E) < \infty \rightarrow 
    \exists \mathcal{I} \in \textsc{Ali}: \forall f \in E: \mathcal{I}(f) = \mathtt{true}$
    \\[0.2cm]
    Eine Menge $M$ von aussagenlogischen Formeln hei\3t \emph{maximal endlich erf\"{u}llbar}
    (abgek\"{u}rzt m.e.e.) genau dann, wenn $M$ endlich erf\"{u}llbar ist und wenn au\3erdem f\"{u}r
    jede aussagenlogische Formel $f$ entweder $f$ oder die Negation $\neg f$ ein Element von $M$ ist:
    \\[0.2cm]
    \hspace*{1.3cm}
    $M$ m.e.e. \quad g.d.w. \quad 
    $M$ e.e.$\;$ und $\;\forall f \el \mathcal{F}: f \el M \,\vee\, (\neg f) \el M$. \qed
  }
\end{Definition}

\begin{Satz} \label{satz28}
{\em
  Es sei $M$ maximal endlich erf\"{u}llbar.  Dann gilt 
  \\[0.2cm]
  \hspace*{1.3cm}
  $(f \wedge g) \el M$ \quad g.d.w. \quad $f \el M$ und $g \el M$.
}
\end{Satz}

\noindent
\textbf{Beweis}: Wir beweisen beide Richtungen des ``g.d.w.'' getrennt.
\begin{description}
\item[``$\Rightarrow$''] Sei $(f \wedge g) \el M$.  Wir f\"{u}hren den Beweis indirekt und
  nehmen an, es gelte $f \notel M$. Da
  $M$ maximal endlich erf\"{u}llbar ist, muss dann die Formel $\neg f$ ein Element der Menge
  $M$ sein.  Damit enth\"{a}lt $M$ aber die endliche Teilmenge 
  \\[0.2cm]
  \hspace*{1.3cm}
  $E := \{ f \wedge g, \neg f\}$, 
  \\[0.2cm]
  die offenbar nicht erf\"{u}llbar ist, denn jede Belegung, die $f \wedge g$ wahr macht, macht
  sicher auch $f$ wahr und damit $\neg f$ falsch.  Die Unerf\"{u}llbarkeit von $E$ steht im
  Widerspruch zur endlichen Erf\"{u}llbarkeit von $M$.  Damit muss die Annahme $f \notel M$
  falsch sein und wir haben $f \el M$ gezeigt. 

  Der Nachweis von $g \el M$ kann analog gef\"{u}hrt werden.
\item[``$\Leftarrow$''] Sei nun $f \el M$ und $g \el M$.  
  Wir f\"{u}hren den Beweis indirekt und nehmen an, dass $(f \wedge g) \notel M$ gelte.  Da $M$
  maximal endlich erf\"{u}llbar ist, muss dann die Formel $\neg (f \wedge g)$ ein Element der
  Menge $M$ sein.  Dann enth\"{a}lt $M$ aber die endliche Teilmenge
  \\[0.2cm]
  \hspace*{1.3cm}
  $E := \{ \neg (f \wedge g), f, g \}$
  \\[0.2cm]
  die offensichtlich nicht erf\"{u}llbar ist, denn jede Belegung, die $f$ und $g$ wahr macht,
  macht auch die Formel $f \wedge g$ wahr und muss damit die Formel $\neg (f \wedge g)$
  falsch machen.  Die Unerf\"{u}llbarkeit von $E$ steht im Widerspruch zur endlichen
  Erf\"{u}llbarkeit von $M$.  Damit muss die Annahme $(f \wedge g) \notel M$ falsch sein und
  es gilt $(f \wedge g) \el M$. \qed
\end{description}

\begin{Satz} \label{satz29}
{\em
  Es sei $M$ maximal endlich erf\"{u}llbar. Dann gilt:
  \begin{enumerate}
  \item $(\neg f) \el M$ \quad g.d.w. $f \notel M$.
  \item $(f \vee g) \el M$ \quad g.d.w. $f \el M$ oder $g \el M$.
  \item $(f \rightarrow g) \el M$ \quad g.d.w. $(\neg f) \el M$ oder $g \el M$.
  \item $(f \leftrightarrow g) \el M$ \quad g.d.w. 
        $(f \rightarrow g) \el M$ und $(g \rightarrow f) \el M$.
  \end{enumerate}
}
\end{Satz}

\noindent
\textbf{Beweis}:  Der Beweis der einzelnen Behauptungen verl\"{a}uft ganz analog zu dem Beweis
von Satz \ref{satz28}.  Exemplarisch zeigen wir den Beweis der vierten Behauptung.
Wir beweisen beide  Richtungen des ``g.d.w.'' getrennt.
\begin{description}
\item[``$\Rightarrow$''] Sei $(f \leftrightarrow g) \el M$.  Wir f\"{u}hren den Beweis indirekt und
  nehmen an, es gelte $(f \rightarrow g) \notel M$. Da
  $M$ maximal endlich erf\"{u}llbar ist, muss dann die Formel $\neg (f \rightarrow g)$ ein Element der Menge
  $M$ sein.  Damit enth\"{a}lt $M$ aber die endliche Teilmenge 
  \\[0.2cm]
  \hspace*{1.3cm}
  $E := \{ f \leftrightarrow g, \neg (f \rightarrow g)\}$, 
  \\[0.2cm]
  die offenbar nicht erf\"{u}llbar ist, denn jede Belegung, die $f \leftrightarrow g$ wahr macht, macht
  sicher auch $f \rightarrow g$ wahr und damit $\neg (f \rightarrow g)$ falsch.  Die
  Unerf\"{u}llbarkeit von $E$ steht im 
  Widerspruch zur endlichen Erf\"{u}llbarkeit von $M$.  Damit muss die Annahme $(f \rightarrow g) \notel M$
  falsch sein und wir haben $(f \rightarrow g) \el M$ gezeigt. 

  Der Nachweis von $(g \rightarrow f) \el M$ kann analog gef\"{u}hrt werden.
\item[``$\Leftarrow$''] Sei nun $(f \rightarrow g) \el M$ und $(g \rightarrow f) \el M$.  
  Wir f\"{u}hren den Beweis indirekt und nehmen an, dass $(f \leftrightarrow g) \notel M$ gelte.  Da $M$
  maximal endlich erf\"{u}llbar ist, muss dann die Formel $\neg (f \leftrightarrow g)$ ein Element der
  Menge $M$ sein.  Dann enth\"{a}lt $M$ aber die endliche Teilmenge
  \\[0.2cm]
  \hspace*{1.3cm}
  $E := \{ \neg (f \leftrightarrow g), f \rightarrow g, g \rightarrow f \}$
  \\[0.2cm]
  die offensichtlich nicht erf\"{u}llbar ist, denn jede Belegung, die $f \rightarrow g$ und $g
  \rightarrow f$ wahr macht,
  macht auch die Formel $f \leftrightarrow g$ wahr und muss damit die Formel $\neg (f \leftrightarrow g)$
  falsch machen.  Die Unerf\"{u}llbarkeit von $E$ steht im Widerspruch zur endlichen
  Erf\"{u}llbarkeit von $M$.  Damit muss die Annahme $(f \leftrightarrow g) \notel M$ falsch sein und
  es gilt $(f \leftrightarrow g) \el M$. \qed
\end{description}

\begin{Satz} \label{satz30}
{\em
  Jede maximal endlich erf\"{u}llbare Menge $M$ ist erf\"{u}llbar.
}
\end{Satz}

\noindent
\textbf{Beweis}:
Die Menge $M$ sei maximal endlich erf\"{u}llbar.  Wir konstruieren eine Variablen-Belegung
$\mathcal{I}$ wie folgt:
\\[0.2cm]
\hspace*{1.3cm}
$\mathcal{I} := \bigl\{ \pair(p, \mathtt{true})  \mid p \el M \bigr\} \cup
                \bigl\{ \pair(p, \mathtt{false}) \mid p \notel M \bigr\}$.
\\[0.2cm]
Wir behaupten, dass diese Belegung  jede Formel $f\el M$ wahr und jede Formel $f \notel M$
falsch macht und beweisen diese
Behauptung durch Induktion nach dem Aufbau der Formel $f$. Wir beweisen also
\\[-0.2cm]
\hspace*{1.3cm}
$f \el M$ \quad g.d.w. \quad $\mathcal{I}(f) = \mathtt{true}$
\\[0.2cm]
durch Induktion nach $f$.
\begin{enumerate}
\item $f = p \el \mathcal{P}$, \ $f$ ist also eine aussagenlogische Variable.
    
      Falls $p \el M$ ist, dann folgt $\mathcal{I}(f) = \mathtt{true}$ aus der Definition
      von $\mathcal{I}$.   Falls $p \notel M$ ist, folgt aus der Definition von
      $\mathcal{I}$ sofort $\mathcal{I}(p) = \mathtt{false}$.
      
\item $f = \neg g$.  

      Sei zun\"{a}chst $f \el M$. Nach Satz \ref{satz29} folgt aus $(\neg g) \el M$ sofort
      $g \notel M$.  Nach Induktions-Voraussetzung gilt dann $\textsl{I}(g) = \mathtt{false}$ und
      daraus folgt sofort $\textsl{I}(f) = \textsl{I}(\neg g) = \mathtt{true}$.
      
      Sei nun $f \notel M$. Wieder nach Satz \ref{satz29} folgt aus $(\neg g) \notel M$ sofort
      $g \el M$.  Nach Induktions-Voraussetzung gilt dann $\textsl{I}(g) = \mathtt{true}$ und
      daraus folgt sofort $\textsl{I}(f) = \textsl{I}(\neg g) = \mathtt{false}$.
\item $f = (g \wedge h)$.

      Sei zun\"{a}chst $f \el M$.  Nach Satz \ref{satz29} folgt aus $(g \wedge h) \el M$ sofort
      $g \el M$ und $h \el M$.  Wenden wir die Induktions-Voraussetzung auf $g$ und $h$ an, so sehen
      wir, dass $\textsl{I}(g) = \mathtt{true}$ und
      $\textsl{I}(h) = \mathtt{true}$ gelten muss, woraus sofort
      $\textsl{I}(f) = \textsl{I}(g \wedge h) = \mathtt{true}$ folgt.
      
      Sei nun $f \notel M$. Nach Satz \ref{satz29} folgt aus $(g \wedge h) \notel M$, dass
      entweder $g \notel M$ oder $h \notel M$ ist.  Wir betrachten nur den Fall $g \notel M$, 
      der Fall $h \notel M$ ist analog.  Aus $g \notel M$ folgt mit der Induktions-Voraussetzung,
      dass $\mathcal{I}(g) = \mathtt{false}$ gilt.  Dass impliziert aber die Behauptung 
      $\textsl{I}(g \wedge h) = \mathtt{false}$, womit wir $\mathcal{I}(f) = \mathtt{false}$ haben.
\item $f = (g \vee h)$,
      $f = (g \rightarrow h)$,
      $f = (g \leftrightarrow h)$.
      
      Da der Beweis dieser F\"{a}lle ganz analog zum Beweis des Falles $f = (f \wedge g)$ verl\"{a}uft,
      k\"{o}nnen wir auf einen expliziten Beweis dieser F\"{a}lle verzichten. \qed
\end{enumerate}

\begin{Satz}[Kompaktheits-Satz]
{\em
  Die Menge $M$ sei endlich erf\"{u}llbar.  Dann ist $M$ erf\"{u}llbar, es gibt also eine aussagenlogische
  Belegung $\mathcal{I}$, so dass gilt:
  \\[0.2cm]
  \hspace*{1.3cm}
  $\forall f \in M: \mathcal{I}(f) = \mathtt{true}$.
} 
\end{Satz}

\noindent
\textbf{Beweis}:  Wir werden $M$ so zu einer maximal endlich erf\"{u}llbaren Menge $\widehat{M}$
erweitern, dass 
\\[0.2cm]
\hspace*{1.3cm}
$M \subseteq \widehat{M}$
\\[0.2cm]
gilt.  Da die Menge $\widehat{M}$ nach Satz \ref{satz30} erf\"{u}llbar ist, ist $M$ als Teilmenge von
$\widehat{M}$ dann erst recht erf\"{u}llbar.
Zu diesem Zweck definieren wir eine Folge $(M_n)_{n\in \mathbb{N}}$ von Mengen, die alle endlich
erf\"{u}llbar sind und die in einem gewissen Sinne gegen die Menge $\widehat{M}$ konvergieren.
Die Definition der Mengen $M_n$ verl\"{a}uft durch Induktion nach der Zahl $n$.  Bevor wir diese
Induktion durchf\"{u}hren k\"{o}nnen, ben\"{o}tigen wir eine \emph{Aufz\"{a}hlung} aller aussagenlogischen Formeln.
Darunter verstehen wir eine Folge aussagenlogischer Formeln $(g_n)_{n \in \mathbb{N}}$, in der jede
aussagenlogische Formel mindestens einmal vorkommt, es soll also gelten:
\\[0.2cm]
\hspace*{1.3cm}
$\forall f \in \mathcal{F}: \exists n \in \mathbb{N}: f = g_n$.
\\[0.2cm]
Falls die Menge der Aussagen-Variablen endlich ist, so kann eine solche Folge zum Beispiel dadurch konstruiert
werden, dass wir erst alle Formeln der L\"{a}nge 1, dann die Formeln der L\"{a}nge 2 und so weiter
aufz\"{a}hlen.  

Die induktive Definition der Mengen $M_n$ verl\"{a}uft nun wie folgt.
\begin{enumerate}
\item[I.A.] $n = 0$:  Wir setzen
  \\[0.2cm]
  \hspace*{1.3cm}
  $M_0 := M$.
  \\[0.2cm]
  Dann ist die Menge $M_0$ nach Voraussetzung endlich erf\"{u}llbar.
\item[I.S.] $n \mapsto n + 1$:  Die Definition von $M_{n+1}$ erfolgt \"{u}ber eine Fall-Unterscheidung:
  \\[0.2cm]
  \hspace*{1.3cm}
  $M_{n+1} := \left\{
  \begin{array}[c]{ll}
    M_n \cup \{ f_n \}      & \mbox{falls $M_n \cup \{ f_n \}$ endlich erf\"{u}llbar ist;} \\[0.2cm]
    M_n \cup \{ \neg f_n \} & \mbox{sonst.}
  \end{array}
  \right.
  $
  \\[0.2cm]
  Wir m\"{u}ssen zeigen, dass $M_{n+1}$ endlich erf\"{u}llbar ist.  Es gibt zwei F\"{a}lle.
  \begin{enumerate}
  \item $M_n \cup \{ f_n \}$ ist endlich erf\"{u}llbar.

        Dann gilt $M_{n+1} = M_n \cup \{ f_n \}$ und die Behauptung ist trivial.
  \item $M_n \cup \{ f_n \}$ ist nicht endlich erf\"{u}llbar.

        Da $M_n$ alleine nach Induktions-Voraussetzung endlich erf\"{u}llbar ist, muss es dann eine
        endliche Teilmenge $E = \{ e_1, \cdots, e_k \} \subseteq M_n$ geben, so dass
        \\[0.2cm]
        \hspace*{1.3cm}
        $E \cup \{ f_n \}$
        \\[0.2cm]
        nicht erf\"{u}llbar ist.  Damit gilt dann
        \\[0.2cm]
        \hspace*{1.3cm}
        $\{ e_1, \cdots, e_m, f_n \} \models \falsum$
        \\[0.2cm]
        und daraus folgt
        \\[0.2cm]
        \hspace*{1.3cm}
        $ \models e_1 \wedge \cdots \wedge e_n \rightarrow \neg f_n$.
        \\[0.2cm]
        Wir f\"{u}hren den Beweis nun indirekt und nehmen an, dass die Menge
        \\[0.2cm]
        \hspace*{1.3cm}
        $M_n \cup \{ \neg f_n \}$ 
        \\[0.2cm]
        nicht endlich erf\"{u}llbar w\"{a}re.  Dann gibt es eine endliche Menge 
        $G = \{ g_1, \cdots, g_k \} \subseteq M_n$,
        so dass gilt:
        \\[0.2cm]
        \hspace*{1.3cm}
        $\{ g_1, \cdots, g_k, \neg f_n \} \models \falsum$.
        \\[0.2cm]
        Daraus folgt aber
        \\[0.2cm]
        \hspace*{1.3cm}
        $\models g_1 \wedge \cdots \wedge g_k \rightarrow f_n$.
        \\[0.2cm]
        Insgesamt haben wir damit aber sowohl
        \\[0.2cm]
        \hspace*{1.3cm}
        $\models e_1 \wedge \cdots \wedge e_m \wedge g_1 \wedge \cdots \wedge g_k \rightarrow f_n$
        \\
        als auch 
        \\[-0.2cm]
        \hspace*{1.3cm}
        $\models e_1 \wedge \cdots \wedge e_m \wedge g_1 \wedge \cdots \wedge g_k \rightarrow \neg f_n$.
        \\[0.2cm]
        Zusammengenommen zeigt dies, dass die Menge $G := \{ e_1, \cdots, e_m, g_1, \cdots, g_k \}$
        nicht erf\"{u}llbar ist:
        \\[0.2cm]
        \hspace*{1.3cm}
        $\{ e_1, \cdots, e_m, g_1, \cdots, g_k \} \models \falsum$.
        \\[0.2cm]
        Dies steht im Widerspruch dazu, dass diese Menge eine endliche Teilmenge von $M_n$ ist und
        $M_n$ ist nach Induktions-Voraussetzung endlich erf\"{u}llbar.  Daher haben wir die Annahme,
        dass $M_n \cup \{ \neg f_n \}$ nicht endlich erf\"{u}llbar ist, widerlegt. 
  \end{enumerate}
  Nach Konstruktion der Mengen $M_n$ gilt offenbar
  \\[0.2cm]
  \hspace*{1.3cm}
  $M_0 \subseteq M_1 \subseteq M_2 \subseteq \cdots \subseteq M_n \subseteq M_{n+1} \subseteq \cdots$.
  \\[0.2cm]
  Daher definieren wir die Menge $\widehat{M}$ nun als den Grenzwert der Folge $(M_n)_{n \in \mathbb{N}}$:
  \\[0.2cm]
  \hspace*{1.3cm}
  $\widehat{M} = \bigcup\limits_{n \in \mathbb{N}} M_n$.
  \\[0.2cm]
  Wir zeigen zun\"{a}chst, dass $\widehat{M}$ endlich erf\"{u}llbar ist.  Sei also $E$ eine endliche
  Teilmenge von $\widehat{M}$:
  \\[0.2cm]
  \hspace*{1.3cm}
  $E = \{ e_1, \cdots, e_m \} \subseteq \bigcup\limits_{n \in \mathbb{N}} M_n$.
  \\[0.2cm]
  Dann gibt es f\"{u}r jedes $i \in \{ 1, \cdots, m \}$ eine Zahl $n(i)$, so dass
  \\[0.2cm]
  \hspace*{1.3cm}
  $e_i \in M_{n(i)}$ 
  \\[0.2cm]
  ist.  Wir definieren
  \\[0.2cm]
  \hspace*{1.3cm}
  $\widehat{n} := \max\bigl(\bigl\{n(1), \cdots, n(m) \bigr\}\bigr)$
  \\[0.2cm]
  Daraus folgt aber sofort $E \subseteq M_{\widehat{n}}$ und da $M_{\widehat{n}}$ endlich erf\"{u}llbar
  ist, muss die Menge $E$ als endliche Teilmenge von $M_{\widehat{n}}$ erf\"{u}llbar sein.
  Damit haben wir gezeigt, dass $\widehat{M}$ endlich erf\"{u}llbar ist.

  Als n\"{a}chstes zeigen wir, dass $\widehat{M}$ maximal endlich erf\"{u}llbar ist.  Betrachten wir eine
  beliebige aussagenlogische Formel $f$.  Wir m\"{u}ssen zeigen, dass entweder $f \el \widehat{M}$ oder 
  $(\neg f) \el \widehat{M}$ gilt.  Wir hatten vorausgesetzt, dass die Folge $(f_n)_{n \in \mathbb{N}}$ 
  die Menge aller aussagenlogischen Formeln aufz\"{a}hlt.  Also gibt es ein $n$, so dass $f = f_n$ ist.
  Nach Definition gilt dann aber 
  \\[0.2cm]
  \hspace*{1.3cm}
  $M_{n+1} = M_n \cup \{ f_n \}$ \quad oder \quad 
  $M_{n+1} = M_n \cup \{ \neg f_n \}$
  \\[0.2cm]
  und da $M_{n+1} \subseteq \widehat{M}$ ist, folgt
  \\[0.2cm]
  \hspace*{1.3cm}
  $f_n \el \widehat{M}$ \quad oder \quad
  $(\neg f_n) \el \widehat{M}$.
  \\[0.2cm]
  Also ist $\widehat{M}$ maximal endlich erf\"{u}llbar und daher ist $\widehat{M}$ nach Satz \ref{satz30} erf\"{u}llbar. \qed
\end{enumerate}

%%% Local Variables: 
%%% mode: latex
%%% TeX-master: "logik"
%%% End: 
