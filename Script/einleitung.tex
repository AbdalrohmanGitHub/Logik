\chapter{Einleitung}
In diesem einführenden Kapitel möchte ich zunächst motivieren, warum wir in der Informatik
nicht darum herum kommen uns mit den formalen Methoden der mathematischen Logik zu beschäftigen.
Anschließend gebe ich einen Überblick über den Rest der Vorlesung.

\section{Motivation}
Informationstechnische Systeme 
(im Folgenden kurz als IT-Systeme bezeichnet) gehören zu den komplexesten Systemen, welche
die Menschheit je entwickelt hat.  Das lässt sich schon an dem Aufwand erkennen,
der bei der Erstellung von IT-Systemen anfällt.  So sind im Bereich der Telekommunikations-Industrie
IT-Projekte, bei denen mehr als 1000 Entwickler über mehrere Jahre zusammenarbeiten,
die Regel.  Es ist offensichtlich, dass ein Scheitern solcher Projekte mit enormen
Kosten verbunden ist.  Einige Beispiele mögen dies verdeutlichen.
\begin{enumerate}
\item Am 9.~Juni 1996 stürzte die Rakete Ariane 5 auf ihrem Jungfernflug ab.
      Ursache war ein Kette von Software-Fehlern:  Ein Sensor im Navigations-System
      der Ariane 5 misst die horizontale Neigung und speichert diese zunächst als Gleitkomma-Zahl
      mit einer Genauigkeit von 64 Bit ab.  Später wird dieser Wert dann in eine 
      16 Bit Festkomma-Zahl konvertiert.
      Bei dieser Konvertierung trat ein Überlauf ein, da die zu konvertierende Zahl
      zu groß war, um als 16 Bit Festkomma-Zahl dargestellt werden zu können.
      In der Folge gab das Navigations-System auf dem Datenbus, der dieses System mit
      der Steuerungs-Einheit verbindet, eine Fehlermeldung aus.
       Die Daten dieser Fehlermeldung wurden von der Steuerungs-Einheit als Flugdaten 
      interpretiert.  Die Steuer-Einheit leitete daraufhin eine Korrektur des
      Fluges ein, die dazu führte, dass die Rakete auseinander brach und die 
      automatische Selbstzerstörung eingeleitet werden musste.
      Die Rakete war mit 4 Satelliten beladen. Der wirtschaftliche Schaden, der durch den Verlust dieser
      Satelliten entstanden ist, lag bei mehreren 100 Millionen Dollar.
      
      Ein vollständiger Bericht über die Ursache des Absturzes des Ariane 5 findet sich
      im Internet unter der Adresse \\[0.1cm]
      \hspace*{1.3cm} 
      \href{http://www.ima.umn.edu/~arnold/disasters/ariane5rep.html}{\texttt{http://www.ima.umn.edu/\symbol{126}arnold/disasters/ariane5rep.html}}
\item Die Therac 25 ist ein medizinisches Bestrahlungs-Gerät, das durch 
      Software kontrolliert wird.  Durch  Fehler in dieser Software erhielten 1985
      mindestens 6 Patienten eine Überdosis an Strahlung.  Drei dieser Patienten sind an den Folgen dieser 
      Überdosierung gestorben, die anderen wurden schwer verletzt.

      Einen detailierten Bericht über diese Unfälle finden Sie unter \\[0.1cm]
      \hspace*{1.3cm} 
      \href{http://courses.cs.vt.edu/~cs3604/lib/Therac_25/Therac_1.html}{\texttt{http://courses.cs.vt.edu/\symbol{126}cs3604/lib/Therac\_25/Therac\_1.html}}
\item Im ersten Golfkrieg konnte eine irakische \textsl{Scud} Rakete von dem \textsl{Patriot} Flugabwehrsystem
      aufgrund eines Programmier-Fehlers in der Kontrollsoftware des Flugabwehrsystems
      nicht abgefangen werden.  28 Soldaten verloren dadurch ihr Leben, 100 weitere wurden
      verletzt. \\[0.1cm]
      \hspace*{1.3cm} 
      \href{http://www.ima.umn.edu/~arnold/disasters/patriot.html}{\texttt{http://www.ima.umn.edu/\symbol{126}arnold/disasters/patriot.html}}
\item Im Internet finden Sie unter \\[0.2cm]
      \hspace*{0.0cm}
      \href{http://www.computerworld.com/article/2515483/enterprise-applications/epic-failures--11-infamous-software-bugs.html}{\texttt{http://www.computerworld.com/\\
      \hspace*{0.3cm}
        article/2515483/enterprise-applications/epic-failures--11-infamous-software-bugs.html}}
      \\[0.2cm]
      eine Auflistung von schweren Unfällen, die auf Software-Fehler zurückgeführt werden konnten.
\end{enumerate}
Diese Beispiele zeigen, dass bei der Konstruktion von IT-Systemen mit großer Sorgfalt
und Präzision gearbeitet werden sollte.  Die Erstellung von IT-Systemen muss auf einer 
wissenschaftlich fundierten Basis erfolgen, denn nur dann ist es möglich, die Korrekheit
solcher Systeme zu \emph{verifizieren}, also mathematisch zu beweisen.
Diese wissenschaftliche Basis für die Entwicklung von IT-Systemen ist die Informatik, 
und diese hat ihre Wurzeln in der mathematischen Logik.  Zur präzisen Definition der
Bedeutung logischer Formeln benötigen wir die Mengenlehre, die in der Mathematik-Vorlesung
vorgestellt wird.  Daher sind die Logik und die Mengenlehre die beiden Gebiete, mit denen 
Informatiker sich zu Beginn Ihres Studiums beschäftigen.  Sowohl die Mengenlehre als auch die Logik
haben unmittelbare praktische Anwendungen in der Informatik.
\begin{enumerate}
\item Die Schnittstellen komplexer Systeme können mit Hilfe logischer Formeln
      exakt spezifiziert werden.  Spezifikationen, die mit Hilfe natürlicher Sprache
      erstellt werden, haben gegenüber formalen Spezifikationen den Nachteil,
      dass sie oft mehrdeutig sind.
\item Die korrekte Funktionsweise digitaler Schaltungen kann mit Hilfe automatischer
      Beweiser verifiziert werden.  
\item Die Mengenlehre und die damit verbundene Theorie der Relationen bietet die Grundlage
      der Theorie der relationalen Datenbanken.  
\end{enumerate}
Die obige Auflistung ließe sich leicht fortsetzen, aber
neben den unmittelbaren Anwendungen von Logik und Mengenlehre hat die Beschäftigung mit
diesen beiden Gebiete noch eine andere, sehr wichtige Funktion:
Ohne die Einführung geeigneter Abstraktionen sind komplexe Systeme nicht beherrschbar.
Kein Mensch ist in der Lage, alle Details eines Software-Systems, dass aus mehreren
$100\,000$ Programm-Zeilen besteht, zu verstehen.   Die einzige Chance um ein solches
System zu beherrschen besteht in der Einführung geeigneter Abstraktionen.
Daher gehört ein überdurchschnittliches Abstraktionsvermögen zu den wichtigsten Werkzeugen
eines Informatikers.  Die Beschäfigung mit Logik und Mengenlehre trainiert gerade dieses
abstrakte Denkvermögen. 

% Schließlich gibt es für Sie noch einen sehr gewichtigen Grund, sich intensiv mit Logik und
% Mengenlehre zu beschäftigen, den ich Ihnen nicht verschweigen möchte:  Es handelt sich
% dabei um die Klausur am Ende des ersten Semesters!  

Aus meiner Erfahrung weiß ich, dass einige der Studenten sich unter dem Thema Informatik etwas Anderes
vorgestellt haben als die Diskussion abstrakter Konzepte.  Für diese Studenten ist die Beherrschung
einer Programmiersprache und einer dazugehörigen Programmierumgebung das Wesentliche der Informatik.
Natürlich ist die Beherrschung einer Programmiersprache für einen Informatiker unabdingbar.  Sie
sollten sich allerdings darüber im klaren sein, dass das damit verbundene Wissen sehr vergänglich
ist, denn niemand kann Ihnen heute sagen, in welcher Programmiersprache Sie in 10 Jahren programmieren werden.
Im Gegensatz dazu sind die mathematischen Grundlagen der Informatik wesentlich beständiger.


\section{Überblick über den Inhalt der Vorlesung} 
Die erste Informatik-Vorlesung legt die Grundlagen, die für das weitere Studium
der Informatik benötigt werden.  Bei der Diskussion dieser Grundlagen werden wir uns in dieser
Vorlesung auf die Logik konzentrieren, denn die Mengenlehre ist bereits Gegenstand der
Mathematik-Vorlesung.  Daher ist diese Vorlesung wie folgt aufgebaut:
\begin{enumerate}
\item Die Programmier-Sprache \textsc{SetlX}.

      \href{http://randoom.org/Software/SetlX}{\textsc{SetlX}} (\underline{set} \underline{l}anguage
      e\underline{x}tended) ist eine \emph{mengenbasierte} 
      Programmiersprache, in der dem Programmierer die Operationen der Mengenlehre zur
      Verfügung gestellt werden.  Zusätzlich unterstützt die Sprache \emph{funktionales Programmieren}.
      Da \textsc{SetlX} mengenbasiert ist, ist es einfach möglich,  Algorithmen in der Sprache der
      Mengenlehre zu formulieren.  Eine solche Formulierung ist oft sowohl klarer als auch kürzer
      als die Implementierung des Algorithmus in einer klassischen Programmier-Sprache wie
      beispielsweise \texttt{C} oder \textsl{Java}.  Daher diskutiert das zweite Kapitel die Sprache
      \textsc{SetlX}.  In den folgenden Kapitel werden wir \textsc{SetlX} zur Implementierung
      verschiedener Algorithmen verwenden.
\item Grenzen der Berechenbarkeit.

      Jeder Informatiker sollte wissen, dass bestimmte Probleme nicht algorithmisch entscheidbar
      sind.  Beispielsweise ist die Frage, ob ein gegebenes Programm mit einer vorgegebenen Eingabe
      irgendwann anhält und eine Ergebnis liefert, oder ob es unendlich lange rechnen würde,
      unentscheidbar.  Diese Frage wird auch als das \emph{Halte-Problem} bezeichnet.
      Dieses Problem ist unentscheidbar, was wir im dritten Kapitel beweisen werden.
\item Aussagen-Logik.

      Die Theorie der Logik unterscheidet zwischen \emph{Aussagen-Logik} und
      \emph{Prädikaten-Logik}.  Die Aussagen-Logik untersucht nur die Junktoren 
      \\[0.2cm]
      \hspace*{1.3cm}
      ``$\neg$'', ``$\wedge$'', ``$\vee$'', ``$\rightarrow$'' und ``$\leftrightarrow$'',
      \\[0.2cm]
      während die Prädikaten-Logik zusätzlich noch die Quantoren
      \\[0.2cm]
      \hspace*{1.3cm}
      ``$\forall$'' und ``$\exists$''
      \\[0.2cm]
      betrachtet. Wir wenden uns im vierten  Kapitel zunächst der \emph{Aussagen-Logik}
      zu.  Die Handhabung  aussagenlogischer Formeln ist einfacher als die 
      Handhabung prädikatenlogischer Formeln.  Daher bietet sich die Aussagen-Logik als
      Trainings-Objekt an um mit den Methoden der Logik vertraut zu werden.  Die Aussagen-Logik hat
      gegenüber der Prädikaten-Logik noch einen weiteren Vorteil: Sie ist \emph{entscheidbar}, d.h.~wir
      können ein Programm schreiben, das als Eingabe eine aussagenlogische Formel verarbeitet und welches
      dann entscheidet, ob diese Formel allgemeingültig ist.  Es ist hingegen nicht möglich, ein Programm zu
      erstellen, das für eine prädikatenlogische Formel entscheiden kann, ob diese allgemeingültig ist.

      Ein weiteres Argument, sich zunächst mit der Aussagenlogik zu beschäftigen liefern die Anwendungen
      der Aussagenlogik:  Es gibt in der Praxis eine Reihe von Problemen, die
      bereits mit Hilfe der Aussagenlogik gelöst werden können.  Beispielsweise lässt sich die Frage nach der
      Korrektheit kombinatorischer digitaler Schaltungen auf die Entscheidbarkeit einer aussagenlogischen
      Formel zurückführen.  Außerdem gibt es eine Reihe kombinatorischer Probleme, die sich auf
      aussagenlogische Probleme reduzieren lassen.  Als ein Beispiel zeigen wir, wie sich das
      8-Damen-Problem mit Hilfe der Aussagenlogik lösen lässt.
\item Prädikaten-Logik.

      Im fünften Kapitel behandeln wir die Prädikatenlogik und analysieren den Begriff
      des prädikatenlogischen Beweises mit Hilfe eines \emph{Kalküls}.  Ein
      \emph{Kalkül} ist dabei ein formales Verfahren, einen mathematischen Beweis zu führen.
      Ein solches Verfahren lässt sich programmieren.  Wir stellen zu diesem Zweck den 
      \emph{Resolutions-Kalkül} vor, mit dem sich Beweise führen lassen.
      Dieser Kalkül ist die Grundlage verschiedener Verfahren zum automatischen Beweisen.

      Als Anwendungen der Prädikaten-Logik werden wir schließlich die Systeme \textsl{Prover9} und
      \textsl{Mace4} diskutieren.  Bei \textsl{Prover9} handelt es sich um einen automatischen
      Beweiser, während \textsl{Mace4} eine Programm ist, das dazu benutzt werden kann, ein
      Gegenbeispiel für eine mathematische Vermutung zu finden.

\item Verifikation von Programmen.

      Die Korrektheit eines Programms lässt sich mathematisch nachweisen.  Wir stellen hier drei
      verschiedene Methoden vor: Der Hoare-Kalkül und die Methode der symbolischen
      Programm-Ausführung eignen sich zur Verifikation \emph{iterativer} Prozeduren.  (Iterative
      Programme sind Programme, die hauptsächlich mit Hilfe von Schleifen arbeiten.)
      Demgegenüber ist die Methode der Wertverlaufs-Induktion besser geeignet um die Korrektheit
      \emph{rekursiver} Funktionen nachzuweisen.
\end{enumerate}
\pagebreak

\remark
Zum Schluss möchte ich hier noch ein Paar Worte zum Gebrauch von neuer und alter
Rechtschreibung und der Verwendung von Spell-Checkern in diesem Skript sagen.
Dieses Skript wurde unter Verwendung strengster wirtschaftlicher Kriterien
erstellt.  Im Klartext heißt das: Zeit ist Geld und als Dozent an der DHBW hat man
weder das eine noch das andere.  Daher ist es sehr wichtig zu wissen, wo eine
zusätzliche Investition von Zeit noch einen für die Studenten nützlichen Effekt
bringt und wo dies nicht der Fall ist.  Ich habe mich an aktuellen
Forschungs-Ergebnissen zum Nutzen der Rechtschreibung orientiert. Diese zeigen,
dass es nicht wichtig ist, in welcher Reihenfolge die Bcushatebn in eniem Wrot
setehn, das eniizge was wihtcig ist, ist dass der esrte und der ltzete Bcusthabe
an der rcihitgen Psoiiton sthet. Der Rset knan ein ttolaer Böldisnn sien,
trtodzem knan man ihn onhe Porbelme lseen. Das ist so, wiel wir nciht jdeen
Buhctsaben eniezln lseen, snoedrn das Wrot als gseatmes.  Wie sie sheen, ist das
tastcähilch der Flal. $\displaystyle\smiley$

Nichtsdestotrotz möchte ich Sie darum bitten, mir Tipp- und sonstige Fehler, die Ihnen in diesem
Skript auffallen, per Email an
\\[0.2cm]
\hspace*{1.3cm}
\texttt{karl.stroetmann@dhbw-mannheim.de}
\\[0.2cm]
zu melden.  Es bringt nichts, wenn Sie mir diese Fehler nach der
Vorlesung mitteilen, denn bis ich dazu komme, die Fehler zu korrigieren, habe ich längst vergessen,
was das Problem war.

%%% Local Variables: 
%%% mode: latex
%%% TeX-master: "logik"
%%% End: 
