\documentclass{article}
\usepackage{ngerman}
\usepackage[latin1]{inputenc}
\usepackage{a4wide}
\usepackage{amssymb}
\usepackage{fancyvrb}
\usepackage{alltt}

\def\pair(#1,#2){\langle #1, #2 \rangle}
%\renewcommand{\labelenumi}{(\alph{enumi})}
\renewcommand{\labelenumi}{\arabic{enumi}.}

\newcounter{aufgabe}

\newcommand{\qed}{\hspace*{\fill} $\Box$}
\newcommand{\next}{\vspace*{0.1cm}

\noindent}

\newcommand{\exercise}{\vspace*{0.3cm}
\stepcounter{aufgabe}

\noindent
\textbf{Aufgabe \arabic{aufgabe}}: }

\newcommand{\solution}{\vspace*{0.3cm}

\noindent
\textbf{L�sung}: }

\pagestyle{empty}

\begin{document}
\noindent
\textbf{\large Exercise: \quad \emph{The Prince and the Tigers}}
\vspace*{0.3cm}

\noindent
Once upon a time there was a mighty king who had a daughter that was beautiful beyond
comparison.  As the king was getting old, he wanted to see his daughter married so that
her husband could take over as a new king.  So he proclaimed to his people that he was
going to marry his daughter and that he was looking for a suitable husband.
As the king had no intentions to marry his daughter to some dump-ass, every prospective
prince had to pass a test:  He was lead in a room that had 10 doors.
The 10th door was labeled exit and lead directly into the hall of shame.
Regarding the other doors, rules were more interesting:
Behind one of these doors was the princess and if the candidate opened that door,
he would be the new king.  The  rooms behind the doors that did not contain the princess
 were either just empty or there was a tiger behind the door.  If the prospective prince
 would open the door to a 
room containing a tiger, the promissing prince would have a most unpleasant experience.
Furthermore, every door had an inscription  on it.  If the king's daughter was in the room
behind a given door, the inscription on that door would surely be true. 
If there was a tiger behind a given door, the inscription would be false instead.  Regarding
the empty rooms, things were more complicated:  Either the inscriptions on all the empty rooms
are all true or all of the inscriptions on empty rooms are false.
The inscriptions were as follows:
\begin{description}
\item[Room 1:] The princess is in a room with an odd number and
               the rooms with an even number do not have a tiger.
\item[Room 2:] This room is empty.
\item[Room 3:] The inscription on room no.~5 is true, the inscription on room no.~7
               is wrong and room no.~3 has a tiger.
\item[Room 4:] The inscription on room no.~1 is false, room no.~8 has a tiger, and the
               inscription on room no.~9 is true.
\item[Room 5:] If the inscription on room no.~2 or room no.~4 are true,
               then there is no tiger in room no.~1.
\item[Room 6:] The inscription on room no.~3 is false, the princess is in room no.~2 and
               room no.~4 does not have a tiger.
\item[Room 7:] The princess is in room no.~1 and the inscription on room no.~5 is true.
\item[Room 8:] This room does not have a tiger and room no.~9 is empty.
\item[Room 9:] Neither this room nor room no.~1 has a tiger and the inscription on
               room number 6 is true.
\end{description}
Your task is to write a program that computes the room of the princess.
Use the framework available at:
\noindent
\\[0.2cm]
\hspace*{1.3cm}
\texttt{http://www.ba-stuttgart.de/\symbol{126}stroetma/SETL2/prince-tiger-frame.stl}
\\[0.2cm]
In order to write the propositional formul\ae\ describing the problem
it is usefull to introduce the following propositional variables:
\begin{enumerate}
\item $p_i$ for i in $\{1, \cdots, n\}$ is true iff the princess is in room $i$.
\item $t_i$ for i in $\{1, \cdots, n\}$ is true iff there is a tiger in room $i$.
\item $r_i$ for i in $\{1, \cdots, n\}$ is true iff the inscription on room  $i$ is true.
\item $e$ is true iff the inscriptions of all empty rooms are true.
\end{enumerate}
\end{document}

%%% Local Variables: 
%%% mode: latex
%%% TeX-master: t
%%% End: 
