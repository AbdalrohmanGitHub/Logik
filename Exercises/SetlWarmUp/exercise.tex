\documentclass{article}
\usepackage{german}
\usepackage[latin1]{inputenc}
\usepackage{a4wide}
\usepackage{amssymb}
\usepackage{fancyvrb}
\usepackage{alltt}

\pagestyle{empty}

\renewcommand{\labelenumi}{(\alph{enumi})}
\renewcommand{\labelenumii}{\arabic{enumii}.}

\newcommand{\club }{\ensuremath{\clubsuit   }}
\newcommand{\spade}{\ensuremath{\spadesuit  }}
\newcommand{\heart}{\ensuremath{\heartsuit  }}
\newcommand{\diamo}{\ensuremath{\diamondsuit}}
\def\pair(#1,#2){\langle #1, #2 \rangle}

\begin{document}


\noindent
\textbf{Exercise 1}: A number $m\in \mathbb{N}$ is a  \emph{proper divisor} of a number 
$n \in \mathbb{N}$ iff $m$ is a divisor of $n$ and, furthermore, $m < n$.
A number $n \in \mathbb{N}$ is called a \emph{perfect number} iff $n$ is equal to the sum of all
its proper divisors.  For example, the number $6$ is a perfect number, as we have
\\[0.2cm]
\hspace*{1.3cm}
 $1 + 2 + 3 = 6$.
\begin{enumerate}
\item Implement a procedure \texttt{properDivisor} such that for a natural number $n$ the call \\
      $\mathtt{properDivisor}(n)$ yields the set of all proper divisors of
      $n$.
\item Implement a procedure \texttt{perfect} such that for a given natural number $B$,
      the call $\mathtt{perfect}(n)$ yields true iff $N$ is a perfect number.
\item Compute the set of all perfect numbers less than $10\,000$.
\end{enumerate}
\vspace{0.3cm}

\noindent
\textbf{Exercise 2}:
\begin{enumerate}
\item Implement a procedure \texttt{cd} such that, given two natural numbers $m$ and $n$,
      the call $\mathtt{cd}(m,n)$ computes the set of all common divisors of 
      $m$ and $n$.
\item Implement a procedure \texttt{gcd} such that, given two natural numbers $m$ and $n$,
      the call $\mathtt{gcd}(m,n)$ computes the greatest common divisors of $m$ and $n$.       
\end{enumerate}
\vspace{0.3cm}

\noindent
\textbf{Exercise 3}:
Implement a procedure \texttt{lcm} such that, given two natural numbers $m$ and $n$, the call
$\mathtt{lcm}(m,n)$ computes the least common multiple of $m$ and $n$. 
 \vspace{0.2cm}

\noindent
\textbf{Hint}: We have $\texttt{lcm}(m,n) \leq m \cdot n$ for all $m,n \in \mathbb{N}$.
\vspace{0.3cm}

\noindent
\textbf{Exercise 4}: A poker player (Texas Hold'em) has been dealt the cards  
 $\pair(8,\spade)$ and $\pair(9,\spade)$.  Answer the following questions.
\begin{enumerate}
\item What is the probability that the flop contains at least two cards that have the
      suit $\spade$?
\item What is the probability that all three cards of the flop have the suit $\spade$?
\end{enumerate}
\vspace{0.3cm}

\noindent
\textbf{Homework}:
After their ship has been sunk, three pirates manage to escape to a small island.
The only thing there is are coconuts and one monkey.  They harvest all coconuts, put them
on a big heap and go to sleep.  At the next morning, they want to distribute the coconuts.
 
The first pirate gets up at  22:00.  As he doesn't trust his colleagues, he divides the
heap into three parts of equal size.  As the number of coconuts is not divisible by three, one
coconut is left over.  He gives this coconut to the monkey and puts all the remaining
coconuts back on the heap.

The second pirate gets up at  24:00.  As he doesn't trust his colleagues, he divides the
heap into three parts of equal size.  As the number of coconuts is again not divisible by three, one
coconut is left over.  He gives this coconut to the monkey and puts all the remaining
coconuts back on the heap.

The third pirate gets up at  2:00.  As he doesn't trust his colleagues, he divides the
heap into three parts of equal size.  As the number of coconuts is again not divisible by three, one
coconut is left over.  He gives this coconut to the monkey and puts all the remaining
coconuts back on the heap.

On the next morning all get up and the remaining heap is divided into parts of equal size.
Again, there is one coconut for the monkey left over.  Assuming there were less that 100
coconuts at the beginning, how many coconuts did the pirates start with?
\vspace{0.2cm}
\end{document}

%%% Local Variables: 
%%% mode: latex
%%% TeX-master: t
%%% End: 
