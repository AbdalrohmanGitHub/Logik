\documentclass{article}
\usepackage{german}
\usepackage[latin1]{inputenc}
\usepackage{a4wide}
\usepackage{amssymb}
\usepackage{fancyvrb}
\usepackage{alltt}
\usepackage{hyperref}
\usepackage{fancyvrb}
\usepackage{fancyhdr}
\usepackage{lastpage} 
\usepackage[all]{hypcap}
\hypersetup{
	colorlinks = true, % comment this to make xdvi work
	linkcolor  = blue,
	citecolor  = red,
        filecolor  = Gold,
        urlcolor   = [rgb]{0.5, 0.1, 0.0},
	pdfborder  = {0 0 0} 
}

\renewcommand*{\familydefault}{\sfdefault}
 
\pagestyle{fancy}
\fancyfoot[C]{--- \thepage/\pageref{LastPage}\ ---}

\renewcommand{\labelenumi}{(\alph{enumi})}
\renewcommand{\labelenumii}{\arabic{enumii}.}

\newcounter{aufgabe}
\newcommand{\exercise}{\vspace*{0.3cm}
\stepcounter{aufgabe}

\noindent
\textbf{Aufgabe \arabic{aufgabe}}: }

\newcommand{\club }{\ensuremath{\clubsuit   }}
\newcommand{\spade}{\ensuremath{\spadesuit  }}
\newcommand{\heart}{\ensuremath{\heartsuit  }}
\newcommand{\diamo}{\ensuremath{\diamondsuit}}
\def\pair(#1,#2){\langle #1, #2 \rangle}

\begin{document}


\exercise
Eine Zahl $m\in \mathbb{N}$ ist ein \emph{echter Teiler} einer Zahl
$n \in \mathbb{N}$ genau dann, wenn $m$ ein Teiler von $n$ ist und wenn au�erdem $m < n$ gilt.

Eine Zahl $n \in \mathbb{N}$ hei�t \emph{perfekt}, wenn $n$ gleich der Summe aller echten Teiler von
$n$ ist. Zum Beispiel ist die Zahl $6$ perfekt, denn  die Menge der echten Teiler
von 6 ist $\{1,2,3\}$ und es gilt $1 + 2 + 3 = 6$.
\begin{enumerate}
\item Implementieren Sie eine Prozedur \texttt{echteTeiler}, so dass der Aufruf
      $\mathtt{echteTeiler}(n)$ f�r eine nat�rliche Zahl $n$ die Menge aller echten Teiler von
      $n$ berechnet.
\item Berechnen Sie die Menge aller perfekten Zahlen, die kleiner als $10\,000$ sind.
\end{enumerate}
\vspace{0.3cm}

\exercise
\begin{enumerate}
\item Implementieren Sie eine Prozedur \texttt{gt}, so dass der Aufruf 
      $\mathtt{gt}(m,n)$ f�r zwei nat�rliche Zahlen $m$ und $n$ die Menge aller gemeinsamen
      Teiler von $m$ und $n$ berechnet.

      \textbf{Hinweis}: Berechnen Sie zun�chst die Menge der Teiler von $m$ und 
      die Menge der Teiler von $n$.  �berlegen Sie, wie die Mengenlehre Ihnen weiterhilft,
      wenn Sie diese beiden Mengen berechnet haben.
\item Implementieren Sie nun eine Prozedur \texttt{ggt}, so dass der Aufruf
      $\mathtt{ggt}(m,n)$ den gr��ten gemeinsamen Teiler der beiden Zahlen $m$ und $n$
      berechnet. 
\end{enumerate}
\vspace{0.3cm}

\exercise
Implementieren Sie  eine Prozedur \texttt{kgv}, so dass der Aufruf
$\mathtt{kgv}(m,n)$ f�r zwei nat�rliche  Zahlen $m$ und $n$ das kleinste gemeinsame
Vielfache der Zahlen $m$ und $n$ berechnet. 
 \vspace{0.2cm}

\noindent
\textbf{Hinweis}: Es gilt $\texttt{kgv}(m,n) \leq m \cdot n$.
\vspace{0.3cm}

\exercise
Nehmen Sie an, ein Spieler hat im Poker (Texas Hold'em) die beiden
Karten $\pair(8,\spade)$ und $\pair(9,\spade)$ bekommen.  Schreiben Sie ein
\textsc{SetlX}-Programm, dass die folgenden Fragen beantworten.
\begin{enumerate}
\item Wie gro� ist die Wahrscheinlichkeit, dass im Flop wenigsten zwei weitere Karten
      der Farbe $\spade$ liegen?
\item Wie gro� ist die Wahrscheinlichkeit, dass alle drei Karten im Flop
      die Farbe $\spade$ haben?
\end{enumerate}
\pagebreak

\exercise
Eine Liste der Form $[a, b, c]$ wird als 
\emph{geordnetes} \href{http://de.wikipedia.org/wiki/Pythagoreisches_Tripel}{\emph{pythagoreisches Tripel}}
bezeichnet, wenn 
\\[0.2cm]
\hspace*{1.3cm}
$a^2 + b^2 = c^2$ \quad und \quad $a < b$
\\[0.2cm]
gilt.  Beispielsweise ist $[3,4,5]$ ein geordnetes pythagoreisches Tripel, denn $3^2 + 4^2 = 5^2$.
\begin{enumerate}
\item Implementieren Sie eine Prozedur \texttt{pythagoras}, so dass der Aufruf
      \\[0.2cm]
      \hspace*{1.3cm}
      $\mathtt{pythagoras}(n)$
      \\[0.2cm]
      die Menge aller geordneten  pythagoreischen Tripel $[a,b,c]$ berechnet, f�r die $c \leq n$ ist.
\item Ein pythagoreisches Tripel $[a,b,c]$ ist ein \emph{reduziertes} Tripel, wenn
      die Zahlen $a$, $b$ und $c$ keinen nicht-trivialen gemeinsamen Teiler haben.
      Implementieren Sie eine Funktion \texttt{isReduced}, die als Argumente drei nat�rliche Zahlen 
      $a$, $b$ und $c$ erh�lt und die genau dann \texttt{true} als Ergebnis zur�ck liefert,
      wenn das Tripel $[a, b, c]$ reduziert ist.
\item Implementieren Sie eine Prozedur \texttt{reducedPythagoras}, so dass der Aufruf
      \\[0.2cm]
      \hspace*{1.3cm}
      $\mathtt{reducedPythagoras}(n)$
      \\[0.2cm]
      die Menge aller geordneten pythagoreischen Tripel $[a,b,c]$ berechnet, die reduziert sind.

      Berechnen Sie mit dieser Prozedur alle reduzierten geordneten pythagoreischen Tripel
      $[a,b,c]$, f�r die $c \leq 50$ ist. 
\end{enumerate}


\end{document}

%%% Local Variables: 
%%% mode: latex
%%% TeX-master: t
%%% End: 
