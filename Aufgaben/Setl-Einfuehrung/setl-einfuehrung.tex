\documentclass{article}
\usepackage{german}
\usepackage[latin1]{inputenc}
\usepackage{a4wide}
\usepackage{amssymb}
\usepackage{fancyvrb}
\usepackage{alltt}

\pagestyle{empty}

\begin{document}
\noindent
{\Large \textbf{Aufgaben-Blatt}:  \textsc{Setl}-Programmierung}
\vspace{0.5cm}

\noindent
\textbf{Aufgabe 1:}
Erstellen Sie eine \textsc{Setl2}-Funktion \textsl{ggt}, so dass f�r zwei positive nat�rliche Zahlen
$a$ und $b$ der Aufruf $\textsl{ggt}(a,b)$ den gr��ten gemeinsamen Teiler der Zahlen $a$ und $b$ berechnet.
\vspace{0.1cm}

\noindent
\textbf{Hinweis}: Es ist sinnvoll, eine Hilfsfunktion $\textsl{teiler}$ zu implementieren, so dass
f�r eine positive nat�rliche Zahl $n$ der Aufruf $\textsl{teiler}(n)$ die Menge aller Teiler von $n$
berechnet. 

\vspace{0.3cm}

\noindent
\textbf{Aufgabe 2:}
F�r eine positive nat�rliche Zahl $n$ definieren wir eine \emph{Permutation} der L�nge $n$ als eine
Liste, die jede der Zahlen von $1$ bis $n$ genau einmal enth�lt.  Implementieren Sie eine Funktion 
$\textsl{permutation}$, so dass der Aufruf $\textsl{permutation}(n)$ f�r eine positive nat�rliche
Zahl $n$ die Menge aller Permutationen der L�nge $n$ berechnet.  Beispielsweise soll der Aufruf
$\textsl{permutation}(3)$ als Ergebnis die Menge
\\[0.2cm]
\hspace*{1.3cm}
$\bigl\{ [1,2,3], [1,3,2], [2,1,3], [2,3,1], [3,1,2], [3,2,1] \bigr\}$
\\[0.2cm]
berechnen.
\vspace{0.1cm}

\noindent
\textbf{Hinweis}: Die einfachste L�sung funktioniert rekursiv.
\vspace{0.3cm}

\noindent
\textbf{Aufgabe 3:}
Nehmen Sie an, dass Sie im Poker-Spiel die Karten $\langle 2, \clubsuit \rangle$ und $\langle 6, \clubsuit \rangle$
erhalten haben.  Berechnen Sie die Wahrscheinlichkeit, dass Sie nach dem Flop eine \emph{Stra�e} haben.
\vspace{0.1cm}

\noindent
\textbf{Hinweis}:
Sie erhalten genau dann eine Stra�e im Flop, wenn die im Flop liegenden Karten die Werte
3, 4 und 5 haben.
\vspace{0.3cm}

\noindent
\textbf{Aufgabe 4:}
Implementieren Sie ein \textsc{Setl2}-Prozedur \textsl{fakultaet}, so dass der Aufruf
$\textsl{fakultaet}(n)$ f�r eine positive nat�rliche Zahl $n$ die Fakult�t $n!$ berechnet.
\vspace{0.1cm}

\noindent
\textbf{Hinweis}:
Die Fakult�t $n!$ ist das Produkt aller nat�rlichen Zahlen kleinergleich $n$, es gilt also
\\[0.2cm]
\hspace*{1.3cm}
$n! = 1 * 2 * 3 * \cdots * n$.
\vspace{0.3cm}

\noindent
\textbf{Aufgabe 5:}
Implementieren Sie eine \textsc{Setl2}-Prozedur \textsl{isFunctional}, so dass f�r eine Relation $R$ und
eine Menge $M$ der Aufruf $\textsl{isFunctional}(R, M)$
genau dann das Ergebnis \texttt{true} zur�ck gibt, wenn die
Relation $R$ funktional auf $M$ ist.
\vspace{0.1cm}

\noindent
\textbf{Hinweis}:
Implementieren Sie geeignete Hilfspr�dikate.
\vspace{0.3cm}

\noindent
\textbf{Aufgabe 6:}
Implementieren Sie eine \textsc{Setl2}-Prozedur \textsl{powerSet}, so dass der Aufruf
$\textsl{powerSet}(S)$ f�r eine gegebene Menge $S$ die Potenz-Menge $2^S$ berechnet.  Sie d�rfen dabei
den Operator \texttt{pow} nicht benutzen.
\vspace{0.1cm}

\noindent
\textbf{Hinweis}:  Am einfachsten ist eine rekursive L�sung.
\vspace{0.3cm}

\noindent
\textbf{Aufgabe 7:}
Implementieren Sie eine \textsc{Setl2}-Prozedur \textsl{justOnce}, so dass der Aufruf
$\textsl{justOnce}(L)$ f�r eine gegebene Liste $L$ genau dann den Wert \texttt{true} liefert,
wenn in der Liste $L$ keine Elemente doppelt auftreten.
\vspace{0.1cm}

\end{document}

%%% Local Variables: 
%%% mode: latex
%%% TeX-master: t
%%% End: 
