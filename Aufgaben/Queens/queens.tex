\documentclass{slides}
\usepackage[latin1]{inputenc}
\usepackage{german}
\usepackage{epsfig}
\usepackage{amssymb}

\pagestyle{empty}
\setlength{\textwidth}{17cm}
\setlength{\textheight}{24cm}
\setlength{\topmargin}{0cm}
\setlength{\headheight}{0cm}
\setlength{\headsep}{0cm}
\setlength{\topskip}{0cm}
\setlength{\oddsidemargin}{0cm}
\setlength{\evensidemargin}{0cm}

\newfont{\chess}{chess30}
\newcommand{\chf}{\baselineskip30pt\lineskip0pt\chess}

\newcommand{\cq}{\symbol{34}}
\newcommand{\Ll}{{\cal L}}
\newcommand{\Rl}{{\cal R}}
\newcommand{\NS}{{\cal N\!S}}
\newcommand{\cl}[1]{{\cal #1}}

\newcommand{\bruch}[2]{\frac{\displaystyle\;#1\;}{\displaystyle\;#2\;}}
\newcommand{\Oh}{\mathcal{O}}

\newcounter{mypage}

\begin{document}

\footnotesize
\begin{center}
\textbf{8--Damen Problem}: 
  
\end{center}
K"onnen 8 Damen so auf einem Schach--Brett
aufgestellt werden, dass keine Dame eine andere Dame schlagen kann?

Eine Dame kann eine andere schlagen falls diese
\begin{itemize}
\item in der selben Reihe steht,
\item in der selben Spalte steht, oder
\item in der selben Diagonale steht.
\end{itemize}
\vspace*{-1.0cm}

\footnotesize
\setlength{\unitlength}{1.5cm}

\begin{picture}(10,10)

\thicklines
\put(1,1){\line(1,0){8}}
\put(1,1){\line(0,1){8}}
\put(1,9){\line(1,0){8}}
\put(9,1){\line(0,1){8}}
\put(0.9,0.9){\line(1,0){8.2}}
\put(0.9,9.1){\line(1,0){8.2}}
\put(0.9,0.9){\line(0,1){8.2}}
\put(9.1,0.9){\line(0,1){8.2}}

\thinlines
\multiput(1,2)(0,1){7}{\line(1,0){8}}
\multiput(2,1)(1,0){7}{\line(0,1){8}}
\put(4.15,6.15){{\chess Q}}
\multiput(5.25,6.5)(1,0){4}{\vector(1,0){0.5}}
\multiput(3.75,6.5)(-1,0){3}{\vector(-1,0){0.5}}
\multiput(5.25,7.25)(1,1){2}{\vector(1,1){0.5}}
\multiput(5.25,5.75)(1,-1){4}{\vector(1,-1){0.5}}
\multiput(3.75,5.75)(-1,-1){3}{\vector(-1,-1){0.5}}
\multiput(3.75,7.25)(-1,1){2}{\vector(-1,1){0.5}}
\multiput(4.5,7.25)(0,1){2}{\vector(0,1){0.5}}
\multiput(4.5,5.75)(0,-1){5}{\vector(0,-1){0.5}}
\end{picture}
\vspace*{-1.0cm}

\scriptsize
\vspace*{\fill}
\tiny \addtocounter{mypage}{1}
\rule{17cm}{1mm}
8-Damen-Problem  \hspace*{\fill} Seite \arabic{mypage}


%%%%%%%%%%%%%%%%%%%%%%%%%%%%%%%%%%%%%%%%%%%%%%%%%%%%%%%%%%%%%%%%%%%%%%%%

\begin{slide}{}
\normalsize

\begin{center}
Repr�sentation des Schach-Bretts als AL-Formel
\end{center}
\vspace*{0.5cm}

\footnotesize
\setlength{\unitlength}{2.0cm}
\hspace*{-2.0cm}
\begin{picture}(10,9)
\thicklines
\put(0.9,0.9){\line(1,0){8.2}}
\put(0.9,9.1){\line(1,0){8.2}}
\put(0.9,0.9){\line(0,1){8.2}}
\put(9.1,0.9){\line(0,1){8.2}}
\put(1,1){\line(1,0){8}}
\put(1,1){\line(0,1){8}}
\put(1,9){\line(1,0){8}}
\put(9,1){\line(0,1){8}}
\thinlines
\multiput(1,2)(0,1){7}{\line(1,0){8}}
\multiput(2,1)(1,0){7}{\line(0,1){8}}

%%  for (i = 1; i <= 8; i = i + 1) {
%%for (j = 1; j <= 8; j = j + 1) \{
%%   \put(\$j.15,<9-$i>.35){{\Large p<$i>\$j}}
%%\}
%%  }

\put(1.15,8.40){{ p11}}
\put(2.15,8.40){{ p12}}
\put(3.15,8.40){{ p13}}
\put(4.15,8.40){{ p14}}
\put(5.15,8.40){{ p15}}
\put(6.15,8.40){{ p16}}
\put(7.15,8.40){{ p17}}
\put(8.15,8.40){{ p18}}
\put(1.15,7.40){{ p21}}
\put(2.15,7.40){{ p22}}
\put(3.15,7.40){{ p23}}
\put(4.15,7.40){{ p24}}
\put(5.15,7.40){{ p25}}
\put(6.15,7.40){{ p26}}
\put(7.15,7.40){{ p27}}
\put(8.15,7.40){{ p28}}
\put(1.15,6.40){{ p31}}
\put(2.15,6.40){{ p32}}
\put(3.15,6.40){{ p33}}
\put(4.15,6.40){{ p34}}
\put(5.15,6.40){{ p35}}
\put(6.15,6.40){{ p36}}
\put(7.15,6.40){{ p37}}
\put(8.15,6.40){{ p38}}
\put(1.15,5.40){{ p41}}
\put(2.15,5.40){{ p42}}
\put(3.15,5.40){{ p43}}
\put(4.15,5.40){{ p44}}
\put(5.15,5.40){{ p45}}
\put(6.15,5.40){{ p46}}
\put(7.15,5.40){{ p47}}
\put(8.15,5.40){{ p48}}
\put(1.15,4.40){{ p51}}
\put(2.15,4.40){{ p52}}
\put(3.15,4.40){{ p53}}
\put(4.15,4.40){{ p54}}
\put(5.15,4.40){{ p55}}
\put(6.15,4.40){{ p56}}
\put(7.15,4.40){{ p57}}
\put(8.15,4.40){{ p58}}
\put(1.15,3.40){{ p61}}
\put(2.15,3.40){{ p62}}
\put(3.15,3.40){{ p63}}
\put(4.15,3.40){{ p64}}
\put(5.15,3.40){{ p65}}
\put(6.15,3.40){{ p66}}
\put(7.15,3.40){{ p67}}
\put(8.15,3.40){{ p68}}
\put(1.15,2.40){{ p71}}
\put(2.15,2.40){{ p72}}
\put(3.15,2.40){{ p73}}
\put(4.15,2.40){{ p74}}
\put(5.15,2.40){{ p75}}
\put(6.15,2.40){{ p76}}
\put(7.15,2.40){{ p77}}
\put(8.15,2.40){{ p78}}
\put(1.15,1.40){{ p81}}
\put(2.15,1.40){{ p82}}
\put(3.15,1.40){{ p83}}
\put(4.15,1.40){{ p84}}
\put(5.15,1.40){{ p85}}
\put(6.15,1.40){{ p86}}
\put(7.15,1.40){{ p87}}
\put(8.15,1.40){{ p88}}
\end{picture}
\\[0.1cm]
\hspace*{0.3cm} 
$\texttt{p}ij = \mathtt{true}$ \quad g.d.w. \quad Dame in Zeile $i$ und Spalte $j$ 

\vspace*{\fill}
\tiny \addtocounter{mypage}{1}
\rule{17cm}{1mm}
8-Damen-Problem  \hspace*{\fill} Seite \arabic{mypage}
\end{slide}

%%%%%%%%%%%%%%%%%%%%%%%%%%%%%%%%%%%%%%%%%%%%%%%%%%%%%%%%%%%%%%%%%%%%%%%%

\begin{slide}{}
\normalsize

\begin{center}
Darstellung in \textsc{Setl2}
\end{center}
\vspace*{0.5cm}

\footnotesize
\begin{verbatim}
board = 
    [  
      [p11, p12, p13, p14, p15, p16, p17, p18], 
      [p21, p22, p23, p24, p25, p26, p27, p28], 
      [p31, p32, p33, p34, p35, p36, p37, p38], 
      [p41, p42, p43, p44, p45, p46, p47, p48], 
      [p51, p52, p53, p54, p55, p56, p57, p58], 
      [p61, p62, p63, p64, p65, p66, p67, p68], 
      [p71, p72, p73, p74, p75, p76, p77, p78], 
      [p81, p82, p83, p84, p85, p86, p87, p88] 
    ]
\end{verbatim}

Zugriff auf Zeile $i$, Spalte $j$: \\[0.3cm]
\hspace*{4.3cm} $\texttt{board}(i)(j)$

Aufsteigende Hauptdiagonale:
\begin{verbatim}
 { p81, p72, p63, p54, p45, p36, p27, p18 }
\end{verbatim}
Invariante: \quad $i + j = 9$

Obere aufsteigende Nebendiagonale:
\begin{verbatim}
 { p71, p62, p53, p44, p35, p26, p17 }
\end{verbatim}
Invariante: \quad $i + j = 8$

Allgemeine Invariante f�r aufsteigende Diagonalen: \\[0.1cm]
\hspace*{1.3cm} $i + j = k$

Allgemeine Invariante f�r absteigende Diagonalen: \\[0.1cm]
\hspace*{1.3cm} $i - j = k$

\vspace*{\fill}
\tiny \addtocounter{mypage}{1}
\rule{17cm}{1mm}
8-Damen-Problem  \hspace*{\fill} Seite \arabic{mypage}
\end{slide}



\end{document}

%%% Local Variables: 
%%% mode: latex
%%% TeX-master: t
%%% End: 
