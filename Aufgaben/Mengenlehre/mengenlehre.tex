\documentclass{article}
\usepackage{ngerman}
\usepackage[latin1]{inputenc}
\usepackage{a4wide}
\usepackage{amssymb}
\usepackage{fancyvrb}
\usepackage{alltt}

\def\pair(#1,#2){\langle #1, #2 \rangle}
\renewcommand{\labelenumi}{(\alph{enumi})}
%\renewcommand{\labelenumi}{\arabic{enumi}.}

\newcounter{aufgabe}

\newcommand{\qed}{\hspace*{\fill} $\Box$}
\newcommand{\next}{\vspace*{0.1cm}

\noindent}

\newcommand{\exercise}{\vspace*{0.3cm}
\stepcounter{aufgabe}

\noindent
\textbf{Aufgabe \arabic{aufgabe}}: }

\newcommand{\solution}{\vspace*{0.3cm}

\noindent
\textbf{L�sung}: }


\begin{document}
\noindent
\textbf{\large Aufgaben zur Mengenlehre}
\vspace*{0.3cm}

\exercise 
Geben Sie die folgende Menge durch explizites Auflisten aller Elemente an: 
\\[0.1cm]
\hspace*{1.3cm}
$M := \{ x \in \mathbb{R} \mid x� - 4 \cdot x + 3 = 0 \}$.

\exercise
Geben Sie f�r die folgenden Mengen eine Definition in der Form 
\\[0.1cm]
\hspace*{1.3cm} 
$M := \{ g(x) \mid F(x) \}$
\\[0.1cm]
an. Dabei ist $g(x)$ ein Term und $F(x)$ ist eine logische Formel, die genau dann wahr ist,
wenn $x$ ein Element der Menge $M$ ist.  Beispiel:  F�r die Menge 
\\[0.1cm]
\hspace*{1.3cm}
$M := \{ 2, 4, 8, \cdots \}$
\\[0.1cm]
lautet die Definition 
\\[0.1cm]
\hspace*{1.3cm}
$M := \{ 2^n \mid n \in \mathbb{N} \wedge n \geq 1 \}$.
\\[0.1cm]
Definieren Sie die folgenden Mengen in der oben gezeigten Form:
\begin{enumerate}
\item $M_1 = \{ 5, 10, 15, \cdots \}$
\item $M_2 = \{ 2, 4, 6, \cdots \} \cap \{ 3, 6, 9, \cdots \}$
\item $M_3 = \{ 2, 4, 6, \cdots \} \cup \{ 3, 6, 9, \cdots \}$
\item $M_4 = \{ 2, 4, 6, \cdots \} \;\backslash\; \{ 3, 6, 9, \cdots \}$
\end{enumerate}

\exercise
Es seien $M$ und $N$ Mengen und $R \subseteq M \times N$ sei eine bin�re Relation.
F�r eine Menge $X \subseteq M$ ist der Ausdruck $R(X)$ wie folgt definiert: 
\\[0.1cm]
\hspace*{1.3cm}
$R(X) := \{ y \in N \mid \exists x \in X : \pair(x,y) \in R \bigr\}$
\\[0.1cm]
Beweisen oder widerlegen Sie die folgenden Aussagen:
\begin{enumerate}
\item F�r beliebe Teilmengen $A \subseteq M$ und $B \subseteq M$ gilt 
      \\[0.1cm]
      \hspace*{1.3cm}
      $R(A \cap B) = R(A) \cap R(B)$.
\item F�r beliebe Teilmengen $A \subseteq M$ und $B \subseteq M$ gilt 
      \\[0.1cm]
      \hspace*{1.3cm}
      $R(A \cup B) = R(A) \cup R(B)$.
\item Falls Sie in Teil (a) oder (b) dieser Aufgabe zu dem Schluss kommen, dass
      die Aussage nicht allgemeing�ltig ist, dann �berlegen Sie,
      welche Eigenschaften die Relation $R$ haben muss, damit 
      die Aussage richtig wird.
\end{enumerate}

\exercise
Beweisen oder widerlegen Sie die folgenden Gleichungen:
\begin{enumerate}
\item $(K \;\backslash\; M) \;\backslash\; N = K \;\backslash\; (M \cup N)$
\item $K \;\backslash\; (M \cap N) = (K \;\backslash\; M) \cap ( K \;\backslash\; N)$
\item $K \;\backslash\; (M \cup N) = (K \;\backslash\; M) \cup (K \;\backslash\; N)$
\item Falls Sie in Teil (b) oder (c) zu dem Schluss kommen,
      dass die Gleichung im Allgemeinen nicht gilt, dann geben Sie an,
      wie Sie die rechte Seite der Gleichung ver�ndern m�ssen,
      damit die Gleichung richtig wird.
\end{enumerate}
\pagebreak


\exercise 
F�r eine endliche Menge $M$ bezeichnet $\textsl{card}(M)$ die Anzahl der Elemente.
\begin{enumerate}
\item Geben Sie eine Formel zur Berechnung von $\textsl{card}(M \times M)$ an.
\item Zeigen Sie, dass f�r die Potenz-Menge $2^M$ gilt 
      \\[0.1cm]
      \hspace*{1.3cm}
      $\textsl{card}\bigl(2^M\bigr) = 2^{\textsl{card}(M)}$.
\item Zeigen Sie, dass die Gleichung 
      \\[0.1cm]
      \hspace*{1.3cm}
      $\textsl{card}(A \cup B) = \textsl{card}(A) + \textsl{card}(B)$
      \\[0.1cm]
      im Allgemeinen falsch ist.
\item K�nnen Sie die rechte Seite der Gleichung in Teil (c) so �ndern,
      dass die Gleichung richtig wird?
\item Geben Sie eine Formel zur Berechnung von $\textsl{card}(A \cup B \cup C)$ an.
\end{enumerate}

\exercise
Auf der Menge der nat�rlichen Zahlen werde die Relation $R$ als 
\\[0.1cm]
\hspace*{1.3cm}
$R := \bigl\{ \pair(n, 2\cdot n) \mid n \in \mathbb{N} \bigr\}$
\\[0.1cm]
definiert.  Berechnen Sie den transitiven Abschluss der Relation $R$.

\exercise
Es sei $M$ eine endliche Menge und 
\\[0.1cm]
\hspace*{1.3cm}
$\Gamma: 2^M \rightarrow 2^M$
\\[0.1cm]
sei eine Funktion, die Teilmengen von $M$ in Teilmengen von $M$ abbildet.
Eine Funktion, die Mengen in Mengen abbildet, wird als \emph{Operator} bezeichnet.
Wir nennen den Operator \emph{monoton}, wenn 
\\[0.1cm]
\hspace*{1.3cm}
$\forall A,B \in 2^M: A \subseteq B \rightarrow \Gamma(A) \subseteq \Gamma(B)$
\\[0.1cm]
gilt.  Wir definieren induktiv eine Folge $(X_n)_{n\in\mathbb{N}}$ von Teilmengen der
Menge $M$ wie folgt:
\renewcommand{\labelenumi}{\arabic{enumi}.}
\begin{enumerate}
\item Induktions-Anfang: $n=0$
      \\[0.1cm]
      \hspace*{1.3cm}
      $X_0 := \emptyset$.
\item Induktions-Schritt: $n \mapsto n+1$
      \\[0.1cm]
      \hspace*{1.3cm}
      $X_{n+1} := \Gamma(X_n)$
\end{enumerate}
\renewcommand{\labelenumi}{(\alph{enumi})}
Nehmen Sie im folgenden an, dass $\Gamma$ monoton ist und bearbeiten Sie die folgenden Teilaufgaben.
\begin{enumerate}
\item Zeigen Sie durch vollst�ndige Induktion 
      \\[0.1cm]
      \hspace*{1.3cm}
      $\forall n \in \mathbb{N}: X_n \subseteq X_{n+1}$.
\item Es gibt ein $k\in \mathbb{N}$ so dass gilt:
      $\forall n \in \mathbb{N}: n \geq k \rightarrow X_n = X_k$

      Die Menge $X_k$ definieren wir dann als den \emph{Grenzwert} der Folge $(X_n)_{n\in\mathbb{N}}$:
      \\[0.1cm]
      \hspace*{1.3cm}
      $\lim\limits_{n\rightarrow\infty} X_n := X_k$.
\item Zeigen Sie, dass es eine Menge $X \subseteq M$ gibt, so dass
      $\Gamma(X) = X$ 
      gilt. 

      Eine Menge $Y$ mit der Eigenschaft $\Gamma(Y) = Y$ hei�t \emph{Fixpunkt} des Operators $\Gamma$.
\item Ist $Y$ ein Fixpunkt des Operators $\Gamma$, so gilt 
      \\[0.1cm]
      \hspace*{1.3cm}
      $\lim\limits_{n\rightarrow\infty} X_n \subseteq Y$.
\end{enumerate}
\textbf{Bemerkung}: 
Die Aufgabe zeigt, dass auf einer endlichen Menge $M$ jeder monotone Operator 
\\[0.1cm]
\hspace*{1.3cm}
$\Gamma: 2^M \rightarrow 2^M$
\\[0.1cm]
einen kleinsten Fixpunkt hat und gibt dar�ber hinaus ein Verfahren, um diesen Fixpunkt zu
berechnen. 
\end{document}

%%% Local Variables: 
%%% mode: latex
%%% TeX-master: t
%%% End: 
