\documentclass{article}
\usepackage{german}
\usepackage[latin1]{inputenc}
\usepackage{a4wide}
\usepackage{amssymb}
\usepackage{fancyvrb}
\usepackage{alltt}
\usepackage{epsfig}
\usepackage{hyperref}
\usepackage{fancyhdr}
\usepackage{lastpage}

\usepackage{color}
\hypersetup{
	colorlinks = true, % comment this to make xdvi work
	linkcolor  = blue,
	citecolor  = red,
        filecolor  = blue,
        urlcolor   = [rgb]{0.5, 0.4, 0.0},
	pdfborder  = {0 0 0} 
}

\renewcommand*{\familydefault}{\sfdefault}

\pagestyle{fancy}

\fancyfoot[C]{--- \thepage/\pageref{LastPage}\ ---}
 
\def\pair(#1,#2){\langle #1, #2 \rangle}
%\renewcommand{\labelenumi}{(\alph{enumi})}
\renewcommand{\labelenumi}{\arabic{enumi}.}


\begin{document}
\noindent
\textbf{\large Aufgabe: \quad \emph{God save the King!}}
\vspace*{0.3cm}

\noindent
Schreiben Sie ein \textsc{SetlX}-Programm, welches das folgende R�tsel l�st.
\vspace*{0.3cm}

\begin{minipage}{0.9\linewidth}
{
Es war einmal ein K�nig, der wog 195 Pfund, und seine Tochter, die Prinzessin, wog 105 Pfund. Der
Diener der Prinzessin war noch ein Kind und wog nur 90 Pfund. Dies alles wird so genau berichtet,
weil es eine gro�e Rolle spielte, als die drei eines Tages fliehen mu�ten. Sie gerieten auf der
Flucht in einen Wald und versteckten sich dort in einem Turm. Ungl�cklicherweise brach beim
Besteigen des Turmes die Treppe unter dem schwergewichtigen K�nig zusammen, so da� die drei
oben festsa�en. Als der K�nig aber zuf�llig aus dem Turmfenster blickte, sah er, da� au�en an der
Mauer ein Mechanismus angebracht war, der den Heuaufz�gen glich, welche die Bauern an ihren Scheunen
haben. In der Tat handelte es sich um einen doppelten Korbaufzug. Hoch oben am Dach des Turmes war
eine Rolle befestigt, �ber die ein dickes Seil lief. Ein Korb fand sich direkt unter dem Fenster,
und der andere ber�hrte unten den Boden. Die Prinzessin entdeckte auch eine Inschrift neben dem
Fenster, welche besagte, da� bei der Personenbef�rderung ein Gewichtsunterschied von maximal
15 Pfund zwischen den beiden K�rben erlaubt sei.  W�re der Unterschied gr��er, so w�rde der
schwerere Korb so schnell nach unten rauschen, dass die Passagiere dies nicht �berleben w�rden.
Gl�cklicherweise befand sich oben auf dem Turm noch eine Stahlkugel, die ein Gewicht von 75
Pfund hatte.  Wie gelang es dem K�nig, den Turm zu verlassen?}
\vspace*{0.2cm}

Beachten Sie, dass die Kugel durchaus alleine in dem Korb nach unten fahren darf.  Au{\ss}erdem
sind alle beteiligten Personen stark genug, um die Kugel in den Korb zu heben oder Sie aus dem Korb zu entfernen. 
\end{minipage}
\vspace*{0.3cm}

\noindent
 Verwenden Sie zur L�sung des Problems die Vorlage, die Sie im Netz unter der Adresse
\\[0.2cm]
\hspace*{0.3cm}
\href{https://github.com/karlstroetmann/Logik/blob/master/Aufgaben/Blatt-12/korb-frame.stlx}{\texttt{https://github.com/karlstroetmann/Logik/blob/master/Aufgaben/Blatt-12/korb-frame.stlx}} 
\\[0.2cm]
finden.  Bei diesem Rahmen gehe ich davon aus, dass Sie die Zust�nde des Suchproblems durch Mengen 
von Personen darstellen.  Die Idee ist, dass ein Zustand durch die Menge der Personen beschrieben
wird, die oben auf dem Turm sind.  Zur Vereinfachung der Aufgabe habe ich bereits die Menge
\texttt{All} aller Objekte, die bei dem R�tsel eine Rolle spielen, definiert.  Zus�tzlich enth�lt
der Rahmen noch die Definition der funktionalen Relation \texttt{Weight}, mit der Sie das Gewicht
eines Objektes berechnen k�nnen.  Dazu m\"ussen Sie die Funktion \texttt{weightSet}, mit der Sie das
Gewicht einer Menge von Personen berechnen k�nnen, implementieren.


\end{document}

%%% Local Variables: 
%%% mode: latex
%%% TeX-master: t
%%% End: 
