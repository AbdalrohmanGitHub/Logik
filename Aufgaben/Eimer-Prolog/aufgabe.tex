\documentclass{article}
\usepackage{german}
\usepackage[latin1]{inputenc}
\usepackage{a4wide}
\usepackage{amssymb}
\usepackage{fancyvrb}
\usepackage{alltt}

\pagestyle{empty}

\newcounter{aufgabe}
\newcommand{\exercise}{\vspace*{0.1cm}
\stepcounter{aufgabe}

\noindent
\textbf{Aufgabe \arabic{aufgabe}}: }

\begin{document}
\noindent
{\Large \textbf{Aufgaben-Blatt}: \textsl{Prolog}}
\vspace{0.5cm}



\exercise \emph{Der Wein-H�ndler} \\[0.2cm]
Ein Wein-H�ndler hat ein gro�es Fass Wein und
zwei Eimer, von denen ein Eimer genau drei Liter fasst, w�hrend 
der andere Eimer 5 Liter fasst.  Ziel ist es, den Eimer, der 
f�nf Liter fasst, mit genau vier Litern zu f�llen.  Erlaubte
Operationen sind:
\begin{enumerate}
\item Ein Eimer kann mit Wein aus dem Fass gef�llt werden.
      Danach ist der Eimer vollst�ndig gef�llt.
\item Der Inhalt eines Eimers kann in das Fass zur�ck gegossen werden.
      Danach ist der Eimer leer.
\item Der Inhalt eines Eimers kann in den anderen Eimer umgef�llt
      werden.  Dabei wird aber h�chstens soviel umgef�llt, bis
      der andere Eimer voll ist.
\end{enumerate}
Erstellen Sie ein \textsl{Prolog}-Programm, das die Aufgabe l�st.  Unter
\\[0.2cm]
\hspace*{1.3cm}
\texttt{http://www.dhbw-stuttgart.de/stroetmann/Logic/Prolog/wein-frame.pl}
\\[0.2cm]
finden Sie ein Programm-Ger�st, in dem Sie nur noch das Pr�dikat \texttt{edge}
implementieren m�ssen um das Problem zu l�sen.
\vspace{0.3cm} 

\exercise \emph{Der Quick-Sort-Algorithmus}
\begin{enumerate}
\item Schreiben Sie ein Pr�dikat \texttt{small}, das mit der Typ-Spezifikation \\[0.1cm]
      \hspace*{1.3cm} 
      \texttt{small(+\textsl{Number}, +\textsl{List}(\textsl{Number}), -\textsl{List}(\textsl{Number}))} 
      \\[0.1cm]
      vertr�glich ist.  Der Aufruf  \\[0.1cm]
      \hspace*{1.3cm}  $\mathtt{small}(x, l, \mathtt{S})$ \\[0.1cm]
      soll f�r eine Zahl $x$  und eine Liste von Zahlen $l$ die Liste aller der Zahlen
      aus $l$ berechnen, die kleiner oder gleich $x$ sind.
\item Schreiben Sie ein Pr�dikat \texttt{big}, das mit der Typ-Spezifikation \\[0.1cm]
      \hspace*{1.3cm} 
      \texttt{big(+\textsl{Number}, +\textsl{List}(\textsl{Number}), -\textsl{List}(\textsl{Number}))} 
      \\[0.1cm]
      vertr�glich ist.  Der Aufruf  \\[0.1cm]
      \hspace*{1.3cm}  $\mathtt{big}(x, l, \mathtt{S})$ \\[0.1cm]
      soll f�r eine Zahl $x$  und eine Liste von Zahlen $l$ die Liste aller der Zahlen
      aus $l$ berechnen, die gr��er als $x$ sind.
\item Schreiben Sie ein Pr�dikat \texttt{quick\_sort}, das mit der Typ-Spezifikation \\[0.1cm]
      \hspace*{1.3cm} 
      \texttt{quick\_sort(+\textsl{List}(\textsl{Number}), -\textsl{List}(\textsl{Number}))} 
      \\[0.1cm]
      vertr�glich ist.  Der Aufruf  \\[0.1cm]
      \hspace*{1.3cm}  $\mathtt{quick\_sort}(l, \mathtt{L})$ \\[0.1cm]
      soll die Liste $l$ sortieren.

      Das Pr�dikat \texttt{quick\_sort} soll nach der \emph{divide-and-conquer}-Methode
      arbeiten:
      \begin{enumerate}
      \item Teilen Sie die zu sortierende Liste $l$ zun�chst mit den Pr�dikaten
            \texttt{small/3} und \texttt{big/3} in zwei Listen $s$ und $b$ auf.
            Die  Liste $s$ soll dabei alle Elemente enthalten, die kleiner als
            das erste Element der Liste $l$ sind, w�hrend die Liste $b$ die Elemente
            aus $l$ enth�lt, die gr��er als das erste Element von $l$ sind.
      \item Sortieren Sie die Listen $s$ und $b$ rekursiv.
      \item Fassen Sie die beiden sortierten Listen zu einer sortierten Liste
            zusammen.  Benutzen Sie dazu das vordefinierte Pr�dikat \texttt{append},
            das zwei Listen aneinander h�ngt.
      \end{enumerate}
\end{enumerate}

%\framebox{
%\framebox{
%\hspace*{0.3cm}
%\begin{minipage}[c]{0.8\linewidth}
%\vspace{0.3cm}


%Auch bei dieser Aufgabe handelt es sich um eine Bonus-Abgabe, die nicht
%abgegeben werden braucht und die dazu dient, den bisher erarbeiteten Stoff zu vertiefen.
%Um die Sache etwas interessanter zu gestalten, werde ich
%die beste bis zum 7.~M�rz abgegebene L�sung (im Rahmen einer feierlichen Zeremonie)
%mit einer Flasche Wein pr�mieren.  Werden mehrere gleichgute L�sungen abgegeben,
%so gewinnt die L�sung, die als erstes abgegeben wurde.  
%\vspace{0.3cm}

%Sie k�nnen auch als Gruppe eine L�sung erarbeiten.  
%\vspace{0.3cm}

%\end{minipage}
%\hspace*{0.3cm}

%\vspace{0.3cm}
%}}

\end{document}

%%% Local Variables: 
%%% mode: latex
%%% TeX-master: t
%%% End: 
