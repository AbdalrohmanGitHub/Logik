\documentclass{article}
\usepackage{german}
\usepackage[latin1]{inputenc}
\usepackage{a4wide}
\usepackage{amssymb}
\usepackage{fancyvrb}
\usepackage{alltt}

%\pagestyle{empty}

\renewcommand{\labelenumi}{(\alph{enumi})}
\renewcommand{\labelenumii}{\arabic{enumii}.}

\newcommand{\club }{\ensuremath{\clubsuit   }}
\newcommand{\spade}{\ensuremath{\spadesuit  }}
\newcommand{\heart}{\ensuremath{\heartsuit  }}
\newcommand{\diamo}{\ensuremath{\diamondsuit}}
\def\pair(#1,#2){\langle #1, #2 \rangle}

\begin{document}


\noindent
\textbf{Aufgabe}: 
Es sei $M$ eine endliche Menge und $R$ sei eine Relation auf $M$.  Wir definieren eine Folge von
Relationen $(S_n)_{n\in \mathbb{N}}$ durch Induktion:
\\[0.2cm]
\hspace*{1.3cm}
$S_1 := R$ \quad und \quad $S_{n+1} = R \cup S_n \circ S_n$ 
\\[0.2cm]
Zeigen Sie, dass die Folge $(S_n)_{n\in \mathbb{N}}$ gegen den transitiven Abschluss
$R^+$ konvergiert.
\vspace*{0.3cm}

\noindent
\textbf{L�sung}:
Wir berechnen zun�chst die Werte $S_2$ und $S_3$.  Es gilt
\\[0.2cm]
\hspace*{1.3cm}
$S_2 = R \cup S_1 \circ S_1 = R \cup R \circ R = R^1 \cup R^2$ \quad und
\\[0.2cm]
\hspace*{1.3cm}
$
\begin{array}[t]{lcl}
S_3  & = & R \cup S_2 \circ S_2 \\
     & = & R \cup (R^1 \cup R^2) \circ (R^1 \cup R^2) \\
     & = & R^1 \cup R^1 \circ R^1 \cup R^1 \circ R^2 \cup R^2 \circ R^1 \cup R^2 \circ R^2 \\
     & = & R^1 \cup R^2 \cup R^3 \cup R^3 \cup R^4 \\
     & = & R^1 \cup R^2 \cup R^3 \cup R^4
\end{array}
$
\\[0.2cm]
An dieser Stelle vermuten wir, dass die Folge $(S_n)_{n\in \mathbb{N}}$ durch die Formel 
\\[0.2cm]
\hspace*{1.3cm}
$\displaystyle S_n = \bigcup\limits_{i=1}^{\displaystyle 2^{n-1}} R^i$ \quad f�r alle $n\in \mathbb{N}$
\\[0.2cm]
beschrieben werden kann.  (Wenn Sie diese Gesetzm��gikeit an dieser Stelle noch nicht erkennen
k�nnen, dann m�ssen Sie hier halt noch $S_4$ und $S_5$ ausrechnen.  Sp�tenstens dann ist das
Bildungsgesetz nicht mehr zu �bersehen.)  Wir beweisen diese Gesetzm��gikeit nun durch Induktion
nach $n$.
\begin{enumerate}
\item[I.A.] $n = 1$:

            $\bigcup\limits_{i=1}^{\displaystyle 2^{n-1}} R^i =
             \bigcup\limits_{i=1}^{\displaystyle 2^{1-1}} R^i =
             \bigcup\limits_{i=1}^{\displaystyle 2^0} R^i = 
             \bigcup\limits_{i=1}^{1} R^i = R^1 = R = S_1$
\item[I.S.] $n \mapsto n + 1$:

            $
            \begin{array}{lcll}
              S_{n+1} & = & R \cup S_n \circ S_n \\[0.1cm]
                      & = & \displaystyle
                            R \cup \bigcup\limits_{i=1}^{\normalsize 2^{n-1}} R^i \circ
                                   \bigcup\limits_{j=1}^{\normalsize 2^{n-1}} R^j 
                          & \mbox{nach Induktions-Voraussetzung} \\[0.5cm]
                      & = & \displaystyle
                            R \cup \bigcup\limits_{i=1}^{\normalsize 2^{n-1}} 
                                   \bigcup\limits_{j=1}^{\normalsize 2^{n-1}} R^i \circ R^j 
                          & \mbox{Distributiv-Gesetz} \\[0.5cm]
                      & = & \displaystyle
                            R \cup \bigcup\limits_{i=1}^{\normalsize 2^{n-1}} 
                                   \bigcup\limits_{j=1}^{\normalsize 2^{n-1}} R^{i + j} 
                          & \mbox{Potenz-Gesetz} \\[0.5cm]
                      & = & \displaystyle
                            R \cup \bigcup\limits_{k=2}^{\normalsize 2^{n-1} + 2^{n-1}} R^k \\[0.5cm]
                      & = & \displaystyle
                            R^1 \cup \bigcup\limits_{k=2}^{\normalsize 2^{n}} R^i \\[0.5cm]
                      & = & \displaystyle
                            \bigcup\limits_{i=1}^{\normalsize 2^{(n+1)-1}} R^i 
            \end{array}
            $
\end{enumerate}
Offenbar ist die Folge $(S_n)_{n \in \mathbb{N}}$ eine monoton steigende Folge in dem Sinne, dass
\\[0.2cm]
\hspace*{1.3cm}
$S_n \subseteq S_{n+1}$ 
\\[0.2cm]
gilt, denn wir haben nach dem eben gezeigten ja
\\[0.2cm]
\hspace*{1.3cm}
$\displaystyle S_{n+1} = S_n \cup \bigcup\limits_{\displaystyle i=2^{n-1}+1}^{\displaystyle 2^n} R^i$
\\[0.2cm]
 Daher gibt es genau
wie bei der in der Vorlesung diskutierten Folge $(T_n)_{n \in \mathbb{N}}$ eine Zahl 
$k \in \mathbb{N}$, so dass
\\[0.2cm]
\hspace*{1.3cm}
$S_n = S_k$ \quad f�r alle $n \geq k$
\\[0.2cm]
gilt und damit gilt genau wie in der Vorlesung
\\[0.2cm]
\hspace*{1.3cm}
$S_k = R^+$. \hspace*{\fill} $\Box$
\vspace*{0.3cm}

\noindent
\textbf{Bemerkung}:  Die Folge $(S_n)_{n \in \mathbb{N}}$ konvergiert wesentlich schneller gegen
$R^+$ als die Folge $(T_n)_{n \in \mathbb{N}}$.


\end{document}

%%% Local Variables: 
%%% mode: latex
%%% TeX-master: t
%%% End: 
