\documentclass{article}
\usepackage{german}
\usepackage[latin1]{inputenc}
\usepackage{a4wide}
\usepackage{amssymb}
\usepackage{fancyvrb}
\usepackage{alltt}

\pagestyle{empty}

\renewcommand{\labelenumi}{(\alph{enumi})}
\renewcommand{\labelenumii}{\arabic{enumii}.}

\newcommand{\club }{\ensuremath{\clubsuit   }}
\newcommand{\spade}{\ensuremath{\spadesuit  }}
\newcommand{\heart}{\ensuremath{\heartsuit  }}
\newcommand{\diamo}{\ensuremath{\diamondsuit}}
\def\pair(#1,#2){\langle #1, #2 \rangle}

\begin{document}


\noindent
\textbf{Aufgabe 1}: Eine Zahl $m\in \mathbb{N}$ ist ein \emph{echter Teiler} einer Zahl
$n \in \mathbb{N}$ genau dann, wenn $m$ ein Teiler von $n$ ist und wenn au�erdem $m < n$ gilt.

Eine Zahl $n \in \mathbb{N}$ hei�t \emph{perfekt}, wenn $n$ gleich der Summe aller echten Teiler von
$n$ ist. Zum Beispiel ist die Zahl $6$ perfekt, denn  die Menge der echten Teiler
von 6 ist $\{1,2,3\}$ und es gilt $1 + 2 + 3 = 6$.
\begin{enumerate}
\item Implementieren Sie eine Prozedur \texttt{echteTeiler}, so dass der Aufruf
      $\mathtt{echteTeiler}(n)$ f�r eine nat�rliche Zahl $n$ die Menge aller echten Teiler von
      $n$ berechnet.
\item Berechnen Sie die Menge aller perfekten Zahlen, die kleiner als $10\,000$ sind.
\end{enumerate}
\vspace{0.3cm}

\noindent
\textbf{Aufgabe 2}:
\begin{enumerate}
\item Implementieren Sie eine Prozedur \texttt{gt}, so dass der Aufruf 
      $\mathtt{gt}(m,n)$ f�r zwei nat�rliche Zahlen $m$ und $n$ die Menge aller gemeinsamen
      Teiler von $m$ und $n$ berechnet.
\item Implementieren Sie nun eine Prozedur \texttt{ggt}, so dass der Aufruf
      $\mathtt{ggt}(m,n)$ den gr��ten gemeinsamen Teiler der beiden Zahlen $m$ und $n$
      berechnet. 
\end{enumerate}
\vspace{0.3cm}

\noindent
\textbf{Aufgabe 3}:
Implementieren Sie  eine Prozedur \texttt{kgv}, so dass der Aufruf
$\mathtt{kgv}(m,n)$ f�r zwei nat�rliche  Zahlen $m$ und $n$ das kleinste gemeinsame
Vielfache der Zahlen $m$ und $n$ berechnet. 
 \vspace{0.2cm}

\noindent
\textbf{Hinweis}: Es gilt $\texttt{kgv}(m,n) \leq m \cdot n$.
\vspace{0.3cm}

\noindent
\textbf{Aufgabe 4}: Nehmen Sie an, ein Spieler hat im Poker (Texas Hold'em) die beiden
Karten $\pair(8,\spade)$ und $\pair(9,\spade)$ bekommen.  Schreiben Sie ein
\textsc{SetlX}-Programm, dass die folgenden Fragen beantworten.
\begin{enumerate}
\item Wie gro� ist die Wahrscheinlichkeit, dass im Flop wenigsten zwei weitere Karten
      der Farbe $\spade$ liegen?
\item Wie gro� ist die Wahrscheinlichkeit, dass alle drei Karten im Flop
      die Farbe $\spade$ haben?
\end{enumerate}
\vspace{0.3cm}

\noindent
\textbf{Hausaufgabe}:
Nach dem ihr Schiff gesunken ist, werden drei Matrose an eine einsame Insel gesp�lt.  Dort
wachsen nur Kokosn�sse. Sie pfl�cken alle Kokosn�sse, legen diese auf einen Haufen
und gehen schlafen.  Es sind insgesamt weniger als 100 Kokosn�sse.
Am n�chsten Morgen wollen Sie die Kokosn�sse aufteilen. 

Der erste Matrose steht nachts um 22:00 auf, und da er seinen Kollegen nicht traut, teilt den
Haufen Kokosn�sse in 3 gleich gro�e Haufen.  Dabei bleibt eine Kokosnuss �brig, die er dem auf
der Insel ans�ssigen Affen schenkt. Seinen eigenen Haufen versteckt der Matrose, aus den
�brigen Kokosn�ssen bildet er wieder einen gro�en Haufen und legt sich schlafen.

Der zweite Matrose steht um 24:00 auf, teilt den verbliebenen Haufen in 3 gleiche Teile, wobei
wieder eine Kokosnuss �brig bleibt, die an den Affen f�llt.  Der zweite Matrose versteckt 
seinen eigenen Haufen, bildet aus den �brigen Kokosn�ssen wieder einen gro�en Haufen und legt sich schlafen.

Der dritte Matrose steht um 2:00 auf, teilt den verbliebenen Haufen  in 3 gleiche Teile, wobei
wieder eine Kokosnuss �brig bleibt, die an den Affen f�llt.  Der dritte Matrose versteckt 
seinen eigenen Haufen, bildet aus den �brigen Kokosn�ssen wieder einen gro�en Haufen und legt sich schlafen.

Am n�chsten Morgen stehen alle wieder auf und teilen gemeinsam den verbliebenen Haufen in 
drei gleich gro�e Teile.  Es bleibt wieder eine Kokosnuss f�r den Affen �brig.  
Schreiben Sie ein \textsc{SetlX}-Programm, das die Zahl der am Anfang vorhandenen
Kokosn�sse berechnet.
\vspace{0.2cm}
\end{document}

%%% Local Variables: 
%%% mode: latex
%%% TeX-master: t
%%% End: 
