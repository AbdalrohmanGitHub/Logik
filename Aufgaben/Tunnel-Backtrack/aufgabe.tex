\documentclass{article}
\usepackage{german}
\usepackage[latin1]{inputenc}
\usepackage{a4wide}
\usepackage{amssymb}
\usepackage{fancyvrb}
\usepackage{alltt}

\pagestyle{empty}

\begin{document}
\noindent
{\Large \textbf{Aufgaben-Blatt}: Der Tunnel}
\vspace{0.5cm}



\noindent
Vier Personen, Albert, Bruno, Claudia und Doris m�ssen eine Tunnel durchqueren.  Der Tunnel
ist so eng, dass immer nur zwei Personen gleichzeitig hindurch k�nnen.  Au�erdem braucht
man zum Durchqueren eine Taschen-Lampe.  Die vier haben aber zusammen nur eine
Taschen-Lampe.  Die Zeiten, die Albert, Bruno, Claudia und Doris zum Durchqueren ben�tigen
sind 5 Minuten, 10 Minuten, 20 Minuten und 25 Minuten.  
\begin{enumerate}
\item Schreiben Sie ein \textsl{Prolog}-Programm, welches das Problem l�st.
\item Schreiben Sie ein \textsl{SetlX}-Programm, welches das Problem l�st.
      Das \textsc{SetlX}-Programm soll die L�sung durch Backtracking berechnen.
      Versuchen Sie dabei analog vorzugehen wie in dem unten gezeigten
      \textsc{SetlX}-Programm zur L�sung des 8-Damen-Problems.
\end{enumerate} 


\begin{Verbatim}[ frame         = lines, 
                  framesep      = 0.3cm, 
                  firstnumber   = 1,
                  labelposition = bottomline,
                  numbers       = left,
                  numbersep     = -0.2cm,
                  xleftmargin   = 0.8cm,
                  xrightmargin  = 0.8cm,
                ]
    solve := procedure(l, n) { 
        if (#l == n) {
            return l;
        }
        for (x in [1 .. n ]) {
            try {
                check(l, x);
                // if this works, try to add more queens
                return solve(l + [x], n);
            } catch (e) {
                // do nothing, next value of x is tried automatically
            }
        }
        // if we haven't found a solution so far, there is none
        throw("no solution");
    };
    
    check := procedure(l, x) {
        if (x in l) {
            throw("same row");
        }
        m := #l;
        if (exists (i in {1 .. m} | i-l(i) == m+1-x || i+l(i) == m+1+x)) {
            throw("same diagonal");
        }
    };

    solveAndPrint := procedure(n) {
        try {
            l := solve([], n);
            print(l);
        } catch (e) {
            print("There is no solution for n = $n$.");
        }
    };
\end{Verbatim}

\end{document}

%%% Local Variables: 
%%% mode: latexosis
%%% TeX-master: t
%%% End: 
