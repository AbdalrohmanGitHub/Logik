\documentclass{article}
\usepackage{german}
\usepackage[latin1]{inputenc}
\usepackage{a4wide}
\usepackage{amssymb}
\usepackage{fancyvrb}
\usepackage{alltt}
\usepackage{fancyhdr}

\renewcommand{\labelenumi}{(\alph{enumi})}
\renewcommand{\labelenumii}{\arabic{enumii}.}

\renewcommand{\headrulewidth}{0.0pt}
\renewcommand{\footrulewidth}{0.0pt}
\fancyhead[CO,CE]{}
\fancyfoot[CO,CE]{-- \thepage\ --}

\pagestyle{fancy}

\begin{document}
\noindent
{\Large \textbf{Aufgaben-Blatt}:  Onkologie-Station}
\vspace{0.5cm}

\noindent
\textbf{Aufgabe 1:}
Bearbeiten Sie die folgenden Teilaufgaben:
\begin{enumerate}
\item Schreiben sie ein Pr�dikat \texttt{myMember}, das der Typ-Spezifikation \\[0.1cm]
      \hspace*{1.3cm} \texttt{myMember(+\textsl{Number}, +\textsl{List}(\textsl{Number}))} \\[0.1cm]
      entspricht.  Der Aufruf  \\[0.1cm]
      \hspace*{1.3cm} $\texttt{myMember}(x, l)$ \\[0.1cm]
      soll genau dann erfolgreich sein, wenn die Zahl $x$ in der Liste $l$ auftritt.

\item Schreiben Sie ein Pr�dikat \texttt{intersect} das der Typ-Spezifikation \\[0.1cm]
      \hspace*{1.3cm} 
      \texttt{intersect(+\textsl{List}(\textsl{Number}), +\textsl{List}(\textsl{Number}), -\textsl{List}(\textsl{Number}))} 
      \\[0.1cm]
      entspricht.  Der Aufruf  \\[0.1cm]
      \hspace*{1.3cm}  $\mathtt{intersect}(l_1, l_2, \mathtt{L})$ \\[0.1cm]
      soll f�r zwei Listen $l_1$ und $l_2$ eine Liste $l$ berechnen, die alle die Elemente
      enth�lt, die sowohl in $l_1$ als auch in $l_2$ auftreten.
\item Schreiben Sie ein Pr�dikat \texttt{small}, das der Typ-Spezifikation \\[0.1cm]
      \hspace*{1.3cm} 
      \texttt{small(+\textsl{Number}, +\textsl{List}(\textsl{Number}), -\textsl{List}(\textsl{Number}))} 
      \\[0.1cm]
      gen�gt.  Der Aufruf  \\[0.1cm]
      \hspace*{1.3cm}  $\mathtt{small}(x, l, \mathtt{S})$ \\[0.1cm]
      soll f�r eine Zahl $x$  und eine Liste von Zahlen $l$ die Liste $S$ aller der Zahlen
      aus $l$ berechnen, die kleiner oder gleich $x$ sind.
\item Schreiben Sie ein Pr�dikat \texttt{big}, das der Typ-Spezifikation \\[0.1cm]
      \hspace*{1.3cm} 
      \texttt{big(+\textsl{Number}, +\textsl{List}(\textsl{Number}), -\textsl{List}(\textsl{Number}))} 
      \\[0.1cm]
      gen�gt.  Der Aufruf  \\[0.1cm]
      \hspace*{1.3cm}  $\mathtt{big}(x, l, \mathtt{B})$ \\[0.1cm]
      soll f�r eine Zahl $x$  und eine Liste von Zahlen $l$ die Liste $B$ aller der Zahlen
      aus $l$ berechnen, die gr��er als $x$ sind.
\item Schreiben Sie ein Pr�dikat \texttt{quick\_sort}, das der Typ-Spezifikation \\[0.1cm]
      \hspace*{1.3cm} 
      \texttt{quick\_sort(+\textsl{List}(\textsl{Number}), -\textsl{List}(\textsl{Number}))} 
      \\[0.1cm]
      gen�gt.  Der Aufruf  \\[0.1cm]
      \hspace*{1.3cm}  $\mathtt{quick\_sort}(l, \mathtt{S})$ \\[0.1cm]
      soll die Liste $l$ sortieren.

      Das Pr�dikat \texttt{quick\_sort} soll nach der \emph{divide-and-conquer}-Methode
      arbeiten:
      \begin{enumerate}
      \item Teilen Sie die zu sortierende Liste $l$ zun�chst mit den Pr�dikaten
            \texttt{small/3} und \texttt{big/3} in zwei Listen $s$ und $b$ auf.
            Die  Liste $s$ soll dabei alle Elemente enthalten, die kleiner als
            das erste Element der Liste $l$ sind, w�hrend die Liste $b$ die Elemente
            aus $l$ enth�lt, die gr��er als das erste Element von $l$ sind.
      \item Sortieren Sie die Listen $s$ und $b$.
      \item Fassen Sie die beiden sortierten Listen zu einer sortierten Liste
            zusammen.  Benutzen Sie dazu das Pr�dikat \texttt{concat} aus der Vorlesung.
      \end{enumerate}
\end{enumerate}
\pagebreak

\noindent
\textbf{Aufgabe 2:}
Auf einer Onkologie-Station liegen f�nf Patienten in nebeneinander liegenden Zimmern.
Bis auf einen  der Patienten hat jeder genau eine Zigaretten-Marke geraucht.
Der Patient, der nicht Zigarette geraucht hat, hat Pfeife geraucht.
Jeder Patient f�hrt genau ein Auto und ist
an genau einer Krebs-Art erkrankt.  Zus�tzlich haben Sie die folgenden Informationen:
\begin{enumerate}
\item Im Zimmer neben Michael wird Camel geraucht.
\item Der Trabant-Fahrer raucht Ernte 23 und liegt im Zimmer neben dem 
      Zungen-Krebs Patienten.
\item Rolf liegt im letzten Zimmer und hat Kehlkopf-Krebs.
\item Der West-Raucher liegt im ersten Zimmer.
\item Der Mazda-Fahrer hat Zungen-Krebs und liegt neben dem Trabant-Fahrer.
\item Der Nissan-Fahrer liegt neben dem Zungen-Krebs Patient.
\item Rudolf w�nscht sich Sterbe-Hilfe und liegt zwischen dem Camel-Raucher und dem Trabant-Fahrer.
\item Der Seat Fahrer hat morgen seinen letzten Geburtstag.
\item Der Luckies Raucher liegt neben dem Patienten mit Lungen-Krebs.
\item Der Camel Raucher liegt neben dem Patienten mit Darm-Krebs.
\item Der Nissan Fahrer liegt neben dem Mazda-Fahrer.
\item Der Mercedes-Fahrer raucht Pfeife und liegt neben dem Camel Raucher.
\item Jens liegt neben dem Luckies Raucher.
\item Der Hodenkrebs-Patient hat gestern seine Eier durchs Klo gesp�lt.
\end{enumerate}
Entwickeln Sie ein \textsl{Prolog}-Programm, das die folgenden Fragen beantwortet:
\begin{enumerate}
\item Was raucht der Darmkrebs-Patient?
\item Was f�hrt Kurt f�r ein Auto?
\end{enumerate}
\pagebreak

\noindent

\end{document}

%%% Local Variables: 
%%% mode: latex
%%% TeX-master: t
%%% End: 
