\documentclass{article}
\usepackage{german}
\usepackage[latin1]{inputenc}
\usepackage{a4wide}
\usepackage{amssymb}
\usepackage{fancyvrb}
\usepackage{alltt}
\usepackage{epsfig}
\usepackage{hyperref}
\usepackage{fancyhdr}
\usepackage{lastpage} 
\usepackage{color}
\hypersetup{
	colorlinks = true, % comment this to make xdvi work
	linkcolor  = blue,
	citecolor  = red,
        filecolor  = blue,
        urlcolor   = [rgb]{0.5, 0.4, 0.0},
	pdfborder  = {0 0 0} 
}

\renewcommand*{\familydefault}{\sfdefault}

\pagestyle{fancy}

\fancyfoot[C]{--- \thepage/\pageref{LastPage}\ ---}
 
\def\pair(#1,#2){\langle #1, #2 \rangle}
%\renewcommand{\labelenumi}{(\alph{enumi})}
\renewcommand{\labelenumi}{\arabic{enumi}.}

\begin{document}
\noindent
{\Large \textbf{Aufgaben-Blatt}: \quad Der Tunnel}
\vspace{0.5cm}



\noindent
Vier Personen, Alice, Britney, Charly und Daniel, m�ssen eine Tunnel durchqueren.  Der Tunnel
ist so eng, dass immer nur zwei Personen gleichzeitig hindurch k�nnen.  Au�erdem braucht
man zum Durchqueren des Tunnels eine Fackel.  Die vier haben aber zusammen nur eine einzige
Fackel.  Die Zeiten, die Alice, Britney, Charly und Daniel zum Durchqueren ben�tigen,
sind 1 Minute, 2 Minuten, 4 Minuten und 5 Minuten.  Berechnen Sie einen Plan zum
Durchqueren des Tunnels, der die ben�tigte Zeit minimiert.
Verwenden Sie zur L�sung des Problems die Vorlage, die Sie im Netz unter der Adresse
\noindent
\\[0.2cm]
\hspace*{0.3cm}
\href{https://github.com/karlstroetmann/Logik/blob/master/Aufgaben/Blatt-9/tunnel-frame.stlx}{\texttt{github.com/karlstroetmann/Logik/blob/master/Aufgaben/Blatt-9/tunnel-frame.stlx}} 
\\[0.2cm]
finden.  Gehen Sie bei der L�sung des Problems in folgenden Schritten vor:
\begin{enumerate}
\item Implementieren Sie in Zeile 85 eine Funktion \texttt{timeSet}, die f�r eine gegebene 
      Menge $s$ von Personen berechnet, wie lange es f�r die Gruppe $s$ dauern w�rde, den Tunnel zu
      durchqueren.  Beachten Sie, dass die Menge $s$ auch das Element
      \texttt{\symbol{34}Torch\symbol{34}} enth�lt.

      \textbf{Hinweis}:  Die Funktion \texttt{timeSet} ist als \texttt{closure} deklariert, damit
      Sie bei der Implementierung auf die funktionale Relation \texttt{timeNeeded} zur�ck greifen
      k�nnen.  Diese Relation ist in Zeile 79 bereits definiert.
\item Implementieren Sie in Zeile 95 eine Funktion
      \\[0.2cm]
      \hspace*{1.3cm}
      $\texttt{timeTransition}(\textsl{stateBefore}, \textsl{stateBefore})$.
      \\[0.2cm]
      Hier bezeichnet \textsl{stateBefore} den Zustand am Eingang des Tunnels bevor eine Gruppe den
      Tunnel in eine der beiden Richtungen durchquert, w�hrend \textsl{stateAfter} den Zustand nach
      der Durchquerung beschreibt.  Zust�nde werden dabei  durch Teilmengen der in Zeile 76
      definierten Menge \texttt{all} kodiert.
\item Implementieren Sie in Zeile 105 eine Funktion \texttt{timePath}, so dass der Aufruf
      \\[0.2cm]
      \hspace*{1.3cm}
      $\texttt{timePath}(l)$
      \\[0.2cm]
      f�r eine Liste $l$ von aufeinander folgenden Zust�nden die Zeit berechnet, die insgesamt f�r alle
      Transitionen ben�tigt wird.  

      \textbf{Beispiel}: Falls die Liste $l$ die Form
      \\[0.2cm]
      \hspace*{1.3cm}
      $l = [s_1, s_2, s_3, s_4]$
      \\[0.2cm]
      h�tte, dann k�nnte $\texttt{timePath}(l)$ durch den Ausdruck
      \\[0.2cm]
      \hspace*{1.3cm}
      $\texttt{timeTransition}(s_1, s_2) + \texttt{timeTransition}(s_2, s_3) +\texttt{timeTransition}(s_3, s_4)$
      \\[0.2cm]
      berechnet werden.
\item Definieren Sie in Zeile 115 die Menge aller Zust�nde.  Ein einzelner Zustand ist dabei eine
      Teilmenge der in Zeile 76 definierten Menge \texttt{all}.
\item Definierten Sie in Zeile 117 die Relation \texttt{r1}, mit der die Zustands�berg�nge
      beschrieben werden, bei denen eine Gruppe von Personen den Tunnel vom Eingang zum Ausgang
      durchquert. 
\item Definierten Sie in Zeile 119 die Relation \texttt{r2}, mit der die Zustands�berg�nge
      beschrieben werden, bei denen eine Gruppe von Personen im  Tunnel vom Ausgang zum Eingang
      zur�ck l�uft. 
\item Definieren Sie den Startzustand in Zeile 122.
\item Definieren Sie den Zustand, der am Ende erreicht werden soll, in Zeile 124.
\item Die L�sung wird nun in Zeile 126 berechnet und in Zeile 129 ausgegeben.  Bei der optimalen
      L�sung werden insgesamt 12 Minuten zur Durchquerung des Tunnels ben�tigt.

      Auf meinem Rechner dauert die Berechnung der L�sung etwas weniger als 7 Sekunden.
\end{enumerate}
\end{document}

%%% Local Variables: 
%%% mode: latex
%%% TeX-master: t
%%% End: 
