\documentclass{article}
\usepackage{german}
\usepackage[latin1]{inputenc}
\usepackage{hyperref}
\usepackage{a4wide}
\usepackage{amssymb}
\usepackage{fancyvrb}
\usepackage{alltt}

\hypersetup{
		colorlinks = true, % comment this to make xdvi work
		linkcolor  = blue,
		citecolor  = red,
        filecolor  = Gold,
        urlcolor   = [rgb]{0.5, 0.1, 0.0},
		pdfborder  = {0 0 0} 
}


\begin{document}
\noindent
{\Large \textbf{Aufgaben-Blatt}: Der Tunnel}
\vspace{0.5cm}

%\noindent
%{\em
%Bei diesem Aufgabenblatt handelt es sich um ein au�erplanm��iges Bonus-Blatt,
%das diejenigen, die den Stoff vertiefen m�chten,
%zu Hause bearbeiten k�nnen.  Ich werde in zwei Wochen eine L�sung ins Netz stellen.
%Wer mag, kann das Blatt auch abgeben, aber dazu besteht keine
%Verpflichtung.
%}
%\vspace{0.5cm}


\noindent
Vier Personen, Anton, Bruno, Charly und Daniel m�ssen eine Tunnel durchqueren.  Der Tunnel
ist so eng, dass immer nur zwei Personen gleichzeitig hindurch k�nnen.  Au�erdem braucht
man zum Durchqueren eine Taschen-Lampe.  Die vier haben aber zusammen nur eine
Taschen-Lampe.  Die Zeiten, die Anton, Bruno, Charly und Daniel zum Durchqueren ben�tigen
sind 1 Minute, 2 Minuten, 4 Minuten und 5 Minuten.  Berechnen Sie einen Plan zum
Durchqueren des Tunnels, der die ben�tigte Zeit minimiert.
     
Da es sich bei diesem Problem nicht um ein reines Suchproblem handelt, m�ssen wir unsere
Begriffs-Bildungen erweitern um es mit Hilfe der Mengenlehre l�sen zu k�nnen.  Ist eine
Menge $P$ von Punkten gegeben, so definieren wir eine \emph{gewichtete Relation} auf $P$ als
eine Menge, die Elemente der Form
\[ \bigl\langle \langle x, y \rangle, d \bigr\rangle \]
enth�lt.  Dabei sind  $x$ und $y$ Elemente der Menge $P$, so dass es einen direkten Weg von $x$
nach $y$ gibt, der die L�nge $d$ hat.  Wir setzen voraus, dass diese L�nge $d$ immer eine
nat�rliche Zahl ist.  Ist $D$ eine gewichtete Relation auf der Menge $P$ , so gilt also
\[ D \subseteq (P \times P) \times \mathbb{N}  \] 
Ein \emph{gewichteter Pfad} auf $P$ ist dann ein Paar der Form
\[ \bigl\langle \textsl{list}, d \bigr\rangle. \]
Dabei ist \textsl{list} eine Liste von Punkten aus $P$ der Form $[x_1,x_2,\cdots,x_n]$,
so dass es eine direkte Verbindung von $x_i$ nach $x_{i+1}$ gibt.  Die Zahl $d$ gibt
die Gesamtl�nge der Verbindung an. 
Unter \\[0.2cm]
\hspace*{0.8cm}      
\href{http://www.dhbw-stuttgart.de/stroetmann/Logic/SetlX/tunnel-frame.stlx}{\texttt{http://www.dhbw-stuttgart.de/stroetmann/Logic/SetlX/tunnel-frame.stlx}}
\\[0.2cm]
finden Sie ein Program-Ger�st, in dem Sie noch verschiedene Teile implementieren m�ssen
um das Problem zu l�sen. 
\begin{enumerate}
\item In Zeile 103 sollen Sie die Menge $p$ der m�glichen Punkte definieren.
       Wir stellen Punkte durch Paare der Form
       $[s, l]$ dar.  
       Dabei ist $s$S die Menge der Personen am Eingang des Tunnels.
       $l$ ist 1, wenn die Taschen-Lampe am Eingang ist.  Falls die Taschen-Lampe
       sich am Ausgang befindet, hat $l$ den Wert 0.
\item In Zeile 64 sollen Sie eine Prozedur \texttt{pruefeDauer} implementieren,
      die folgende Argumente bekommt:
      \begin{enumerate}
      \item \textsl{dauer} ist eine funktionale Relation die f�r jede Person
            angibt, wie lange diese Person ben�tigt, um den Tunnel zu durchqueren.
            Die Relation \textsl{dauer} ist bereits in Zeile 97 definiert.
      \item $x$ und $y$ sind zwei Punkte aus der in Aufgabe 1 definierten Menge $p$.
      \item $d$ ist eine nat�rliche Zahl.
      \end{enumerate}
      Die Prozedur gibt als Ergebnis \texttt{true} zur�ck, wenn
      der Zustand $y$ aus dem Zustand $x$ dadurch erreicht werden kann, dass eine Gruppe
      von Personen den Tunnel durchquert und wenn diese Gruppe daf�r die Zeit $d$ ben�tigt.
\item In Zeile 113 sollen Sie dann unter Benutzung der Prozedur \texttt{pruefeDauer}
      eine gewichtete Relation $r$ berechnen, die das zu l�sende Problem beschreibt.
\item In Zeile 121 und 123 sollen Sie den Startzustand und den Endzustand spezifizieren.
\item In Zeile 125 werden alle die gewichteten Pfade berechnet, die aus h�chstens
      7 Schritten bestehen.       
\item In Zeile 127 m�ssen Sie aus der Menge \texttt{allPathes} alle die Pfade ausw�hlen,
      die zum Ziel f�hren.
\item In Zeile 129 berechnen Sie die minimale Dauer, mit der das Ziel erreicht werden kann.
\item In Zeile 132 w�hlen Sie aus allen m�glichen Pfaden, die vom Start zum Ziel f�hren,
      einen Pfad mit minimaler Dauer aus.
\end{enumerate}

\end{document}

%%% Local Variables: 
%%% mode: latex
%%% TeX-master: t
%%% End: 
