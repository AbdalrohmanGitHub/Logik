\documentclass{article}
\usepackage{german}
\usepackage[latin1]{inputenc}
\usepackage{a4wide}
\usepackage{amssymb}
\usepackage{fancyvrb}
\usepackage{alltt}

\pagestyle{empty}

\begin{document}
\noindent
{\Large \textbf{Aufgaben-Blatt}: Der Tunnel}
\vspace{0.5cm}



\noindent
Vier Personen, Anton, Bruno, Charly und Daniel, m�ssen eine Tunnel durchqueren.  Der Tunnel
ist so eng, dass immer nur zwei Personen gleichzeitig hindurch k�nnen.  Au�erdem braucht
man zum Durchqueren eine Taschen-Lampe.  Die vier haben aber zusammen nur eine
Taschen-Lampe.  Die Zeiten, die Anton, Bruno, Charly und Daniel zum Durchqueren ben�tigen
sind 1 Minute, 2 Minuten, 4 Minuten und 5 Minuten.  Berechnen Sie einen Plan zum
Durchqueren des Tunnels, der die ben�tigte Zeit minimiert.
     
Da es sich bei diesem Problem nicht um ein reines Suchproblem handelt, m�ssen wir unsere
Begriffs-Bildungen erweitern um es mit Hilfe der Mengenlehre l�sen zu k�nnen.  Ist eine
Menge $P$ von Punkten gegeben, so definieren wir eine \emph{gewichtete Relation} auf $P$ als
eine Menge, die Elemente der Form
\[ \bigl\langle \langle x, y \rangle, d \bigr\rangle \]
enth�lt.  Dabei sind  $x$ und $y$ Elemente der Menge $P$, so dass es einen direkten Weg von $x$
nach $y$ gibt, der die L�nge $d$ hat.  Wir setzen voraus, dass diese L�nge $d$ immer eine
nat�rliche Zahl ist.  Ist $D$ eine gewichtete Relation auf der Menge $P$, so gilt also
\[ D \subseteq (P \times P) \times \mathbb{N}  \]
Wir werden $D$ als Funktion auffassen und anstelle von
\[ \bigl\langle \langle x, y \rangle,  d \bigr\rangle \in D \qquad \mbox{die Schreibweise} \qquad d = D(x,y) \]
verwenden.  Ein \emph{gewichteter Pfad} auf $P$ ist dann ein Paar der Form
\[ \bigl\langle [x_1,x_2,\cdots,x_n], d \bigr\rangle. \]
Dabei ist $[x_1,x_2,\cdots,x_n]$ eine Liste von Punkten aus $P$,
so dass es f�r alle $i=1,\cdots,n-1$ eine direkte Verbindung von $x_i$ nach $x_{i+1}$ gibt.  Die Zahl $d$ gibt
die Gesamtl�nge der Verbindung an, es gilt also
\[ d = D(x_1, x_2) + D(x_2, x_3) + \cdots + D(x_{n-1}, x_n). \]
Versuchen Sie ein Programm zu schreiben, dass die gestellte Aufgabe l�st.
Am einfachsten ist es, wenn Sie dabei von dem Programm zur L�sung des
\emph{Wolf-Ziege-Kohl}-Problem ausgehen.

Zu diesem Zweck ist es erfoderlich, dass Sie die dort vorhandene Funktion
$\textsl{reachable}()$ s ab�ndern, dass dort anstelle einer Relation $R$ eine gewichtete
Relation $D$ verwendet wird.  Zus�tzlich ist es sinnvoll die Berechnung von
$\textsl{reachable}()$ durch Angabe eine Schranke $l$ so zu begrenzen, dass nur Wege, die
k�rzer als $l$ sind, berechnet werden.  Au�erdem muss die Hilfs-Funktion
$\textsl{path\_product}()$ ver�ndert werden, denn beim Aufruf
\\[0.2cm]
\hspace*{1.3cm}
$\textsl{path\_product}(P, D)$
\\[0.2cm]
ist $P$ nun eine Menge von gewichteten Pfaden, w�hrend $D$ eine gewichtete Relation ist.
\vspace{0.5cm}

\framebox{
\framebox{
\hspace*{0.3cm}
\begin{minipage}[c]{12cm}
\vspace*{0.3cm}

Die beste bis zum 17.~Februar um 21:00 abgegebene L�sung wird von mir
(im Rahmen einer feierlichen Zeremonie)
mit einer Flasche Wein pr�miert.  Gehen mehrere gleich gute L�sungen ein, 
so gewinnt die L�sung, die als erstes abgegeben wurde.  Die L�sungen sollten per Email
an folgende Adresse geschickt werden:
\\[0.2cm]
\hspace*{1.3cm} \texttt{stroetmann\symbol{64}ba-stuttgart.de}
\vspace{0.3cm}

\end{minipage}
\hspace*{0.3cm}
}}

\end{document}

%%% Local Variables: 
%%% mode: latex
%%% TeX-master: t
%%% End: 
