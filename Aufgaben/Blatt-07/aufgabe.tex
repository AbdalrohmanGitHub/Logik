\documentclass{article}
\usepackage{german}
\usepackage[latin1]{inputenc}
\usepackage{a4wide}
\usepackage{amssymb}
\usepackage{fancyvrb}
\usepackage{alltt}
\usepackage{epsfig}
\usepackage{hyperref}
\usepackage{fancyhdr}
\usepackage{lastpage} 
\usepackage{color}
\hypersetup{
	colorlinks = true, % comment this to make xdvi work
	linkcolor  = blue,
	citecolor  = red,
        filecolor  = blue,
        urlcolor   = [rgb]{1.0, 0.0, 0.0},
	pdfborder  = {0 0 0} 
}

\renewcommand*{\familydefault}{\sfdefault}

\pagestyle{fancy}

\fancyfoot[C]{--- \thepage/\pageref{LastPage}\ ---}
 
\def\pair(#1,#2){\langle #1, #2 \rangle}
%\renewcommand{\labelenumi}{(\alph{enumi})}
\renewcommand{\labelenumi}{\arabic{enumi}.}


\begin{document}
\noindent
\textbf{\large Aufgabe: \quad \emph{Eine Logelei}}
\vspace*{0.3cm}

\noindent
Ein Teil der Familie Meier besucht die Familie M�ller.  Ziel der Aufgabe ist es herauszufinden,
wer genau der Familie M�ller einen Besuch abstattet.  Die folgenden Fakten sind gegeben:
\begin{enumerate}
\item Wenn Herr Meier kommt, bringt er auch Frau Meier mit.
\item Mindestens eines der beiden Kinder Walter und Katrin wird kommen.
\item Entweder kommt Frau Meier oder Franziska, aber nicht beide.
\item Entweder kommen Fransizka und Katrin zusammen oder beide kommen nicht.
\item Wenn Walter kommt, dann kommen auch Katrin und Herr Meier.
\end{enumerate}


\noindent
 Verwenden Sie zur L�sung des Problems die Vorlage, die Sie im Netz unter der Adresse
\noindent
\\[0.2cm]
\hspace*{0.3cm}
\href{https://github.com/karlstroetmann/Logik/blob/master/Aufgaben/Blatt-07/logelei.stlx}{\texttt{github.com/karlstroetmann/Logik/blob/master/Aufgaben/Blatt-07/logelei-frame.stlx}} 
\\[0.2cm]
finden.  Dieser Rahmen enth�lt bereits eine Implementierung der Funktion
\\[0.2cm]
\hspace*{1.3cm}
$\texttt{evaluate}(f, I)$,
\\[0.2cm]
die eine aussagenlogische Formel $f$ unter einer gegebenen Variablen-Belegung $I$ auswertet.  Gehen Sie �hnlich vor
wie bei dem Programm
\\[0.2cm]
\hspace*{0.3cm}
\href{https://github.com/karlstroetmann/Logik/blob/master/SetlX/watson.stlx}{\texttt{github.com/karlstroetmann/Logik/blob/master/SetlX/watson.stlx}}, 
\\[0.2cm]
das wir in der Vorlesung besprochen haben.
\end{document}

%%% Local Variables: 
%%% mode: latex
%%% TeX-master: t
%%% ispell-local-dictionary: "deutsch8"
%%% End: 
