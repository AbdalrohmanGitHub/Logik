\documentclass{article}
\usepackage{german}
\usepackage[latin1]{inputenc}
\usepackage{a4wide}
\usepackage{amssymb}
\usepackage{fancyvrb}
\usepackage{alltt}

\pagestyle{empty}

\begin{document}
\noindent
{\Large \textbf{Aufgaben-Blatt}: Der Wein-H�ndler}
\vspace{1.0cm}



\noindent
Ein Wein-H�ndler hat ein gro�es Fass Wein und
zwei Eimer, von denen ein Eimer genau drei Liter fasst, w�hrend 
der andere Eimer 5 Liter fasst.  Ziel ist es, den Eimer, der 
f�nf Liter fasst, mit genau vier Litern zu f�llen.  Erlaubte
Operationen sind:
\begin{enumerate}
\item Ein Eimer kann mit Wein aus dem Fass gef�llt werden.
      Danach ist der Eimer vollst�ndig gef�llt.
\item Der Inhalt eines Eimers kann in das Fass zur�ck gegossen werden.
      Danach ist der Eimer leer.
\item Der Inhalt eines Eimers kann in den anderen Eimer umgef�llt
      werden.  Dabei wird aber h�chstens soviel umgef�llt, bis
      der andere Eimer voll ist.
\end{enumerate}
Erstellen Sie ein \textsc{SetlX}-Programm, das die Aufgabe l�st.
Sie k�nnen sich an dem \textsc{SetlX}-Programm orientieren, mit dem wir das Problem 
``\emph{Bauer, Wolf, Ziege, Kohl}'' gel�st haben.
\vspace{1.5cm} 

%\framebox{
%\framebox{
%\hspace*{0.3cm}
%\begin{minipage}[c]{0.8\linewidth}
%\vspace{0.3cm}


%Auch bei dieser Aufgabe handelt es sich um eine Bonus-Abgabe, die nicht
%abgegeben werden braucht und die dazu dient, den bisher erarbeiteten Stoff zu vertiefen.
%Um die Sache etwas interessanter zu gestalten, werde ich
%die beste bis zum 7.~M�rz abgegebene L�sung (im Rahmen einer feierlichen Zeremonie)
%mit einer Flasche Wein pr�mieren.  Werden mehrere gleichgute L�sungen abgegeben,
%so gewinnt die L�sung, die als erstes abgegeben wurde.  
%\vspace{0.3cm}

%Sie k�nnen auch als Gruppe eine L�sung erarbeiten.  
%\vspace{0.3cm}

%\end{minipage}
%\hspace*{0.3cm}

%\vspace{0.3cm}
%}}

\end{document}

%%% Local Variables: 
%%% mode: latex
%%% TeX-master: t
%%% End: 
