\documentclass{article}
\usepackage{german}
\usepackage[latin1]{inputenc}
\usepackage{a4wide}
\usepackage{amssymb}
\usepackage{fancyvrb}
\usepackage{alltt}
\usepackage{hyperref}

\pagestyle{empty}

\begin{document}
\noindent
{\Large \textbf{Aufgaben-Blatt}: Missionare und Kannibalen}
\vspace{0.5cm}


\noindent
Drei Missionare und drei Kannibalen wollen zusammen einen Fluss 
�berqueren. Sie haben nur ein Boot, indem maximal zwei Passagiere fahren k�nnen.  
Sowohl die Kannibalen als auch die Missionare k�nnen rudern.
Die Kannibalen sind hungrig, wenn die Missionare an einem der Ufer in der Unterzahl sind, 
haben sie ein Problem.  Die Aufgabe besteht darin, einen Fahrplan zu 
erstellen, so dass hinterher alle das andere  Ufer erreichen und die
Missionare zwischendurch kein Problem bekommen.                                                          
Unter 
\\[0.2cm]
\hspace*{-0.8cm}      
\href{https://github.com/karlstroetmann/Logik/blob/master/Aufgaben/Blatt-2/missionare-frame.stlx}{\texttt{https://github.com/karlstroetmann/Logik/blob/master/Aufgaben/Blatt-2/missionare-frame.stlx}} 
\\[0.2cm]
finden Sie ein Program-Ger�st, in dem Sie noch verschiedene Teile implementieren m�ssen
um das Problem zu l�sen.   F�r Sie relevant ist erst der Code ab Zeile 124.
\begin{enumerate}
\item In Zeile 137 ist die Menge \texttt{p} aller m�glichen Zust�nde zu definieren.  Die Idee
      ist, dass jeder Zustand durch ein Tripel der Form
      \\[0.2cm]
      \hspace*{1.3cm}
      \texttt{[$m$, $k$, $b$]}
      \\[0.2cm]
      dargestellt wird.  Dabei ist $m$ die Anzahl der Missionare am linken Ufer, $k$ ist
      die Zahl der Kannibalen und $b$ ist die Zahl der Boote.
\item Definieren Sie in Zeile 142 eine Prozedur $\texttt{problem}(m, k)$. Dabei ist $m$ die Zahl der
      Missionare  und $k$ ist die Zahl der Kannibalen.
      Die Prozedur soll genau dann \texttt{true} liefern, wenn es f�r die Missionare ein Problem gibt.
\item Definieren Sie  eine Relation $r_{1}$ auf der Menge der Zust�nde $p$.
      Ein Paar 
      \[ \bigl\langle \langle m_1, k_1, b_1 \rangle, \langle m_2, k_2, b_2 \rangle \bigr\rangle \]
      soll dann in $R_{1}$ liegen, wenn der Zustand $\langle m_2, k_2, b_2\rangle$ aus dem Zustand 
      $\langle m_1, k_1, b_1\rangle$
      dadurch hervorgeht, dass das Boot vom linken Ufer zum rechten Ufer mit maximal zwei
      Passagieren �bersetzt und zus�tzlich die Missionare  in dem neuen Zustand 
      kein Problem haben.

      (Zeile 149 in \texttt{missionare-frame.stlx})
\item Definieren Sie analog zu Teilaufgabe 3 eine Relationen $r_{2}$,
      die das �bersetzen des Bootes vom rechten Ufer zum linken Ufer beschreibt.

      (Zeile 159 in \texttt{missionare-frame.stlx}.)
\item Tragen Sie in Zeile 165 und 167 die korrekten Werte f�r den Start-Zustand
      und den Ziel-Zustand ein.
\end{enumerate}
Wenn Sie alle Teilaufgaben bis hierher richtig gel�st haben, dann kann das Programm nun
die L�sung des Problems berechnen.
\end{document}

%%% Local Variables: 
%%% mode: latex
%%% TeX-master: t
%%% End: 
