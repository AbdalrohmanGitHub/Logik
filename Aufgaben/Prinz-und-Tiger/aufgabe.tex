\documentclass{article}
\usepackage{ngerman}
\usepackage[latin1]{inputenc}
\usepackage{a4wide}
\usepackage{amssymb}
\usepackage{fancyvrb}
\usepackage{alltt}

\def\pair(#1,#2){\langle #1, #2 \rangle}
%\renewcommand{\labelenumi}{(\alph{enumi})}
\renewcommand{\labelenumi}{\arabic{enumi}.}

\newcounter{aufgabe}

\newcommand{\qed}{\hspace*{\fill} $\Box$}
\newcommand{\next}{\vspace*{0.1cm}

\noindent}

\newcommand{\exercise}{\vspace*{0.3cm}
\stepcounter{aufgabe}

\noindent
\textbf{Aufgabe \arabic{aufgabe}}: }

\newcommand{\solution}{\vspace*{0.3cm}

\noindent
\textbf{L�sung}: }

\pagestyle{empty}

\begin{document}
\noindent
\textbf{\large Aufgabe: \quad \emph{Der Prinz und die Tiger}}
\vspace*{0.3cm}

\noindent
Es war einmal ein K�nig, der seine Tochter mit einem Prinzen verm�hlen wollte.  Er lie� im
ganzen Land verk�nden, dass er einen Gemahl f�r seine Tochter suche.  Eines Tages kam nun
ein Prinz vorbei, um sich zu bewerben.  Da der K�nig seine Tochter  nicht mit
irgendeiner Dumpfbacke verm�hlen wollte, f�hrte der K�nig den Prinzen in einen Raum mit 9
T�ren.  Der K�nig teilte dem Prinzen mit, dass die Prinzessin sich in einem der Zimmer befinden
w�rde,  dass es aber andere Zimmer g�be, hinter denen hungrige Tiger warten w�rden.
Einige Zimmer sind auch leer.  Wenn nun der Prinz eine T�r mit einem Tiger dahinter �ffnen
w�rde, so m��te er zur Strafe an der DHBW Informatik studieren.  
Weiter sagte der K�nig, dass an allen
T�ren Schilder
angebracht w�ren, auf denen eine Aussage steht.  In den Zimmern, wo ein Tiger drin w�re,
da w�re die Aussage allerdings falsch. In dem Zimmern, in dem sich die Prinzessin befand, w�re die
Aussage richtig.  Was die leeren Zimmer anginge, so w�re der Sachverhalt so, dass entweder
die Aufschriften an allen leeren Zimmern wahr w�ren, oder dass die Aufschriften an allen leeren
Zimmern falsch w�ren.  Daraufhin la� der Prinz die Aufschriften.  Diese waren wie folgt:
\begin{enumerate}
\item Zimmer: Die Prinzessin ist in einem Zimmer mit ungerader Zimmernummer.
      In den Zimmern mit gerader Nummer ist kein Tiger.
\item Zimmer: Dieses Zimmer ist leer.
\item Zimmer: Die Aufschrift an Zimmer Nr.~5 ist wahr, die Aufschrift an Zimmer Nr.~7 
      ist falsch und in Zimmer Nr.~3 ist ein Tiger. 
\item Zimmer: Die Aufschrift an Zimmer Nr.~1 ist falsch, in Zimmer Nr.~8 ist kein Tiger,
      und die Aufschrift an Zimmer Nr.~9 ist wahr.
\item Zimmer: Wenn die Aufschrift an Zimmer Nr.~2 oder an Zimmer Nr.~4 wahr ist,
      dann ist kein Tiger im Zimmer Nr.~1.
\item Zimmer: Die Aufschrift an Zimmer Nr.~3 ist falsch, die Prinzessin ist im Zimmer Nr.~2 
      und im Zimmer Nr.~2 ist kein Tiger.
\item Zimmer: Die Prinzessin ist im Zimmer Nr.~1 und die Aufschrift an Zimmer Nr.~5 ist wahr.

\item Zimmer: In diesem Zimmer ist kein Tiger und Zimmer Nr.~9 ist leer.

\item Zimmer: Weder in diesem Zimmer noch in Zimmer Nr.~1 ist ein Tiger und au�erdem ist
      die Aufschrift an Zimmer Nr.~6 wahr.
\end{enumerate}
Schreiben Sie ein Programm, das f�r den Prinzen ausrechnet, hinter welcher T�r die
Prinzessin ist.  Verwenden Sie dazu die Vorlage, die Sie im Netz unter der Adresse
\noindent
\\[0.2cm]
\hspace*{1.3cm}
\texttt{http://www.dhbw-stuttgart.de/stroetmann/Logic/SetlX/prince-frame.stlx}
\\[0.2cm]
finden.  Diese Vorlage bindet die beiden Programme
\\[0.2cm]
\hspace*{1.3cm}
\texttt{davis-putnam.stlx} \quad und \quad
\texttt{knf.stlx} 
\\[0.2cm]
ein, die Sie dem selben Verzeichnis finden wie die Datei \texttt{prince-frame.stlx}.
Ihre Aufgabe besteht darin, die obigen Aussagen als aussagenlogische Formel zu formalisieren und in einer
Menge von Klauseln zusammen zu fassen, f�r die Sie dann eine m�gliche L�sung berechnen. 
\end{document}

%%% Local Variables: 
%%% mode: latex
%%% TeX-master: t
%%% End: 
