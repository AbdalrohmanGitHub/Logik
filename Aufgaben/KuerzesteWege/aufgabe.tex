\documentclass{article}
\usepackage{german}
\usepackage[latin1]{inputenc}
\usepackage{a4wide}
\usepackage{amssymb}
\usepackage{fancyvrb}
\usepackage{alltt}

\pagestyle{empty}

\begin{document}
\noindent
{\Large \textbf{Aufgaben-Blatt}: Berechnung des k�rzesten Weges}
\vspace{0.5cm}

\noindent
\textbf{Definition}: Ein Pfad-Relation $\textsl{PR}$ ist eine Menge von 3-Tupeln der Form 
 $\bigl\langle \langle x, y \rangle, p, l\bigr\rangle$ so dass gilt:
\begin{enumerate}
\item $\langle x, y \rangle$ ist ein Paar von Punkten.
\item $p$ ist eine Liste von Punkten.  Der erste Punkt der Liste ist $x$, der letzte Punkt
      ist $y$, in \textsc{Setl2}-Notation gilt also: \quad  $x = p(1)$ \quad und \quad $y = p(\#p)$. \\[0.1cm]
      Die Liste $p$ wird interpretiert als ein \emph{Pfad}, der von $x$ nach $y$ f�hrt.
\item $l$ ist eine positive nat�rliche Zahl.  Diese Zahl gibt die L�nge des Pfades $p$ an.
\end{enumerate}
Die \emph{Komposition} einer Abstands-Funktion $D$ mit einer Pfad-Relation $\textsl{PR}$
kann in der Mengenlehre wie folgt definiert werden: \\[0.1cm]
\hspace*{1.3cm} 
$D \circ \textsl{PR} := 
 \Bigl\{ \bigl\langle \langle x, z \rangle,\, [ x ] + p, \, D\bigl(\langle x, y \rangle\bigr) + l \bigr\rangle \mid 
         \langle x, y \rangle \in \textsl{dom}(D) \wedge \bigl\langle \langle y, z \rangle, p, l \bigr\rangle \in \textsl{PR} \Bigr\}$ \\[0.1cm]
Die Notation $[x] + p$ bezeichnet dabei den Pfad, den wir erhalten, wenn 
wir den Punkt $x$ vorne an den Pfad $p$ anf�gen.
\vspace{0.3cm}

\noindent
\textbf{Hinweis}: Versuchen Sie bei der L�sung der nachfolgenden Aufgaben
m�glichst mit Mengen-Konstruktionen  und nicht mit Kontroll-Strukturen wie
 \texttt{for} oder \texttt{while} zu arbeiten.
\vspace{0.3cm}

\noindent 
\textbf{Aufgabe 1}:  
Implementieren Sie eine Prozedur \texttt{compose}, so dass der Aufruf  $\mathtt{compose}(D, \textsl{PR})$ 
f�r eine Abstands-Funktion $D$ und  eine Pfad-Relation $\textsl{PR}$ die Komposition 
$D \circ \textsl{PR}$ berechnet.
\vspace{0.3cm}

\noindent
\textbf{Aufgabe 2}:
Implementieren Sie eine Prozedur \texttt{cyclic}, so dass der Aufruf 
 $\mathtt{cyclic}(p)$ 
f�r einen Pfad $p$ genau dann den Wert \texttt{true} zur�ck liefert,
wenn der Pfad $p$ zyklisch ist, das hei�t dass die Liste $p$ einen Punkt mehrfach enth�lt. 
\vspace{0.1cm}

\noindent
\textbf{Hinweis}: Konvertieren Sie die Liste $p$ in eine Menge und �berlegen Sie, wie sich 
die Anzahl der Elemente dieser Menge zu der L�nge der Liste verh�lt.
\vspace{0.3cm}

\noindent
\textbf{Aufgabe 3}: Implementieren Sie eine Prozedur \texttt{eliminateCycles}, so dass der Aufruf  
\\[0.1cm]
\hspace*{1.3cm}  $\mathtt{eliminateCycles}(\textsl{PR})$ \\[0.1cm]
als Ergbnis die  Pfad-Relation berechnet, die Sie erhalten, wenn Sie aus der Pfad-Relation
\textsl{PR} alle die 3-Tupel $\bigl\langle \langle x, y \rangle, p, l \bigr\rangle$ entfernen, f�r die der Pfad $p$ einen Zyklus enth�lt.
\vspace{0.3cm}

\noindent
\textbf{Aufgabe 4}: Implementieren Sie eine Prozedur \texttt{createPathRelation}, so dass der Aufruf \\[0.1cm]
\hspace*{1.3cm} $\texttt{createPathRelation}(D)$ \\[0.1cm]
aus einer Abstands-Funktion $D$ eine Pfad-Relation erzeugt.
\vspace{0.3cm}

\noindent
\textbf{Aufgabe 5}: �ndern Sie die in der Vorlesung entwickelte Prozedur \texttt{closure} so ab,
dass der Aufruf \\[0.1cm]
\hspace*{1.3cm} $\mathtt{closure}(D)$ \\[0.1cm]
aus einer gegebenen Abstands-Funktion $D$ eine Pfad-Relation erzeugt, die alle zyklen-freien m�glichen Verbindungen
zwischen zwei Punkten enth�lt.  
\vspace{0.1cm}

\noindent
\textbf{Hinweis}: Die in der Vorlesung verwendete Prozedur \texttt{minimum} ben�tigen Sie dazu nicht.
\noindent
\vspace{0.3cm}

\noindent
\textbf{Aufgabe 6}: Entwickeln Sie eine Prozedur \texttt{minimize}, so dass der Aufruf 
 $\mathtt{minimize}(\textsl{PR})$ 
aus einer gegebenen Pfad-Relation \textsl{PR} alle die 3-Tupel  $\bigl\langle \langle x, y \rangle, p, l \bigr\rangle$ 
entfernt, f�r die der Pfad $p$ nicht die k�rzest m�gliche Verbindung zwischen $x$ und $y$ ist.
\end{document}

%%% Local Variables: 
%%% mode: latex
%%% TeX-master: t
%%% End: 
