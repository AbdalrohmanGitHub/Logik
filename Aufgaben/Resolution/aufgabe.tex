\documentclass{article}
\usepackage{german}
\usepackage[latin1]{inputenc}
\usepackage{a4wide}
\usepackage{amssymb}
\usepackage{fancyvrb}
\usepackage{alltt}

\pagestyle{empty}

\newcounter{aufgabe}
\newcommand{\exercise}{\vspace*{0.1cm}
\stepcounter{aufgabe}

\noindent
\textbf{Aufgabe \arabic{aufgabe}}: }

\begin{document}
\noindent
{\Large \textbf{Aufgaben-Blatt}: \textsl{Resolution}}
\vspace{0.5cm}



\exercise  \\[0.2cm]
Nehmen Sie an, dass die folgenden Aussagen richtig sind:
\begin{enumerate}
\item Jack hat einen Hund.
\item Jemand, der einen Hund hat, liebt Tiere.
\item Jemand, der Tiere liebt, t�tet keine Katze.
\item Garfield ist eine Katze.
\item Jack oder Jim hat Garfield get�tet.
\end{enumerate}
Zeigen Sie, dass aus diesen Axiomen folgt, dass Jim den Kater Garfield gt�tet hat.
Wandeln Sie dazu zun�chst die obigen Aussagen in pr�dikatenlogische Formeln um und benutzen Sie
dann den Resolutions-Kalk�l um die Behauptung zu beweisen.
Benutzen Sie in Ihrem Beweis die folgenden Pr�dikatszeichen:
\begin{enumerate}
\item $\texttt{dog}(x)$ ist wahr, wenn $x$ ein  Hund  ist.
\item $\texttt{cat}(y)$ ist wahr, wenn $y$ eine Katze ist.
\item $\texttt{owns}(x, y)$ ist wahr, wenn $x$ Besitzer von $y$ ist.
\item $\texttt{animalLover}(x)$ ist wahr, wenn $x$ Tiere liebt.
\item $\texttt{killed}(x, y)$ ist wahr, wenn $x$ der M�rder von $y$ ist.
\end{enumerate}

\end{document}

%%% Local Variables: 
%%% mode: latex
%%% TeX-master: t
%%% End: 
