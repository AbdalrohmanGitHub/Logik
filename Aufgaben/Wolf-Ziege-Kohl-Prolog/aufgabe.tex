\documentclass{article}
\usepackage{german}
\usepackage[latin1]{inputenc}
\usepackage{a4wide}
\usepackage{amssymb}
\usepackage{fancyvrb}
\usepackage{alltt}

\pagestyle{empty}

\begin{document}
\noindent
{\Large \textbf{Aufgaben-Blatt} Bauer, Wolf, Ziege und Kohl}
\vspace{0.5cm}

\noindent
Ein Bauer will mit einem Wolf, einer Ziege und einem Kohl �ber einen Flu� �bersetzen, um
diese als Waren auf dem Markt zu verkaufen.
Das Boot ist aber so klein, dass er nicht zwei Waren gleichzeitig mitnehmen kann.
Wenn er den Wolf mit der Ziege allein l�sst, dann frisst der Wolf die Ziege und wenn er die
Ziege mit dem Kohl allein l�sst, dann frisst die Ziege den Kohl. Unter \\[0.1cm]
\hspace*{1.3cm} \texttt{http://www.dhbw-stuttgart.de/\symbol{126}stroetma/Prolog/wgc-frame.pl} \\[0.1cm]
finden Sie ein Program-Ger�st, in dem Sie noch verschiedene Teile implementieren m�ssen
um das Problem zu l�sen.  Repr�sentieren Sie die Situation am linken Ufer durch einen Term
der Form \\[0.1cm]
\hspace*{1.3cm} $\mathtt{side}(W,\;G,\;C,\;F)$. \\[0.1cm]
Dabei gibt $W$ die Zahl der W�lfe, $G$ die Zahl der Ziegen, $C$ die Zahl der Kohlk�pfe und
$F$ die Zahl der Bauern am linken Ufer an.  Orientieren Sie sich bei der L�sung der
folgenden Aufgabe an dem im Unterricht besprochenen Programm \texttt{missionare.pl}.
\vspace{0.3cm}

\noindent
\textbf{Aufgabe 1}:
 Setzen Sie in dem Pr�dikat \texttt{solve/0} f�r ``\texttt{?Start?}''
 und ``\texttt{?Goal?}'' die Terme ein, die den Startpunkt und den Zielpunkt repr�sentieren.

\noindent
 (Zeile 8 in \texttt{wgc-frame.stl})
\vspace{0.3cm}

\noindent
\textbf{Aufgabe 2}:
Implementieren Sie das Pr�dikat \texttt{edge/2} so, dass der Aufruf \\[0.1cm]
\hspace*{1.3cm} $\mathtt{edge}(\; \textsl{Point}, \textsl{Next} )$ \\[0.1cm]
f�r einen gegebenen Term \textsl{Point} einen Term \textsl{Next} berechnet, der 
eine Situation repr�sentiert, die durch eine �berfahrt von der durch \textsl{Point}
repr�sentierten Situation erreicht werden kann.
\vspace{0.3cm}

\noindent
\textbf{Aufgabe 3}:
Implementieren sie das Pr�dikat \texttt{switch/2} so, dass der Aufruf \\[0.1cm]
\hspace*{1.3cm} $\mathtt{switch}(\mathtt{side}(W,\;G,\;C,\;F), \textsl{Other})$ \\[0.1cm]
die Situation \textsl{Other} am rechten Ufer berechnet, wenn die Situation am linken Ufer
durch den Term $\mathtt{side}(W,\;G,\;C,\;F)$ repr�sentiert wird.
\vspace{0.3cm}

\noindent
\textbf{Aufgabe 4}:
Implementieren sie das Pr�dikat \texttt{problem/3} so, dass der Aufruf \\[0.1cm]
\hspace*{1.3cm} $\mathtt{problem}(W,\;G,\;C)$ \\[0.1cm]
genau dann gelingt, wenn es an einem Ufer, das durch den Term
$\mathtt{side}(W,\;G,\;C,\;0)$ beschrieben wird, ein Problem gibt.

\end{document}

%%% Local Variables: 
%%% mode: latex
%%% TeX-master: t
%%% End: 
