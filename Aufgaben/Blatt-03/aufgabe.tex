\documentclass{article}
\usepackage{german}
\usepackage[latin1]{inputenc}
\usepackage{a4wide}
\usepackage{amssymb}
\usepackage{fancyvrb}
\usepackage{alltt}
\usepackage{hyperref}

\pagestyle{empty}
\renewcommand*{\familydefault}{\sfdefault}

\begin{document}
\noindent
{\Large \textbf{Aufgaben-Blatt}: Japanischer IQ-Test}
\vspace{0.5cm}


\noindent
Im Internet finden Sie unter der Adresse
\\[0.2cm]
\hspace*{1.3cm}
\href{http://thomas-foerster.com/japanischer-iq-test/}{\texttt{http://thomas-foerster.com/japanischer-iq-test/}}
\\[0.2cm]
eine Animation einer Denksport-Aufgabe, die Sie mit Hilfe eines Programmes l�sen sollen.
Es geht bei der Aufgabe wieder darum, dass Personen �ber einen Fluss �bersetzen sollen
und daf�r nur ein Boot haben, in dem maximal zwei Personen Platz haben.
Bei den Personen handelt es sich um eine Mutter mit zwei T�chtern, einen Vater mit zwei
S�hnen, einen Polizisten und einen Verbrecher.
Bei der �berfahrt sind die folgenden Nebenbedingungen zu beachten:
\begin{enumerate}
\item Der Vater darf nicht ohne die Mutter mit einer der T�chter an einem Ufer sein.
\item Die Mutter darf nicht ohne den Vater mit einem der S�hne an einem Ufer sein.
\item Wenn der Verbrecher nicht allein ist, dann muss der Polizist auf ihn aufpassen.
      Der Verbrecher darf aber alleine sein, denn seine Fu{\ss}fesseln verhindern, 
      dass er weglaufen kann.
\item Nur der Vater, die Mutter und der Polizist k�nnen das Boot steuern.
\end{enumerate}
Unter 
\\[0.2cm]
\hspace*{0.0cm}      
\href{https://github.com/karlstroetmann/Logik/blob/master/Aufgaben/Blatt-03/japanese-frame.stlx}{\texttt{github.com/karlstroetmann/Logik/blob/master/Aufgaben/Blatt-03/japanese-frame.stlx}} 
\\[0.2cm]
finden Sie ein Programm-Ger�st, in dem Sie noch verschiedene Teile implementieren m�ssen
um das Problem mit Hilfe der im Unterricht entwickelten Methoden zu l�sen.
\begin{enumerate}
\item Zun�chst wird in Zeile 84 die Menge \texttt{All} aller Personen definiert.
      Der Einfachheit halber enth�lt diese Menge auch das Boot.
\item Definieren Sie eine Prozedur $\textsl{verboten}(S)$, die f�r eine Menge
      $S$ dann den Wert \texttt{true} zur�ck liefert, wenn diese Menge einen Zustand beschreibt, bei
      dem am linken Ufer ein Problem auftritt.
\item Die Zust�nde beschreiben wir durch gewisse Teilmengen der Menge \texttt{All}.
      Diese Teilmengen geben die Personen an, die sich am linken Ufer befinden.
      In dieser Menge wollen wir nur solche Zust�nde aufnehmen,
      bei denen es auf keiner Seite des Ufers ein Problem gibt.
\item Definieren Sie eine Prozedur $\textsl{bootOK}(B)$, die als Eingabe
      eine Menge $B$ von Personen enth�lt, die im Boot sitzen.  Die Prozedur soll genau
      dann den Wert \texttt{true} zur�ck geben, wenn die oben formulierten Bedingungen an
      das Boot erf�llt sind.
      Zus\"atzlich soll die Menge $B$ auch das Boot enthalten.
\item Definieren Sie nun  die Relation \texttt{R1}, die �berfahrten vom linken
      zum rechten Ufer beschreibt.
\item Definieren Sie anschlie�end die Relation \texttt{R2}, die �berfahrten vom rechten
      zum linken Ufer beschreibt.
\end{enumerate}
\textbf{Hinweise}:  
\begin{enumerate}
\item Die Menge der Zust�nde \texttt{P} enth�lt 140 verschiedene Zust�nde.
\item Die Relation \texttt{R} enth�lt 200 verschiedene Elemente.
\item Die berechnete L�sung \texttt{Path} hat die L�nge 18, es gibt also 17 \"Uberfahrten.
\end{enumerate}
\end{document}

%%% Local Variables: 
%%% mode: latex
%%% TeX-master: t
%%% End: 
