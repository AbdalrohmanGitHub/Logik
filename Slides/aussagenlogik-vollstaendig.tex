\documentclass{slides}
\usepackage[latin1]{inputenc}
\usepackage{german}
\usepackage{epsfig}
\usepackage{amssymb}

\pagestyle{empty}
\setlength{\textwidth}{17cm}
\setlength{\textheight}{24cm}
\setlength{\topmargin}{0cm}
\setlength{\headheight}{0cm}
\setlength{\headsep}{0cm}
\setlength{\topskip}{0.2cm}
\setlength{\oddsidemargin}{0.5cm}
\setlength{\evensidemargin}{0.5cm}

\newfont{\chess}{chess20}
\newfont{\bigchess}{chess30}
\newcommand{\chf}{\baselineskip20pt\lineskip0pt\chess}

\newcommand{\qed}{\hspace*{\fill} $\Box$}
\newcommand{\exend}{\hspace*{\fill} $\diamond$}
\newcommand{\setl}{\textsc{Setl2}}
\newcommand{\struct}{\mathcal{S}}
\newcommand{\FV}{\textsl{FV}}
\newcommand{\id}{\textrm{id}}
\newcommand{\dom}{\textrm{dom}}
\newcommand{\rng}{\textrm{rng}}
\newcommand{\BV}{\textsl{BV}}
\newcommand{\var}{\textsl{Var}}
\newcommand{\el}{\!\in\!}
\newcommand{\I}{\mathcal{I}}
\newcommand{\verum}{\top}
\newcommand{\falsum}{\bot}
\newcommand{\gentzen}{\vdash}
\newcommand{\komplement}[1]{\overline{#1}}
\newcommand{\mathquote}[1]{\mbox{``}\mathtt{#1}\mbox{''}}

\newcommand{\circneg}{\mbox{$\bigcirc\hspace*{-0.3cm}\neg$}}
\newcommand{\circwedge}{\mbox{$\bigcirc\hspace*{-0.3cm}\wedge$}}
\newcommand{\circvee}{\mbox{$\bigcirc\hspace*{-0.3cm}\vee$}}
\newcommand{\circright}{\mbox{$\bigcirc\hspace*{-0.45cm}\rightarrow$}}
\newcommand{\circleftright}{\mbox{$\bigcirc\hspace*{-0.45cm}\leftrightarrow$}}


\def\pair(#1,#2){\langle #1, #2 \rangle}

\newcommand{\is}{\;|\;}
\newcommand{\schluss}[2]{\frac{\displaystyle\quad \rule[-8pt]{0pt}{18pt}#1 \quad}{\displaystyle\quad \rule{0pt}{14pt}#2 \quad}}
\newcommand{\vschlus}[1]{{\displaystyle\rule[-6pt]{0pt}{12pt} \atop \rule{0pt}{10pt}#1}}

\newcounter{mypage}

\newtheorem{Definition}{Definition}
\newtheorem{Notation}[Definition]{Notation}
\newtheorem{Korollar}[Definition]{Korollar}
\newtheorem{Lemma}[Definition]{Lemma}
\newtheorem{Satz}[Definition]{Satz}
\newtheorem{Theorem}[Definition]{Theorem}

\begin{document}

%%%%%%%%%%%%%%%%%%%%%%%%%%%%%%%%%%%%%%%%%%%%%%%%%%%%%%%%%%%%%%%%%%%%%%%%

\begin{slide}
\begin{center}
Widerlegungs-Vollst�ndigkeit der Schnitt-Regel
\end{center}
\vspace*{0.5cm}

\footnotesize
\begin{Definition}[$\vdash$] \hspace*{\fill} \\[0.3cm]
{\em
    \hspace*{1cm} $M$: Menge von Klauseln \\[0.3cm]
    \hspace*{1cm} $f$: \ Klausel 

    Induktive Definition der Relation \quad  $M \vdash f$
    \begin{enumerate}
    \item $M \vdash \verum$
    \item Wenn $f \el M$, dann $M \vdash f$.
    \item Wenn $M \vdash k_1 \cup \{p\}$ und $M \vdash \{ \neg p \} \cup k_2$, \\[0.3cm]
          dann $M \vdash k_1 \cup k_2$.
    \end{enumerate}
}
\end{Definition}
\vspace*{0.5cm}

\begin{Definition}[Folgerungs-Begiff] \hspace*{\fill} \\[-0.9cm]
  \begin{tabbing}
  \textbf{Vor.:} \= $M$: \= Menge von AL-Formeln  \\[0.3cm]
                 \> $f$: \> AL-Formel                \\[0.3cm]
  \textbf{Def.:} \> $M \models f$ \quad g.d.w.    \\[0.3cm]
                 \> $\forall \I \in \textsc{Ali}: \biggl(
                     \bigl(\forall g \in M: 
                     \I(g) = \mathtt{true}\bigr) \rightarrow   \I(f) = \mathtt{true}\biggr)$
  \end{tabbing}
\end{Definition}
\vspace*{0.5cm}

\begin{Satz}[Korrektheits-Satz] \hspace*{\fill} \\[-0.9cm]
  \begin{tabbing}
  \textbf{Vor.:} \= $\{k_1, \cdots, k_n \}$: \= Menge von Klauseln \\[0.3cm]
                 \> $k$:                     \> Klausel \\[0.3cm]
  \textbf{Beh.:} \> 
  $\{k_1, \cdots, k_n \} \vdash k \quad\Rightarrow\quad \{k_1, \cdots, k_n\} \models k$.  
  \end{tabbing}
\end{Satz}

\vspace*{\fill}
\tiny \addtocounter{mypage}{1} 
\rule{17cm}{1mm}
Widerlegungs-Vollst�ndigkeit  \hspace*{\fill} Seite \arabic{mypage}
\end{slide}

%%%%%%%%%%%%%%%%%%%%%%%%%%%%%%%%%%%%%%%%%%%%%%%%%%%%%%%%%%%%%%%%%%%%%%%%

\begin{slide}
\begin{center}
Widerlegungs-Vollst�ndigkeit der Schnitt-Regel
\end{center}
\vspace*{0.5cm}

\footnotesize
\begin{Satz}[Widerlegungs-Vollst�ndigkeits-Satz] \hspace*{\fill} \\[-0.9cm]
  \begin{tabbing}
  \textbf{Vor.:} \= $\{k_1, \cdots, k_n \}$: \= Menge von Klauseln \\[0.3cm]
                 \> $k$:                     \> Klausel \\[0.3cm]
  \textbf{Beh.:} \> 
  $\{k_1, \cdots, k_n \} \models \falsum \quad\Rightarrow\quad \{k_1, \cdots, k_n\} \vdash \falsum$.  
  \end{tabbing}
\end{Satz}
\vspace*{0.5cm}

\begin{Definition}[Redukt]  \hspace*{\fill} \\[-0.9cm]
  \begin{tabbing}
  \textbf{Vor.:} \= $M$: \= Menge von Klauseln \\[0.3cm]
                 \> $k$: \>  Klausel \\[0.3cm]
                 \> $l$: \> Literal  \\[0.3cm]
                 
  \textbf{Def.:} \> $\textsl{redukt}(k,l)$ \quad \= $:= \left\{ 
     \begin{array}{ll}
        \verum                &  \mathrm{falls}\; l \el k; \\
        k \backslash \left\{ \komplement{\,l\,} \right\}  &
   \mathrm{falls}\; l\not\in k\; \mathrm{und}\; \komplement{\,l\,} \el k; \\
        k                     &  \mathrm{sonst}.
     \end{array}
     \right.
  $  \\[0.3cm]
   \> $\textsl{Redukt}(M, l)$ \> $:= \biggl\{\; \textsl{redukt}(k,l) \;|\; k \el M \;\biggr\}$.
  \end{tabbing}
\end{Definition}
\textbf{Idee}: Vereinfache $k$ unter Annahme $l$
\vspace*{0.5cm}

\begin{Satz} (Semantische Eigenschaften von \textsl{Redukt})\hspace*{\fill} \\[-0.9cm]
  \begin{tabbing}
    \textbf{Vor.:} \ \= $M \subseteq \mathcal{K}$ \\[0.3cm]
                     \> $l \el \mathcal{L}$       \\[0.3cm]
    \textbf{Beh.:}   \> $M \models \falsum \quad \Rightarrow \quad \textsl{Redukt}(M, l) \models \falsum$
  \end{tabbing}
\end{Satz}

\vspace*{\fill}
\tiny \addtocounter{mypage}{1} 
\rule{17cm}{1mm}
Widerlegungs-Vollst�ndigkeit  \hspace*{\fill} Seite \arabic{mypage}
\end{slide}

%%%%%%%%%%%%%%%%%%%%%%%%%%%%%%%%%%%%%%%%%%%%%%%%%%%%%%%%%%%%%%%%%%%%%%%%

\begin{slide}
\begin{center}
Widerlegungs-Vollst�ndigkeit der Schnitt-Regel
\end{center}
\vspace*{0.5cm}

\footnotesize
\begin{Satz}  (Syntaktische Eigenschaften von \textsl{Redukt}) \hspace*{\fill} \\[-0.9cm]
  \begin{tabbing}
    \textbf{Vor.:} \ \= $M$: \= Menge von Klauseln \\[0.3cm]
                     \> $f$: \> Klausel            \\[0.3cm]
                     \> $l$: \> Literal            \\[0.3cm]
    \textbf{Beh.:}   \> $\textsl{Redukt}(M,l) \gentzen f \quad \Rightarrow \quad 
                         M \gentzen f \;\;\mbox{oder}\;\;
                         M \gentzen f \cup \bigl\{\komplement{\,l\,}\bigr\}$
  \end{tabbing}
\end{Satz}

\normalsize
\begin{center}
Davis-Putnam Verfahren
\end{center}
\vspace{0.5cm}

\footnotesize
\textbf{Geg.}:  $K$ Menge von Klauseln

\textbf{Ges.}:  $\mathcal{I}$ aussagenlogische Belegung mit 
                \[\forall k \in K:\mathcal{I}(k) = \mathtt{true} \]
\textbf{Algorithmus}
\begin{enumerate}
\item F�hre alle m�gl. Schnitte mit Unit-Klauseln durch.
\item Vereinfache mit Subsumption. 
\item Falls $K$ trivial: fertig!
\item W�hle aussagenlogische Variable $p$ aus $K$.
      \begin{enumerate}
      \item Suche rekursiv L�sung f�r $\mathcal{I}$ f�r $K \cup \biggl\{\{p\}\biggr\}$.

      \item Falls (a) erfolglos ist:
      
            Suche rekursiv L�sung f�r $\mathcal{I}$ f�r $K \cup \biggl\{\{\neg p\}\biggr\}$
      \end{enumerate}
\end{enumerate}



\vspace*{\fill}
\tiny \addtocounter{mypage}{1} 
\rule{17cm}{1mm}
Davis-Putnam  \hspace*{\fill} Seite \arabic{mypage}
\end{slide}

%%%%%%%%%%%%%%%%%%%%%%%%%%%%%%%%%%%%%%%%%%%%%%%%%%%%%%%%%%%%%%%%%%%%%%%%

\begin{slide}
\begin{center}
8-Damen-Problem
\end{center}
\vspace*{0.5cm}

\footnotesize
K�nnen 8 Damen so auf einem Schach--Brett
aufgestellt werden, dass keine Dame eine andere Dame schlagen kann?

Eine Dame kann eine andere schlagen falls diese
\begin{itemize}
\item in derselben Reihe steht,
\item in derselben Spalte steht, oder
\item in derselben Diagonale steht.
\end{itemize}
\vspace*{-1.0cm}

\footnotesize
\setlength{\unitlength}{1.5cm}

\begin{picture}(10,10)

\thicklines
\put(1,1){\line(1,0){8}}
\put(1,1){\line(0,1){8}}
\put(1,9){\line(1,0){8}}
\put(9,1){\line(0,1){8}}
\put(0.9,0.9){\line(1,0){8.2}}
\put(0.9,9.1){\line(1,0){8.2}}
\put(0.9,0.9){\line(0,1){8.2}}
\put(9.1,0.9){\line(0,1){8.2}}

\thinlines
\multiput(1,2)(0,1){7}{\line(1,0){8}}
\multiput(2,1)(1,0){7}{\line(0,1){8}}
\put(4.25,6.25){{\chess Q}}
\multiput(5.25,6.5)(1,0){4}{\vector(1,0){0.5}}
\multiput(3.75,6.5)(-1,0){3}{\vector(-1,0){0.5}}
\multiput(5.25,7.25)(1,1){2}{\vector(1,1){0.5}}
\multiput(5.25,5.75)(1,-1){4}{\vector(1,-1){0.5}}
\multiput(3.75,5.75)(-1,-1){3}{\vector(-1,-1){0.5}}
\multiput(3.75,7.25)(-1,1){2}{\vector(-1,1){0.5}}
\multiput(4.5,7.25)(0,1){2}{\vector(0,1){0.5}}
\multiput(4.5,5.75)(0,-1){5}{\vector(0,-1){0.5}}
\end{picture}


\vspace*{\fill}
\tiny \addtocounter{mypage}{1} 
\rule{17cm}{1mm}
8-Damen-Problem  \hspace*{\fill} Seite \arabic{mypage}
\end{slide}

%%%%%%%%%%%%%%%%%%%%%%%%%%%%%%%%%%%%%%%%%%%%%%%%%%%%%%%%%%%%%%%%%%%%%%%%

\begin{slide}
\begin{center}
Repr�sentation des Problems
\end{center}
\vspace*{0.5cm}

\footnotesize
\setlength{\unitlength}{2.0cm}
\hspace*{-2cm}
\begin{picture}(10,9)
\thicklines
\put(0.9,0.9){\line(1,0){8.2}}
\put(0.9,9.1){\line(1,0){8.2}}
\put(0.9,0.9){\line(0,1){8.2}}
\put(9.1,0.9){\line(0,1){8.2}}
\put(1,1){\line(1,0){8}}
\put(1,1){\line(0,1){8}}
\put(1,9){\line(1,0){8}}
\put(9,1){\line(0,1){8}}
\thinlines
\multiput(1,2)(0,1){7}{\line(1,0){8}}
\multiput(2,1)(1,0){7}{\line(0,1){8}}

%%  for (i = 1; i <= 8; i = i + 1) {
%%for (j = 1; j <= 8; j = j + 1) \{
%%   \put(\$j.15,<9-$i>.35){{\Large p<$i>\$j}}
%%\}
%%  }

\put(1.15,8.40){{ p11}}
\put(2.15,8.40){{ p12}}
\put(3.15,8.40){{ p13}}
\put(4.15,8.40){{ p14}}
\put(5.15,8.40){{ p15}}
\put(6.15,8.40){{ p16}}
\put(7.15,8.40){{ p17}}
\put(8.15,8.40){{ p18}}
\put(1.15,7.40){{ p21}}
\put(2.15,7.40){{ p22}}
\put(3.15,7.40){{ p23}}
\put(4.15,7.40){{ p24}}
\put(5.15,7.40){{ p25}}
\put(6.15,7.40){{ p26}}
\put(7.15,7.40){{ p27}}
\put(8.15,7.40){{ p28}}
\put(1.15,6.40){{ p31}}
\put(2.15,6.40){{ p32}}
\put(3.15,6.40){{ p33}}
\put(4.15,6.40){{ p34}}
\put(5.15,6.40){{ p35}}
\put(6.15,6.40){{ p36}}
\put(7.15,6.40){{ p37}}
\put(8.15,6.40){{ p38}}
\put(1.15,5.40){{ p41}}
\put(2.15,5.40){{ p42}}
\put(3.15,5.40){{ p43}}
\put(4.15,5.40){{ p44}}
\put(5.15,5.40){{ p45}}
\put(6.15,5.40){{ p46}}
\put(7.15,5.40){{ p47}}
\put(8.15,5.40){{ p48}}
\put(1.15,4.40){{ p51}}
\put(2.15,4.40){{ p52}}
\put(3.15,4.40){{ p53}}
\put(4.15,4.40){{ p54}}
\put(5.15,4.40){{ p55}}
\put(6.15,4.40){{ p56}}
\put(7.15,4.40){{ p57}}
\put(8.15,4.40){{ p58}}
\put(1.15,3.40){{ p61}}
\put(2.15,3.40){{ p62}}
\put(3.15,3.40){{ p63}}
\put(4.15,3.40){{ p64}}
\put(5.15,3.40){{ p65}}
\put(6.15,3.40){{ p66}}
\put(7.15,3.40){{ p67}}
\put(8.15,3.40){{ p68}}
\put(1.15,2.40){{ p71}}
\put(2.15,2.40){{ p72}}
\put(3.15,2.40){{ p73}}
\put(4.15,2.40){{ p74}}
\put(5.15,2.40){{ p75}}
\put(6.15,2.40){{ p76}}
\put(7.15,2.40){{ p77}}
\put(8.15,2.40){{ p78}}
\put(1.15,1.40){{ p81}}
\put(2.15,1.40){{ p82}}
\put(3.15,1.40){{ p83}}
\put(4.15,1.40){{ p84}}
\put(5.15,1.40){{ p85}}
\put(6.15,1.40){{ p86}}
\put(7.15,1.40){{ p87}}
\put(8.15,1.40){{ p88}}
\end{picture}


\vspace*{\fill}
\tiny \addtocounter{mypage}{1} 
\rule{17cm}{1mm}
8-Damen-Problem  \hspace*{\fill} Seite \arabic{mypage}
\end{slide}


%%%%%%%%%%%%%%%%%%%%%%%%%%%%%%%%%%%%%%%%%%%%%%%%%%%%%%%%%%%%%%%%%%%%%%%%

\end{document}

%%% Local Variables: 
%%% mode: latex
%%% TeX-master: t
%%% End: 
