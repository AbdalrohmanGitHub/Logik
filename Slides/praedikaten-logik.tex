\documentclass{slides}
\usepackage[latin1]{inputenc}
\usepackage{german}
\usepackage{epsfig}
\usepackage{amssymb}
\usepackage{landscape}

\pagestyle{empty}
\setlength{\textwidth}{17cm}
\setlength{\textheight}{24cm}
\setlength{\topmargin}{0cm}
\setlength{\headheight}{0cm}
\setlength{\headsep}{0cm}
\setlength{\topskip}{0.2cm}
\setlength{\oddsidemargin}{0.5cm}
\setlength{\evensidemargin}{0.5cm}

\newcommand{\is}{\;|\;}
\newcommand{\schluss}[2]{\frac{\displaystyle\quad \rule[-8pt]{0pt}{18pt}#1 \quad}{\displaystyle\quad \rule{0pt}{14pt}#2 \quad}}
\newcommand{\vschlus}[1]{{\displaystyle\rule[-6pt]{0pt}{12pt} \atop \rule{0pt}{10pt}#1}}
\newcommand{\verum}{\top}
\newcommand{\falsum}{\bot}
\newcommand{\gentzen}{\vdash}
\newcommand{\komplement}[1]{\overline{#1}}

\def\pair(#1,#2){\langle #1, #2 \rangle}

\newcounter{mypage}

\begin{document}

\begin{slide}{}
\normalsize
\begin{center}
Pr�dikatenlogik --- �berblick
\end{center}
\vspace*{1cm}

\footnotesize
\begin{enumerate}
\item \emph{Terme}:  Bezeichnungen f�r Objekte 

      Terme aufbauen aus \emph{Variablen} und \emph{Funktions-Zeichen}
      \[ \textsl{vater}(x),\quad \textsl{mutter}(\textsl{isaac}), \quad x+7 \]
\item \emph{Pr�dikats-Zeichen}: setzen Objekte in Beziehung
      \begin{enumerate}
      \item $\textsl{istBruder}\bigl(\textsl{albert}, \textsl{vater}(\textsl{bruno})\bigr)$
      \item $x+7 < x \cdot 7$
      \item $n \in \mathbb{N}$
      \end{enumerate}
      Pr�dikats-Zeichen erzeugen \emph{atomare} Formeln.
\item Verkn�pfung atomarer Formeln durch Junktoren 
      \[ x > 1 \rightarrow x + 7 < x \cdot  7 \]
\item \emph{Quantoren}: legen Bedeutung von Variablen fest
      \begin{enumerate}
      \item Allquantor: 
        \\[0.2cm]
        \hspace*{1.3cm}
        $\forall x \in \mathbb{R}: x \cdot x > 0$
      \item Existenz-Quantor:
        \\[0.2cm]
        \hspace*{1.3cm}
        $\exists r \in \mathbb{R}: x \cdot x = 2$
      \item Quantoren k�nnen verschachtelt werden
            \\[0.2cm]
            \hspace*{1.3cm}
            $\forall x \in \mathbb{R}: \exists n \in \mathbb{N}: x < n$
      \end{enumerate}
\end{enumerate}



\vspace*{\fill}
\tiny \addtocounter{mypage}{1} 
\rule{17cm}{1mm}
Pr�dikatenlogik  \hspace*{\fill} Seite \arabic{mypage}
\end{slide}

%%%%%%%%%%%%%%%%%%%%%%%%%%%%%%%%%%%%%%%%%%%%%%%%%%%%%%%%%%%%%%%%%%%%%%%%

\begin{slide}{}
\normalsize
\begin{center}
Pr�dikatenlogik --- Syntax
\end{center}
\vspace*{1cm}

\footnotesize
\textbf{Definition}: Signatur ist  4-Tupel \\[0.3cm]
  \hspace*{1.3cm} 
  $\Sigma = \langle \mathcal{V}, \mathcal{F}, \mathcal{P}, \textsl{arity} \rangle$
  \begin{enumerate}
  \item $\mathcal{V}$: Menge der Variablen
  \item $\mathcal{F}$: Menge der Funktions-Zeichen
  \item $\mathcal{P}$: Menge der Pr�dikats-Zeichen.
  \item $\textsl{arity}$: Stelligkeit von Funktions- und Pr�dikats-Zeichen
        \\[0.2cm]
        \hspace*{1.3cm} 
        $\textsl{arity}: \mathcal{F} \cup \mathcal{P} \rightarrow \mathbb{N}$
  \item $\mathcal{V} \cap \mathcal{F} = \{\}$, \quad
        $\mathcal{V} \cap \mathcal{P} = \{\}$, \quad und \quad
        $\mathcal{F} \cap \mathcal{P} = \{\}$.
  \end{enumerate}



\vspace*{\fill}
\tiny \addtocounter{mypage}{1} 
\rule{17cm}{1mm}
Pr�dikatenlogik  \hspace*{\fill} Seite \arabic{mypage}
\end{slide}

%%%%%%%%%%%%%%%%%%%%%%%%%%%%%%%%%%%%%%%%%%%%%%%%%%%%%%%%%%%%%%%%%%%%%%%%

\begin{slide}{}
\normalsize
\begin{center}
Terme --- Beispiele
\end{center}
\vspace*{1cm}

\footnotesize
\begin{enumerate}
\item Variablen: $\mathcal{V} = \{ x, y, z \}$
\item Funktions-Zeichen: $\mathcal{F} = \{ 0, 1, \mathtt{+}, \mathtt{-}, * \}$
\item Pr�dikats-Zeichen: $\mathcal{P} = \{\mathtt{=}, \leq\}$
\item Stelligkeit: 

      $\textsl{arity} = \bigl\{ \pair(0,0), \pair(1,0), \pair(\mathtt{+},2),
      \pair(\mathtt{-},2), \pair(*,2), \pair(=,2), \pair(\leq,2) \bigr\}$

\item Signatur: $\Sigma_\mathrm{arith} =
       \langle \mathcal{V}, \mathcal{F}, \mathcal{P}, \textsl{arity} \rangle$ 
\item $x, y, z \in \mathcal{T}_{\Sigma_{\mathrm{arith}}}$
\item $0, 1 \in \mathcal{T}_{\Sigma_{\mathrm{arith}}}$
\item $\mathtt{+}(0,x) \in \mathcal{T}_{\Sigma_{\mathrm{arith}}}$
\item $*(\mathtt{+}(0,x),1) \in \mathcal{T}_{\Sigma_{\mathrm{arith}}}$
\end{enumerate}
\vspace*{\fill}
\tiny \addtocounter{mypage}{1} 
\rule{17cm}{1mm}
Pr�dikatenlogik  \hspace*{\fill} Seite \arabic{mypage}
\end{slide}

%%%%%%%%%%%%%%%%%%%%%%%%%%%%%%%%%%%%%%%%%%%%%%%%%%%%%%%%%%%%%%%%%%%%%%%%


\end{document}

%%% Local Variables: 
%%% mode: latex
%%% TeX-master: t
%%% End: 
