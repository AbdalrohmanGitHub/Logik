\documentclass{slides}
\usepackage[latin1]{inputenc}
\usepackage{german}
\usepackage{epsfig}
\usepackage{amssymb}

\pagestyle{empty}
\setlength{\textwidth}{17cm}
\setlength{\textheight}{24cm}
\setlength{\topmargin}{0cm}
\setlength{\headheight}{0cm}
\setlength{\headsep}{0cm}
\setlength{\topskip}{0.2cm}
\setlength{\oddsidemargin}{0.5cm}
\setlength{\evensidemargin}{0.5cm}

\newcommand{\is}{\;|\;}
\newcommand{\schluss}[2]{\frac{\displaystyle\quad \rule[-8pt]{0pt}{18pt}#1 \quad}{\displaystyle\quad \rule{0pt}{14pt}#2 \quad}}
\newcommand{\vschlus}[1]{{\displaystyle\rule[-6pt]{0pt}{12pt} \atop \rule{0pt}{10pt}#1}}
\newcommand{\verum}{\top}
\newcommand{\falsum}{\bot}
\newcommand{\gentzen}{\vdash}
\newcommand{\komplement}[1]{\overline{#1}}

\newcounter{mypage}

\begin{document}

\begin{slide}{}
\normalsize
\begin{center}
  Schluss-Regeln
\end{center}
\vspace*{1cm}

\footnotesize
Eine \emph{Schluss-Regel} ist eine Paar \\[0.5cm]
\hspace*{1.3cm} $\langle \{f_1, \cdots, f_n\}, k \rangle$ \\[0.5cm]
mit $f_1, \cdots, f_n, k \in \mathcal{F}$.  

$f_1$, $\cdots$, $f_n$: \emph{Pr�missen}.

$k$:  \emph{Konklusion}.

Schreibweise: $\schluss{f_1 \quad \cdots \quad f_n}{k}$  

\textbf{Beispiele} f�r Schluss-Regeln:
\begin{enumerate}
\item ``\emph{Modus Ponens}'':\\[0.3cm]
      \hspace*{1.3cm} $\schluss{p \quad\quad p \rightarrow q}{q}\;(\textsl{MP})$
\item ``\emph{Modus Ponendo Tollens}'': \\[0.3cm]
      \hspace*{1.3cm} $\schluss{\neg q \quad\quad p \rightarrow q}{\neg p}\;(\textsl{MPT})$
\item ``\emph{Modus Tollendo Tollens}'': \\[0.3cm]
      \hspace*{1.3cm} $\schluss{\neg p \quad\quad p \rightarrow q}{\neg q}\;(\textsl{MTT})$
\end{enumerate}
\framebox{\framebox{
Frage:  Wann sind Schluss-Regeln korrekt?}}

\vspace*{\fill}
\tiny \addtocounter{mypage}{1} 
\rule{17cm}{1mm}
Der Beweis-Begriff  \hspace*{\fill} Seite \arabic{mypage}
\end{slide}

%%%%%%%%%%%%%%%%%%%%%%%%%%%%%%%%%%%%%%%%%%%%%%%%%%%%%%%%%%%%%%%%%%%%%%%%
 
\begin{slide}{}
\normalsize
\begin{center}
  Erf�llbarkeit
\end{center}
\vspace{0.5cm}


\footnotesize
\textbf{Geg}.: $M = \{ k_1, \cdots, k_n \}$ Menge von Klauseln
\begin{enumerate}
\item \textbf{Frage}: Wann ist $M$ Tautologie? 

      Formal: Wann gilt $\models k_1 \wedge \cdots \wedge k_n$?
      \vspace{0.3cm}

      \textbf{Antwort}: 
      $$
      \begin{array}{cll}
                     & \models M \\[0.3cm]
      \mbox{g.d.w.}  & \models k_i            &\mbox{f�r alle $i=1,\cdots,n$}  \\[0.3cm]
      \mbox{g.d.w.}  & k_i \;\;\mbox{trivial} &\mbox{f�r alle $i=1,\cdots,n$}  
      \end{array}
      $$
      Zufriedenstellende Antwort.
\item \textbf{Frage}: Wann ist $M$ erf�llbar? 

      Formal: Wann gibt es Belegung $\mathcal{I}$ so dass gilt:  \\[0.3cm]
      \hspace*{1.3cm} 
      $\mathtt{eval}(k_i,I) = \mathtt{true}$ \quad f�r alle $i=1,\cdots,n$?      
      \vspace{0.3cm}

      \textbf{Antwort} ist schwieriger:
      $$
      \begin{array}{cll}
                     & \texttt{$M$ unerf�llbar} \\[0.3cm]
      \mbox{g.d.w.}  & \texttt{aus $M$ ist $\falsum$ herleitbar} \\[0.3cm]
      \mbox{g.d.w.}  & M \vdash \falsum \\[0.3cm]
      \end{array}
      $$
      Wir ben�tigen den
      \begin{center}
        \emph{Herleitungs-Begriff}        
      \end{center}
      zur Beantwortung der Frage.
\end{enumerate}



\vspace*{\fill}
\tiny \addtocounter{mypage}{1} 
\rule{17cm}{1mm}
Aussagenlogik  \hspace*{\fill} Seite \arabic{mypage}
\end{slide}

%%%%%%%%%%%%%%%%%%%%%%%%%%%%%%%%%%%%%%%%%%%%%%%%%%%%%%%%%%%%%%%%%%%%%%%% 
 
\begin{slide}{}
\normalsize
\begin{center}
Spezialf�lle der Schnitt-Regel
\end{center}
\vspace{0.5cm}

\footnotesize
\begin{enumerate}
\item Setze $k_1 := \emptyset$, $l := p$ und $k_2 := \{q\}$: \\[0.5cm]
      $\schluss{\{\} \cup \{p\} \quad\quad \{\neg p\} \cup \{ q \} }{ \{\} \cup \{q\} }$ \\[0.5cm]
     Interpretation von Mengen als Disjunktionen liefert: \\[0.5cm]
     $\schluss{p \quad\quad \neg p \vee q }{ q }$ \\[0.5cm]
     Ber�cksichtigung von $\neg p \vee q \;\leftrightarrow\; p \rightarrow q$ liefert: \\[0.5cm]
     $\schluss{p \quad\quad p \rightarrow q }{ q }$ (\textsl{Modus Ponens})

\item Setze $k_1 := \emptyset$, $l := \neg q$ und $k_2 := \{ \neg p \}$: \\[0.5cm]
      $\schluss{\{\} \cup \{ \neg q \} \quad\quad \{ q \} \cup \{ \neg p \} }{ \{\} \cup \{ \neg p \} }$ \\[0.5cm]
     Interpretation von Mengen als Disjunktionen liefert: \\[0.5cm]
     $\schluss{\neg q \quad\quad q \vee \neg p }{ \neg p }$ \\[0.5cm]
     Ber�cksichtigung von $q \vee \neg p \;\leftrightarrow\; p \rightarrow q$ liefert: \\[0.5cm]
     $\schluss{\neg q \quad\quad p \rightarrow q }{ \neg p }$ (\textsl{Modus Ponendo Tollens})

\item $\schluss{\neg p \quad\quad p \rightarrow q}{\neg q}$ (\textsl{Modus Tollendo Tollens}) \\[0.5cm]
     Ist \textsl{MTT} Spezialfall der Schnitt-Regel?
\end{enumerate}

\vspace*{\fill}
\tiny \addtocounter{mypage}{1}
\rule{17cm}{1mm}
Der Beweis-Begriff  \hspace*{\fill} Seite \arabic{mypage}
\end{slide}

%%%%%%%%%%%%%%%%%%%%%%%%%%%%%%%%%%%%%%%%%%%%%%%%%%%%%%%%%%%%%%%%%%%%%%%%

\begin{slide}{}
\normalsize
\begin{center}
Beweis-Begriff: $M \vdash f$
\end{center}
\vspace{0.5cm}

\footnotesize
\textbf{Vor.}: $M \subseteq \mathcal{F}$, $f \in \mathcal{F}$.

Schreibweise: $M \vdash f$  ($M$ beweist $f$)

Definition induktiv:
\begin{enumerate}
\item $M \vdash \top$
\item Wenn $f \in M$ ist, dann gilt $M \vdash f$.
\item Es gelte:
  \begin{enumerate}
  \item $M \vdash k_1 \cup \{p\}$,
  \item $M \vdash \{\neg p\} \cup k_2$.
  \end{enumerate}
  Dann gilt auch \\[0.3cm]
  \hspace*{1.3cm} $M \vdash k_1 \cup k_2$.
\end{enumerate}
\vspace{0.5cm}

\textbf{Korrektheits-Satz}:  Es gilt:\\[0.3cm]
\hspace*{1.3cm}  $M \vdash f  \quad\Rightarrow \quad M \models f$


\vspace*{\fill}
\tiny \addtocounter{mypage}{1}
\rule{17cm}{1mm}
Der Beweis-Begriff  \hspace*{\fill} Seite \arabic{mypage}
\end{slide}

%%%%%%%%%%%%%%%%%%%%%%%%%%%%%%%%%%%%%%%%%%%%%%%%%%%%%%%%%%%%%%%%%%%%%%%%

\begin{slide}{}
\normalsize
\begin{center}
Widerlegungs-Vollst�ndigkeit
\end{center}
\vspace{0.5cm}

\footnotesize
\textbf{Theorem}: (Widerlegungs-Vollst�ndigkeit von $G$) \\[0.3cm]
 Sei $\{ k_1, \cdots, k_n \} \subseteq \mathcal{K}$. Dann gilt\\[0.3cm]
\hspace*{1.3cm}  $\{ k_1, \cdots, k_n \} \models \falsum$ \quad$\Rightarrow$\quad $\{ k_1, \cdots, k_n \} \vdash \falsum$.

\textbf{Frage}: Wann folgt $f$ aus Klausel-Menge $\{ k_1, \cdots, k_n \}$?

\textbf{Antwort}: Falls gilt: $\{ k_1, \cdots, k_n \} \cup \textsl{knf}(\neg f) \vdash \falsum$
\vspace{0.5cm}

\textbf{Definition}: Sei $k \in \mathcal{K}$, $M \subseteq \mathcal{K}$  und $l \in \mathcal{L}$. \\[0.5cm]
\hspace*{1.3cm} $\textsl{redukt}(k,l) := \left\{ 
     \begin{array}{ll}
        \verum                &  \mathrm{falls}\; l \in k; \\
        k \backslash \Bigl\{ \komplement{\,l\,} \Bigr\}  &
   \mathrm{falls}\; l\not\in k\; \mathrm{und}\; \komplement{\,l\,} \in k; \\
        k                     &  \mathrm{sonst}.
     \end{array}
     \right.$ \\[0.5cm]
  \hspace*{1.3cm} $\textsl{Redukt}(M, l) := \bigl\{\; \textsl{redukt}(k,l) \;|\; k \in M \;\bigr\}$.
\vspace{0.5cm}

\textbf{Satz}: Ist $M \subseteq \mathcal{K}$ und $l \in \mathcal{L}$, so gilt \\[0.3cm]
\hspace*{1.3cm} $M \models \falsum \quad \Rightarrow \quad \textsl{Redukt}(M, l) \models \falsum$.
\vspace{0.5cm}

\textbf{Satz}: Ist $M \subseteq \mathcal{K}$, $f \in \mathcal{K}$ und $l \in \mathcal{L}$, so gilt: \\[0.3cm]
\hspace*{0.3cm}  $\textsl{Redukt}(M,l) \gentzen f$  $\;\Rightarrow\;$ 
                 $M \gentzen f$  oder  $M \gentzen f \cup \bigl\{\komplement{\,l\,}\bigr\}$.

\vspace*{\fill}
\tiny \addtocounter{mypage}{1}
\rule{17cm}{1mm}
Der Beweis-Begriff  \hspace*{\fill} Seite \arabic{mypage}
\end{slide}

\end{document}

%%% Local Variables: 
%%% mode: latex
%%% TeX-master: t
%%% End: 
