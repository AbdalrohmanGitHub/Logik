\documentclass{slides}
\usepackage{german}
\usepackage[latin1]{inputenc}
\usepackage{epsfig}
\usepackage{amssymb}

\pagestyle{empty}
\setlength{\textwidth}{17cm}
\setlength{\textheight}{24cm}
\setlength{\topmargin}{0cm}
\setlength{\headheight}{0cm}
\setlength{\headsep}{0cm}
\setlength{\topskip}{0.2cm}
\setlength{\oddsidemargin}{0.5cm}
\setlength{\evensidemargin}{0.5cm}

\newcommand{\is}{\;|\;}
\newcommand{\schluss}[2]{\frac{\displaystyle\quad \rule[-8pt]{0pt}{18pt}#1 \quad}{\displaystyle\quad \rule{0pt}{14pt}#2 \quad}}
\newcommand{\vschlus}[1]{{\displaystyle\rule[-6pt]{0pt}{12pt} \atop \rule{0pt}{10pt}#1}}
\newcommand{\verum}{\top}
\newcommand{\falsum}{\bot}
\newcommand{\var}{\textsl{Var}}
\newcommand{\struct}{\mathcal{S}}
\newcommand{\el}{\!\in\!}
\newcommand{\FV}{\textsl{FV}}
\newcommand{\BV}{\textsl{BV}}
\newcommand{\gentzen}{\vdash_\mathcal{G}}
\newcommand{\komplement}[1]{\overline{#1}}

\newcounter{mypage}

\begin{document}

%%%%%%%%%%%%%%%%%%%%%%%%%%%%%%%%%%%%%%%%%%%%%%%%%%%%%%%%%%%%%%%%%%%%%%%%

\begin{slide}{}
\normalsize
\begin{center}
Vision des logischen Programmierens 
\end{center}
\vspace{0.5cm}

\footnotesize
\begin{enumerate}
\item Prolog: ``\emph{\underline{pro}gramming in \underline{log}ic}''.

\item Beschreibe gegebenes Problem durch  Formeln.

      Prolog: Beschreibung durch 
      \begin{enumerate}
      \item Fakten,
      \item Regeln und
      \end{enumerate}
\item Benutze Automatischen Beweiser (Inferenz-Maschine) zur Beantwortung
      einer Anfrage.

      \begin{center}
      \framebox{\framebox{Kein Algorithmus erforderlich!}}        
      \end{center}

\item Das funktioniert nur in einfachen F�llen.
  
      Im allgemeinen muss man den Suchalgorithmus des Beweisers verstehen,
      um komplexe Probleme l�sen zu k�nnen.
\item Trotzdem vorteilhaft:  

     Programme sind Formeln, haben deshalb \emph{deklarative Semantik} und sind
     daher (nach Gew�hnungsphase) leicht zu verstehen.
\end{enumerate}

\vspace*{\fill}
\tiny \addtocounter{mypage}{1}
\rule{17cm}{1mm}
Prolog  \hspace*{\fill} Seite \arabic{mypage}
\end{slide}


\begin{slide}{}
\normalsize
\begin{center}
Prolog --- Syntax
\end{center}
\vspace{0.5cm}

\footnotesize
\begin{enumerate}
\item Variablen bestehen aus 
      \begin{enumerate}
      \item Buchstaben
      \item Ziffern
      \item ``\texttt{\_}''
      \end{enumerate}
      Erster Buchstabe: Gro\3buchstabe oder ``\texttt{\_}''.

      Beispiele: \\[0.3cm]
     \hspace*{1.3cm} ``\texttt{X}'', ``\texttt{\_}'', ``\texttt{A\_Long\_Variable\_Name}'',
                     ``\texttt{ABC}'',  ``\texttt{\_Y}''. 
\item Funktions--Zeichen bestehen aus
      \begin{enumerate}
      \item Buchstaben
      \item Ziffern
      \item ``\texttt{\_}''
      \end{enumerate}
      Erster Buchstabe: Kleinbuchstabe

      Beispiele: \\[0.3cm]
      \hspace*{1.3cm}  ``\texttt{abc2}'', ``\texttt{a\_XY}'', ``\texttt{etc}'', ``\texttt{x}'', ``\texttt{q}'', 
                 ``\texttt{q2}'', ``\texttt{+}''.


      Zus"atzlich: Operator-Symbole \\[0.3cm]
     \hspace*{1.3cm} ``\texttt{+}'', ``\texttt{-}'', ``\texttt{*}'', ``\texttt{/}'', ``\texttt{.}'', $\cdots$
 
      Au\3erdem: Zahlen 
      \\[0.3cm]
      \hspace*{1.3cm} 1, -32, 0.5, 1.4e-6
\end{enumerate}


\vspace*{\fill}
\tiny \addtocounter{mypage}{1}
\rule{17cm}{1mm}
Prolog  \hspace*{\fill} Seite \arabic{mypage}
\end{slide}

%%%%%%%%%%%%%%%%%%%%%%%%%%%%%%%%%%%%%%%%%%%%%%%%%%%%%%%%%%%%%%%%%%%%%%%%

\begin{slide}{}
\normalsize
\begin{center}
Syntax
\end{center}
\vspace{0.5cm}

\footnotesize
\begin{enumerate}
\item[3.] Pr�dikats--Zeichen bestehen aus

      Buchstaben, Ziffern, ``\texttt{\_}''

      Erster Buchstabe: Kleinbuchstabe

      Beispiele: \\[0.3cm]
      \hspace*{1.3cm} ``\texttt{abc2}'', ``\texttt{a\_XY}'', ``\texttt{etc}'', ``\texttt{x}'', ``\texttt{q}'', 
                      ``\texttt{q2}''.

      Au�erdem: Operator-Symbole 
      \\[0.3cm]
      \hspace*{1.3cm}      
      ``\texttt{=}'', ``\texttt{$\backslash$=}'', ``\texttt{<}'', ``\texttt{>}'', ``\texttt{=<}'', ``\texttt{>=}'', $\cdots$

      Kontext: Unterscheidung \\[0.3cm]
      \hspace*{1.3cm} Pr�dikats--Zeichen --- Funktions--Zeichen 
\item[4.] Fakten 

      $p\big(s_1, \cdots, s_m\big)$.

      $p$: Pr�dikats-Zeichen, \quad $s_i$: Terme
\item[5.] Klauseln (Regeln)

      $p\big(s_1, \cdots, s_m\big)$ \texttt{:-} $p_1\big(s^{1}_1, \cdots, s^{1}_{m(1)}\big)$\texttt{,} $\cdots$\texttt{,} $p_n(s^{n}_1, \cdots, s^{n}_{m(n)}\big)$.

      Interpretation

      $p_1\big(s^{(1)}_1, \cdots, s^{(1)}_{m(1)}\big) \wedge \cdots \wedge p_n(s^{(n)}_1,\cdots, s^{(n)}_{m(n)}\big) \rightarrow p\big(s_1, \cdots, s_m\big)$.  

      Ersetzung:
      \begin{enumerate}
      \item ``$\leftarrow$'' durch ``\texttt{:-}''
      \item ``$\wedge$'' durch ``\texttt{,}''
      \end{enumerate}
\end{enumerate}

\vspace*{\fill}
\tiny \addtocounter{mypage}{1}
\rule{17cm}{1mm}
Prolog  \hspace*{\fill} Seite \arabic{mypage}
\end{slide}

%%%%%%%%%%%%%%%%%%%%%%%%%%%%%%%%%%%%%%%%%%%%%%%%%%%%%%%%%%%%%%%%%%%%%%%%

\begin{slide}{}
\normalsize
\begin{center}
Programm: Beispiel
\end{center}
\vspace{0.5cm}

\footnotesize
\begin{verbatim}
    gallier(asterix).
    gallier(obelix).

    stark(X) :- gallier(X).

    maechtig(X) :- stark(X).
    maechtig(X) :- kaiser(X).

    kaiser(caesar).
    roemer(caesar).

    spinnt(X) :- roemer(X).
\end{verbatim}

Interpretation:
\begin{enumerate}
\item Asterix ist ein Gallier.
\item Obelix ist ein Gallier.
\item Gallier sind stark.
\item Wer stark ist, ist m�chtig.
\item Wer Kaiser ist, ist m�chtig.
\item C�sar ist ein Kaiser.
\item C�sar ist ein R�mer.
\item Die R�mer spinnen. 
\end{enumerate}

\vspace*{\fill}
\tiny \addtocounter{mypage}{1}
\rule{17cm}{1mm}
Prolog  \hspace*{\fill} Seite \arabic{mypage}
\end{slide}

%%%%%%%%%%%%%%%%%%%%%%%%%%%%%%%%%%%%%%%%%%%%%%%%%%%%%%%%%%%%%%%%%%%%%%%%

\begin{slide}{}
\normalsize
\begin{center}
Knobeleien mit Prolog
\end{center}
\vspace{0.5cm}

\footnotesize
\begin{enumerate}
\item Drei Freunde belegen den ersten, zweiten und dritten Platz bei einem
      Programmier-Wettbewerb.
\item Jeder der drei hat genau einen Vornamen, genau ein Auto und hat sein Programm
      in genau einer Programmier-Sprache geschrieben.
\item Michael programmiert in \textsl{Setl} und war besser als der Audi-Fahrer.
\item Julia, die einen Ford Mustang f�hrt, war besser als der Java-Programmierer.
\item Das Prolog-Programm war am besten.
\item Wer f�hrt Toyota?
\item In welcher Sprache programmiert Thomas?
\end{enumerate}

Darstellung von Personen durch Pr�dikat \texttt{person/3}
\[ \texttt{person}(\textsl{Name}, \textsl{Car}, \textsl{Language}) \]
\begin{enumerate}
\item $\textsl{Name} \in \{ \mathtt{julia}, \mathtt{thomas}, \mathtt{michael} \}$, 
\item $\textsl{Car} \in \{ \mathtt{ford}, \mathtt{toyota}, \mathtt{audi} \}$, 
\item $\textsl{Language} \in \{ \mathtt{java}, \mathtt{prolog}, \mathtt{setl} \}$.
\end{enumerate}

\vspace*{\fill}
\tiny \addtocounter{mypage}{1}
\rule{17cm}{1mm}
Prolog  \hspace*{\fill} Seite \arabic{mypage}
\end{slide}

%%%%%%%%%%%%%%%%%%%%%%%%%%%%%%%%%%%%%%%%%%%%%%%%%%%%%%%%%%%%%%%%%%%%%%%%

\begin{slide}{}
\normalsize
\begin{center}
Vision des logischen Programmierens 
\end{center}
\vspace{0.5cm}

\footnotesize
\begin{enumerate}
\item Beschreibe gegebenes Problem durch  Formeln.

      Prolog: Beschreibung durch 
      \begin{enumerate}
      \item Fakten,
      \item Regeln und
      \end{enumerate}
\item Benutze Automatischen Beweiser (Inferenz-Maschine) zur Beantwortung
      einer Anfrage.

      \begin{center}
      \framebox{\framebox{Kein Algorithmus erforderlich!}}        
      \end{center}

      Leider ist das nur eine Vision.
\end{enumerate}

Vorheriges Beispiel:  \\[0.3cm]
\hspace*{1.3cm} Gibt es einen M�chtigen, der spinnt?

Anfrage an Prolog:
\begin{verbatim}
    $ pl
    Welcome to SWI-Prolog (Version 5.0.8)

    1 ?- consult(gallier).
    % gallier compiled 0.00 sec, 1,712 bytes

    Yes
    2 ?- maechtig(X), spinnt(X).

    X = caesar 
\end{verbatim}%$

\vspace*{\fill}
\tiny \addtocounter{mypage}{1}
\rule{17cm}{1mm}
Prolog  \hspace*{\fill} Seite \arabic{mypage}
\end{slide}

%%%%%%%%%%%%%%%%%%%%%%%%%%%%%%%%%%%%%%%%%%%%%%%%%%%%%%%%%%%%%%%%%%%%%%%%

\begin{slide}{}
\normalsize
\begin{center}
Listen-Notation in Prolog
\end{center}
\vspace{0.5cm}

\footnotesize
Spezielle Funktions-Zeichen: 
\begin{enumerate}
\item $\texttt{.}: \textsl{Term} \times \textsl{Term} \rightarrow \textsl{Term}$
\item $\texttt{[]}: \textsl{Term}$
\end{enumerate}
Interpretation:
\begin{enumerate}
\item $l = \texttt{[]}$: $l$ ist leere Liste
\item $l = \texttt{.}(x, r)$: 
      \begin{enumerate}
      \item $x$ erstes Element von $l$
      \item $r$ restliche Elemente von $l$
      \end{enumerate}
\end{enumerate}
Kurzschreibweisen:
\begin{enumerate}
\item \texttt{[a,b,c] := .(a, .(b, .(c, [])))} 
\item \texttt{[ $x$ | $r$ ] := .($x$, $r$)}
\end{enumerate}
Beispiel:
\begin{verbatim}
  concat( [], L, L ).
  concat( [ X | L1 ], L2, [ X | L3 ] ) :- 
       concat( L1, L2, L3 ).
\end{verbatim}

\vspace*{\fill}
\tiny \addtocounter{mypage}{1}
\rule{17cm}{1mm}
Prolog  \hspace*{\fill} Seite \arabic{mypage}
\end{slide}

%%%%%%%%%%%%%%%%%%%%%%%%%%%%%%%%%%%%%%%%%%%%%%%%%%%%%%%%%%%%%%%%%%%%%%%%

\begin{slide}{}
\normalsize
\begin{center}
Prolog -- Was passiert
\end{center}
\vspace{0.5cm}

\footnotesize
Prolog = Implementierung von $\leadsto_\mathcal{P}$:

\textbf{Gegeben}:
\begin{enumerate}
\item Prolog-Programm $\mathcal{P}$
\item Ziel $G$
\end{enumerate}
\textbf{Vorgehen}: $\langle G, [] \rangle \leadsto_\mathcal{P}^* \langle \verum,\tau\rangle$

$\tau$ ist \emph{berechnete Antwort}.

\textbf{Problem}: $\leadsto_\mathcal{P}$ ist nicht-deterministisch:
\begin{verbatim}
p(X) :- q1(X).
p(X) :- q2(X).

q1(a).
q2(b).
\end{verbatim}
Betrachte Ziel \texttt{p(X)}:

$\langle \texttt{p(X)}, [] \rangle \leadsto_\mathcal{P} \langle \texttt{q1(X)}, [] \rangle \leadsto \langle \verum, [ \mathtt{X} \mapsto \mathtt{a} ] \rangle$ 

$\langle \texttt{p(X)}, [] \rangle \leadsto_\mathcal{P} \langle \texttt{q2(X)}, [] \rangle \leadsto \langle \verum, [ \mathtt{X} \mapsto \mathtt{b} ] \rangle$ 

Die Substitutionen $[\texttt{X} \mapsto \texttt{a}]$  und $[\texttt{X} \mapsto \texttt{b}]$ sind beide richtig.

\textbf{Frage}: Welche liefert Prolog?

\textbf{Antwort}: Prolog w�hlt Klauseln in der Reihenfolge aus,
in der sie im Programm stehen!

\vspace*{\fill}
\tiny \addtocounter{mypage}{1}
\rule{17cm}{1mm}
Prolog  \hspace*{\fill} Seite \arabic{mypage}
\end{slide}

%%%%%%%%%%%%%%%%%%%%%%%%%%%%%%%%%%%%%%%%%%%%%%%%%%%%%%%%%%%%%%%%%%%%%%%%

\begin{slide}{}
\normalsize
\begin{center}
Scheitern von Zielen
\end{center}
\vspace{0.5cm}

\footnotesize
\textbf{Definition}: \\  Ziel $G_1$ \emph{scheitert unmittelbar} 
(bez�glich Programm $\mathcal{P}$) g.d.w. \\[0.1cm]
\hspace*{1.3cm} $\neg \exists G_2 : G_1 \leadsto_\mathcal{P} G_2$ 

\textbf{Beispiel}: Gegebenes Programm $\mathcal{P}$:
\begin{verbatim}
  q(a).
  q(b).
\end{verbatim}
Ziel: $G = \texttt{q(c)}$.
\vspace{0.5cm}

\textbf{Problem}: Unendliche Reduktionen

\textbf{Beispiel}: Gegebenes Programm \texttt{loop}:
\begin{verbatim}
  loop :- loop.
  loop.
\end{verbatim}

Es gilt \\[0.1cm]
\hspace*{1.3cm} $\langle \texttt{loop}, [] \rangle \leadsto_\mathcal{P} \langle \texttt{loop}, [] \rangle$

\begin{verbatim}
8 ?- consult(loop).
% loop compiled 0.00 sec, 652 bytes

Yes
9 ?- loop.
ERROR: Out of local stack
\end{verbatim}

\vspace*{\fill}
\tiny \addtocounter{mypage}{1}
\rule{17cm}{1mm}
Prolog  \hspace*{\fill} Seite \arabic{mypage}
\end{slide}

%%%%%%%%%%%%%%%%%%%%%%%%%%%%%%%%%%%%%%%%%%%%%%%%%%%%%%%%%%%%%%%%%%%%%%%%

\begin{slide}{}
\normalsize
\begin{center}
Prolog-Algorithmus
\end{center}
\vspace{0.5cm}

\footnotesize
\textbf{Gegeben}: 
\begin{enumerate}
\item Ziel $G$
\item Prolog-Programm $\mathcal{P}$
\end{enumerate}
\textbf{Gesucht}: Substitution $\tau$ mit\\[0.3cm]
\hspace*{1.3cm} $\mathcal{P} \models G\tau$
\begin{enumerate}
\item Setze $G\!L := [ \langle G, [] \rangle ]$.
\item Falls $G\!L$ == [] ist, ist Ziel $G$ \emph{endg�ltig gescheitert}.

\item Es sei $G\!L = [ Z | Zs ]$ 

      Fallunterscheidung:
      \begin{enumerate}
      \item $Z = \langle \verum, \tau \rangle$.

            Gebe Ergebnis $\tau$ aus.
      \item Seien  \\[0.1cm]
            \hspace*{1.3cm} $Z \leadsto_\mathcal{P} Z_i$  f�r $i=1,\cdots,n$ \\[0.3cm]
            alle m�gliche Reduktionen von $Z$.  Dann setze \\[0.3cm]
            \hspace*{1.3cm}  $G\!L := [ Z_1, \cdots, Z_n | Zs ]$. \\[0.3cm]
            Gehe zu Schritt 2.
      \end{enumerate}
\end{enumerate}

\vspace*{\fill}
\tiny \addtocounter{mypage}{1}
\rule{17cm}{1mm}
Prolog  \hspace*{\fill} Seite \arabic{mypage}
\end{slide}

%%%%%%%%%%%%%%%%%%%%%%%%%%%%%%%%%%%%%%%%%%%%%%%%%%%%%%%%%%%%%%%%%%%%%%%%

\begin{slide}{}
\normalsize
\begin{center}
Probleme des Prolog-Algorithmus
\end{center}
\vspace{0.5cm}

\footnotesize
\begin{enumerate}
\item[I.] Nicht vollst�ndig wegen unendlicher Reduktionen! \\
      Beispiel:
      \begin{verbatim}
      loop :- loop.
      loop.
      \end{verbatim}
      \vspace{-0.8cm}

      Es gilt $\mathcal{P} \models \texttt{loop}$, \\
      aber Prolog-Algorithmus terminiert nicht!
\item[II.] Unifikation falsch: 1. Martelli-Montanari-Regel:

      \begin{enumerate}
      \item[1.] Falls $y\in\mathcal{V}$ und \fbox{$x \not\in\FV(t)$} so gilt: \\[0.3cm]
           \hspace*{1.3cm} $\Big\langle E \cup \big\{ y \doteq t \big\}, \sigma \Big\rangle \;\leadsto\; \Big\langle E[y \mapsto t], \sigma\big[ y \mapsto t \big] \Big\rangle$.
      \end{enumerate}
      \vspace{0.3cm}

      \emph{Occur-Check}: $x \not\in\FV(t)$ fehlt in Prolog.
      \vspace{0.5cm}

      Beispiel: Programm \texttt{occur}:
     \begin{verbatim}
    equal(Y,Y).
     \end{verbatim}
\begin{verbatim}
    1 ?- consult(occur).
    % occur compiled 0.00 sec, 504 bytes
    
    Yes
    2 ?- equal(X, s(X)).
    
    X = s(s(s(s(s(s(s(s(s(s(...)))))))))) 
\end{verbatim}
\end{enumerate}


\vspace*{\fill}
\tiny \addtocounter{mypage}{1}
\rule{17cm}{1mm}
Prolog  \hspace*{\fill} Seite \arabic{mypage}
\end{slide}


\end{document}

%%% Local Variables: 
%%% mode: latex
%%% TeX-master: t
%%% End: 
