\documentclass{slides}
\usepackage[latin1]{inputenc}
\usepackage{german}
\usepackage{epsfig}
\usepackage{amssymb}

\pagestyle{empty}
\setlength{\textwidth}{17cm}
\setlength{\textheight}{24cm}
\setlength{\topmargin}{0cm}
\setlength{\headheight}{0cm}
\setlength{\headsep}{0cm}
\setlength{\topskip}{0.2cm}
\setlength{\oddsidemargin}{0.5cm}
\setlength{\evensidemargin}{0.5cm}

\newcommand{\verum}{\top}
\newcommand{\falsum}{\bot}
\newcommand{\schluss}[2]{\frac{\displaystyle\quad \rule[-6pt]{0pt}{12pt}#1 \quad}{\displaystyle\quad \rule{0pt}{10pt}#2 \quad}}
\newcommand{\gentzen}{\vdash}
\newcommand{\komplement}[1]{\overline{#1}}
\newcommand{\mathquote}[1]{\mbox{``}\mathtt{#1}\mbox{''}}
\newcommand{\is}{\;|\;}
\newcommand{\circneg}{\mbox{$\bigcirc\hspace*{-0.5cm}\neg$}}
\newcommand{\circwedge}{\mbox{$\bigcirc\hspace*{-0.5cm}\wedge$}}
\newcommand{\circvee}{\mbox{$\bigcirc\hspace*{-0.5cm}\vee$}}
\newcommand{\circright}{\mbox{$\bigcirc\hspace*{-0.8cm}\rightarrow$}}
\newcommand{\circleftright}{\mbox{$\bigcirc\hspace*{-0.8cm}\leftrightarrow$}}

\newcounter{mypage}

\begin{document}

%%%%%%%%%%%%%%%%%%%%%%%%%%%%%%%%%%%%%%%%%%%%%%%%%%%%%%%%%%%%%%%%%%%%%%%%

\begin{slide}{}
\normalsize
\begin{center}
Aussagen-Logik
\end{center}
\vspace{0.5cm}

\footnotesize
\begin{enumerate}
     \item Einfache Aussagen: S�tze, die
       \begin{enumerate}
       \item Tatbestand ausdr�cken, (wahr oder falsch) 
       \item keine \emph{Junktoren} enthalten.

         Junktoren: ``\emph{und}'', ``\emph{oder}'', ``\emph{nicht}'', \\
         ``\emph{wenn $\cdots$, dann}'', und ``\emph{genau dann, wenn}''
       \end{enumerate}
       Beispiele f�r einfache Aussagen:
       \begin{enumerate}
       \item ``{\em Die Sonne scheint.}''
       \item ``{\em Es regnet.}''
       \item ``{\em Am Himmel ist ein Regenbogen.}''
       \end{enumerate}
       einfache Aussagen = \emph{atomare} Aussagen
\item Zusammengesetzte Aussagen
  \begin{center}
    ``\emph{\underline{Wenn} die Sonne scheint \underline{und} es regnet, \\
            \underline{dann} ist ein Regenbogen am Himmel}.''
  \end{center}
\item Logische Schl�sse
$$ \schluss{
  \begin{array}{ll}
  1. &\mathtt{SonneScheint} \\ 
  2. & \mathtt{EsRegnet}     \\
  3. & \mathtt{SonneScheint} \wedge \mathtt{EsRegnet} \rightarrow \mathtt{Regenbogen}
  \end{array}
   }{\mathtt{Regenbogen}} 
$$
     
\end{enumerate}

\vspace*{\fill}
\tiny \addtocounter{mypage}{1}
\rule{17cm}{1mm}
Aussagenlogik  \hspace*{\fill} Seite \arabic{mypage}
\end{slide}

%%%%%%%%%%%%%%%%%%%%%%%%%%%%%%%%%%%%%%%%%%%%%%%%%%%%%%%%%%%%%%%%%%%%%%%%

\begin{slide}{}
\normalsize
\begin{center}
Schluss-Regeln
\end{center}
\vspace{0.5cm}

\footnotesize
\begin{enumerate}
\item  Junktoren als Abk�rzungen:
\begin{enumerate}
\item $\neg a$ \quad\quad\ f�r \quad \emph{nicht} $a$ 
\item $a \wedge b$ \,\quad\ f�r \quad $a$ \emph{und} $b$
\item $a \vee b$ \,\quad\ f�r \quad $a$ \emph{oder} $b$
\item $a \rightarrow b$   \quad f�r \quad \emph{wenn} $a$, \emph{dann} $b$
\item $a \leftrightarrow b$ \quad f�r \quad  $a$ \emph{genau dann, wenn} $b$
\end{enumerate}

\item Konkreter Schluss:
$$ \schluss{
  \begin{array}{ll}
  1. &\mathtt{SonneScheint} \\ 
  2. & \mathtt{EsRegnet}     \\
  3. & \mathtt{SonneScheint} \wedge \mathtt{EsRegnet} \rightarrow \mathtt{Regenbogen}
  \end{array}
   }{\mathtt{Regenbogen}} 
$$
\item Schluss-Regel:
$$ \frac{\;p \quad q \quad p \wedge q \rightarrow r\;}{\;r\;}  $$

\textbf{Aufgabe:} Welche  Schluss-Regel wird in dem folgenden Argument verwendet?
\begin{center}
\begin{minipage}[c]{14.2cm}
{\em  ``Wenn es regnet, ist die Stra�e nass.  Es regnet nicht.  Also ist die Stra�e nicht nass.''}
\end{minipage}  
\end{center}

\end{enumerate}

\vspace*{\fill}
\tiny \addtocounter{mypage}{1}
\rule{17cm}{1mm}
Aussagenlogik  \hspace*{\fill} Seite \arabic{mypage}
\end{slide}

%%%%%%%%%%%%%%%%%%%%%%%%%%%%%%%%%%%%%%%%%%%%%%%%%%%%%%%%%%%%%%%%%%%%%%%%

\begin{slide}{}
\normalsize
\begin{center}
Kalk�l, Herleitungsbegriff
\end{center}
\vspace{0.5cm}

\footnotesize
\begin{enumerate}
\item $\textsl{Kalk�l} = \textsl{Menge von Schluss-Regeln} $
\item Scheibweise:  \quad $M \vdash r$
     
      lese:  Kalk�l $\vdash$ leitet $r$ aus $M$ her.    
      \begin{enumerate}
      \item $\vdash$: Kalk�l
      \item $M$: Menge von aussagenlogischen Formeln
      \item $r$: aussagenlogische Formel
      \end{enumerate}
\end{enumerate}

\normalsize
\begin{center}
  Folgerungsbegriff
\end{center}

\footnotesize
\begin{enumerate}
\item $M \models r$  \qquad (lese: aus $M$ folgt $r$)

      Interpretation: 
      \begin{center}
\begin{minipage}[c]{11.2cm}
      \emph{Wenn alle Formeln aus $M$ wahr sind, dann ist auch $r$ wahr.}
\end{minipage}          
      \end{center}
\end{enumerate}

\normalsize
\begin{center}
Ziel: korrekter und vollst�ndiger Kalk�l 
\end{center}

\footnotesize
\begin{enumerate}
\item Korrektheit:
  \begin{center}
      Aus $M \vdash r$ folgt $M \models r$.     
  \end{center}
\item Vollst�ndigkeit:
  \begin{center}
      Aus $M \models r$ folgt $M \vdash r$.     
  \end{center}
\end{enumerate}


\vspace*{\fill}
\tiny \addtocounter{mypage}{1}
\rule{17cm}{1mm}
Aussagenlogik  \hspace*{\fill} Seite \arabic{mypage}
\end{slide}

%%%%%%%%%%%%%%%%%%%%%%%%%%%%%%%%%%%%%%%%%%%%%%%%%%%%%%%%%%%%%%%%%%%%%%%%

\begin{slide}{}
\normalsize
\begin{center}
Anwendung der Aussagenlogik 
\end{center}
\vspace{0.5cm}

\footnotesize
\begin{enumerate}
\item Analyse und Design digitaler Schaltungen.

      Pentium \texttt{IV}, Northwood Kernel: 55 Millionen Gatter
  \begin{enumerate}
  \item Schaltungsvergleich: Magma offeriert \\
        \textsl{Quartz Formal} 
        zum Preis von $150\,000\, \symbol{36}$ pro \\
        Lizenz 
  \item Timing Analyse
  \item $\cdots$
  \end{enumerate}
\item Erstellung von Stundenpl�nen.

      (diskrete Mathematik $\mapsto$ Aussagenlogik)

\item Verschlusspl�ne f�r Weichen und Signale.

      (Einstellung von Fahrstra�en)
\item Kombinatorische Puzzle

      (Beispiel: 8--Damen--Problem).
\end{enumerate}

\vspace*{\fill}
\tiny \addtocounter{mypage}{1}
\rule{17cm}{1mm}
Aussagenlogik  \hspace*{\fill} Seite \arabic{mypage}
\end{slide}

%%%%%%%%%%%%%%%%%%%%%%%%%%%%%%%%%%%%%%%%%%%%%%%%%%%%%%%%%%%%%%%%%%%%%%%%

\begin{slide}{}
\normalsize
\begin{center}
Extensional vs. Intensional Interpretation
\end{center}
\vspace{0.5cm}

\footnotesize
Interpretation  aussagenlogischer Junktoren  \emph{extensional}:

Berechnung von  
\begin{center}
\hspace*{1.3cm}
$\mathcal{I}(f \odot g)$ \quad mit $\odot \in \{\wedge, \vee, \rightarrow,\leftrightarrow \}$
\end{center}
\begin{enumerate}
\item Werte $\mathcal{I}(f)$ und $\mathcal{I}(g)$ reichen aus,
\item $f$ und $g$ nicht ben�tigt!
\end{enumerate}


\textbf{Problem}:  Umgangssprache
\begin{center}
Kausale Bedeutung von ``\emph{wenn $\cdots$, dann}'' 
\end{center}

\textbf{Beispiel}:
\begin{center}
``\emph{Wenn $3 \cdot 3 = 8$, dann schneit es Morgen.}''  
\end{center}
\begin{enumerate}
\item Extensional: wahr
\item Intensional: Unsinn, da kein Zusammenhang besteht.
\end{enumerate}

\textbf{Erkenntnis}
\begin{itemize}
\item extensionale Interpretation ist \emph{Abstraktion}

      kausale Zusammenh�nge bleiben unber�cksichtigt
\item mathematische Praxis
  \begin{enumerate}
  \item extensionale Interpretation ausreichend
  \item intensionale Interpretation zu kompliziert
  \end{enumerate}
\end{itemize}


\vspace*{\fill}
\tiny \addtocounter{mypage}{1}
\rule{17cm}{1mm}
Aussagenlogik  \hspace*{\fill} Seite \arabic{mypage}
\end{slide}

%%%%%%%%%%%%%%%%%%%%%%%%%%%%%%%%%%%%%%%%%%%%%%%%%%%%%%%%%%%%%%%%%%%%%%%%

\begin{slide}{}
\normalsize
\begin{center}
Aussagenlogische Formeln
\end{center}
\vspace{0.5cm}

\footnotesize
Gegeben: $\mathcal{P}$ Menge 0--stelliger Pr�dikats--Zeichen\\[0.3cm]
\hspace*{1.3cm} (Aussage-Variablen)
\begin{enumerate}
\item $\top \in \mathcal{F}$ und $\mathtt{\bot} \in \mathcal{F}$.

      $\top$:  \emph{Verum},  immer wahr.

      $\bot$:  \emph{Falsum}, immer falsch. 
\item Wenn $p \in \mathcal{P}$, dann $p \in \mathcal{F}$.
\item Wenn $f \in \mathcal{F}$, dann $\neg f \in \mathcal{F}$.
\item Wenn $f_1, f_2 \in \mathcal{F}$, dann $(f_1 \vee f_2) \in \mathcal{F}$.
\item Wenn $f_1, f_2 \in \mathcal{F}$, dann $(f_1 \wedge f_2) \in \mathcal{F}$.
\item Wenn $f_1, f_2 \in \mathcal{F}$, dann $(f_1 \rightarrow f_2) \in \mathcal{F}$.
\item Wenn $f_1, f_2 \in \mathcal{F}$, dann $(f_1 \leftrightarrow f_2) \in \mathcal{F}$.
\end{enumerate}
Beispiele: Sei $\mathcal{P} = \{ p, q, r \}$
\begin{itemize}
\item $(\neg p \rightarrow q)$
\item $((p \wedge q) \rightarrow q))$
\item $(p \leftrightarrow (q \wedge (q \wedge p)))$
\end{itemize}


\vspace*{\fill}
\tiny \addtocounter{mypage}{1}
\rule{17cm}{1mm}
Aussagenlogik  \hspace*{\fill} Seite \arabic{mypage}
\end{slide}

%%%%%%%%%%%%%%%%%%%%%%%%%%%%%%%%%%%%%%%%%%%%%%%%%%%%%%%%%%%%%%%%%%%%%%%%

\begin{slide}{}
\normalsize
\begin{center}
Klammern Sparen 
\end{center}
\vspace{0.5cm}

\footnotesize
\begin{enumerate}
\item �u�ere Klammern werden weggelassen:\\[0.3cm]
      \hspace*{1.3cm} $p \wedge q$ \quad statt \quad $(p \wedge q)$.
\item ``$\vee$'' und ``$\wedge$'' werden links geklammert:  \\[0.3cm]
      \hspace*{1.3cm} $p \wedge q \wedge r$ \quad statt \quad $(p \wedge q) \wedge r$.
\item ``$\rightarrow$'' wird rechts geklammert: \\[0.3cm]
      \hspace*{1.3cm} $p \rightarrow q \rightarrow r$ \quad statt \quad $p \rightarrow (q \rightarrow r)$.
\item ``$\vee$'' und ``$\wedge$'' binden st�rker als ``$\rightarrow$'': \\[0.3cm]
      \hspace*{1.3cm} $p \wedge q \rightarrow r$ \quad statt \quad $(p \wedge q) \rightarrow r$
\item ``$\rightarrow$'' bindet st�rker als ``$\leftrightarrow$'': \\[0.3cm]
      \hspace*{1.3cm} $p \rightarrow q \leftrightarrow r$ \quad statt \quad $(p \rightarrow q) \leftrightarrow r$.
\end{enumerate}
Beispiele:
\begin{itemize}
\item $(\neg p \rightarrow q)$ \quad wird zu \quad $\neg p \rightarrow q$
\item $((p \wedge q) \rightarrow (q \vee r))$  \quad wird zu \quad $p \wedge q \rightarrow
  q \vee r$
\item $(p \leftrightarrow ((q \wedge q) \wedge p))$ \quad wird zu \quad
      $p \leftrightarrow q \wedge q \wedge p$
\end{itemize}


\vspace*{\fill}
\tiny \addtocounter{mypage}{1}
\rule{17cm}{1mm}
Aussagenlogik  \hspace*{\fill} Seite \arabic{mypage}
\end{slide}


%%%%%%%%%%%%%%%%%%%%%%%%%%%%%%%%%%%%%%%%%%%%%%%%%%%%%%%%%%%%%%%%%%%%%%%%

\begin{slide}{}
\normalsize
\begin{center}
Wahrheits--Tafel
\end{center}
\vspace{0.5cm}

\footnotesize

  \begin{tabular}{|l|l|l|l|l|l|l|}
\hline
   $p$            & $q$            &  $\neg p$      &  $p \vee q$    & $p \wedge q$   & $p \rightarrow q$ & $p \leftrightarrow q$
   \\
\hline
\hline
   \texttt{true}  & \texttt{true}  & \texttt{false} & \texttt{true}  & \texttt{true}  & \texttt{true}     & \texttt{true}  \\
\hline
   \texttt{true}  & \texttt{false} & \texttt{false} & \texttt{true}  & \texttt{false} & \texttt{false}    & \texttt{false}  \\
\hline
   \texttt{false} & \texttt{true}  & \texttt{true}  & \texttt{true}  & \texttt{false} & \texttt{true}     & \texttt{false} \\
\hline
   \texttt{false} & \texttt{false} & \texttt{true}  & \texttt{false} & \texttt{false} & \texttt{true}     & \texttt{true}  \\
\hline
  \end{tabular}
\vspace{0.5cm}

Interpretiere Junktoren als Funktionen \\
gem�� Wahrheits--Tafel:

\begin{tabular}{|l|l|}
\hline
$\circneg: \mathbb{B} \rightarrow \mathbb{B}$                           &                                                             \\
\hline
$\circwedge: \mathbb{B} \times \mathbb{B} \rightarrow \mathbb{B}$       & $\circvee: \mathbb{B} \times \mathbb{B} \rightarrow \mathbb{B}$ \\
\hline
$\circright: \mathbb{B} \times \mathbb{B} \rightarrow \mathbb{B}$  & $\circleftright: \mathbb{B} \times \mathbb{B} \rightarrow \mathbb{B}$ \\
\hline
\end{tabular}

\textbf{Definition:}  Semantik der Aussagenlogik \\[0.3cm]
Gegeben: \ \quad $\mathcal{I}: \mathcal{P} \rightarrow \mathbb{B}$ \\[0.3cm]
Erweitere: \quad $\mathcal{I}: \mathcal{F} \rightarrow \mathbb{B}$
\begin{enumerate}
\item $\mathcal{I}(\neg f)              :=\; \circneg       \left( \mathcal{I}(f)                 \rule{0pt}{14pt}\right)$
\item $\mathcal{I}(f \wedge g)          :=\; \circwedge     \left( \mathcal{I}(f),\; \mathcal{I}(g) \rule{0pt}{16pt}\right)$
\item $\mathcal{I}(f \vee g)            :=\; \circvee       \left( \mathcal{I}(f),\; \mathcal{I}(g) \rule{0pt}{16pt}\right)$
\item $\mathcal{I}(f \rightarrow g)     :=\; \circright     \left( \mathcal{I}(f),\; \mathcal{I}(g) \rule{0pt}{16pt}\right)$
\item $\mathcal{I}(f \leftrightarrow g) :=\; \circleftright \left( \mathcal{I}(f),\; \mathcal{I}(g) \rule{0pt}{16pt}\right)$
\end{enumerate}

\vspace*{\fill}
\tiny \addtocounter{mypage}{1}
\rule{17cm}{1mm}
Aussagenlogik  \hspace*{\fill} Seite \arabic{mypage}
\end{slide}

%%%%%%%%%%%%%%%%%%%%%%%%%%%%%%%%%%%%%%%%%%%%%%%%%%%%%%%%%%%%%%%%%%%%%%%%

\begin{slide}{}
\normalsize
\begin{center}
Wahrheits--Tafeln 
\end{center}
\vspace{0.5cm}

\footnotesize

Prinzip: 
\begin{enumerate}
\item Eine Spalte pro Teilformel.
\item Teilformeln ordnen nach Komplexit�t.
\item Eine Zeile pro aussagenlogische Interpretation.

      ($n$ aussagenlogische Variablen $\Rightarrow$ $2^n$ Zeilen.)
\end{enumerate}

Wahrheits--Tafel f�r  $(\neg p \rightarrow q) \rightarrow q$

Teilformeln: $\{ p, q, \neg p, \neg p \rightarrow q, (\neg p \rightarrow q) \rightarrow q \}$
\vspace{0.5cm}

\begin{tabular}{|l|l|l|l|l|l|}
\hline
   $p$ & $q$ & $\neg p$ & $\neg p \rightarrow q$ & $(\neg p \rightarrow q) \rightarrow q$  
   \\
\hline
\hline
   \texttt{true}  & \texttt{true}  & \texttt{false} & \texttt{true}  & \texttt{true}   \\
\hline
   \texttt{true}  & \texttt{false} & \texttt{false} & \texttt{true} & \texttt{false}  \\
\hline
   \texttt{false} & \texttt{true}  & \texttt{true}  & \texttt{true} & \texttt{true}  \\
\hline
   \texttt{false} & \texttt{false} & \texttt{true}  & \texttt{false} & \texttt{true}   \\
\hline
  \end{tabular}


\textbf{Aufgabe}: Wahrheits--Tafel f�r $(p \rightarrow \neg p) \rightarrow \neg p$

\textbf{L�sung}: \\
 Teilformeln $\{p, \neg p, p \rightarrow \neg p, (p \rightarrow \neg p) \rightarrow \neg p\}$

\begin{tabular}{|l|l|l|l|}
\hline
                $p$ & $\neg p$             & $p \rightarrow \neg p$         & $ (p \rightarrow \neg p) \rightarrow \neg p $  
   \\
\hline
\hline
   \texttt{true}  & \texttt{false}  & \texttt{false} & \texttt{true}  \\
\hline
   \texttt{false} & \texttt{true}   & \texttt{true}  & \texttt{true}  \\
\hline
  \end{tabular}


\vspace*{\fill}
\tiny \addtocounter{mypage}{1}
\rule{17cm}{1mm}
Aussagenlogik  \hspace*{\fill} Seite \arabic{mypage}
\end{slide}

%%%%%%%%%%%%%%%%%%%%%%%%%%%%%%%%%%%%%%%%%%%%%%%%%%%%%%%%%%%%%%%%%%%%%%%%

\begin{slide}{}
\normalsize
\begin{center}
Anwendung
\end{center}
\vspace{0.5cm}

\footnotesize
Nach einem Einbruch: drei Verd�chtige \\[0.3cm]
\hspace*{2cm} Anton, Bruno, Claus 
\begin{enumerate}
\item Einer der drei ist schuldig: \\[0.1cm]
      \hspace*{1.3cm} 
      $f_1 := a \vee b \vee c$.
\item Wenn Anton schuldig ist, dann hat er genau einen Komplizen. 
      \begin{enumerate}
      \item Wenn Anton schuldig, dann hat er mindestens einen Komplizen: \\[0.1cm]
            \hspace*{1.3cm} $f_2 := a \rightarrow b \vee c$ 
      \item Wenn Anton schuldig ist, dann hat er h�ch\-stens einen Komplizen: \\[0.1cm]
           \hspace*{1.3cm} $f_3 := a \rightarrow \neg (b \wedge c)$
      \end{enumerate}
\item Wenn Bruno unschuldig ist, dann Claus auch: \\[0.1cm]
      \hspace*{1.3cm} $f_4 :=  \neg b \rightarrow \neg c$ 
\item Wenn genau zwei schuldig, dann Claus schuldig. 
      \\[0.1cm]
      \hspace*{1.3cm} $f_5 := \neg ( \neg c  \wedge a \wedge b )$ 
\item Wenn Claus unschuldig ist, ist Anton schuldig. \\[0.1cm]
      \hspace*{1.3cm} $f_6 := \neg c \rightarrow a$
\end{enumerate}

\vspace*{\fill}
\tiny \addtocounter{mypage}{1}
\rule{17cm}{1mm}
Aussagenlogik  \hspace*{\fill} Seite \arabic{mypage}
\end{slide}

%%%%%%%%%%%%%%%%%%%%%%%%%%%%%%%%%%%%%%%%%%%%%%%%%%%%%%%%%%%%%%%%%%%%%%%%


\begin{slide}{}
\normalsize
\begin{center}
�quivalenzen
\end{center}
\vspace{0.5cm}

\footnotesize
\begin{enumerate}
\item $\models \neg \bot \leftrightarrow \top$ \quad und \quad $\models \neg \top \leftrightarrow \bot$
\item Tertium--non--Datur \\[0.3cm]
      \hspace*{1.3cm} $\models p \vee   \neg p \leftrightarrow \top$ \\[0.3cm]
      \hspace*{1.3cm} $\models p \wedge \neg p \leftrightarrow \bot$ 
\item Neutrales Element \\[0.3cm]
      \hspace*{1.3cm} $\models p \vee   \bot \leftrightarrow p$ \\[0.3cm]
      \hspace*{1.3cm} $\models p \wedge \top  \leftrightarrow p$ \\[0.3cm]
      \hspace*{1.3cm} $\models p \vee   \top  \leftrightarrow \top$ \\[0.3cm]
      \hspace*{1.3cm} $\models p \wedge \bot \leftrightarrow \bot$ 
\item Idempotenz \\[0.3cm]
      \hspace*{1.3cm} $\models p \wedge p \leftrightarrow p$ \\[0.3cm]
      \hspace*{1.3cm} $\models p \vee p \leftrightarrow p$ 
\item Kommutativit�t \\[0.3cm]
      \hspace*{1.3cm} $\models p \wedge q \leftrightarrow q \wedge p$ \\[0.3cm]
      \hspace*{1.3cm} $\models p \vee   q \leftrightarrow q \vee p$ 
\item Assoziativit�t \\[0.3cm]
      \hspace*{1.3cm} $\models (p \wedge q) \wedge r \leftrightarrow p \wedge (q \wedge r) $ \\[0.3cm]
      \hspace*{1.3cm} $\models (p \vee   q) \wedge r \leftrightarrow p \vee   (q \wedge r) $ 
\end{enumerate}

\vspace*{\fill}
\tiny \addtocounter{mypage}{1}
\rule{17cm}{1mm}
Aussagenlogik  \hspace*{\fill} Seite \arabic{mypage}
\end{slide}

%%%%%%%%%%%%%%%%%%%%%%%%%%%%%%%%%%%%%%%%%%%%%%%%%%%%%%%%%%%%%%%%%%%%%%%%

\begin{slide}{}
\normalsize
\begin{center}
�quivalenzen
\end{center}
\vspace{0.5cm}

\footnotesize
\begin{enumerate}
\item[7.] Elimination der Doppelnegation \\[0.3cm]
      \hspace*{1.3cm} $\models \neg \neg p \leftrightarrow p$
\item[8.] DeMorgan'sche Regeln \\[0.3cm]
      \hspace*{1.3cm} $\models \neg (p \wedge q) \leftrightarrow  \neg p \vee   \neg q$ \\[0.3cm]
      \hspace*{1.3cm} $\models \neg (p \vee   q) \leftrightarrow  \neg p \wedge \neg q$ 
\item[9.] Absorption \\[0.3cm]
      \hspace*{1.3cm} $\models p \wedge (p \vee q)   \leftrightarrow p$ \\[0.3cm]
      \hspace*{1.3cm} $\models p \vee   (p \wedge q) \leftrightarrow p$ 
\item[10.] Distributivit�t \\[0.3cm]
      \hspace*{1.3cm} $\models p \wedge (q \vee r)   \leftrightarrow (p \wedge q) \vee   (p \wedge r)$ \\[0.3cm]
      \hspace*{1.3cm} $\models p \vee   (q \wedge r) \leftrightarrow (p \vee q)   \wedge (p \vee   r)$ 
\item[11.] Elimination von $\rightarrow$ \\[0.3cm]
      \hspace*{1.3cm} $\models (p \rightarrow q) \leftrightarrow \neg p \vee q$
\item[12.] Elimination von $\leftrightarrow$ \\[0.3cm]
      \hspace*{1.3cm} $\models (p \leftrightarrow q) \leftrightarrow (\neg p \vee q) \wedge (\neg q \vee p)$
\end{enumerate}

\vspace*{\fill}
\tiny \addtocounter{mypage}{1}
\rule{17cm}{1mm}
Aussagenlogik  \hspace*{\fill} Seite \arabic{mypage}
\end{slide}

%%%%%%%%%%%%%%%%%%%%%%%%%%%%%%%%%%%%%%%%%%%%%%%%%%%%%%%%%%%%%%%%%%%%%%%%

\begin{slide}{}
\normalsize
\begin{center}
Konjunktive Normalform
\end{center}
\vspace{0.5cm}

\footnotesize
\textbf{Definition}:
  $f \in \mathcal{F}$  ist \emph{Literal} g.d.w. 
  \begin{itemize}
  \item $f = \top$ oder $f = \bot$, \quad oder
  \item $f = p$ \quad mit $p \in \mathcal{P}$, \quad oder

  \item $f = \neg p$ \quad mit $p \in \mathcal{P}$.
  \end{itemize}
\vspace{0.5cm}

\textbf{Definition}:
  $f \in \mathcal{F}$ ist \emph{Klausel} g.d.w. \\[0.3cm]
\hspace*{1.3cm} $f = L_1 \vee \cdots \vee L_r$ \\[0.3cm]
mit Literalen $L_1$, $\cdots$, $L_r$. 

Mengenschreibweise: \\[0.3cm]
\hspace*{1.3cm} $f = \{ L_1, \cdots, L_r \}$ \quad 
(statt $f = L_1 \vee \cdots \vee L_r$)
\vspace{0.5cm}

\textbf{Definition}: \\
$f \in \mathcal{F}$ ist in \emph{konjunktiver Normalform} (kurz KNF)
g.d.w. \\[0.3cm]
\hspace*{1.3cm} $f = k_1 \wedge \cdots \wedge k_n$, \\[0.3cm]
mit Klauseln $k_i$ f�r $i=1,\cdots,n$.
\vspace{0.5cm}

Mengenschreibweise: \\[0.3cm]
\hspace*{1.3cm} $f = \{ k_1, \cdots, k_n \}$ \quad
(statt $f = k_1 \wedge \cdots \wedge k_n$)
\vspace{0.5cm}


\vspace*{\fill}
\tiny \addtocounter{mypage}{1}
\rule{17cm}{1mm}
Aussagenlogik  \hspace*{\fill} Seite \arabic{mypage}
\end{slide}

%%%%%%%%%%%%%%%%%%%%%%%%%%%%%%%%%%%%%%%%%%%%%%%%%%%%%%%%%%%%%%%%%%%%%%%%

\begin{slide}{}
\normalsize
\begin{center}
�berf�hrung in KNF
\end{center}
\vspace{0.5cm}

\footnotesize
\begin{enumerate}
\item Eliminiere  ``$\leftrightarrow$'' mit  \\[0.3cm]
      \hspace*{1.3cm} $(p \leftrightarrow q) \leftrightarrow (\neg p \vee q) \wedge (\neg q \vee p)$
\item Eliminiere  ``$\rightarrow$'' mit \\[0.3cm]
      \hspace*{1.3cm} $(p \rightarrow q) \leftrightarrow \neg p \vee q$
\item Schiebe ``$\neg$'' nach innen mit 
  \begin{enumerate}
  \item $\neg \bot \leftrightarrow \top$
  \item $\neg \top \leftrightarrow \bot$
  \item $\neg \neg p \leftrightarrow p$
  \item $\neg (p \wedge q) \leftrightarrow  \neg p \vee   \neg q$ 
  \item $\neg (p \vee   q) \leftrightarrow  \neg p \wedge \neg q$ 
  \end{enumerate}
      Ergebnis in \emph{Negations--Normalform}:

      ``$\neg$'' steht nur noch vor Aussage-Variablen.
\item {Ausmultiplizieren} mit \\[0.3cm]
      \hspace*{1.3cm}       $p \vee (q \wedge r) \leftrightarrow (p \vee q) \wedge (p
\vee r)$ 
       \vspace{0.2cm}

      ``$\vee$'' steht nur noch innen.
\item Mengen-Schreibweise:
  \begin{enumerate}
  \item Klausel: Menge von Literalen
  \item Formel: Menge von Klauseln
  \end{enumerate}
\end{enumerate}


\vspace*{\fill}
\tiny \addtocounter{mypage}{1}
\rule{17cm}{1mm}
Aussagenlogik  \hspace*{\fill} Seite \arabic{mypage}
\end{slide}

%%%%%%%%%%%%%%%%%%%%%%%%%%%%%%%%%%%%%%%%%%%%%%%%%%%%%%%%%%%%%%%%%%%%%%%%

\begin{slide}{}
\normalsize
\begin{center}
Beispiel zur �berf�hrung in KNF 
\end{center}
\vspace{0.5cm}

\footnotesize
\[
\begin{array}[t]{cl}
                  & (p \rightarrow q) \rightarrow (\neg p \rightarrow \neg q)       \\[0.3cm]
  \Leftrightarrow & \neg (p \rightarrow q) \vee (\neg p \rightarrow \neg q)        \\[0.3cm]
  \Leftrightarrow & \neg (\neg p \vee q) \vee (\neg \neg p \vee \neg q)            \\[0.3cm]
  \Leftrightarrow & (\neg \neg p \wedge \neg q) \vee (\neg \neg p \vee \neg q)            \\[0.3cm]
  \Leftrightarrow & (p \wedge \neg q) \vee (p \vee \neg q)                         \\[0.3cm]
  \Leftrightarrow & (p \vee (p \vee \neg q)) \wedge (\neg q \vee (p \vee \neg q))  \\[0.3cm]
  \Leftrightarrow & \biggl\{ \{p, p, \neg q\},\, \{\neg q,  p,  \neg q\}\biggr\}   \\[0.3cm]
  \Leftrightarrow & \biggl\{ \{p, \neg q\},\, \{\neg q,  p \} \biggr\}             \\[0.3cm]
  \Leftrightarrow & \biggl\{ \{p, \neg q\} \biggr\}
\end{array}
\]

\vspace*{\fill}
\tiny \addtocounter{mypage}{1}
\rule{17cm}{1mm}
Aussagenlogik  \hspace*{\fill} Seite \arabic{mypage}
\end{slide}


\end{document}

%%% Local Variables: 
%%% mode: latex
%%% TeX-master: t
%%% End: 
