%%%%%%%%%%%%%%%%%%%%%%%%%%%%%%%%%%%%%%%%%%%%%%%%%%%%%%%%%%%%%%%%%%%%%%%%

\begin{slide}{}
\normalsize
\begin{center}
Informatik I 
\end{center}
\vspace{0.5cm}

\footnotesize

"Uberblick: 
\begin{enumerate}
\item Motivation: Warum formale Modelle?
\item Sprache der Pr"adikaten--Logik
\item Naive Mengenlehre / Algebraische Strukturen
\item Formale Logik (Semantik, Beweise)
  \begin{enumerate}
        \item Aussagen--Logik

              Entscheidbarkeit
        \item Pr"adikaten--Logik

              Semi--Entscheidbarkeit
  \end{enumerate}
\item Algorithmen--Theorie
  \begin{enumerate}
  \item Berechenbarkeit

        Unentscheidbarkeit des Halte--Problems
  \item Korrektheit 

        Programm--Verifikation
  \item Komplexit"at 

        Resourcen--Verbrauch von Algorithmen \\
        $O$--Notation
  \end{enumerate}
\item Logische Programmierung (Prolog)
\end{enumerate}

\setcounter{page}{1}
\vspace*{\fill}
\tiny \addtocounter{mypage}{1}
\rule{15cm}{1mm}
Informatik I  \hspace*{\fill} Seite \arabic{mypage}
\end{slide}

%%%%%%%%%%%%%%%%%%%%%%%%%%%%%%%%%%%%%%%%%%%%%%%%%%%%%%%%%%%%%%%%%%%%%%%%

\begin{slide}{}
\normalsize
\begin{center}
Motivation: Warum formale Modelle?
\end{center}
\vspace{0.5cm}

\footnotesize
Analogie: Br"uckenbau $\leftrightarrow$ Software

\begin{tabular}[c]{|l|l|}
\hline
Architekt / Statiker  \rule{0pt}{22pt}  &    Informatiker  \\[0.3cm]
\hline
Br"ucke    \rule{0pt}{22pt}  &    IT--System  \\[0.3cm]
\hline
Maurer / Schwei\3er   \rule{0pt}{22pt}  &    Programmierer \\[0.3cm]
\hline
Physik / Mathematik / Statik \rule{0pt}{22pt}  &  Informatik / Logik \\[0.3cm]
\hline
Modell / Bauplan  \rule{0pt}{22pt}  &  Modell \\[0.3cm]
\hline
\end{tabular}
\vspace*{0.3cm}

Katastrophale Software-- und Hardware--Fehler
\begin{enumerate}
\item Ariane 5: \quad (9. Juni 1996)
      
      Fehler in der Kontroll--Software \\
      34 Sekunden nach Start wird Selbstzerst"orung aktiviert \\
      4 Satelliten verloren
\item Therac 25: \quad medizinisches Bestrahlungs--Ger"at

      Statt 200 {\tt rad} Dosis von 25.000 {\tt rad} \\
      Resultat: 4 Tote, mehrere Verletzte

\item 9. November 1979:  Roter Alarm im Pentagon.
      
\item Fehler im Pentium Chip \\
      ca. 400 000 000 Dollar Schaden f"ur Intel
\end{enumerate}


\vspace*{\fill}
\tiny \addtocounter{mypage}{1}
\rule{15cm}{1mm}
Informatik I  \hspace*{\fill} Seite \arabic{mypage}
\end{slide}

%%%%%%%%%%%%%%%%%%%%%%%%%%%%%%%%%%%%%%%%%%%%%%%%%%%%%%%%%%%%%%%%%%%%%%%%

\begin{slide}{}
\normalsize
\begin{center}
Warum Formeln?
\end{center}
\vspace{0.5cm}

\footnotesize
Umgangssprachliche Beschreibung: 
\begin{center}
\begin{minipage}[c]{12cm}
{\em 
  Addieren wir zwei Zahlen und bilden dann das Quadrat dieser Summe, so ist das Ergebnis das
  selbe, wie wenn wir zun"achst beide Zahlen einzeln quadrieren und dann das Produkt der beiden
  Zahlen zweifach hinzu addieren.
}
\end{minipage}
\end{center}
\vspace*{0.5cm}

\begin{tabular}[c]{ll}
Frage:   & Was wird ausgedr"uckt? \\[0.5cm]
Antwort: & 1. Binomischer Lehrsatz!
\end{tabular}
$$  (x + y )^2 = x^2 + y^2 + 2*x*y $$

Formalisierung hat drei Vorteile:
\begin{enumerate}
\item Bessere Verst"andlichkeit.
\item Formeln sind manipulierbar:
  \begin{enumerate}
  \item Wir k"onnen mit Formeln \underline{rechnen}.
  \item Formeln im Rechner darstellbar.

        Automatisches Beweisen m"oglich!
  \end{enumerate}
\item Klare Semantik.
\end{enumerate}

\vspace*{\fill}
\tiny \addtocounter{mypage}{1}
\rule{15cm}{1mm}
Informatik I  \hspace*{\fill} Seite \arabic{mypage}
\end{slide}

%%%%%%%%%%%%%%%%%%%%%%%%%%%%%%%%%%%%%%%%%%%%%%%%%%%%%%%%%%%%%%%%%%%%%%%%

\begin{slide}{}
\normalsize
\begin{center}
Formeln als Kurzschreibweise
\end{center}
\vspace{0.5cm}

\footnotesize
All--Quantor: \\[0.3cm]
\hspace*{1.3cm} F"ur alle ganzen Zahlen $z$ gilt: $z^2 \geq 0$ \\[0.3cm]
\hspace*{1.3cm} $\forall z \in \mathbb{Z} : z^2 \geq 0$
\vspace{0.5cm}

Existenz--Quantor: \\[0.3cm]
\hspace*{1.3cm} F"ur alle  $u, v \in \mathbb{R}$ gibt es $w$ aus $\mathbb{R}$ mit 
                $u + w = v$. \\[0.3cm]
\hspace*{1.3cm} $\forall u,v \in \mathbb{Z}:  \exists w \in \mathbb{Z} : u + w = v$
\vspace{0.5cm}

Implikation: \\[0.3cm]
\hspace*{1.3cm} F"ur alle $u,v \in \mathbb{N}$: wenn $u < v$ dann $u^2 < v^2$. \\[0.3cm]
\hspace*{1.3cm} $\forall u, v \in \mathbb{N}: u < v \rightarrow u^2 < v^2$
\vspace{0.5cm}

Konjunktion: \\[0.3cm]
\hspace*{1.3cm} $2 < 7$ und $7 < 10$ \\[0.3cm] 
\hspace*{1.3cm} $2 < 7 \wedge 7 < 10$
\vspace{0.5cm}

Disjunktion: \\[0.3cm]
\hspace*{1.3cm} F"ur alle $u,v \in \mathbb{N}$ gilt $u \leq v$ oder $v \leq u$. \\[0.3cm] 
\hspace*{1.3cm} $\forall u,v \in \mathbb{N}: u \leq v \vee v \leq u$.
\vspace{0.5cm}

Negation: \\[0.3cm]
\hspace*{1.3cm} F"ur alle $q \in \mathbb{Q}$ ist $q^2$ von 2 verschieden.\\[0.3cm]
\hspace*{1.3cm} $\forall q \in \mathbb{Q}: \neg\; x^2 = 2$.

\vspace*{\fill}
\tiny \addtocounter{mypage}{1}
\rule{15cm}{1mm}
Informatik I  \hspace*{\fill} Seite \arabic{mypage}
\end{slide}

%%%%%%%%%%%%%%%%%%%%%%%%%%%%%%%%%%%%%%%%%%%%%%%%%%%%%%%%%%%%%%%%%%%%%%%%

\begin{slide}{}
\normalsize
\begin{center}
Formeln als Kurzschreibweise
\end{center}
\vspace{0.5cm}

\footnotesize
\begin{center}
Tabelle der Quantoren und Junktoren

\begin{tabular}[c]{|l|l|l|}
\hline
            & Zeichen    &  Bedeutung  \\
\hline
\cline{2-3}  Quantoren   & $\forall$ & f"ur alle \\
\cline{2-3}              & $\exists$ & es  gibt \\
\hline
\cline{2-3} Junktoren    & $\rightarrow$ & wenn, dann \\
\cline{2-3}              & $\wedge$  & und \\
\cline{2-3}              & $\vee$    & oder \\
\cline{2-3}              & $\neg$    & nicht \\
\cline{2-3}              & $\leftrightarrow$    & genau dann, wenn \\
\hline
\end{tabular}
  
\end{center}
\vspace{1.0cm}

Aufgabe: Formalisieren sie die folgende Aussage.
\begin{center}
\framebox{\em 
\begin{minipage}{12cm}
F"ur alle nat"urlichen Zahlen $x$ und $y$ gilt: Wenn $x > 0$ ist, dann gibt es
nat"urliche Zahlen $q$ und $r$, so da\3   $y * q + r = x$ ist und au\3erdem  $r < y$ ist.
\end{minipage}  
}
\end{center}
\vspace{1.0cm}

L"osung: 
$$ \forall x, y \in \mathbb{N}: x > 0 \rightarrow \exists q, r \in \mathbb{N}: 
   y * q + r = x \wedge r < y $$

\vspace*{\fill}
\tiny \addtocounter{mypage}{1}
\rule{15cm}{1mm}
Informatik I  \hspace*{\fill} Seite \arabic{mypage}
\end{slide}

%%%%%%%%%%%%%%%%%%%%%%%%%%%%%%%%%%%%%%%%%%%%%%%%%%%%%%%%%%%%%%%%%%%%%%%%

\begin{slide}{}
\normalsize
\begin{center}
0--stellige Pr"adikate (Aussagen)
\end{center}
\vspace{0.5cm}

\footnotesize
0--stellige Pr"adikate:
\begin{enumerate}
\item einfache Aussagen "uber Zustand,
\item keine Referenz auf Objekte. 
\end{enumerate}
Beispiele: 
\begin{enumerate}
\item Es regnet.  

        Formalisierung: {\tt regnet}
\item Sonne scheint. 

        Formalisierung: {\tt Sonne\_scheint}
\item Man sieht einen Regenbogen. 

        Formalisierung: {\tt Regenbogen}
\end{enumerate}
Regel: W"ahle beliebige,  sinnvolle Namen f"ur 0--stellige Pr"adikate.
$$ \mathtt{regnet} \wedge \mathtt{Sonne\_scheint} \rightarrow \mathtt{Regenbogen} $$

\vspace*{\fill}
\tiny \addtocounter{mypage}{1}
\rule{15cm}{1mm}
Informatik I  \hspace*{\fill} Seite \arabic{mypage}
\end{slide}

%%%%%%%%%%%%%%%%%%%%%%%%%%%%%%%%%%%%%%%%%%%%%%%%%%%%%%%%%%%%%%%%%%%%%%%%

\begin{slide}{}
\normalsize
\begin{center}
1--stellige Pr"adikate
\end{center}
\vspace{0.5cm}

\footnotesize
1--stellige Pr"adikate
\begin{enumerate}
\item machen Aussagen "uber  Eigenschaft eines  Objektes,
\item Pr"adikat trifft zu, wenn Objekt Eigenschaft besitzt.
\end{enumerate}

Beispiele:
\begin{enumerate}
\item $2$ ist gerade.

      Formalisierung: $\mathtt{gerade}(2)$.
\item Die Ampel $a$ ist gr"un.

      Formalisierung: $\mathtt{gruen}(a)$.
\end{enumerate}
Regel: W"ahle f"ur vorgegeben Eigenschaft einen sinnvollen Namen $p$.  
Schreibweise: \\[0.3cm]
\hspace*{1.3cm} 
$p(x)$ g.d.w. Eigenschaft $p$ trifft auf $x$ zu.

Beispiel: {\em Wenn die Fu\3g"anger--Ampel gr"un ist, dann ist die 
Autofahrer--Ampel rot.}
$$ \mathtt{gruen}(\mathtt{ampel\_fussgaenger}) \rightarrow 
   \mathtt{rot}(\mathtt{ampel\_auto}) $$

\vspace*{\fill}
\tiny \addtocounter{mypage}{1}
\rule{15cm}{1mm}
Informatik I  \hspace*{\fill} Seite \arabic{mypage}
\end{slide}

%%%%%%%%%%%%%%%%%%%%%%%%%%%%%%%%%%%%%%%%%%%%%%%%%%%%%%%%%%%%%%%%%%%%%%%%

\begin{slide}{}
\normalsize
\begin{center}
2--stellige Pr"adikate
\end{center}
\vspace{0.5cm}

\footnotesize
2--stellige Pr"adikate
\begin{enumerate}
\item machen Aussagen "uber Beziehungen zwischen 2 Objekten,
\item Pr"adikat trifft zu, wenn die beiden Objekte in gegebener Beziehungen stehen.
\end{enumerate}

Beispiele:
\begin{enumerate}
\item $2$ ist kleiner als $5$.

      Formalisierung: $<(2,5)$. 
      
      Oft Infix--Schreibweise:  $2 < 5$.
\item Hans ist Bruder von Petra.

      Formalisierung: $\mathtt{bruder}(\mathtt{Hans}, \mathtt{Petra})$.
\end{enumerate}
Regel: W"ahle f"ur gegebene Beziehung sinnvollen Namen $b$.
Schreibweise: \\[0.3cm]
\hspace*{1.3cm} 
$b(x,y)$ g.d.w. das Objekt $x$ in der Beziehung \\[0.1cm]
\hspace*{1.3cm} $b$ zu $y$ steht.

Beispiel: Wenn $x$ eine Bruder von $y$ ist und $y$ ist eine Bruder von $z$, dann ist auch
$x$ ein Bruder von $z$.

Formalisierung:
$$ \mathtt{bruder}(x, y) \wedge \mathtt{bruder}(y, z) \rightarrow \mathtt{bruder}(x,z) $$

\vspace*{\fill}
\tiny \addtocounter{mypage}{1}
\rule{15cm}{1mm}
Informatik I  \hspace*{\fill} Seite \arabic{mypage}
\end{slide}

%%%%%%%%%%%%%%%%%%%%%%%%%%%%%%%%%%%%%%%%%%%%%%%%%%%%%%%%%%%%%%%%%%%%%%%%

\begin{slide}{}
\normalsize
\begin{center}
$n$--stellige Pr"adikate
\end{center}
\vspace{0.5cm}

\footnotesize
\footnotesize
n--stellige Pr"adikate f"ur $n\in \mathbb{N}$
\begin{enumerate}
\item machen Aussagen "uber Beziehungen zwischen $n$ Objekten,
\item Pr"adikat trifft zu, wenn die $n$ Objekte in gegebener Beziehung stehen.
\end{enumerate}

Beispiele:
\begin{enumerate}
\item $3$ ist gemeinsamer Teiler von $12$ und  $15$.

      Formalisierung: $\mathtt{gemeinsamer\_teiler}(3,12,15)$. 
      
\item Hans und Petra sind die Eltern von Gerhard.

      Formalisierung: $\mathtt{eltern}(\mathtt{Hans}, \mathtt{Petra}, \mathtt{Gerhard})$.
\end{enumerate}
Regel: W"ahle f"ur gegebene Beziehung sinnvollen Namen $b$.
Schreibweise: \\[0.3cm]
\hspace*{1.3cm} 
$b(x_1,x_2, \cdots, x_n)$ g.d.w. die Objekte $x_1$, $\cdots$, $x_n$
\hspace*{1.3cm}  in der Beziehung $b$ stehen.

Beispiel: Wenn $x$ und $y$ die Eltern von $z$ sind und $v$ ein Bruder von $x$ ist, dann
ist $v$ ein Onkel von $z$.

Formalisierung:
$$ \mathtt{eltern}(x, y, z) \wedge \mathtt{bruder}(x, v) \rightarrow \mathtt{onkel}(v, z)$$


\vspace*{\fill}
\tiny \addtocounter{mypage}{1}
\rule{15cm}{1mm}
Informatik I  \hspace*{\fill} Seite \arabic{mypage}
\end{slide}

%%%%%%%%%%%%%%%%%%%%%%%%%%%%%%%%%%%%%%%%%%%%%%%%%%%%%%%%%%%%%%%%%%%%%%%%

\begin{slide}{}
\normalsize
\begin{center}
Aufgaben
\end{center}
\vspace{0.5cm}

\footnotesize
1. Formalisieren folgende Aussage:

Wenn $x$ der Vater von $y$ ist und $y$ die Mutter von $z$ ist, dann ist $x$ ein
      Gro\3vater von $z$.

L"osung: 
$$ \mathtt{vater}(x, y) \wedge \mathtt{mutter}(y, z) \rightarrow \mathtt{grossvater}(x,z) $$


2. Geben Sie eine vollst"andige Charakterisierung des Pr"adikats Gro\3vater.

L"osung: \\[0.3cm]
\hspace*{0.3cm} $\mathtt{grossvater}(x,z) \quad \leftrightarrow$ \\
\hspace*{1.3cm} $\left( \rule{0pt}{18pt}\exists y \in \mathtt{Person}: (\mathtt{vater}(x,
  y) \wedge \mathtt{mutter}(y, z)) \quad \vee \right.$ \\
\hspace*{1.6cm} $\left. \rule{0pt}{18pt} \exists y \in \mathtt{Person}: (\mathtt{vater}(x,
  y) \wedge \mathtt{vater}(y, z)) \quad\quad
\right) $

3. Geben Sie eine vollst"andige Charakterisierung des Pr"adikats \\[0.1cm]
\hspace*{1.3cm} {\em $x$ ist gr"o\3ter gemeinsamer Teiler von $y$ und $z$}.

L"osung: 
\begin{enumerate}
\item $\mathtt{teiler}(x, y) \leftrightarrow \exists z \in \mathbb{N}: x * z = y$
\item $\mathtt{gt}(x, y, z) \leftrightarrow  \left(
       \rule{0pt}{16pt}  \mathtt{teiler}(x,y) \wedge \mathtt{teiler}(x,z) \right)$
\item $\mathtt{ggt}(x, y, z) \leftrightarrow $ \\[0.1cm]
\hspace*{1.3cm} $\mathtt{gt}(x, y, z) \quad \wedge$  \\
\hspace*{1.3cm} $\forall u \in \mathbb{N}: \mathtt{gt}(u,y,z) \rightarrow u \leq x$
\end{enumerate}


\vspace*{\fill}
\tiny \addtocounter{mypage}{1}
\rule{15cm}{1mm}
Informatik I  \hspace*{\fill} Seite \arabic{mypage}
\end{slide}


%%%%%%%%%%%%%%%%%%%%%%%%%%%%%%%%%%%%%%%%%%%%%%%%%%%%%%%%%%%%%%%%%%%%%%%%

\begin{slide}{}
\normalsize
\begin{center}
Terme (Bezeichnung von Objekten)
\end{center}
\vspace{0.5cm}

\footnotesize
Konstanten: (O--stellige Funktionszeichen) 
\vspace{0.5cm}

Beispiele: {\tt 0}, \texttt{Hans}, \texttt{ampel\_auto}
\vspace{0.5cm}

1--stellige Funktionszeichen
\begin{enumerate}
\item $\mathtt{square}: \mathbb{N} \rightarrow \mathbb{N}$

      $\mathtt{square}(3)$
\item $\mathtt{vater}: \mathtt{Person} \rightarrow \mathtt{Person}$

      $\mathtt{vater}(\mathtt{Hans})$
\end{enumerate}

2--stellige Funktionszeichen
\begin{enumerate}
\item $\mathtt{kinder}: \mathtt{Maenner} \times \mathtt{Frauen} \rightarrow 2^\mathtt{Person}$

      $\mathtt{kinder}(\mathtt{Hans}, \mathtt{Ute})$
\item $+: \mathbb{N} \times \mathbb{N} \rightarrow \mathbb{N}$
      
      $+(4, 5)$ \\
         wird meist in Infix--Notation geschrieben: $4 + 5$
\item $\mathtt{Preis}: \mathtt{Ware} \times \mathbb{R} \rightarrow \mathbb{N}$
      
      $\mathtt{Preis}( \mathtt{Spinat}, 0.6 )$
\end{enumerate}

\vspace*{\fill}
\tiny \addtocounter{mypage}{1}
\rule{15cm}{1mm}
Informatik I  \hspace*{\fill} Seite \arabic{mypage}
\end{slide}

%%%%%%%%%%%%%%%%%%%%%%%%%%%%%%%%%%%%%%%%%%%%%%%%%%%%%%%%%%%%%%%%%%%%%%%%

\begin{slide}{}
\normalsize
\begin{center}
Signaturen
\end{center}
\vspace{0.5cm}

\footnotesize
Sinnlose Funktions--Anwendung:
\begin{enumerate}
\item $\mathtt{vater}(5)$
\item $\mathtt{square}(\mathtt{Hans})$
\item $\mathtt{kinder}(\mathtt{Hans}, 3)$
\end{enumerate}
Problem: Typen stimmen nicht.

L"osung: \emph{Signatur} beschreibt Definitions-- und Wertebereich einer Funktion.
Beispiele:
\begin{enumerate}
\item $5: \mathbb{N}$
\item $\mathtt{Hans}: \mathtt{Person}$
\item $\mathtt{square}: \mathbb{N} \rightarrow \mathbb{N}$
\item $\mathtt{vater}: \mathtt{Person} \rightarrow \mathtt{Person}$
\item $\mathtt{kinder}: \mathtt{Maenner} \times \mathtt{Frauen} \rightarrow 2^\mathtt{Person}$
\item $+: \mathbb{N} \times \mathbb{N} \rightarrow \mathbb{N}$
\item $\mathtt{Preis}: \mathtt{Ware} \times \mathbb{R} \rightarrow \mathbb{N}$
\end{enumerate}

\vspace*{\fill}
\tiny \addtocounter{mypage}{1}
\rule{15cm}{1mm}
Informatik I  \hspace*{\fill} Seite \arabic{mypage}
\end{slide}

%%%%%%%%%%%%%%%%%%%%%%%%%%%%%%%%%%%%%%%%%%%%%%%%%%%%%%%%%%%%%%%%%%%%%%%%

\begin{slide}{}
\normalsize
\begin{center}
Signaturen (Formale Definition)
\end{center}
\vspace{0.5cm}

\footnotesize
Gegeben: Menge von Typ--Bezeichnern $\mathbb{T}$.

Z.B. \quad $\mathbb{T} = \{ \mathbb{N}, \mathtt{Person}, \mathbb{R}, \mathtt{Ware}, \cdots \}$

{\bf Definition}: Eine {\em Signatur}  hat die Form \\[0.3cm]
\hspace*{1.3cm} $\sigma_1 \times \sigma_2 \times \cdots \times \sigma_n \rightarrow \tau$. \\[0.3cm]
Dabei gilt:
\begin{enumerate}
\item F"ur alle $i=1,\cdots,n$ ist $\sigma_i$ ein Typ--Bezeichner.

      $\sigma_i$ beschreibt den Typ des $i$--ten Arguments.
\item $\tau$ ist ein Typ--Bezeichner.

      $\tau$ beschreibt den Typ des Ergebnisses.
\end{enumerate}
\begin{center}

\fbox{\fbox{ $\Sigma$: Menge der Signaturen \ }}
  

\end{center}

\vspace*{\fill}
\tiny \addtocounter{mypage}{1}
\rule{15cm}{1mm}
Informatik I  \hspace*{\fill} Seite \arabic{mypage}
\end{slide}

%%%%%%%%%%%%%%%%%%%%%%%%%%%%%%%%%%%%%%%%%%%%%%%%%%%%%%%%%%%%%%%%%%%%%%%%%%%%%%%%

\begin{slide}{}
\normalsize
\begin{center}
Terme (Formale Definition)
\end{center}
\vspace{0.5cm}

\footnotesize
Gegeben: 
\begin{enumerate}
\item ${\cal F}$: \quad Menge von Funktionszeichen.
\item $\mathtt{sign}: {\cal F} \rightarrow \Sigma$ 

      ordnet jedem Funktions--Zeichen $f \in {\cal F}$ eine Signatur zu.
      Statt \\[0.3cm]
      \hspace*{1.3cm} $\mathtt{sign}(f) = \sigma_1 \times \cdots \times \sigma_n \rightarrow \tau$ \\[0.3cm]
      schreiben wir \\[0.3cm]
      \hspace*{1.3cm} $f: \sigma_1 \times \cdots \times \sigma_n \rightarrow \tau$.
\item ${\cal V}_\tau$: \quad F"ur jeden Typ--Bezeichner $\tau \in \mathbb{T}$ ist 
      ${\cal V}_\tau$ eine Menge von Variablen.
\end{enumerate}
Induktive Definition der Terme  ${\cal T}_\tau$ vom Typ $\tau$:
\begin{enumerate}
\item F"ur jede Variable $x \in {\cal V}_\tau$ gilt:\\[0.1cm]
      \hspace*{1.3cm} $x \in {\cal T}_\tau$.
\item Falls
  \begin{enumerate}
  \item $f \in {\cal F}$ mit $f: \sigma_1 \times \cdots \times \sigma_n \rightarrow \tau$
  \item $t_i \in {\cal T}_{\sigma_i}$ f"ur alle $i = 1, \cdots, n$
  \end{enumerate}
   dann gilt: \\[0.1cm]
   \hspace*{1.3cm} $f(t_1,\cdots, t_n) \in {\cal T}_{\tau}$.
\end{enumerate}

\vspace*{\fill}
\tiny \addtocounter{mypage}{1}
\rule{15cm}{1mm}
Informatik I  \hspace*{\fill} Seite \arabic{mypage}
\end{slide}

%%%%%%%%%%%%%%%%%%%%%%%%%%%%%%%%%%%%%%%%%%%%%%%%%%%%%%%%%%%%%%%%%%%%%%%%

\begin{slide}{}
\normalsize
\begin{center}
Terme (Beispiele)
\end{center}
\vspace{0.5cm}

\footnotesize
Gegeben seien folgende Funktions--Zeichen mit\\ Signaturen:
\begin{enumerate}
\item $5: \mathbb{N}$
\item $1.0: \mathbb{R}$
\item $\mathtt{Hans}: \mathtt{Person}$
\item $\mathtt{Kartoffeln}: \mathtt{Ware}$
\item $\mathtt{square}: \mathbb{N} \rightarrow \mathbb{N}$
\item $\mathtt{vater}: \mathtt{Person} \rightarrow \mathtt{Maenner}$
\item $\mathtt{kinder}: \mathtt{Maenner} \times \mathtt{Frauen} \rightarrow 2^\mathtt{Person}$
\item $+: \mathbb{N} \times \mathbb{N} \rightarrow \mathbb{N}$
\item $\mathtt{Preis}: \mathtt{Ware} \times \mathbb{R} \rightarrow \mathbb{N}$
\end{enumerate}

Ferner sei $\mathcal{V} = \{x, y, z\}$ eine Menge von Variablen.

Beispiele f"ur Terme:
\begin{enumerate}
\item $\mathtt{square}(+(x,5))$ \quad Term vom Typ $\mathbb{N}$
\item $\mathtt{kinder}(\mathtt{vater}(\mathtt{Hans}), y)$ \quad Term vom Typ $2^\mathtt{Person}$
\item $\mathtt{Preis}(\mathtt{Kartoffeln}, 1.0)$ \quad Term vom Typ $\mathbb{N}$
\end{enumerate}

\vspace*{\fill}
\tiny \addtocounter{mypage}{1}
\rule{15cm}{1mm}
Informatik I  \hspace*{\fill} Seite \arabic{mypage}
\end{slide}

%%%%%%%%%%%%%%%%%%%%%%%%%%%%%%%%%%%%%%%%%%%%%%%%%%%%%%%%%%%%%%%%%%%%%%%%

\begin{slide}{}
\normalsize
\begin{center}
Atomare Formeln
\end{center}
\vspace{0.5cm}

\footnotesize
Gegeben:
\begin{enumerate}
\item ${\cal T}$: \quad Menge der Terme
\item ${\cal P}$: \quad Menge von Pr"adikatszeichen
\item $\mathtt{sign}: {\cal P} \rightarrow \Sigma$ \\[0.3cm]
      ordnet jedem Pr"adikatszeichen eine Signatur der Form \\[0.1cm]
      \hspace*{1.3cm} $\sigma_1 \times \cdots \times \sigma_n \rightarrow \mathbb{B}$ \\[0.3cm]
      zu.  Hierbei ist $\mathbb{B}$ der Typ--Bezeichner f"ur die Menge $\{ \mathtt{true}, \mathtt{false} \}$
      der Wahrheitswerte.
      
\end{enumerate}
Definition der atomaren Formeln ${\cal A}$:  Falls
\begin{enumerate}
\item $p\in {\cal P}$ mit $p: \sigma_1 \times \cdots \times \sigma_n \rightarrow \mathbb{B}$ und
\item $t_i \in {\cal T}_{\sigma_i}$ f"ur alle $i=1, \cdots, n$,
\end{enumerate}
dann gilt: \\[0.1cm]
\hspace*{1.3cm} $p(t_1, \cdots, t_n) \in{\cal A}$.

Notation: Zweistellige Pr"adikate ``$=$'', ``$\leq$'', etc. werden {\em Infix} geschrieben.  
Beispiel: statt ``$=(t_1, t_2)$'' schreiben wir ``$t_1 = t_2$''.

Beispiele:
\begin{enumerate}
\item $\mathtt{square}(\mathtt{plus}(x,y)) = z$
\item $\mathtt{grossvater}(\mathtt{vater}(\mathtt{mutter}(x)), x)$
\end{enumerate}


\vspace*{\fill}
\tiny \addtocounter{mypage}{1}
\rule{15cm}{1mm}
Informatik I  \hspace*{\fill} Seite \arabic{mypage}
\end{slide}

%%%%%%%%%%%%%%%%%%%%%%%%%%%%%%%%%%%%%%%%%%%%%%%%%%%%%%%%%%%%%%%%%%%%%%%%

\begin{slide}{}
\normalsize
\begin{center}
Pr"adikatenlogische Formeln
\end{center}
\vspace{0.5cm}

\footnotesize
Gegeben: 
\begin{enumerate}
\item Atomare Formeln "uber Variablen ${\cal V}_\tau$.
\item Menge $\mathbb{T}$ von Typ--Bezeichnern.

      Z.B.: $\mathbb{T} = \{ \mathbb{N}, \mathtt{Person}, \cdots \}$
\end{enumerate}

Definition der pr"adikatenlogischen Formeln $\mathbb{F}$.
\begin{enumerate}
\item Falls $f \in {\cal A}$, dann $f \in \mathbb{F}$.
\item Falls $f \in \mathbb{F}$, dann $\neg f \in \mathbb{F}$.
\item Falls $f_1, f_2 \in \mathbb{F}$, dann
  \begin{enumerate}
  \item $(f_1 \wedge f_2) \in \mathbb{F}$,
  \item $(f_1 \vee f_2) \in \mathbb{F}$,
  \item $(f_1 \rightarrow f_2) \in \mathbb{F}$,
  \item $(f_1 \leftrightarrow f_2) \in \mathbb{F}$.
  \end{enumerate}
\item Falls $x \in {\cal V}_\tau$, $f \in \mathbb{F}$, und $\tau \in \mathbb{T}$, dann 
  \begin{enumerate}
  \item $(\forall x \in \tau: f) \in \mathbb{F}$,
  \item $(\exists x \in \tau: f) \in \mathbb{F}$.
  \end{enumerate}
\end{enumerate}


\vspace*{\fill}
\tiny \addtocounter{mypage}{1}
\rule{15cm}{1mm}
Informatik I  \hspace*{\fill} Seite \arabic{mypage}
\end{slide}

%%%%%%%%%%%%%%%%%%%%%%%%%%%%%%%%%%%%%%%%%%%%%%%%%%%%%%%%%%%%%%%%%%%%%%%%

\begin{slide}{}
\normalsize
\begin{center}
Schreibweise von Formeln
\end{center}
\vspace{0.5cm}

\footnotesize
Wir vereinbaren Vereinfachungen zur Schreibweise:
\begin{enumerate}
\item "Au\3ere Klammern werden weggelassen.
\item $\wedge$ und $\vee$ sind links--assoziativ:

      Also: statt 
      $$        ((p \wedge q) \wedge r)    $$
      schreiben wir 
      $$        p \wedge q \wedge r        $$
\item $\wedge$ und $\vee$ binden st"arker als $\rightarrow$.
\item $\rightarrow$ bindet st"arker als $\leftrightarrow$.

      Also: statt
      $$  (p \wedge q) \rightarrow (u \vee v)  $$
      schreiben wir
      $$  p \wedge q \rightarrow u \vee v  $$
      Statt
      $$  (p \rightarrow q) \leftrightarrow (\neg p \vee q)  $$
      schreiben wir     
      $$  p \rightarrow q \quad \leftrightarrow \quad \neg p \vee q  $$
\item Gleiche Quantoren werden zusammengefa\3t: Statt
      $$  \forall x \in \mathbb{N}: (\forall y \in \mathbb{N}: f(x) = y) $$
      schreiben wir:
      $$  \forall x, y \in \mathbb{N}: f(x) = y $$
\end{enumerate}

\vspace*{\fill}
\tiny \addtocounter{mypage}{1}
\rule{15cm}{1mm}
Informatik I  \hspace*{\fill} Seite \arabic{mypage}
\end{slide}

%%%%%%%%%%%%%%%%%%%%%%%%%%%%%%%%%%%%%%%%%%%%%%%%%%%%%%%%%%%%%%%%%%%%%%%%

\begin{slide}{}
\normalsize
\begin{center}
Aufgaben
\end{center}
\vspace{0.5cm}

\footnotesize
\textbf{Aufgabe 1}: Definieren Sie das Pr"adikat\\[0.3cm]
\hspace*{1.3cm} {\em $x$ ist das kleinste gemeinsame Vielfache \\
\hspace*{1.3cm}  der Zahlen $y$ und $z$.} \\[0.3cm]
Hinweis: Sie d"urfen das Pr"adikat $\mathtt{teiler}(x, y)$ als gegeben voraussetzen.

L"osung:
\begin{enumerate}
\item definiere gemeinsames Vielfaches:

      $\forall x,y \in \mathbb{N}: \mathtt{gv}(x, y, z) \leftrightarrow \left( \mathtt{teiler}(y, x) \wedge \mathtt{teiler}(z, x) \right)$
\item definiere kleinstes gemeinsames Vielfaches:

      $\forall x,y,z \in \mathbb{N}: \mathtt{kgv}(x,y,z) \leftrightarrow$ \\[0.1cm]
\hspace*{1.3cm} $\mathtt{gv}(x, y, z) \wedge (\forall u \in \mathbb{N}: \mathtt{gv}(u, y, z) \rightarrow x \leq u)$
\end{enumerate}

\textbf{Aufgabe 2}: Sei $f: \mathbb{N} \rightarrow \mathbb{N}$. Formalisieren Sie:
\begin{enumerate}
\item $f$ is {\em injektiv}, d.h. f"ur jedes $y$ gibt es h"ochstens ein $x$, so da\3
      $f(x) = y$ ist.
\item $f$ is {\em surjektiv}, d.h. f"ur jedes $y$ gibt es mindestens ein $x$, so da\3
      $f(x) = y$ ist.
\end{enumerate}

L"osung:
\begin{enumerate}
\item $\forall x_1, x_2 \in \mathbb{N}: f(x_1) = f(x_2) \rightarrow x_1 = x_2$.
\item $\forall y \in \mathbb{N}: \exists x \in \mathbb{N}: f(x) = y$.
\end{enumerate}

\vspace*{\fill}
\tiny \addtocounter{mypage}{1}
\rule{15cm}{1mm}
Informatik I  \hspace*{\fill} Seite \arabic{mypage}
\end{slide}

%%%%%%%%%%%%%%%%%%%%%%%%%%%%%%%%%%%%%%%%%%%%%%%%%%%%%%%%%%%%%%%%%%%%%%%%%%%%%%%%

\begin{slide}{}
\normalsize
\begin{center}
Naive Mengenlehre
\end{center}
\vspace{0.5cm}

\footnotesize
Frage:   Was sind Mengen? \\
Antwort: Ansammlung verschiedener Objekte. 

Beispiele:
\begin{enumerate}
\item Menge der Zahlen 1, 2 und 5: \quad $\{1, 2, 5 \}$.
\item Menge der Buchstaben:       \quad $\{a, b, c, \cdots, x, y, z \}$.
\item Menge der nat"urlichen Zahlen: $\mathbb{N} = \{ 0, 1, 2, 3, \cdots \}$.
\end{enumerate}
Notation:  Statt \\[0.3cm]
\hspace*{1.3cm} ``Die Zahl $2$ ist eine Element der Menge $\{1, 2, 3\}$'' \\[0.3cm]
schreiben wir:  $2 \in \{ 1, 2, 3 \}$.

\begin{enumerate}
\item Reihenfolge spielt keine Rolle.\\[0.1cm]
      \hspace*{1.3cm} $\{ 1, 2, 3 \} = \{ 3, 2, 1 \}$
\item Eine Menge kann jedes Element h"ochstens einmal enthalten: \\[0.1cm]
\hspace*{1.3cm} $\{ 1, 2, 2, 3 \} = \{ 1, 1, 2, 3, 3 \}$.
\end{enumerate}

\begin{center}
\framebox{\framebox{
  \begin{minipage}[c]{10cm}
{\bf Extensionalit"ats--Prinzip}: \\
Zwei Mengen sind gleich, wenn sie dieselben Elemente enthalten:
  \end{minipage}
}}  
\end{center}
Formalisiert:
$$ M = N \leftrightarrow (\forall x: x \in M \leftrightarrow x \in N) $$

\vspace*{\fill}
\tiny \addtocounter{mypage}{1}
\rule{15cm}{1mm}
Informatik I  \hspace*{\fill} Seite \arabic{mypage}
\end{slide}

%%%%%%%%%%%%%%%%%%%%%%%%%%%%%%%%%%%%%%%%%%%%%%%%%%%%%%%%%%%%%%%%%%%%%%%%

\begin{slide}{}
\normalsize
\begin{center}
Teilmenge 
\end{center}
\vspace{0.5cm}

\footnotesize
Gegeben: Mengen $M_1$ und $M_2$ \\[0.3cm]
\textbf{Definition}: $M_1$ \emph{Teilmenge} von $M_2$ g.d.w. \\[0.3cm]
\hspace*{1.3cm} $\forall x \in M_1: x \in M_2$ \\[0.3cm]
\textbf{Schreibweise}: $M_1 \subseteq M_2$


\normalsize
\begin{center}
Definition von Mengen 
\end{center}
\vspace{0.5cm}

\footnotesize
Problem: nicht alle Mengen sind explizit angebbar.  \\
Beispiel: Menge der geraden Zahlen. \\[0.3cm]
\hspace*{1.3cm} $\{0, 2, 4, 6, \cdots \}$ \\[0.3cm]
``P"unktchen--Schreibweise'' nicht eindeutig.

L"osung:  Operationen zur Bildung von Mengen
\begin{enumerate}
\item Selektion 

      Auswahl einer Menge von Elementen aus einer gegebenen Menge
\item Vereinigung von Mengen
\item Schnitt von Mengen
\item Kartesische Produkte
\item Potenz--Menge
\item Abbildungen von Mengen 
\end{enumerate}


\vspace*{\fill}
\tiny \addtocounter{mypage}{1}
\rule{15cm}{1mm}
Informatik I  \hspace*{\fill} Seite \arabic{mypage}
\end{slide}

%%%%%%%%%%%%%%%%%%%%%%%%%%%%%%%%%%%%%%%%%%%%%%%%%%%%%%%%%%%%%%%%%%%%%%%%

\begin{slide}{}
\normalsize
\begin{center}
Selektion
\end{center}
\vspace{0.5cm}

\footnotesize
Selektion: Beispiel \\[0.3cm]
\hspace*{1.3cm} $\{ x \in \mathbb{N} \is \exists y \in \mathbb{N}: x = 2 * y \}$  \\[0.6cm]
beschreibt die Menge der geraden Zahlen.


Allgemein: Sei
\begin{enumerate}
\item $M$ gegebene Menge
\item $p$ eine Formel, in der Variable $x$ auftritt.
\end{enumerate}
Dann bezeichnet \\[0.3cm]
\hspace*{1.3cm} $\{ x \in M : p \}$ \\[0.3cm]
Menge aller $x$ aus $M$, f"ur die Eigenschaft $p$ zutrifft.

\begin{center}
\normalsize
  Schnitt--Menge
\end{center}
\footnotesize

Gegeben: Zwei Mengen $M_1$ und $M_2$.  \\
Definition: \\[0.3cm]
\hspace*{1.3cm} $M_1 \cap M_2 = \{ x \in M_1 \is x \in M_2 \}$  \\[0.3cm]
hei\3t Schnitt--Menge.

Beispiel:  $\{1, 3, 5, 6, 7 \} \cap \{4, 5, 6 \} = \{ 5, 6 \}$

\vspace*{\fill}
\tiny \addtocounter{mypage}{1}
\rule{15cm}{1mm}
Informatik I  \hspace*{\fill} Seite \arabic{mypage}
\end{slide}

%%%%%%%%%%%%%%%%%%%%%%%%%%%%%%%%%%%%%%%%%%%%%%%%%%%%%%%%%%%%%%%%%%%%%%%%

\begin{slide}{}

\begin{center}
\normalsize
  Vereinigungs--Menge
\end{center}
\footnotesize

Gegeben: Drei Mengen $M_1$, $M_2$ und $N$ mit \\[0.3cm]
\hspace*{1.3cm} $M_1 \subseteq N$ \quad und \quad $M_2 \subseteq N$ \\[0.3cm]
Definition: \\[0.3cm]
\hspace*{1.3cm} $M_1 \cup M_2 = \{ x \in N \is x \in M_1 \vee x \in M_2 \}$  \\[0.3cm]
hei\3t Vereinigungs--Menge.

Beispiel:  $\{1, 3, 5, 6, 7 \} \cup \{4, 5, 6 \} = \{ 1, 3, 4, 5, 6, 7 \}$

\normalsize
\begin{center}
Mengen--Komplement
\end{center}
\vspace{0.5cm}

\footnotesize
Gegeben: Mengen $M_1$ und $M_2$. \\[0.3cm]
Definition: $M_1 \backslash M_2 = \{ x \in M_1 \is x \not\in M_2 \}$

Beispiel:  $\{1, 3, 5, 6, 7 \} \backslash \{4, 5, 6 \} = \{ 1, 3, 7 \}$

\textbf{Satz}: F"ur beliebige Mengen $M$ und $N$ gilt:
\begin{enumerate}
\item $M \cap N \subseteq M$
\item $M \subseteq M \cup N$
\item $M \backslash N \subseteq M$
\item $M \backslash (M \backslash N) = M \cap N$
\item $N \backslash (M \backslash N) = N$
\end{enumerate}

\vspace*{\fill}
\tiny \addtocounter{mypage}{1}
\rule{15cm}{1mm}
Informatik I  \hspace*{\fill} Seite \arabic{mypage}
\end{slide}

%%%%%%%%%%%%%%%%%%%%%%%%%%%%%%%%%%%%%%%%%%%%%%%%%%%%%%%%%%%%%%%%%%%%%%%%

\begin{slide}{}
\normalsize
\begin{center}
    Mengen--Algebra
\end{center}
\footnotesize
\begin{enumerate}
\item Idempotenz:\quad      $M \cup M = M$,     $M \cap M = M$
\item Neutrales Element:    $M \cup \emptyset = M$, $M \cap \emptyset = \emptyset$

\item Kommutativ--Gesetze

        $M_1 \cup M_2 = M_2 \cup M_1$

        $M_1 \cap M_2 = M_2 \cap M_1$
\item Assoziativ--Gesetze

      $(M_1 \cup M_2) \cup M_3 = M_1 \cup (M_2 \cup M_3)$

      $(M_1 \cap M_2) \cap M_3 = M_1 \cap (M_2 \cap M_3)$

\item Distributiv--Gesetze

      $(M_1 \cup M_2) \cap N = (M_1 \cap N) \cup (M_2 \cap N)$

      $(M_1 \cap M_2) \cup N = (M_1 \cup N) \cap (M_2 \cup N)$
\item DeMorgan--Gesetze

      $N \backslash (M_1 \cup M_2) = (N \backslash M_1) \cap (N \backslash M_2)$

      $N \backslash (M_1 \cap M_2) = (N \backslash M_1) \cup (N \backslash M_2)$
\end{enumerate}
\vspace*{\fill}
\tiny \addtocounter{mypage}{1}
\rule{15cm}{1mm}
Informatik I  \hspace*{\fill} Seite \arabic{mypage}
\end{slide}

%%%%%%%%%%%%%%%%%%%%%%%%%%%%%%%%%%%%%%%%%%%%%%%%%%%%%%%%%%%%%%%%%%%%%%%%%%%%%%%%

\begin{slide}{}
\normalsize
\begin{center}
  Aufgaben zur Mengen--Bildung
\end{center}
\vspace{0.5cm}

\footnotesize
\textbf{Aufgabe 1}: Berechnen Sie \\[0.3cm]
\hspace*{1.3cm} $M = \{ x \in \mathbb{N} \is (\exists y \in \mathbb{N}: y^2 = x) \wedge x < 10 \}$.

L"osung: \\[0.3cm]
\hspace*{1.3cm} $M = \{0, 1, 4, 9\}$
\vspace{0.5cm}

\textbf{Aufgabe 2}: Gegeben seien die folgenden Mengen: \\[0.3cm]
\hspace*{1.3cm} $M_1 = \{ x \in \mathbb{N} \is x^2 - 5 * x + 6 = 0 \}$ \\[0.3cm]
\hspace*{1.3cm} $M_2 = \{ x \in \mathbb{N} \is x^3 > 8 \}$ \\[0.3cm]
Berechnen Sie $M_1 \cap M_2$.

L"osung: \\[0.3cm]
\hspace*{1.3cm} $M_1 \cap M_2 = \{ 3 \}$
\vspace{0.5cm}

\textbf{Aufgabe 3}: Geben Sie einen Ausdruck f"ur die Menge $P$ aller Primzahlen an.

L"osung: \\[0.3cm]
\hspace*{0.0cm} $P = \{ x \in \mathbb{N} \is x \geq 2 \;\wedge$ \\
\hspace*{3.3cm} $(\forall u, v \in \mathbb{N}: u > 1 \wedge v > 1 \rightarrow u * v \not= x) \}$

\textbf{Aufgabe 4}: Geben Sie einen Ausdruck f"ur die folgende Menge an: \\[0.3cm]
\hspace*{1.3cm} $M = \{ 2, 4, 5, 6, 8, 10, 12, 14, 15, 16, 18, 20, 22, \cdots \}$

L"osung: \\[0.3cm]
\hspace*{1.3cm} $M = \{ 2, 4, 6, \cdots \} \cup \{ 5, 15, 20, \cdots \} =$ \\[0.3cm]
\hspace*{2.9cm} $\{ x \in \mathbb{N} \is \exists y \in \mathbb{N}: x = 2 * y \} \; \cup$ \\
\hspace*{2.9cm} $\{ x \in \mathbb{N} \is \exists y \in \mathbb{N}: x = 5 * y \}$

\vspace*{\fill}
\tiny \addtocounter{mypage}{1}
\rule{15cm}{1mm}
Informatik I  \hspace*{\fill} Seite \arabic{mypage}
\end{slide}

%%%%%%%%%%%%%%%%%%%%%%%%%%%%%%%%%%%%%%%%%%%%%%%%%%%%%%%%%%%%%%%%%%%%%%%%

\begin{slide}{}
\normalsize
\begin{center}
Tupel (Endliche Folgen)
\end{center}
\vspace{0.5cm}

\footnotesize
Notation: $\langle x_1, x_2, \cdots, x_n \rangle$ hei\3t $n$--Tupel.

Sprechweise:
\begin{enumerate}
\item $n$ ist die L"ange des Tupels.
\item $x_i$ ist die $i$--te Komponente.
\end{enumerate}

Gleichheit: Es gilt \\[0.3cm]
\hspace*{1.3cm} $\langle x_1, x_2, \cdots, x_n \rangle = \langle y_1, y_2, \cdots, y_m \rangle$ \\[0.3cm]
g.d.w.~folgendes gilt:
\begin{enumerate}
\item $n = m$

      Die L"ange der Tupel ist gleich.
\item $\forall i \in \mathbb{N}: i \leq n \rightarrow x_i = y_i$

      Die Komponenten stimmen paarweise "uberein.
\end{enumerate}

\normalsize
\begin{center}
  Kartesisches Produkt
\end{center}

\footnotesize
Gegeben:    Mengen $M_1$ und $M_2$ \\[0.3cm]
Definition: $M_1 \times M_2 = \{ \langle x, y \rangle \is x \in M_1 \wedge y \in M_2 \}$ \\[0.3cm]
Sprechweise: $M_1 \times M_2$ hei\3t {\em kartesisches Produkt} der Mengen $M_1$ und $M_2$.

\vspace*{\fill}
\tiny \addtocounter{mypage}{1}
\rule{15cm}{1mm}
Informatik I  \hspace*{\fill} Seite \arabic{mypage}
\end{slide}

%%%%%%%%%%%%%%%%%%%%%%%%%%%%%%%%%%%%%%%%%%%%%%%%%%%%%%%%%%%%%%%%%%%%%%%%

\begin{slide}{}
\normalsize
\begin{center}
Verallgemeinertes kartesisches Produkt
\end{center}
\vspace{0.5cm}

\footnotesize
Wir identifizieren \\[0.3cm]
\hspace*{1.3cm} $\langle \langle x, y \rangle, z \rangle = \langle x, \langle y, z \rangle \rangle = \langle x, y, z \rangle$

Daher gilt: \\[0.3cm]
\hspace*{1.3cm} $(M_1 \times M_2) \times M_3 = M_1 \times (M_2 \times M_3)$ \\[0.3cm]
\hspace*{6.2cm} $ = M_1 \times M_2 \times M_3$.

Bezeichnung: $M_1 \times M_2 \times \cdots \times M_n$ hei\3t 
\begin{center}
    \fbox{\fbox{verallgemeinertes kartesisches Produkt}}
\end{center}

\normalsize
\begin{center}
    Bin"are Relation
\end{center}
\footnotesize
Gilt $R \subseteq M \times M$, so hei\3t $R$ eine \emph{bin"are Relation} auf $M$.

Beispiele:
\begin{enumerate}
\item $\{ \langle x, x \rangle \in \mathbb{N} \times \mathbb{N} \is x \in \mathbb{N} \}$.
\item $\{ \langle x, y \rangle \in \mathbb{N} \times \mathbb{N} \is x < y \}$.
\item $\{ \langle x, y \rangle \in \mathbb{N} \times \mathbb{N} \is y = x * x \}$.
\end{enumerate}

Schreibweise: \quad $M^2 := M \times M$ \\
Allgemein wird $M^n$ per Induktion nach $n$ definiert:
\begin{enumerate}
\item $M^1 := M$ 
\item $M^{n+1} := M^n \times M$
\end{enumerate}

\vspace*{\fill}
\tiny \addtocounter{mypage}{1}
\rule{15cm}{1mm}
Informatik I  \hspace*{\fill} Seite \arabic{mypage}
\end{slide}

%%%%%%%%%%%%%%%%%%%%%%%%%%%%%%%%%%%%%%%%%%%%%%%%%%%%%%%%%%%%%%%%%%%%%%%%

\begin{slide}{}
\normalsize
\begin{center}
  Eigenschaften bin"arer Relationen
\end{center}

\footnotesize
\textbf{Infix--Schreibweise}: Schreibe $x R y$ statt $\langle x, y \rangle \in R$.

\textbf{Definition}:  $R \subseteq M^2$ ist \emph{reflexiv} g.d.w. \\[0.3cm]
\hspace*{1.3cm} $\forall x \in M: x R x$.

\textbf{Definition}:  $R \subseteq M^2$ ist \emph{symmetrisch} g.d.w. \\[0.3cm]
\hspace*{1.3cm}  $\forall x, y \in M: x R y \rightarrow  y R x$. 

\textbf{Definition}:  $R \subseteq M^2$ ist \emph{asymmetrisch} g.d.w. \\[0.3cm]
\hspace*{1.3cm}  $\forall x, y \in M: x R y \rightarrow \neg y R x$. 

\textbf{Definition}:  $R \subseteq M^2$ ist \emph{transitiv} g.d.w. \\[0.3cm]
\hspace*{0.3cm} $\forall x, y, z \in M: x R y \wedge y R z \rightarrow x R z$. 

\textbf{Definition}: \\
 $R \subseteq M^2$  \emph{Ordnungs--Relation} im Sinne von $\leq$ g.d.w. 
\begin{enumerate}
\item $R$ ist reflexiv.
\item $R$ ist transitiv.
\item $\forall x, y \in M: x R y \wedge y R x \rightarrow x = y$. 
\end{enumerate}

\textbf{Definition}:  $R \subseteq M^2$ ist \emph{"Aquivalenz--Relation} g.d.w. 
\begin{enumerate}
\item $R$ ist reflexiv,
\item $R$ ist symmetrisch und
\item $R$ ist transitiv.
\end{enumerate}

\vspace*{\fill}
\tiny \addtocounter{mypage}{1}
\rule{15cm}{1mm}
Informatik I  \hspace*{\fill} Seite \arabic{mypage}
\end{slide}

%%%%%%%%%%%%%%%%%%%%%%%%%%%%%%%%%%%%%%%%%%%%%%%%%%%%%%%%%%%%%%%%%%%%%%%%

\begin{slide}{}
\normalsize
\begin{center}
Beispiele
\end{center}
\vspace{0.5cm}

\footnotesize
\begin{enumerate}
\item $R_1 = \{ \langle x, y \rangle \in \mathbb{N}^2 \is x \leq y \}$ 

      ist Ordnungs--Relation im       Sinne von $\leq$.
\item $R_2 = \{ \langle x, y \rangle \in \mathbb{N}^2 \is \exists z \in \mathbb{N}: x * z = y \}$ 

      ist Ordnungs--Relation im       Sinne von $\leq$.

\item Sei $f: \mathbb{N} \rightarrow \mathbb{N}$ bel. Funktion.  Dann gilt: \\[0.3cm]
      \hspace*{1.3cm} $R_3 = \{ \langle x, y \rangle \in \mathbb{N}^2 \is f(x) = f(y) \}$ \\[0.3cm]
      ist "Aquivalenz--Relation.
\end{enumerate}

\textbf{Definition}: "Aquivalenz--Klasse \\[0.3cm]
Sei $\sim\; \subseteq M^2$ "Aquivalenz--Relation. F"ur alle $x \in M$ bezeichnet \\[0.3cm]
\hspace*{1.3cm} $[x]_{\sim} = \{ y \in M \is x \sim y \}$ \\[0.3cm]
die von $x$ generierte \emph{"Aquivalenz--Klasse}.

\textbf{Satz}: Es gilt
\begin{enumerate}
\item $\forall x, y \in M: x \sim y \rightarrow [x]_{\sim} = [y]_{\sim}$
\item $\forall x, y \in M: \neg x \sim y \rightarrow [x]_{\sim} \cap [y]_{\sim} = \emptyset$
\end{enumerate}

\vspace*{\fill}
\tiny \addtocounter{mypage}{1}
\rule{15cm}{1mm}
Informatik I  \hspace*{\fill} Seite \arabic{mypage}
\end{slide}

%%%%%%%%%%%%%%%%%%%%%%%%%%%%%%%%%%%%%%%%%%%%%%%%%%%%%%%%%%%%%%%%%%%%%%%%

%\begin{slide}{}
%\normalsize
%\begin{center}
%   Korrektur
%\end{center}
%\vspace*{1.5cm}

%\footnotesize
%\textbf{Definition}:  $R \subseteq M^2$ ist \emph{asymmetrisch} g.d.w. \\[0.3cm]
%\hspace*{1.3cm}  $\forall x, y \in M: x R y \rightarrow \neg y R x$. 

%\textbf{Definition}:  $R \subseteq M^2$ ist \emph{anti--symmetrisch} g.d.w. \\[0.3cm]
%\hspace*{1.3cm}  $\forall x, y \in M: x R y \wedge y R x \rightarrow  x = y$. 
%\vspace*{0.5cm}

%\textbf{Definition}: \\
% $R \subseteq M^2$  \emph{Ordnungs--Relation} im Sinne von $\leq$ g.d.w. 
%\begin{enumerate}
%\item $R$ ist reflexiv.
%\item $R$ ist transitiv.
%\item $R$ ist anti--symmetrisch. 
%\end{enumerate}

%\vspace*{\fill}
%\tiny \addtocounter{mypage}{1}
%\rule{15cm}{1mm}
%Informatik I  \hspace*{\fill} 
%\end{slide}

%%%%%%%%%%%%%%%%%%%%%%%%%%%%%%%%%%%%%%%%%%%%%%%%%%%%%%%%%%%%%%%%%%%%%%%%

\begin{slide}{}
\normalsize
\begin{center}
Aufgaben
\end{center}
\vspace{0.5cm}

\footnotesize
\textbf{Aufgabe 1}: Welche der folgenden Relationen auf $S = \{1, 2, 3\}$ sind "Aquivalenz--Relationen?
\begin{enumerate}
\item[(a)] $R_1 = \{ \langle 1, 1 \rangle,
                     \langle 2, 2 \rangle,
                     \langle 3, 3 \rangle,
                     \langle 1, 2 \rangle,
                     \langle 2, 1 \rangle,
                     \langle 2, 3 \rangle,
                     \langle 3, 2 \rangle \}$
\item[(b)] $R_1 = \{ \langle 2, 2 \rangle,
                     \langle 3, 3 \rangle,
                     \langle 1, 2 \rangle,
                     \langle 2, 1 \rangle,
                     \langle 2, 3 \rangle,
                     \langle 1, 1 \rangle,
                     \langle 1, 3 \rangle,
                     \langle 3, 1 \rangle \}$
\end{enumerate}

\textbf{Aufgabe 2}: Wie sieht die kleinste "Aquivalenz--Relation auf $S = \{1, 2, 3 \}$ aus?

\textbf{Aufgabe 3}: Sei $M$ die Menge aller Menschen.
\begin{enumerate}
\item[(a)] Ist  $R_1 = \{ \langle x, y \rangle \in M^2 \is \mathtt{vater}(x) = \mathtt{vater}(y) \}$ \\[0.3cm]
           "Aquivalenz--Relation?
\item[(b)] Gegeben $x \in M$. Dann bezeichne \\[0.1cm]
           \hspace*{1.3cm} $\mathtt{vorfahre}(x) = \{ y \in M \is y \;\mbox{ist vorfahre von}\; x \}$ \\[0.3cm]
           die Menge aller Vorfahren von $x$. Ist  \\[0.3cm]
           \hspace*{0.5cm} 
           $R_2 = \{ \langle x, y \rangle \in M^2 \is y = x \;\vee\; y \in
           \mathtt{vorfahre}(x) \}$ \\[0.3cm]
           eine Ordnungs--Relation im Sinne von $\leq$?
\end{enumerate}

\textbf{Aufgabe 4}: Ist \\[0.3cm]
\hspace*{1.3cm}  $R = \{ \langle x, y \rangle \in \mathbb{N} \is x \geq y \}$ \\[0.3cm]
eine Ordnungs--Relation im Sinne von $\leq$?


\vspace*{\fill}
\tiny \addtocounter{mypage}{1}
\rule{15cm}{1mm}
Informatik I  \hspace*{\fill} Seite \arabic{mypage}
\end{slide}

%%%%%%%%%%%%%%%%%%%%%%%%%%%%%%%%%%%%%%%%%%%%%%%%%%%%%%%%%%%%%%%%%%%%%%%%%%%%%%%%

%\begin{slide}{}
%\normalsize
%\begin{center}
%Eigenschaften bin"arer Relationen
%\end{center}
%\vspace{0.5cm}

%\footnotesize
%\textbf{Definition}:  $R \subseteq M \times N$ ist \emph{linkseindeutig} g.d.w. \\[0.3cm]
%\hspace*{1.3cm} $\forall x_1, x_2 \in M: \forall y \in N: 
%                 x_1 R y \wedge x_2 R y \rightarrow x_1 = x_2$ 

%\textbf{Definition}:  $R \subseteq M \times N$ ist \emph{rechtseindeutig} g.d.w. \\[0.3cm]
%\hspace*{1.3cm} $\forall x \in M: \forall y_1, y_2 \in N: 
%                 x R y_1 \wedge x R y_2 \rightarrow y_1 = y_2$ 

%\textbf{Definition}:  Sei $R \subseteq M \times N$. \\[0.3cm]
%\hspace*{1.3cm} $R^{-1} := \{ \langle x, y \rangle \in M \times N \is \langle y, x \rangle \in R \}$
%\vspace{0.5cm}

%\normalsize
%\begin{center}
%Relationales Produkt
%\end{center}
%\vspace{0.5cm}

%\footnotesize
%\textbf{Gegeben}:    \quad $R_1\in M_1 \times M_2$, $R_2 \in M_2 \times M_3$ \\[0.3cm]
%\textbf{Definition}: \quad $R_2 \circ R_1 :=$ \\[0.3cm]
%\hspace*{1.3cm}   $\{ \langle x, z \rangle \in M_1 \times M_3 \is \exists y \in M_2: \langle x, y \rangle \in R_1\; \wedge$ \\
%\hspace*{9.7cm}   $ \langle y, z \rangle \in R_2 \quad\}$
%\vspace{0.5cm}

%\textbf{Aufgabe 1}: $R_1 = \{ \langle 1, 2 \rangle, \langle 2, 3 \rangle, \langle 3, 4 \rangle \}$, \\[0.2cm]
%\hspace*{3.5cm}    $R_2 = \{ \langle 1, 1 \rangle, \langle 2, 4 \rangle, \langle 3, 9 \rangle \}$. \\[0.3cm]
%\hspace*{3.5cm}    Berechnen Sie $R_1 \circ R_2$ und $R_2 \circ R_1$.

%\textbf{Aufgabe 2}: Seien $R_1\in M_1 \times M_2$, $R_2 \in M_2 \times M_3$. \\[0.3cm]
%\hspace*{3.5cm}    Zeigen Sie:  $(R_1 \circ R_2)^{-1} = R_2^{-1} \circ R_1^{-1}$

%\vspace*{\fill}
%\tiny \addtocounter{mypage}{1}
%\rule{15cm}{1mm}
%Informatik I  \hspace*{\fill} Seite \arabic{mypage}
%\end{slide}

%%%%%%%%%%%%%%%%%%%%%%%%%%%%%%%%%%%%%%%%%%%%%%%%%%%%%%%%%%%%%%%%%%%%%%%%%%%%%%%%%

%\begin{slide}{}
%\normalsize
%\begin{center}
%Funktionen als Relationen
%\end{center}
%\vspace{0.5cm}

%\footnotesize
%Gegeben: $f: M \rightarrow N$. \\[0.3cm]
%Definiere: $\texttt{Graph}(f) = \{ \langle x, y \rangle \in M \times N \is y = f(x) \}$
%\vspace{0.5cm}

%Gegeben: $R: M \rightarrow N$. \\[0.3cm]
%Definiere: $\texttt{Dom}(R) = \{ x \in M \is \exists y \in N: \langle x, y \rangle \in R \}$
%\vspace{0.5cm}

%Gegeben: $R: M \rightarrow N$. \\[0.3cm]
%Definiere: $\texttt{Bild}(R) = \{ y \in N \is \exists x \in M: \langle x, y \rangle \in R \}$
%\vspace{0.5cm}

%\textbf{Definition}: Sei $R: M \rightarrow N$. \\[0.1cm]
%\hspace*{3.5cm} $R$ ist \emph{total} falls $\mathtt{dom}(R) = M$.
%\vspace{0.5cm}


%\textbf{Satz}: Sei $f: M \rightarrow N$. Dann gilt:
%\begin{enumerate}
%\item $\texttt{Graph}(f)$ ist rechtseindeutig.
%\item $\mathtt{Graph}(f)$ ist total.
%\end{enumerate}

%\textbf{Satz}: Sei $R \subseteq M \times N$ und gelte
%\begin{enumerate}
%\item $R$ rechtseindeutig.
%\item $R$ total.
%\end{enumerate}
%Dann gibt es genau ein $f: M \rightarrow N$ mit \\[0.3cm]
%\hspace*{1.3cm} $R = \mathtt{graph}(f)$.

%\vspace*{\fill}
%\tiny \addtocounter{mypage}{1}
%\rule{15cm}{1mm}
%Informatik I  \hspace*{\fill} Seite \arabic{mypage}
%\end{slide}

%%%%%%%%%%%%%%%%%%%%%%%%%%%%%%%%%%%%%%%%%%%%%%%%%%%%%%%%%%%%%%%%%%%%%%%%%

%\begin{slide}{}
%\normalsize
%\begin{center}
%Komposition von Funktionen
%\end{center}
%\vspace{0.5cm}

%\footnotesize
%\textbf{Bemerkung}: $\langle x, y \rangle \in \mathtt{graph}(f) \leftrightarrow y = f(x)$
%\vspace{0.5cm}

%\textbf{Gegeben}: $f: K \rightarrow M$ und $g: M \rightarrow N$ \\[0.3cm]
%\textbf{Definiere}: $g \circ f:K \rightarrow N$ durch \\[0.3cm]
%\hspace*{1.3cm} $(g \circ f)(x) := g(f(x))$
%\vspace{0.5cm}

%\textbf{Satz}: Sei $f: K \rightarrow M$ und $g: M \rightarrow N$. Dann gilt: \\[0.3cm]
%\hspace*{1.3cm} $\mathtt{graph}(g \circ f) = \mathtt{graph}(g) \circ \mathtt{graph}(f)$

%\textbf{Beweis}: Es gilt \\[0.3cm]
%\hspace*{1.1cm} $\mathtt{graph}(g) \circ \mathtt{graph}(f)$ \\[0.3cm]
%\hspace*{0.3cm} $= \{ \langle x, z \rangle \in K \times N \is \exists y \in M : \langle x, y \rangle \in \mathtt{graph}(f)\; \wedge$ \\[0.3cm]
%\hspace*{8.6cm} $\langle y, z \rangle \in \mathtt{graph}(g) \;\; \}$ \\[0.3cm]
%\hspace*{0.3cm} $= \{ \langle x, z \rangle \in K \times N \is \exists y \in M : y = f(x) \wedge z = g(y) \}$ \\[0.3cm]
%\hspace*{0.3cm} $= \{ \langle x, z \rangle \in K \times N \is z = g(f(x)) \}$ \\[0.3cm]
%\hspace*{0.3cm} $= \{ \langle x, z \rangle \in K \times N \is z = (g \circ f)(x)$ \\[0.3cm]
%\hspace*{0.3cm} $= \mathtt{graph}(g \circ f)$
%\vspace{0.5cm}


%\vspace*{\fill}
%\tiny \addtocounter{mypage}{1}
%\rule{15cm}{1mm}
%Informatik I  \hspace*{\fill} Seite \arabic{mypage}
%\end{slide}

%%%%%%%%%%%%%%%%%%%%%%%%%%%%%%%%%%%%%%%%%%%%%%%%%%%%%%%%%%%%%%%%%%%%%%%%%%%%%%%%%

%\begin{slide}{}
%\normalsize
%\begin{center}
%Eigenschaften von Kompositionen
%\end{center}
%\vspace{0.5cm}

%\footnotesize
%\textbf{Definition}: Sei $f: M \rightarrow N$ bijektiv. Dann bezeichnet 
%$f^{-1}: N \rightarrow M$ die Umkehrfunktion.
%\vspace{0.5cm}

%\textbf{Satz}: Sei $f: M \rightarrow N$ bijektiv. Dann gilt: \\[0.3cm]
%\hspace*{1.3cm} $\mathtt{graph}(f^{-1}) = (\mathtt{graph}(f))^{-1}$
%\vspace{0.5cm}

%\textbf{Satz}: Seien $f: K \rightarrow M$ und $g: M \rightarrow N$ surjektiv.
%Dann ist $g \circ f$ surjektiv.
%\vspace{0.5cm}

%\textbf{Satz}: Seien $f: K \rightarrow M$ und $g: M \rightarrow N$ injektiv.
%Dann ist $g \circ f$ injektiv.
%\vspace{0.5cm}

%\textbf{Korollar}: Seien $f: K \rightarrow M$ und $g: M \rightarrow N$ bijektiv.
%Dann ist $g \circ f$ bijektiv.
%\vspace{0.5cm}

%\textbf{Satz}: Seien $f: K \rightarrow M$ und $g: M \rightarrow N$ bijektiv.
%Dann gilt $(g \circ f)^{-1} = f^{-1} \circ g^{-1}$.

%\vspace*{\fill}
%\tiny \addtocounter{mypage}{1}
%\rule{15cm}{1mm}
%Informatik I  \hspace*{\fill} Seite \arabic{mypage}
%\end{slide}

%%%%%%%%%%%%%%%%%%%%%%%%%%%%%%%%%%%%%%%%%%%%%%%%%%%%%%%%%%%%%%%%%%%%%%%%%%%%%%%%

\begin{slide}{}
\normalsize
\begin{center}
Aufgaben
\end{center}
\vspace{0.5cm}

\footnotesize
\textbf{Aufgabe 1}: Wie sieht die kleinste Ordnungs--Relation im Sinne von $\leq$ auf 
$S = \{1, 2, 3\}$ aus?
\vspace{0.5cm}

\textbf{Definition}: Sei $R \subseteq M \times M$ eine Ordnungs--Relation im Sinne von
$\leq$.  Die Ordnung $R$ hei\3t \underline{\emph{total}} g.d.w. \\[0.3cm]
\hspace*{1.3cm} $\forall x,y \in M: x R y \vee y R x$.

\textbf{Beispiel}: Die Relation $\{ \langle x, y \rangle \in \mathbb{N}^2 \is x \leq y \}$
ist total.

\textbf{Gegenbeispiel}: Die Relation \\[0.3cm]
\hspace*{1.3cm} $\mathtt{Teiler} = \{ \langle x, y \rangle \in \mathbb{N}^2 \is \exists z \in \mathbb{N}: x * z = y \}$ \\[0.3cm]
ist \underline{nicht} total.
\vspace{0.5cm}

\textbf{Aufgabe 2}: Geben Sie eine \underline{totale} Ordnungs--Relation im Sinne von $\leq$ auf 
$S = \{1, 2, 3\}$ an.
\vspace{0.5cm}

\textbf{Definition}: Sei $R_1 \subseteq K \times M$ und $R_2 \subseteq M \times N$
$$  R_2 \circ R_1 = \{ \langle x, z \rangle \in K \times N \is \exists y \in M: \langle x, y \rangle \in R_1\; \wedge
                                                                                \langle y, z \rangle \in R_2 \}$$

\textbf{Aufgabe 3}: Seien $R_1, R_2 \subseteq M \times M$. \\[0.3cm]
Beweisen oder widerlegen Sie:
\begin{enumerate}
\item Falls $R_1$ und $R_2$ reflexiv sind, \\[0.3cm]
      dann ist auch $R_1 \circ R_2$ reflexiv.
\item Falls $R_1$ und $R_2$ symmetrisch sind, \\[0.3cm]
      dann ist auch $R_1 \circ R_2$ symmetrisch.
\item Falls $R_1$ und $R_2$ transitiv sind, \\[0.3cm]
      dann ist auch $R_1 \circ R_2$ transitiv.
\end{enumerate}



\vspace*{\fill}
\tiny \addtocounter{mypage}{1}
\rule{15cm}{1mm}
Informatik I  \hspace*{\fill} Seite \arabic{mypage}
\end{slide}

%%%%%%%%%%%%%%%%%%%%%%%%%%%%%%%%%%%%%%%%%%%%%%%%%%%%%%%%%%%%%%%%%%%%%%%%

\begin{slide}{}
\normalsize
\begin{center}
Potenz--Menge
\end{center}
\vspace{0.5cm}

\footnotesize
\textbf{Schreibweise}: Sei $f: M \rightarrow N$ Funktion.  \\[0.3cm]
\hspace*{1.3cm} $\{ f(x) \is x \in M \} := \{ y \in N \is \exists x \in M: y = f(x) \}$ \\[0.3cm]
\hspace*{3.9cm} $f(M) := \{ y \in N \is \exists x \in M: y = f(x) \}$

\footnotesize
\textbf{Definition}: Ist $M$ eine Menge, so bezeichnet \\[0.3cm]
\hspace*{1.3cm} $2^M = \{ N \is N \subseteq M \}$ \\[0.3cm]
die \emph{Potenz--Menge} von $M$.
\vspace{0.5cm}

\textbf{Definition}: Ist $M$ endliche Menge, so bezeichnet \\[0.3cm]
\hspace*{1.3cm} $\mathtt{card}(M)$ \\[0.3cm]
die Anzahl der Elemente von $M$.

\textbf{Beispiele}:
\begin{enumerate}
\item $M = \emptyset$. \\[0.3cm]
      \hspace*{1.3cm} $2^M = \{ \emptyset \}$, \quad $\mathtt{card}(2^M) = 1$.
\item $M = \{1\}$. \\[0.3cm]
      \hspace*{1.3cm} $2^M = \left\{ \emptyset, \{1\} \rule{0pt}{16pt} \right\}$, \quad $\mathtt{card}(2^M) = 2$.
\item $M = \{1,2\}$ \\[0.3cm]
      \hspace*{1.3cm} $2^M = \left\{ \emptyset, \{1\}, \{2\}, \{1,2\} \rule{0pt}{16pt} \right\}$, \\[0.3cm]
      \hspace*{1.3cm} $\mathtt{card}(2^M) = 4$.
\end{enumerate}
\vspace{0.5cm}

\textbf{Satz}: Ist $M$ endlich, so gilt 
$$\mathtt{card}(2^M) = 2^{\mathtt{card}(M)}$$

\vspace*{\fill}
\tiny \addtocounter{mypage}{1}
\rule{15cm}{1mm}
Informatik I  \hspace*{\fill} Seite \arabic{mypage}
\end{slide}

%%%%%%%%%%%%%%%%%%%%%%%%%%%%%%%%%%%%%%%%%%%%%%%%%%%%%%%%%%%%%%%%%%%%%%%%

\begin{slide}{}
\normalsize
\begin{center}
Funktionen als Mengen
\end{center}
\vspace{0.5cm}

\footnotesize
\textbf{Definition}: Sei $f: M \rightarrow N$. \\[0.3cm]
\hspace*{1.3cm} $\texttt{graph}(f) := \left\{ \rule{0pt}{16pt} \langle x, y \rangle \is y = f(x) \right\}$
\vspace{0.5cm}

\textbf{Definition}: $M$ und $N$ seien Mengen 
$$  N^M := \left\{ \mathtt{graph}(f) \is f: M \rightarrow N \rule{0pt}{16pt} \right\} $$


\textbf{Satz}: $M$ und $N$ endlich.  Dann gilt
$$  \mathtt{card}(N^M) = \mathtt{card}(N)^{\mathtt{card}(M)} $$

\textbf{Satz}:  $M$ und $N$ endlich.  Dann gilt
$$  \mathtt{card}(M \times N) = \mathtt{card}(M) * \mathtt{card}(N) $$

\vspace*{\fill}
\tiny \addtocounter{mypage}{1}
\rule{15cm}{1mm}
Informatik I  \hspace*{\fill} Seite \arabic{mypage}
\end{slide}

%%%%%%%%%%%%%%%%%%%%%%%%%%%%%%%%%%%%%%%%%%%%%%%%%%%%%%%%%%%%%%%%%%%%%%%%

\begin{slide}{}
\normalsize
\begin{center}
M"achtigkeit von Mengen
\end{center}
\vspace{0.5cm}

\footnotesize
\textbf{Definition}:  $M$ und $N$  \emph{gleichm"achtig} g.d.w.
\begin{enumerate}
\item es existiert eine \underline{injektive} Funktion \\[0.3cm]
      \hspace*{1.3cm} $f: M \rightarrow N$ \quad und
\item es existiert eine \underline{injektive} Funktion \\[0.3cm]
      \hspace*{1.3cm} $g: N \rightarrow M$
\end{enumerate}
\textbf{Schreibweise}: \\[0.3cm]
\hspace*{1.3cm} $M \approx_{card} N$.

\textbf{Satz}: 
\begin{enumerate}
\item $M \approx_{card} M$
\item $K \approx_{card} M \wedge M \approx_{card} N \rightarrow K \approx_{card} N$
\end{enumerate}

\textbf{Beispiele}:
\begin{enumerate}
\item $\{1,2,3\} \approx_{card} \{a,b,c\}$
\item $\mathbb{N} \approx_{card} \mathbb{Z}$
\item $\mathbb{N} \approx_{card} \mathbb{N} \times \mathbb{N}$
\end{enumerate}
\vspace{0.5cm}

\vspace*{\fill}
\tiny \addtocounter{mypage}{1}
\rule{15cm}{1mm}
Informatik I  \hspace*{\fill} Seite \arabic{mypage}
\end{slide}


%%%%%%%%%%%%%%%%%%%%%%%%%%%%%%%%%%%%%%%%%%%%%%%%%%%%%%%%%%%%%%%%%%%%%%%%

\begin{slide}{}
\normalsize
\begin{center}
M"achtigkeit von Mengen
\end{center}
\vspace{0.5cm}

\footnotesize
\textbf{Satz} (\textsl{Schr"oder--Bernstein})  Falls $M \approx_{card} N$, dann gibt es
eine \underline{bijektive} Funktion \\[0.3cm]
\hspace*{1.3cm} $h: M \rightarrow N$.
\vspace{0.5cm}


\textbf{Definition}: $N$ \emph{m"achtiger} als $M$ g.d.w.
\begin{enumerate}
\item es existiert eine injektive Funktion \\[0.3cm]
      \hspace*{1.3cm} $f: M \rightarrow N$ \quad und
\item es existiert \underline{keine} injektive Funktion \\[0.3cm]
      \hspace*{1.3cm} $g: N \rightarrow M$.
\end{enumerate}
\textbf{Schreibweise}: $M \prec_{card} N$

\textbf{Beispiel}: $\{1,2,3\} \prec_{card} \{a,b,c,d\}$
\vspace{0.5cm}

\textbf{Satz}:  $K \prec_{card} M \wedge M \prec_{card} N \rightarrow K \prec_{card} N$
\vspace*{0.5cm}


\textbf{Definition}: Falls $M \approx_{card} \mathbb{N}$, so hei\3t $M$ \\
\emph{abz"ahlbar unendlich}.

\textbf{Definition}: Falls $\mathbb{N} \prec_{card} M$, so hei\3t $M$ \\ 
\emph{"uberabz"ahlbar}.

\vspace*{\fill}
\tiny \addtocounter{mypage}{1}
\rule{15cm}{1mm}
Informatik I  \hspace*{\fill} Seite \arabic{mypage}
\end{slide}

%%%%%%%%%%%%%%%%%%%%%%%%%%%%%%%%%%%%%%%%%%%%%%%%%%%%%%%%%%%%%%%%%%%%%%%%

%\begin{slide}{}
%  \normalsize
%  \begin{center}
% Eigenschaften von  $\approx_{card}$ und $\prec_{card}$
%  \end{center}
%\vspace*{0.5cm}


%\footnotesize
%\textbf{Erinnerung}: F"ur $f: M \rightarrow N$ und $L \subseteq M$ ist\\[0.3cm]
%\hspace*{1.3cm} $f(L) := \{ y \in N \is \exists x \in M: y = f(x) \}$
%\vspace*{0.5cm}

%\textbf{Satz}: 
%$$    A \approx_{card} B \rightarrow 2^{A} \approx_{card} 2^{B}$$

%\textbf{Beweis}:  Sei $f: A \rightarrow B$ injektiv. Definiere \\[0.3cm]
%\hspace*{1.3cm} $F: 2^A \rightarrow  2^B$ durch \\[0.3cm]
%\hspace*{1.3cm} $F(M) := f(M)$. \\[0.3cm]
%Zeige: $F$ ist injektiv.  Sei also $F(M_1) = F(M_2)$.  Dann ist z.Z: \\[0.3cm]
%\hspace*{1.3cm} $M_1 = M_2$. \\[0.3cm]
%Sei also $x_1 \in M_1$. \\[0.3cm]
%\hspace*{1.3cm} $\Rightarrow \quad f(x_1) \in f(M_1) = f(M_2)$. \\[0.3cm]
%\hspace*{1.3cm} $\Rightarrow \quad \exists x_2 \in M_2: f(x_2) = f(x_1)$ \\[0.3cm]
%\hspace*{1.3cm} $\Rightarrow \quad x_2 = x_1$ \quad weil $f$ injektiv \\[0.3cm]
%\hspace*{1.3cm} $\Rightarrow \quad x_1 \in M_2$. \\[0.3cm]
%\hspace*{1.3cm} Genauso: $x_2 \in M_2 \rightarrow x_2 \in M_1$ \\[0.3cm]
%\hspace*{1.3cm} Also $M_1 = M_2$.

%\vspace*{\fill}
%\tiny \addtocounter{mypage}{1}
%\rule{15cm}{1mm}
%Informatik I  \hspace*{\fill} Seite \arabic{mypage}
%\end{slide}

%%%%%%%%%%%%%%%%%%%%%%%%%%%%%%%%%%%%%%%%%%%%%%%%%%%%%%%%%%%%%%%%%%%%%%%%

\begin{slide}{}
\normalsize
\textbf{Satz}: (Cantor) \\[0.3cm]
\hspace*{1.3cm} $\mathbb{N} \prec_{card} \mathbb{N}^\mathbb{N}$
\vspace{0.5cm}

\footnotesize
\textbf{Beweis}:
Annahme: $\mathbb{N} \approx_{card} \mathbb{N}^\mathbb{N}$

Schr"oder--Bernstein: existiert $f:\mathbb{N} \rightarrow \mathbb{N}^\mathbb{N}$ bijektiv.

Definiere\\[0.3cm]
\hspace*{1.3cm}  $\mathtt{diag}: \mathbb{N} \rightarrow \mathbb{N}$ \\[0.3cm]
\hspace*{1.3cm}  $\mathtt{diag}(n) := \left(\rule{0pt}{16pt}f(n)\right)(n) + 1$.\\[0.3cm]
$$ \mathtt{diag} \in \mathbb{N}^\mathbb{N} \quad 
      \mbox{also existiert}\; d \in \mathbb{N}\; \mbox{mit}\; f(d) = \mathtt{diag} $$
Konsequenz: 
$$ 
\begin{array}{lcl}
 \texttt{diag}(d) & = & \left(\rule{0pt}{16pt}f(d)\right)(d) + 1 \\
           & = & \texttt{diag}(d) + 1
\end{array}
$$        
Widerspruch!
      
\begin{center}
\framebox{\framebox{ $\;\displaystyle \mathbb{N}^\mathbb{N}$ ist "uberabz"ahlbar.\ }}  
\end{center}

\vspace*{\fill}
\tiny \addtocounter{mypage}{1}
\rule{15cm}{1mm}
Informatik I  \hspace*{\fill} Seite \arabic{mypage}
\end{slide}

%%%%%%%%%%%%%%%%%%%%%%%%%%%%%%%%%%%%%%%%%%%%%%%%%%%%%%%%%%%%%%%%%%%%%%%%

\begin{slide}{}
\normalsize
\begin{center}
  Nicht berechenbare Funktionen  
\end{center}


\textbf{Definition}: \\[0.3cm]
\hspace*{1.3cm} $\mathbb{P}$ \quad := \quad ``Menge aller \texttt{C}--Programme''

\footnotesize
\textbf{Satz}: \quad $\mathbb{P} \approx_{card} \mathbb{N}$

\textbf{Beweis}: Jedes \texttt{C}--Programme $P$ ist endliche Folge von Bytes $B_i$: \\[0.3cm]
\hspace*{1.3cm} $P = B_0 B_1 B_2 \cdots B_n$ \\[0.3cm]
Interpretiere $P$ als Zahl: \\[0.3cm]
\hspace*{1.3cm} $\sum\limits_{i=0}^n B_i * 256^i$. \\[0.3cm]
Dann gilt $\mathbb{P} \subseteq \mathbb{N}$.

\textbf{Korollar}: Es gibt Funktionen in $\mathbb{N}^\mathbb{N}$, die nicht durch ein
\texttt{C}--Programm berechenbar sind!

\vspace*{\fill}
\tiny \addtocounter{mypage}{1}
\rule{15cm}{1mm}
Informatik I  \hspace*{\fill} Seite \arabic{mypage}
\end{slide}

%%%%%%%%%%%%%%%%%%%%%%%%%%%%%%%%%%%%%%%%%%%%%%%%%%%%%%%%%%%%%%%%%%%%%%%%%%%%%%%%

%%%%%%%%%%%%%%%%%%%%%%%%%%%%%%%%%%%%%%%%%%%%%%%%%%%%%%%%%%%%%%%%%%%%%%%%

\begin{slide}{}
\normalsize
\textbf{Aufgabe}: Zeigen Sie 
$$   \mathbb{N} \prec_{card} 2^\mathbb{N} $$
\vspace{0.5cm}

%\footnotesize
%\textbf{Beweis}:
%Annahme: $\mathbb{N} \approx_{card} 2^\mathbb{N}$

%Dann existiert $f:\mathbb{N} \rightarrow 2^\mathbb{N}$ bijektiv.

%Definiere\\[0.3cm]
%\hspace*{1.3cm}    $C := \{ x \in \mathbb{N} \is x \not\in f(x) \}$.\\[0.3cm]
%\hspace*{0.5cm} $\Rightarrow C \in 2^\mathbb{N}$

%Da $f$ bijektiv, existiert $d\in \mathbb{N}$ mit \\[0.3cm]
%\hspace*{1.3cm} $C = f(d)$.  \\[0.3cm]
%Dann gilt: \\[0.3cm]
%\hspace*{1.3cm} $
%\begin{array}{ll}
%                 & d \in C \\[0.3cm]
% \Leftrightarrow & d \in \{ x \in \mathbb{N} \is x \not\in f(x) \} \\[0.3cm]
% \Leftrightarrow & d \not\in f(d) \\[0.3cm]
% \Leftrightarrow & d \not\in C \\[0.3cm]
%\end{array}$ \\[0.3cm]
%Widerspruch!

\vspace*{\fill}
\tiny \addtocounter{mypage}{1}
\rule{15cm}{1mm}
Informatik I  \hspace*{\fill} Seite \arabic{mypage}
\end{slide}


%%% Local Variables: 
%%% mode: latex
%%% TeX-master: "informatik-1"
%%% End: 
