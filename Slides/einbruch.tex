\documentclass{slides}
\usepackage[latin1]{inputenc}
\usepackage{german}
\usepackage{epsfig}
\usepackage{amssymb}

\pagestyle{empty}
\setlength{\textwidth}{17cm}
\setlength{\textheight}{24cm}
\setlength{\topmargin}{0cm}
\setlength{\headheight}{0cm}
\setlength{\headsep}{0cm}
\setlength{\topskip}{0.2cm}
\setlength{\oddsidemargin}{0.5cm}
\setlength{\evensidemargin}{0.5cm}

\newcommand{\is}{\;|\;}
\newcommand{\schluss}[2]{\frac{\displaystyle\quad \rule[-8pt]{0pt}{18pt}#1 \quad}{\displaystyle\quad \rule{0pt}{14pt}#2 \quad}}
\newcommand{\vschlus}[1]{{\displaystyle\rule[-6pt]{0pt}{12pt} \atop \rule{0pt}{10pt}#1}}
\newcommand{\verum}{\top}
\newcommand{\falsum}{\bot}
\newcommand{\gentzen}{\vdash}
\newcommand{\komplement}[1]{\overline{#1}}

\newcounter{mypage}

\begin{document}

\begin{slide}{}
\normalsize
\begin{center}
  Aufkl�rung Einbruch
\end{center}
\vspace*{1cm}

\footnotesize

\begin{enumerate}
\item Drei Verd�chtige: Anton, Bruno und Claus.
\item Einer der drei Verd�chtigen muss die Tat begangen haben: \\[0.2cm]
      \hspace*{1.3cm} 
      $f_1 := a \vee b \vee c$.
\item Wenn Anton schuldig ist, so hat er genau einen Komplizen. 

      \begin{enumerate}
      \item Wenn Anton schuldig ist, dann hat er mindestens einen Komplizen: \\[0.2cm]
            \hspace*{1.3cm} $f_2 := a \rightarrow b \vee c$ 
      \item Wenn Anton schuldig ist, dann hat er h�chstens einen Komplizen: \\[0.2cm]
           \hspace*{1.3cm} $f_3 := a \rightarrow \neg (b \wedge c)$
      \end{enumerate}
\item Wenn Bruno unschuldig ist, dann ist auch Claus unschuldig: \\[0.2cm]
      \hspace*{1.3cm} $f_4 :=  \neg b \rightarrow \neg c$ 
\item Wenn genau zwei schuldig sind, dann ist Claus einer von ihnen. \\[0.2cm]
      \hspace*{1.3cm} $f_5 := \neg ( \neg c  \wedge a \wedge b )$ 
\item Wenn Claus unschuldig ist, ist Anton schuldig. \\[0.2cm]
      \hspace*{1.3cm} $f_6 := \neg c \rightarrow a$
\end{enumerate}

\vspace*{\fill}
\tiny \addtocounter{mypage}{1} 
\rule{17cm}{1mm}
  \hspace*{\fill} Seite \arabic{mypage}
\end{slide}

%%%%%%%%%%%%%%%%%%%%%%%%%%%%%%%%%%%%%%%%%%%%%%%%%%%%%%%%%%%%%%%%%%%%%%%%

\end{document}

%%% Local Variables: 
%%% mode: latex
%%% TeX-master: t
%%% End: 
