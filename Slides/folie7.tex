\documentclass{slides}
\usepackage[latin1]{inputenc}
\usepackage{german}
\usepackage{epsfig}
\usepackage{amssymb}

\pagestyle{empty}
\setlength{\textwidth}{17cm}
\setlength{\textheight}{24cm}
\setlength{\topmargin}{0cm}
\setlength{\headheight}{0cm}
\setlength{\headsep}{0cm}
\setlength{\topskip}{0.2cm}
\setlength{\oddsidemargin}{0.5cm}
\setlength{\evensidemargin}{0.5cm}

\newcommand{\is}{\;|\;}
\newcommand{\schluss}[2]{\frac{\displaystyle\quad \rule[-8pt]{0pt}{18pt}#1 \quad}{\displaystyle\quad \rule{0pt}{14pt}#2 \quad}}
\newcommand{\vschlus}[1]{{\displaystyle\rule[-6pt]{0pt}{12pt} \atop \rule{0pt}{10pt}#1}}
\newcommand{\verum}{\top}
\newcommand{\falsum}{\bot}
\newcommand{\var}{\textsl{Var}}
\newcommand{\struct}{\mathcal{S}}
\newcommand{\el}{\!\in\!}
\newcommand{\FV}{\textsl{FV}}
\newcommand{\BV}{\textsl{BV}}
\newcommand{\gentzen}{\vdash_\mathcal{G}}
\newcommand{\komplement}[1]{\overline{#1}}

\newcounter{mypage}

\begin{document}

\begin{slide}
\begin{center}
Unifikation
\end{center}
\vspace*{0.5cm}

\footnotesize
\textbf{Definition}: Gegeben sei Signatur\\[0.3cm]
\hspace*{1.3cm} $\Sigma = \langle \mathbb{T}, \textsl{Fz}, \textsl{Pz}, \mathtt{sign}_{F}, \mathtt{sign}_{P}, \mathcal{V}, \textsl{var} \rangle$ \\[0.3cm]
$\Sigma$--Substitution:  endliche Menge von Paaren der Art\\[0.3cm]
\hspace*{1.3cm} $\sigma = \Bigg\{ \langle x_1, t_1 \rangle, \cdots, \langle x_n, t_n \rangle \Bigg\}$. \\[0.3cm]
mit
\begin{enumerate}
\item $x_i \in \mathcal{V}$ \quad f"ur alle $i\in\{1,\cdots,n\}$

      Die $x_i$ sind Variablen.
\item $x_i \in \textsl{var}(\tau) \Rightarrow t_i \in \mathcal{T_{\tau}}$ \quad f"ur alle $i\in\{1,\cdots,n\}$

      Die Terme $t_i$ haben den gleichen Typ wie $x_i$.
\item $i\not=j \Rightarrow x_i \not= x_j$ \quad f"ur alle $i,j\in\{1,\cdots,n\}$

      Die $x_i$ sind paarweise verschieden.
\end{enumerate}
Schreibweise:  \\[0.3cm]
\hspace*{1.3cm} $\sigma = \bigl[ x_1 \mapsto t_1, \cdots, x_n \mapsto t_n \bigr]$. \\[0.5cm]
\textbf{Definition}: \emph{Domain} von $\sigma$ \\[0.3cm]
\hspace*{1.3cm} $\textsl{dom}(\sigma) := \{x_1,\cdots,x_n\}$.


\vspace*{\fill}
\tiny \addtocounter{mypage}{1} 
\rule{17cm}{1mm}
Unifikation \hspace*{\fill} Seite \arabic{mypage}
\end{slide}

%%%%%%%%%%%%%%%%%%%%%%%%%%%%%%%%%%%%%%%%%%%%%%%%%%%%%%%%%%%%%%%%%%%%%%%%

\begin{slide}{}
\normalsize
\begin{center}
Substitutions--Anwendung
\end{center}
\vspace{0.5cm}

\footnotesize
\textbf{Gegeben}:
\begin{enumerate}
\item $t$: \quad Term
\item $\sigma = \bigl[ x_1 \mapsto s_1, \cdots, x_n \mapsto s_n \bigr]$: \quad Substitution
\end{enumerate}
\emph{Anwendung} von $\sigma$ auf $t$: \\[0.3cm]
\hspace*{1.3cm} Ersetze $x_i$ durch $s_i$.

\textbf{Induktive Definition}: $t\sigma$ (\emph{Anwendung} von $\sigma$ auf $t$) 
\begin{enumerate}
\item  $t$ ist Variable:
  \begin{enumerate}
  \item $x_i\sigma := s_i$.
  \item $y\sigma := y$, \quad falls $y \in \mathcal{V}$, aber $y \not\in \{x_1,\cdots,x_n\}$. 
  \end{enumerate}
\item $f(t_1, \cdots, t_n)\sigma := f(t_1\sigma, \cdots, t_n\sigma)$.
\end{enumerate}
Sei $p(t_1,\cdots,t_n)$ atomare Formel

\hspace*{1.3cm}
$p(t_1,\cdots,t_n) \sigma := p(t_1\sigma, \cdots, t_n\sigma)$.  

Sei $f$ quantorenfreie Formel.

\textbf{Induktive Definition} von  $f\sigma$
\begin{enumerate}
\item $(\neg f)\sigma := \neg (f\sigma)$.
\item $(f_1 \wedge f_2)\sigma := f_1\sigma \wedge f_2\sigma$.
\item $(f_1 \vee f_2)\sigma   := f_1\sigma \vee   f_2\sigma$.
\end{enumerate}

\vspace*{\fill}
\tiny \addtocounter{mypage}{1}
\rule{17cm}{1mm}
Unifikation  \hspace*{\fill} Seite \arabic{mypage}
\end{slide}

%%%%%%%%%%%%%%%%%%%%%%%%%%%%%%%%%%%%%%%%%%%%%%%%%%%%%%%%%%%%%%%%%%%%%%%%

\begin{slide}{}
\normalsize
\begin{center}
Komposition von Substitutionen
\end{center}
\vspace{0.5cm}

\footnotesize
\textbf{Gegeben}:
\begin{enumerate}
\item $\sigma = \big[ x_1 \mapsto s_1, \cdots, x_m \mapsto s_m \big]$
\item $\tau = \big[ y_1 \mapsto t_1, \cdots, y_n \mapsto t_n \big]$
\item $\textsl{dom}(\sigma) \cap \textsl{dom}(\tau) = \emptyset$
\end{enumerate}
\textbf{Definition}: $\sigma\tau$ (\emph{Komposition} von $\sigma$ und $\tau$) \\[0.3cm]
\hspace*{1.3cm} $\sigma\tau := \big[ x_1 \mapsto s_1\tau, \cdots, x_m \mapsto s_m\tau,\; y_1 \mapsto t_1, \cdots, y_n \mapsto t_n \big]$
\vspace{0.5cm}

\textbf{Satz}: Gegeben
\begin{enumerate}
\item $t$: Term
\item $\sigma$, $\tau$: Substitutionen
\end{enumerate}
Dann gilt:  $(t\sigma)\tau = t (\sigma\tau)$
\vspace{0.5cm}

\textbf{Definition}: Gegeben
\begin{enumerate}
\item $s$, $t$: Terme vom selben Typ, oder
\item $s$, $t$: atomare Formeln.
\end{enumerate}
Dann ist $s \doteq t$ eine \emph{syntaktische Gleichung}.

\hspace*{1.3cm} 
$E = \{ s_1 \doteq t_1, \cdots, s_n \doteq t_n \}$

hei�t \emph{syntaktisches Gleichungssystem}

\vspace*{\fill}
\tiny \addtocounter{mypage}{1}
\rule{17cm}{1mm}
Unifikation  \hspace*{\fill} Seite \arabic{mypage}
\end{slide}


%%%%%%%%%%%%%%%%%%%%%%%%%%%%%%%%%%%%%%%%%%%%%%%%%%%%%%%%%%%%%%%%%%%%%%%%

\begin{slide}{}
\normalsize
\begin{center}
Unifikator
\end{center}
\vspace{0.5cm}

\footnotesize
\textbf{Gegeben}: 
\begin{enumerate}
\item Syntaktisches Gleichungssystem \\[0.3cm]
      \hspace*{1.3cm} $E = \{ s_1 \doteq t_1, \cdots, s_n \doteq t_n \}$,
\item Substitution $\sigma$.
\end{enumerate}
\textbf{Definition}: $\sigma$ ist \emph{Unifikator} von $E$ g.d.w. \\[0.1cm]
\hspace*{1.3cm} $s_i\sigma = t_i\sigma$ \quad f�r alle $i=1,\cdots,n$.
\vspace{0.5cm}

\noindent
\textbf{Beobachtung}: Sei
\begin{enumerate}
\item $E$: \quad  syntaktisches Gleichungssystem, 
\item $\sigma$: \quad Substitution
\item $\sigma$ sei Unifikator von $E$,
\item $\tau$: \quad Substitution mit $\textsl{dom}(\tau) \cap \textsl{dom}(\sigma) = \emptyset$
\end{enumerate}
\textbf{Behauptung}: $\sigma\tau$ Unifikator von $E$
\vspace{0.5cm}

\textbf{Definition}: Unifikator $\sigma$ ist \emph{allgemeiner} als Unifikator $\sigma\tau$.

\vspace*{\fill}
\tiny \addtocounter{mypage}{1}
\rule{17cm}{1mm}
Unifikation  \hspace*{\fill} Seite \arabic{mypage}
\end{slide}


%%%%%%%%%%%%%%%%%%%%%%%%%%%%%%%%%%%%%%%%%%%%%%%%%%%%%%%%%%%%%%%%%%%%%%%%

\begin{slide}{}
\normalsize
\begin{center}
Signatur $\Sigma_\textsl{Stack}$ f�r Stacks
\end{center}
\vspace{0.5cm}

\footnotesize
$\Sigma_\textsl{Stack} := \langle \mathbb{T}, \textsl{Fz}, \textsl{Pz}, \mathtt{sign}_{F}, \mathtt{sign}_{P}, \mathcal{V}, \textsl{var} \rangle$ 

\begin{enumerate}
\item $\mathbb{T} := \{ \mathbb{B}, \mathbb{N}, \textsl{Stack} \}$
\item $\textsl{Fz} := \{ 0, s, \textsl{nil}, \textsl{push}, \textsl{pop}, \textsl{top} \}$
\item $\textsl{Pz} := \{ \textsl{empty}, =_O, =_S \}$

      Typ--Spezifikationen:
      \begin{enumerate}
      \item $0: \mathbb{N}$
      \item $s: \mathbb{N} \rightarrow \mathbb{N}$

            Interpretation: $s(n) = n + 1$
      \item $nil: \textsl{Stack}$
      \item $\textsl{push}: \mathbb{N} \times \textsl{Stack} \rightarrow \textsl{Stack}$
      \item $\textsl{pop}: \textsl{Stack} \rightarrow \textsl{Stack}$
      \item $\textsl{top}: \textsl{Stack} \rightarrow \mathbb{N}$
      \item $\textsl{empty}: \textsl{Stack} \rightarrow \mathbb{B}$
      \item $=_O: \mathbb{N} \times \mathbb{N} \rightarrow \mathbb{B}$
      \item $=_S: \textsl{Stack} \times \textsl{Stack} \rightarrow \mathbb{B}$
      \end{enumerate}
\item $\mathcal{V} := \{ x_i | i \in \mathbb{N} \} \cup \{ s_i | i \in \mathbb{N} \}$.
\item $\textsl{var}(\mathbb{N})   := \{ x_i | i \in \mathbb{N} \}$.
\item $\textsl{var}(\textsl{Stack}) := \{ s_i | i \in \mathbb{N} \}$.
\end{enumerate}


\vspace*{\fill}
\tiny \addtocounter{mypage}{1}
\rule{17cm}{1mm}
Unifikation  \hspace*{\fill} Seite \arabic{mypage}
\end{slide}


%%%%%%%%%%%%%%%%%%%%%%%%%%%%%%%%%%%%%%%%%%%%%%%%%%%%%%%%%%%%%%%%%%%%%%%%

\begin{slide}{}
\normalsize
\begin{center}
L�sung syntaktischer Gleichungen
\end{center}
\vspace{0.5cm}

\footnotesize
\textbf{Definition}: Ein Gleichungssystem der Form \\[0.3cm]
\hspace*{1.3cm} $\{ x_1 \doteq t_1, \cdots, x_n \doteq t_n \}$ \\[0.3cm]
ist \emph{trivial} g.d.w.
\begin{enumerate}
\item $x_i \in \mathcal{V}$ \quad f�r alle $i \in \{1,\cdots,n\}$,
\item $x_i \not\in \textsl{var}(t_j)$ \quad f�r alle $i,j \in \{1,\cdots,n\}$ und
\item $i \not= j \Rightarrow x_i \not= x_j$ \quad f�r alle $i,j \in \{1,\cdots,n\}$.
\end{enumerate}

\textbf{Satz}:  Sei $E := \{ x_1 \doteq t_1, \cdots, x_n \doteq t_n \}$ trivial. \\[0.3cm]
Dann ist $\sigma := [ x_1 \mapsto t_1, \cdots, x_n \mapsto t_n ]$ L�sung von $E$. \\[0.3cm]
Dann \textbf{definiere}: \quad $\textsl{Subst}(E) := \sigma$

\vspace{0.5cm}
\textbf{Definition}:  Eine syntaktische Gleichung $e$ ist 
\begin{center}
 \emph{offensichlich unl�sbar}
\end{center}
falls einer der folgenden F�lle vorliegt:
\begin{enumerate}
\item $e = (x \doteq t)$ \quad mit $t \not=x$ und $x\in \var(t)$
\item $e = (\;g(s_1, \cdots, s_m) \doteq f(t_1, \cdots, t_n)\;)$ \quad mit $f \not=g$.
\end{enumerate}

\textbf{Satz}: Sei
\begin{enumerate}
\item $s \doteq t$ offensichlich unl�sbar und
\item $\sigma$ beliebige Substitution.
\end{enumerate}
Dann gilt: \hspace*{1.3cm} $s\sigma \not= t\sigma$.

\vspace*{\fill}
\tiny \addtocounter{mypage}{1}
\rule{17cm}{1mm}
Unifikation  \hspace*{\fill} Seite \arabic{mypage}
\end{slide}


%%%%%%%%%%%%%%%%%%%%%%%%%%%%%%%%%%%%%%%%%%%%%%%%%%%%%%%%%%%%%%%%%%%%%%%%

\begin{slide}{}
\normalsize
\begin{center}
Martelli--Montanari--Regeln 
\end{center}
\vspace{0.5cm}

\footnotesize
\begin{enumerate}
\item Falls $y\in\mathcal{V}$ mit $y \not\in\textsl{Var}(t)$:
      \[ \Big\langle E \cup \big\{ y \doteq t \big\}, \sigma \Big\rangle \quad\leadsto\quad 
         \Big\langle E[y \mapsto t], \sigma\big[ y \mapsto t \big] \Big\rangle 
      \]
\item Falls $y\in\mathcal{V}$ mit $y \in\textsl{Var}(t)$:
      \[ \Big\langle E \cup \big\{ y \doteq t \big\}, \; \sigma \Big\rangle \;\leadsto\; \Omega. \]
\item Falls $y\in\mathcal{V}$ und $t \not\in \mathcal{V}$:
      \[ \Big\langle E \cup \big\{ t \doteq y \big\}, \sigma \Big\rangle \quad\leadsto\quad 
         \Big\langle E \cup \big\{ y \doteq t \big\}, \sigma \Big\rangle.
      \]   
\item Falls $x \in \mathcal{V}$:
      \[ \Big\langle E \cup \big\{ x \doteq x \big\}, \sigma \Big\rangle \quad\leadsto\quad
         \Big\langle E, \sigma \Big\rangle.
      \]   
\item Falls $f$ \ $n$-stelliges Funktions-Zeichen:
      \[
      \begin{array}[t]{cl}
         & \Big\langle E \cup \big\{ f(s_1,\cdots,s_n) \doteq f(t_1,\cdots,t_n) \big\}, \sigma \Big\rangle 
         \\[0.3cm]
         \;\leadsto\; &
         \Big\langle E \cup \big\{ s_1 \doteq t_1, \cdots, s_n \doteq t_n\}, \sigma \Big\rangle.        
      \end{array}
      \]   
\item Falls $f \not= g$;
      \[ \Big\langle E \cup \big\{ f(s_1,\cdots,s_m) \doteq g(t_1,\cdots,t_n) \big\}, \sigma \Big\rangle 
         \;\leadsto\; \Omega. 
      \]
\end{enumerate}



\vspace*{\fill}
\tiny \addtocounter{mypage}{1}
\rule{17cm}{1mm}
Unifikation  \hspace*{\fill} Seite \arabic{mypage}
\end{slide}


%%%%%%%%%%%%%%%%%%%%%%%%%%%%%%%%%%%%%%%%%%%%%%%%%%%%%%%%%%%%%%%%%%%%%%%%

\begin{slide}{}
\normalsize
\begin{center}
Eigenschaften der Martelli--Montanari--Regeln
\end{center}
\vspace{0.5cm}

\footnotesize
\textbf{Satz}: Invariante

\textbf{Vor}.: $\langle E_1, F_2 \rangle \leadsto \langle E_2, F_2 \rangle$

\textbf{Beh}:   $\sigma$ l�st $E_1 \cup F_1$ \quad g.d.w. \\[0.3cm]
\hspace*{1.5cm} $\sigma$ l�st $E_2 \cup F_2$.
\vspace{0.5cm}

\textbf{Satz}: Terminierung

\textbf{Beh}.: Es gibt keine unendliche Folge $\langle E_n, F_n \rangle$ mit \\[0.3cm]
\hspace*{1.7cm} $\langle E_n, F_n \rangle \leadsto \langle E_{n+1}, F_{n+1} \rangle$ \quad f�r alle $n \in \mathbb{N}$.
\vspace{0.5cm}

\textbf{Definition}: $\langle E_1, F_1 \rangle$ ist \emph{maximal reduziert}
g.d.w. \\[0.3cm]
\hspace*{3.4cm}
 es gibt kein $\langle E_2, F_2 \rangle$ mit \\[0.3cm]
\hspace*{4.8cm} $\langle E_1, F_1 \rangle \;\leadsto\; \langle E_2, F_2 \rangle$

\textbf{Satz}: Sei $E_1 := E$, \quad $F_1 := \emptyset$ 
\begin{enumerate}
\item $\langle E_1, F_1 \rangle \;\leadsto\; \langle E_2, F_2 \rangle \;\leadsto\; \cdots \leadsto\; \langle E_n, F_n \rangle$
\item $\langle E_n, F_n \rangle$ maximal reduziert
\end{enumerate}
Dann gilt $E_n = \emptyset$ und entweder
\begin{enumerate}
\item $F_n$ trivial und $\textsl{Subst}(F_n)$ L�sung von $E$ oder
\item $F_n$ offensichlich unl�sbar und $E$ unl�sbar.
\end{enumerate}

\vspace*{\fill}
\tiny \addtocounter{mypage}{1}
\rule{17cm}{1mm}
Unifikation  \hspace*{\fill} Seite \arabic{mypage}
\end{slide}

\end{document}

%%% Local Variables: 
%%% mode: latex
%%% TeX-master: t
%%% End: 
