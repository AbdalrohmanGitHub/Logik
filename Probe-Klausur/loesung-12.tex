\documentclass{article}
\usepackage[latin1]{inputenc}
\usepackage{fleqn}
\usepackage{ngerman}
\usepackage{a4wide}
\usepackage{amssymb}
\usepackage{fancyvrb}

\setlength{\topmargin}{-0.5cm}
\setlength{\mathindent}{1.3cm}
%\setlength{\evensidemargin}{-0.5cm}
%\setlength{\oddsidemargin}{-0.5cm}
%\setlength{\textwidth}{17cm}

\def\pair(#1,#2){\langle #1, #2 \rangle}

\renewcommand{\labelenumi}{(\alph{enumi})}
\renewcommand{\labelenumii}{\arabic{enumii}.}

\newcommand{\exercise}{\vspace*{0.1cm}
\stepcounter{aufgabe}
\vspace*{0.3cm}

\noindent
\textbf{L�sung Aufgabe \arabic{aufgabe}}: }

\newcommand{\solution}{
\vspace*{0.3cm}

\noindent
\textbf{L�sung}: }

\newcounter{aufgabe}

\newcommand{\schluss}[2]{\frac{\displaystyle\quad \rule[-6pt]{0pt}{12pt}#1 \quad}{\displaystyle\quad \rule{0pt}{10pt}#2 \quad}}
\newcommand{\vschlus}[1]{{\displaystyle\rule[-6pt]{0pt}{12pt} \atop \rule{0pt}{10pt}#1}}

\newcommand{\punkte}[1]{\hspace*{\fill} \mbox{(#1 Punkte)}}
\newcommand{\struct}{\mathcal{S}}
\newcommand{\FV}{\textsl{FV}}
\newcommand{\BV}{\textsl{BV}}
\newcommand{\var}{\textsl{Var}}
\newcommand{\qed}{\hspace*{\fill} $\Box$}
\newcommand{\el}{\!\in\!}
\newcommand{\verum}{\top}
\newcommand{\falsum}{\bot}
\newcommand{\gentzen}{\vdash_{G}}
\newcommand{\komplement}[1]{\overline{#1}}
\newcommand{\mathquote}[1]{\mbox{``}\mathtt{#1}\mbox{''}}

\begin{document}

\noindent
{\large L�sungen zu den Aufgaben zur Vorlesung ``{\sl Grundlagen der Informatik}''}
\vspace{0.3cm}

\exercise
\begin{enumerate}
\item Folgende Prozedur leistet das Gew�nschte:
      \begin{Verbatim}[ frame         = lines, 
                        framesep      = 0.3cm, 
                        labelposition = bottomline,
                        numbers       = left,
                        numbersep     = -0.2cm,
                        xleftmargin   = 0.8cm,
                        xrightmargin  = 0.8cm,
                      ]

    both := procedure(l1, l2) {
        return { x in l1 | x in l2 };
    };
      \end{Verbatim}
\item Folgende Prozedur leistet das Gew�nschte:
      \begin{Verbatim}[ frame         = lines, 
                        framesep      = 0.3cm, 
                        labelposition = bottomline,
                        numbers       = left,
                        numbersep     = -0.2cm,
                        xleftmargin   = 0.8cm,
                        xrightmargin  = 0.8cm,
                      ]
    procedure pythagoras(n);
        return { [ x, y, z ] : x in {1..n}, y in {1..n}, z in {1..n} 
                             | x*x + y*y = z*z 
               };
    end pythagoras;
      \end{Verbatim}
\item Folgende Prozedur leistet das Gew�nschte:
      \begin{Verbatim}[ frame         = lines, 
                        framesep      = 0.3cm, 
                        firstnumber   = 1,
                        labelposition = bottomline,
                        numbers       = left,
                        numbersep     = -0.2cm,
                        xleftmargin   = 0.8cm,
                        xrightmargin  = 0.8cm,
                      ]
    subsets := procedure(n, k) {
        return { m in pow({1 .. n }) | #m == k };
    };
      \end{Verbatim}
\item Folgende Prozedur leistet das Gew�nschte:
      \begin{Verbatim}[ frame         = lines, 
                        framesep      = 0.3cm, 
                        firstnumber   = 1,
                        labelposition = bottomline,
                        numbers       = left,
                        numbersep     = -0.2cm,
                        xleftmargin   = 0.8cm,
                        xrightmargin  = 0.8cm,
                      ]
    % pythagoras(+Number, -List(Triple)).   
    pythagoras(N, L) :-
    	pythagoras(N, [], L).
    
    % pythagoras(+Number, +List(Triple), -List(Triple)).
    pythagoras(N, Lin, Lout) :-
    	pythagorasTriple(N, T),
    	\+ member(T, Lin),
    	!,
    	pythagoras(N, [ T | Lin ], Lout).    
    pythagoras(_, L, L).
    
    % pythagorasTriple(+Number, -Triple) :-
    pythagorasTriple(N, [X, Y, Z]) :-
    	M is N - 1,
    	between(1, M, X),
    	between(1, M, Y),
    	between(1, M, Z),
    	Lhs is X * X + Y * Y,
    	Rhs is Z * Z,
    	Lhs == Rhs.
      \end{Verbatim}


\end{enumerate}
\pagebreak

\exercise
\begin{enumerate}
\item Folgende Prozedur leistet das Gew�nschte:
      \begin{Verbatim}[ frame         = lines, 
                        framesep      = 0.3cm, 
                        labelposition = bottomline,
                        numbers       = left,
                        numbersep     = -0.2cm,
                        xleftmargin   = 0.0cm,
                        xrightmargin  = 0.0cm,
                      ]
    compose := procedure(a1, a2) {
        return { [[x,z], d1 + d2] : [[x,y1], d1] in a1, [[y2,z], d2] in a2 
                                  | y1 == y2 
               };
    };
      \end{Verbatim}
\item Folgende Prozedur leistet das Gew�nschte:
      \begin{Verbatim}[ frame         = lines, 
                        framesep      = 0.3cm, 
                        labelposition = bottomline,
                        numbers       = left,
                        numbersep     = -0.2cm,
                        xleftmargin   = 0.0cm,
                        xrightmargin  = 0.0cm,
                      ]
    closure := procedure(a) {
        c    := a;
        oldC := {};
        while (c != oldC) {
            oldC := c;
            c    := c + compose(c, c);
        }
        return c;
    };
      \end{Verbatim}
\end{enumerate}


\exercise
Die folgende Implementierung leistet das Gew�nschte:
\begin{Verbatim}[ frame         = lines, 
                  framesep      = 0.3cm, 
                  labelposition = bottomline,
                  numbers       = left,
                  numbersep     = -0.2cm,
                  xleftmargin   = 0.8cm,
                  xrightmargin  = 0.8cm,
                ]
    toLists := procedure(m) {
        if (m == {}) {
            return { [] };
        }
        return { [ x ] + l : x in m, l in toLists(m - {x}) };
    };
\end{Verbatim}
\pagebreak

\exercise
\begin{enumerate}
\item Wir f�hren die folgenden aussagenlogischen Variablen als Abk�rzungen ein.
      \begin{enumerate}
      \item Linker Motor f�llt aus: $l$
      \item Rechter Motor f�llt aus: $r$
      \item Flugzeug st�rzt ab: $a$
      \end{enumerate}
      Die gesuchte Schluss-Regel ist:
      \[ \schluss{l \vee r \rightarrow a \qquad  a \qquad \neg l}{r} \]
\item Die Schluss-Regel ist genau dann korrekt, wenn die Formel
      \[ (l \vee r \rightarrow a) \wedge a \wedge \neg l \rightarrow r \]
      eine Tautologie ist.  Wir formen diese Formel in KNF um:
      \[
      \begin{array}[t]{ll}
                      & (l \vee r \rightarrow a) \wedge a \wedge \neg l \rightarrow r          \\[0.1cm]
      \Leftrightarrow & \neg\bigl((l \vee r \rightarrow a) \wedge a \wedge \neg l\bigr) \vee r \\[0.1cm]
      \Leftrightarrow & \neg\bigl((\neg (l \vee r) \vee a) \wedge a \wedge \neg l\bigr) \vee r \\[0.1cm]
      \Leftrightarrow & \neg(\neg (l \vee r) \vee a) \vee \neg a \vee  l \vee r         \\[0.1cm]
      \Leftrightarrow & ((l \vee r) \wedge \neg a) \vee \neg a \vee  l \vee r         \\[0.1cm]
      \Leftrightarrow & (l \vee r \vee \neg a \vee l \vee r) \wedge        
                        (\neg a \vee \neg a \vee l \vee r)         \\[0.1cm]
      \Leftrightarrow & (l \vee r \vee \neg a)     \\[0.1cm]
      \end{array}
      \]
      Diese Formel ist keine Tautologie.  Ein Gegenbeispiel ist durch die Belegung
      \\[0.2cm]
      \hspace*{1.3cm}
      $\mathcal{I} := 
       \bigl\{ \pair(l, \mathtt{false}), \pair(r, \mathtt{false}), \pair(a, \mathtt{true}) \bigr\}
      $
      \\[0.2cm]
      gegeben.  Diese Belegung h�tte man auch ohne die Rechnung angeben d�rfen.
      Anschaulich ist das Ergebnis klar, denn das Flugzeug kann auch abst�rzen,
      wenn beide Motoren funktionieren.  Nur mit der Pr�misse
      \\[0.2cm]
      \hspace*{1.3cm}
      $l \vee r \leftrightarrow a$
      \\[0.2cm]
      h�tte man aus dem Absturz auch auf den Ausfall eines  Motors schlie�en k�nnen.
\end{enumerate}

\exercise
\begin{enumerate}
\item Die gesuchte Formel ist
      \[ (q \wedge p) \wedge (r \vee \neg q) \rightarrow r \wedge p. \]
\item Wir formen diese Formel in konjunktive Normalform um:
      \[ 
      \begin{array}[t]{ll}
                      & (q \wedge p) \wedge (r \vee \neg q) \rightarrow r \wedge p                \\[0.1cm]
      \Leftrightarrow & \neg \bigl((q \wedge p) \wedge (r \vee \neg q)\bigr) \vee (r \wedge p)    \\[0.1cm]
      \Leftrightarrow & \bigl(\neg (q \wedge p) \vee \neg (r \vee \neg q)\bigr) \vee (r \wedge p) \\[0.1cm]
      \Leftrightarrow & \bigl((\neg q \vee \neg p) \vee (\neg r \wedge q)\bigr) \vee (r \wedge p) \\[0.1cm]
      \Leftrightarrow & \bigl((\neg q \vee \neg p \vee \neg r) \wedge 
                              (\neg q \vee \neg p \vee q) \bigr) \vee (r \wedge p)                  \\[0.1cm]
      \Leftrightarrow & \bigl((\neg q \vee \neg p \vee \neg r) \wedge \top \bigr) \vee (r \wedge p) \\[0.1cm]
      \Leftrightarrow & (\neg q \vee \neg p \vee \neg r) \vee (r \wedge p) \\[0.1cm]
      \Leftrightarrow & (\neg q \vee \neg p \vee \neg r \vee r) \wedge 
                        (\neg q \vee \neg p \vee \neg r \vee p) \\[0.1cm]
      \Leftrightarrow & \top \wedge \top \\[0.1cm]
      \Leftrightarrow & \top             \\[0.1cm]
      \Leftrightarrow & \{\}
      \end{array}
      \]
\end{enumerate}

\exercise
\begin{enumerate}
\item Die folgende Prozedur leistet das Gew�nschte:
      \begin{Verbatim}[ frame         = lines, 
                        framesep      = 0.3cm, 
                        labelposition = bottomline,
                        numbers       = left,
                        numbersep     = -0.2cm,
                        xleftmargin   = 0.0cm,
                        xrightmargin  = 0.0cm,
                      ]
    allVars := procedure(e) {
        operators := {"+", "-", "*", "/"};
        switch {
            case isString(e):       return { e };
            case e(2) in operators: return allVars(e(1)) + allVars(e(3));
            default:                abort("Error in allVars($e$)");
        }
    };
      \end{Verbatim}
\item Die folgende Prozedur leistet das Gew�nschte:
      \begin{Verbatim}[ frame         = lines, 
                        framesep      = 0.3cm, 
                        labelposition = bottomline,
                        numbers       = left,
                        numbersep     = -0.2cm,
                        xleftmargin   = -0.5cm,
                        xrightmargin  = 0.0cm,
                      ]
    countVars := procedure(e, x) {
        operators := {"+", "-", "*", "/"};
        switch {
            case x == e:            return 1;
            case isString(e):       return 0;
            case e(2) in operators: return countVars(e(1), x) + countVars(e(3), x);
            default:                abort("Error in countVars($e$)");
        }
    };
      \end{Verbatim}
\item Die folgende Prozedur leistet das Gew�nschte:
      \begin{Verbatim}[ frame         = lines, 
                        framesep      = 0.3cm, 
                        labelposition = bottomline,
                        numbers       = left,
                        numbersep     = -0.2cm,
                        xleftmargin   = 0.0cm,
                        xrightmargin  = 0.0cm,
                      ]
    singleVars := procedure(e) {
        return { x in allVars(e) | countVars(e,x) == 1 };
    };
      \end{Verbatim}
\end{enumerate}
\pagebreak

\exercise
Die folgende Kette von Anwendungen der Schnitt-Regel weist die Behauptung nach:
\renewcommand{\labelenumi}{\arabic{enumi}.}
\begin{enumerate}
\item $\{ \neg t, \neg p \}, \{  \neg t, p \} \vdash  \{ \neg t \}$,
\item $\{ \neg t \}, \{ t,  \neg q \}         \vdash  \{ \neg q \}$,
\item $\{ \neg q \}, \{ q,  \neg p \}         \vdash  \{ \neg p \}$,
\item $\{ \neg p \}, \{ t, p, \neg r \}       \vdash  \{ t, \neg r \}$,
\item $\{ t, \neg r \}, \{ \neg t \}          \vdash  \{ \neg r \}$,
\item $\{ \neg r \}, \{ t, q, r \}            \vdash  \{ t, q \}$,
\item $\{ t, q \}, \{ \neg t \}               \vdash  \{ q \}$,
\item $\{ q \}, \{ \neg q \}                  \vdash  \{ \}$.
\end{enumerate}


\exercise
Da die Menge $M$ keine Unit-Klausel enth�lt, beginnen wir mit einer Fall-Unterscheidung.
Wir f�hren die Fall-Unterscheidung bez�glich $p$ durch:
\begin{enumerate}
\item Als erstes betrachten wir daher die Menge $M_0 = M \cup \bigl\{\{p\}\bigr\}$.
      Wir berechnen
      \[ M_1 := \textsl{reduce}(M_0, p) =
         \bigl\{ \;
                    \{ r,  \neg t \},
                    \{ \neg r, q \},
                    \{ \neg q, s \},
                    \{ \neg r,  \neg s \},
                    \{ \neg s \},
                    \{ q, t  \},
                    \{ p \} 
         \; \bigr\}.
      \]
      $M_1$ enth�lt die neue Unit-Klausel $\{ \neg s \}$.  Daher berechnen wir nun
      \[ M_2 := \textsl{reduce}(M_1, \neg s) =
         \bigl\{ \;
                    \{ r,  \neg t \},
                    \{ \neg r, q \},
                    \{ \neg q \},
                    \{ q, t  \},
                    \{ p \}, 
                    \{ \neg s \} 
         \; \bigr\}.
      \]
      $M_2$ enth�lt die neue Unit-Klausel $\{ \neg q \}$.  Daher berechnen wir nun
      \[ M_3 := \textsl{reduce}(M_2, \neg q) =
         \bigl\{ \;
                    \{ r,  \neg t \},
                    \{ \neg r \},
                    \{ t \},
                    \{ p \}, 
                    \{ \neg s \}, 
                    \{ \neg q \}
         \; \bigr\}.
      \]
      $M_3$ enth�lt die neue Unit-Klausel $\{ t \}$.  Daher berechnen wir nun
      \[ M_4 := \textsl{reduce}(M_3,  t) =
         \bigl\{ \;
                    \{ r \},
                    \{ \neg r \},
                    \{ p \}, 
                    \{ \neg s \}, 
                    \{ \neg q \},
                    \{ t \}
         \; \bigr\}.
      \]
      Da $M_4$ sowohl die Unit-Klausel $\{r \}$ als auch die Unit-Klausel $\{\neg r \}$
      enth�lt, ist $M_4$ nicht l�sbar.
\item Nun betrachten wir die Menge $M \cup \bigl\{\{ \neg p\}\bigr\}$.
      \[ M_5 := \textsl{reduce}(M_0, \neg p) =
         \bigl\{ \;
                    \{ q \},
                    \{ r, s \},
                    \{ r, \neg s \},
                    \{ \neg q, s \},
                    \{ \neg r,  \neg s\},
                    \{ \neg p \}
         \; \bigr\}
      \]
      $M_5$ enth�lt die neue Unit-Klausel $\{ q \}$.  Daher berechnen wir nun
      \[ M_6 := \textsl{reduce}(M_5, q) =
         \bigl\{ \;
                    \{ r, s \},
                    \{ r, \neg s \},
                    \{ s \},
                    \{ \neg r, \neg s\},
                    \{ \neg p \},
                    \{ q \}
         \; \bigr\}
      \]
      $M_6$ enth�lt die neue Unit-Klausel $\{ s \}$.  Daher berechnen wir nun
      \[ M_7 := \textsl{reduce}(M_6, s) =
         \bigl\{ \;
                    \{ r \},
                    \{ \neg r \},
                    \{ \neg p \},
                    \{ q \},
                    \{ s \}
         \; \bigr\}
      \]
      Da $M_7$ sowohl die Unit-Klausel $\{ r \}$ als auch die Unit-Klausel $\{ \neg r \}$
      enth�lt, ist $M_7$ nicht l�sbar.
\end{enumerate}
Da sowohl  $M \cup \bigl\{\{ p \}\bigr\}$ als auch $M \cup \bigl\{\{ \neg p \}\bigr\}$ unl�sbar sind, ist
auch die Menge $M$ unl�sbar. \qed
\pagebreak

\renewcommand{\labelenumi}{(\alph{enumi})}

\exercise
Wir setzen $\mathcal{U} := \{ a, b \}$ und definieren die Interpretation $p^\mathcal{J}$
des Pr�dikats-Zeichens $p$ als
\[ p^\mathcal{J} := \bigl\{ \pair(a,a), \pair(b,b) \bigr\}. \]
Dann gibt es in $\mathcal{U}$ zwar f�r jedes $x$ ein $y$, so dass $p(x,y)$ gilt, denn wir k�nnen
$y = x$ setzen.  Aber es gibt kein $y$, so dass f�r alle $x$ die Formel $p(x,y)$ gilt.
Wollten wir dies mit dem in der Vorlesung gezeigten Programm testen, so k�nnten wir folgendes 
definieren:
\begin{Verbatim}[ frame         = lines, 
                  framesep      = 0.3cm, 
                  labelposition = bottomline,
                  numbers       = left,
                  numbersep     = -0.2cm,
                  xleftmargin   = 0.0cm,
                  xrightmargin  = 0.0cm,
                ]
    a := "a";
    b := "b";  
    u := { a, b };  // the universe
    // pJ is the interpretation of the predicate symbol "p".
    pJ := { [ a, a ], [ b, b ] };
    // The structure consists of the universe and the interpretation of
    // the function and predicate symbols.
    j := { [ "p", pJ ]};
    s := [ u, j ];
    // I is a variable assignment.
    i := { [ "x", a ], [ "y", b ] }; 
    f := parse("(forall x: exists y: p(x,y)) -> (exists y: forall x: p(x,y))");
    print(f);
    print(evalFormula(f, s, i));        
\end{Verbatim}
Dann w�rde der Aufruf ``\texttt{evalFormula(f, S, I)}'' den Wert \texttt{false} liefern.

\exercise
\[  
\begin{array}[t]{lll}
                & \Big(\forall x: \neg \exists y: \big(q(x) \rightarrow p(x,y)\big) \Big) \rightarrow 
                  \big(\forall z: q(z)\big) 
                  \\[0.1cm]
\Leftrightarrow & \neg \Big(\forall x: \neg \big(\exists y: q(x) \rightarrow p(x,y)\big) \Big) \vee
                  \big(\forall z: q(z)\big) 
                  \\[0.1cm]
\Leftrightarrow & \Big(\exists x: \exists y: \bigl( q(x) \rightarrow p(x,y)\bigr) \Big) \vee
                  \big(\forall z: q(z)\big) 
                  \\[0.1cm]
\Leftrightarrow & \exists x: \Big( \exists y: \bigl( q(x) \rightarrow p(x,y)\bigr)  \vee
                  \big(\forall z: q(z)\big) \Big)
                  \\[0.1cm]
\Leftrightarrow & \exists x: \exists y: \Big( \bigl( q(x) \rightarrow p(x,y)\bigr)  \vee
                  \big(\forall z: q(z)\big) \Big)
                  \\[0.1cm]
\Leftrightarrow & \exists x: \exists y: \forall z: \Big( \bigl( q(x) \rightarrow p(x,y)\bigr) \vee q(z) \Big)
                  \\[0.1cm]
\approx_e       & \exists y: \forall z: \Big( \bigl( q(s_1) \rightarrow p(s_1,y)\bigr) \vee q(z) \Big)
                & \mbox{mit der Skolem-Konstante $s_1$ f�r $x$} 
                  \\[0.1cm]
\approx_e       & \forall z: \Big( \bigl( q(s_1) \rightarrow p(s_1,s_2)\bigr) \vee q(z) \Big)
                & \mbox{mit der Skolem-Konstante $s_2$ f�r $y$} 
                  \\[0.1cm]
\Leftrightarrow & \forall z: \neg q(s_1) \vee p(s_1,s_2) \vee q(z) 
                  \\[0.1cm]
\Leftrightarrow & \bigl\{ \{ \neg q(s_1),\; p(s_1,s_2),\; q(z) \} \bigr\} 
\end{array}
\]

\pagebreak


\exercise
\begin{Verbatim}[ frame         = lines, 
                  framesep      = 0.3cm, 
                  labelposition = bottomline,
                  numbers       = left,
                  numbersep     = -0.2cm,
                  xleftmargin   = 0.0cm,
                  xrightmargin  = 0.0cm,
                ]
    % is_contained(+List(T), +List(T)).
    
    is_contained([], L).
    
    is_contained([ X | R ], L) :-
        member(X, L),
        is_contained(R, L).
\end{Verbatim}


\exercise
\begin{Verbatim}[ frame         = lines, 
                  framesep      = 0.3cm, 
                  labelposition = bottomline,
                  numbers       = left,
                  numbersep     = -0.2cm,
                  xleftmargin   = 0.0cm,
                  xrightmargin  = 0.0cm,
                ]
    % all_lists(+Number, +Number, -List(Number)).
    
    all_lists(_N, 0, []).
    
    all_lists(N, K, [ X | T ]) :-
    	K > 0,
    	K1 is K - 1,
    	all_lists(N, K1, T),
    	between(1, N, X),
    	\+ member(X, T).
\end{Verbatim}


\end{document}

%%% Local Variables: 
%%% mode: latex
%%% TeX-master: t
%%% End: 
