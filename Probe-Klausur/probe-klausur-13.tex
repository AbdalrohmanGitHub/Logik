\documentclass{article}
\usepackage[latin1]{inputenc}
\usepackage{fleqn}
\usepackage{ngerman}
\usepackage{a4wide}
\usepackage{amssymb}

\setlength{\topmargin}{-0.5cm}
\setlength{\mathindent}{1.3cm}
%\setlength{\evensidemargin}{-0.5cm}
%\setlength{\oddsidemargin}{-0.5cm}
%\setlength{\textwidth}{17cm}

\def\pair(#1,#2){\langle #1, #2 \rangle}


\renewcommand{\labelenumi}{(\alph{enumi})}
\renewcommand{\labelenumii}{\arabic{enumii}.}


\newcommand{\exercise}{\vspace*{0.1cm}
\stepcounter{aufgabe}
\vspace*{0.3cm}

\noindent
\textbf{Aufgabe \arabic{aufgabe}}: }

\newcounter{aufgabe}

\newcommand{\schluss}[2]{\frac{\displaystyle\quad \rule[-6pt]{0pt}{12pt}#1 \quad}{\displaystyle\quad \rule{0pt}{10pt}#2 \quad}}
\newcommand{\vschlus}[1]{{\displaystyle\rule[-6pt]{0pt}{12pt} \atop \rule{0pt}{10pt}#1}}

\newcommand{\punkte}[1]{\hspace*{\fill} \mbox{(#1 Punkte)}}
\newcommand{\struct}{\mathcal{S}}
\newcommand{\FV}{\textsl{FV}}
\newcommand{\BV}{\textsl{BV}}
\newcommand{\var}{\textsl{Var}}
\newcommand{\el}{\!\in\!}
\newcommand{\verum}{\top}
\newcommand{\falsum}{\bot}
\newcommand{\gentzen}{\vdash_{G}}
\newcommand{\komplement}[1]{\overline{#1}}
\newcommand{\mathquote}[1]{\mbox{``}\mathtt{#1}\mbox{''}}

\begin{document}

\noindent
{\large Aufgaben zur Vorlesung  ``{\sl Grundlagen der Informatik}''}
\vspace{0.3cm}

\exercise
F�r zwei Mengen $A$ und $B$ werde die \emph{symmetrische Differenz} $A \oplus B$ durch die Gleichung
\\[0.2cm]
\hspace*{1.3cm}
$A \oplus B := (A \cup B) \backslash (A \cap B)$
\\[0.2cm]
definiert.  \underline{Beweisen} Sie die folgenden Eigenschaften der symmetrischen Differenz:
\begin{enumerate}
\item $(A \oplus B) \oplus C = A \oplus (B \oplus C)$.
\item $A \oplus B = A \oplus C \rightarrow B = C$.
\end{enumerate}
Beweisen oder widerlegen Sie die folgenden Gleichungen:
\begin{enumerate}
\setcounter{enumi}{2}
\item $A \cap (B \oplus C) = (A \cap B) \oplus (A \cap C)$.
\item $A \cup (B \oplus C) = (A \cup B) \oplus (A \cup C)$.
\item $\textsl{card}(A \oplus B) = \textsl{card}(A) + \textsl{card}(B) - \textsl{card}(A \cap B)$.
\end{enumerate}

\exercise
Die Menge $A$ werde definiert als $A := \{1,2,3\}$.  
\begin{enumerate}
\item Geben Sie eine Relation $R \subseteq A \times A$ an, so dass gilt:
      \\[0.2cm]
      \hspace*{1.3cm}
      $R$ ist sowohl symmetrisch als auch anti-symmetrisch.
\item Geben Sie eine Relation $R \subseteq A \times A$ an, so dass gilt:
      \\[0.2cm]
      \hspace*{1.3cm}
      $R$ ist weder symmetrisch noch anti-symmetrisch.
\item Geben Sie eine Relation $R \subseteq A \times A$ an, so dass gilt:
      \\[0.2cm]
      \hspace*{1.3cm}
      $R$ ist transitiv, aber $R \cup R^{-1}$ ist nicht transitiv.
\end{enumerate}

\exercise
Es sei $M$ eine beliebige Menge.  Beweisen Sie die folgende Behauptung: 
\\[0.2cm]
\hspace*{1.3cm}
Wenn eine Relation $R \subseteq M \times M$
transitiv ist, dann ist auch die Relation $R \circ R$ transitiv.


\exercise
\begin{enumerate}
\item Es seien $l_1$ und $l_2$ Listen von ganzen Zahlen.
      Geben Sie eine \textsc{SetlX}-Prozedur $\texttt{both}()$ an, so dass der Aufruf 
      $\mathtt{both}(l_1, l_2)$ die Menge aller der Zahlen berechnet,
      die sowohl in $l_1$ als auch in $l_2$ auftreten.
      \punkte{4}
\item Eine Tripel $\langle x, y, z \rangle \in \mathbb{N}^3$ mit $x > 0$, $y > 0$ und $z > 0$
      ist ein \emph{pythagor�isches Tripel} falls 
      \\[0.1cm]
      \hspace*{1.3cm}      
            $x^2 + y^2 = z^2$
      \\[0.1cm]
      gilt.  Beispielsweise ist $\langle 3, 4, 5 \rangle$ ein pythagor�isches Tripel,
      denn es gilt $3^2 + 4^2 = 5^2$.
      Geben Sie eine \textsc{SetlX}-Prozedur $\texttt{pythagoras}()$ an,
      so dass der Aufruf $\texttt{pythagoras}(n)$ die Menge aller der pythagor�ischen
      Tripel $\langle x, y, z \rangle$ berechnet, f�r die $x$, $y$ und $z$ alle
      kleiner als $n$ sind.
      \punkte{6}
\item Schreiben Sie eine \textsc{SetlX}-Prozedur $\texttt{subsets}()$, so dass der Aufruf
      \\[0.2cm]
      \hspace*{1.3cm}
      $\textsl{subsets}(n, k)$
      \\[0.2cm]
      f�r zwei nat�rliche Zahlen $n$ und $k$ die Menge aller der Teilmengen der Menge
      $\{1, \cdots, n\}$ berechnet, die genau $k$ Elemente enthalten.
      \punkte{8}
\item Implementieren Sie ein \textbf{Prolog}-Pr�dikat $\mathtt{pythagoras}()$, so dass der Aufruf
      \\[0.2cm]
      \hspace*{1.3cm}
      $\mathtt{pythagoras}(n, L)$
      \\[0.2cm]
      f�r eine nat�rliche Zahl $n$ eine Liste $L$ berechnet, die alle Listen der Form $[x,y,z]$ enth�lt, f�r
      die $\langle x, y, z \rangle$ ein pythagor�isches Tripel ist, bei dem $x < n$, $y < n$ und $z < n$ gilt.
      \punkte{16}
\end{enumerate}

\exercise
Wir definieren eine Abstands-Relation $A$ als eine Menge von
Paaren der Form $\bigl\langle \langle x, y \rangle, d \bigr\rangle$ wobei gilt: 
\begin{enumerate}
\item $\langle x, y \rangle \el P \times P$,  \quad wobei $P$ eine Menge von Punkten ist,
\item $d \in \mathbb{N}$ und es gilt $d \geq 1$.
\end{enumerate}
Wir interpretieren eine Abstands-Relation $A$ wie folgt: Ist 
$\bigl\langle \langle x, y\rangle, d \bigr\rangle \in A$, so gibt es eine 
direkte Verbindung zwischen den Punkten $x$ und $y$ und diese Verbindung hat die L�nge
$d$.  Die Komposition $A_1 \odot A_2$ zweier Abstands-Relationen sei so definiert,
dass das Paar $\bigl\langle \langle x, z\rangle, d \bigr\rangle \in A$ genau dann ein
Element der Komposition $A_1 \odot A_2$ ist, wenn es einen Punkt $y\el P$ gibt,
so dass es einerseits in $A_1$ eine direkte Verbindung von $x$ nach $y$ einer L�nge $d_1$
und es andererseits in $A_2$ eine direkte Verbindung von $y$ nach $z$ der L�nge $d_2$ gibt,
so dass $d = d_1 + d_2$ gilt:
\\[0.2cm]
\hspace*{1.3cm}
$A_1 \odot A_2 := \bigl\{ \bigl\langle \langle x,z \rangle, d_1 + d_2 \bigr\rangle \mid \exists y
 \in P: \bigl(\bigl\langle \langle x,y \rangle, d_1 \bigr\rangle \in A_1 \wedge \bigl\langle
 \langle y,z \rangle, d_2 \bigr\rangle \in A_2\bigr) \bigr\}
$.
\begin{enumerate}
\item[(a)] Implementieren Sie eine \textsc{SetlX}-Prozedur $\texttt{compose}()$,
           so dass der Aufruf $\texttt{compose}(A_1, A_2)$ f�r zwei 
           Abstands-Relationen $A_1$ und $A_2$ die Komposition $A_1 \circ A_2$ 
           berechnet.            
           \punkte{6}
\item[(b)] Implementieren Sie eine Prozedur \texttt{closure}, so dass der Aufruf
           $\texttt{closure}(A)$ f�r eine Abstands-Relation $A$
           die Menge aller Paare  $\bigl\langle \langle x_1, x_n \rangle, d \bigr\rangle$
           berechnet, so dass es eine endliche Folge $x_1, x_2, x_3, \cdots, x_n$ von
           Punkten aus $P$ gibt, so dass einerseits 
           \\[0.1cm]
           \hspace*{1.3cm}      
            $\bigl\langle \langle x_i, x_{i+1} \rangle, d_i \bigr\rangle \in A$      
           \\[0.1cm]
           und andererseits $d = d_1 + d_2 + \cdots + d_{n-1}$ gilt.
           \punkte{10}

           \noindent
           \textbf{Hinweis}: Sie d�rfen voraussetzen, dass es nur endlich viele Pfade gibt.
\end{enumerate}

\exercise
Implementieren Sie eine \textsc{SetlX}-Prozedur $\texttt{toLists}()$, so dass der Aufruf
$\texttt{toLists}(M)$ f�r eine Menge $M$ die Menge aller Listen berechnet, die jedes
Element der Menge $M$ genau einmal enthalten.  Beispielsweise soll der Aufruf
\[ \texttt{toLists}(\{1,2,3\}) \]
als Ergebnis die Menge
\[ \bigl\{\; [1,2,3],\; [1,3,2],\; [2,1,3],\; [2,3,1],\; [3,1,2],\; [3,2,1]\; \bigr\} \]
berechnen.
\punkte{8}
\vspace{0.1cm}

\noindent
\textbf{Hinweis}: Implementieren Sie die Prozedur rekursiv.

\exercise
\begin{enumerate}
\item Formalisieren Sie die Schlu�-Regel, die in dem folgenden Argument verwendet wird. 
      \begin{center}
      \begin{minipage}[c]{12cm}
      {\em  
           Wenn der linke Motor ausf�llt oder wenn der rechte Motor ausf�llt, st�rzt
           das Flugzeug ab.  Das Flugzeug  st�rzt ab. Der linke Motor f�llt nicht
           aus. Also ist der rechte Motor ausgefallen.
      } 
      \end{minipage}  
      \end{center}
      \textbf{Hinweis}: Die verwendete Schlu�-Regel hat drei Pr�missen.
      \punkte{4}
\item �berpr�fen  Sie, ob diese Schlu�-Regel korrekt ist.
      \punkte{6}
\end{enumerate}
\pagebreak

\exercise
\begin{enumerate}
\item Betrachten Sie die folgende Schlu�--Regel: \\[0.1cm]
      \hspace*{1.3cm} 
      $\schluss{q \wedge p \quad\quad r \vee \neg q}{r \wedge p}$ \\[0.1cm]
      Geben Sie  eine aussagenlogische
      Formel $f$ an, die genau dann eine Tautologie ist, wenn die obige Schlu�--Regel korrekt ist.
      \punkte{2}
\item Berechnen Sie die konjunktive Normalform der in (a) berechneten Formel.
      \punkte{8} 
\end{enumerate}


\exercise
Die Menge $M$ sei wie folgt definiert: \\[0.1cm] 
           \hspace*{1.3cm} $M := \bigl\{ 
           \{ t, q, r \}, 
           \{ t, p, \neg  r \}, 
           \{  \neg t, p, r \},
           \{ t,  \neg q  \},
           \{  \neg t, p \},
           \{ q,  \neg p  \},
           \{  \neg t,  \neg p \}
           \bigr\}$.
            \\[0.1cm]
           Zeigen Sie durch mehrfache Anwendung der Schnitt-Regel, dass $M \vdash \{\}$ gilt.
           \punkte{8}

\exercise
Die Menge $M$ sei wie folgt definiert: \\[0.1cm]
\hspace*{1.3cm} $M := \{ \; \{p, q\},
                         \{ \neg p, r,  \neg t\},
                         \{r, s, p \},
                         \{ \neg r, q,  \neg p\},
                         \{r,  \neg s, p\},$ \\[0.1cm]
\hspace*{2.6cm} $        \{ \neg p,  \neg q, s\},
                         \{p,  \neg q, s\},
                         \{ \neg r,  \neg s\},
                         \{ \neg p,  \neg s\},
                         \{q, \neg p, t\} \bigr\}$ \\[0.1cm]
Zeigen Sie mit dem Verfahren von Davis-Putnam, dass $M$ nicht l�sbar ist.
\punkte{12}

\exercise
Geben Sie eine pr�dikatenlogische Struktur $\mathcal{S}$ an, in der die 
Formel
\[ (\forall x\colon \exists y\colon p(x,y)) \rightarrow \exists y\colon \forall x\colon p(x,y) \]
nicht gilt.  Die pr�dikatenlogische Struktur soll wie im Skript mit Hilfe
von \textsc{SetlX} beschrieben werden.  \punkte{10}

\exercise
Wandeln Sie die folgende Formel in pr�dikatenlogische Klausel-Normalform um
und geben Sie das Ergebnis in Mengen-Schreibweise an. \\[0.1cm]
\hspace*{1.3cm} 
$\Big(\forall x\colon \neg \exists y\colon \big(q(x) \rightarrow p(x,y)\big) \Big) \rightarrow 
 \big(\forall z\colon q(z)\big)
$ 
\punkte{8}


\exercise
Implementieren Sie ein \textsl{Prolog}-Pr�dikat ``$\mathtt{is\_contained}$'' mit der Typ-Spezifikation
\\[0.1cm]
\hspace*{1.3cm} $\mathtt{is\_contained}( +\textsl{List}(\textsl{Number}), +\textsl{List}(\textsl{Number}) )$. \\[0.1cm]
Ein Aufruf von $\mathtt{is\_contained}$ soll die Form \\[0.1cm]
\hspace*{1.3cm} $\mathtt{is\_contained}(l_1, l_2)$ \\[0.1cm]
haben. Dabei sind $l_1$ und $l_2$ Listen von Zahlen.  Der Aufruf soll genau dann
erfolgreich sein,  wenn alle Elemente von $l_1$ in der Liste $l_2$ enthalten sind.
\punkte{8}
\vspace{0.1cm}

\noindent
\textbf{Hinweis}: Sie d�rfen das eingebaute Pr�dikat $\mathtt{member}$ benutzen.

\exercise
Implementieren Sie ein \textsl{Prolog}-Pr�dikat ``$\mathtt{all\_lists}$'' 
mit der Typ-Spezifikation \\[0.1cm]
\hspace*{1.3cm} 
$\mathtt{all\_lists}( +\textsl{Number}, +\textsl{Number}, -\textsl{List}(\textsl{Number}) )$. 
\\[0.1cm]
Der Aufruf            
\\[0.2cm]
\hspace*{1.3cm}
$\textsl{all\_lists}(N, K, L)$
\\[0.2cm]
soll f�r zwei nat�rliche Zahlen $N$ und $K$ eine Liste $L$ erzeugen, deren Elemente
aus der Menge $\{1, \cdots, N\}$ stammen und die die L�nge $K$ hat.   Die Liste darf kein
Element mehrfach enthalten.   Au�erdem soll das Pr�dikat beim Backtracking alle m�glichen Listen mit
der oben beschriebenen Eigenschaft erzeugen.
\punkte{8}
\end{document}

%%% Local Variables: 
%%% mode: latex
%%% TeX-master: t
%%% End: 
