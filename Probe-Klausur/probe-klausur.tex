\documentclass{article}
\usepackage[latin1]{inputenc}
\usepackage{fleqn}
\usepackage{ngerman}
\usepackage{a4wide}
\usepackage{amssymb}

\setlength{\topmargin}{-0.5cm}
\setlength{\mathindent}{1.3cm}
%\setlength{\evensidemargin}{-0.5cm}
%\setlength{\oddsidemargin}{-0.5cm}
%\setlength{\textwidth}{17cm}

\def\pair(#1,#2){\langle #1, #2 \rangle}

\newcommand{\exercise}{\vspace*{0.1cm}
\stepcounter{aufgabe}
\vspace*{0.3cm}

\noindent
\textbf{Aufgabe \arabic{aufgabe}}: }

\newcounter{aufgabe}

\newcommand{\schluss}[2]{\frac{\displaystyle\quad \rule[-6pt]{0pt}{12pt}#1 \quad}{\displaystyle\quad \rule{0pt}{10pt}#2 \quad}}
\newcommand{\vschlus}[1]{{\displaystyle\rule[-6pt]{0pt}{12pt} \atop \rule{0pt}{10pt}#1}}

\newcommand{\punkte}[1]{\hspace*{\fill} \mbox{(#1 Punkte)}}
\newcommand{\struct}{\mathcal{S}}
\newcommand{\FV}{\textsl{FV}}
\newcommand{\BV}{\textsl{BV}}
\newcommand{\var}{\textsl{Var}}
\newcommand{\el}{\!\in\!}
\newcommand{\verum}{\top}
\newcommand{\falsum}{\bot}
\newcommand{\gentzen}{\vdash_{G}}
\newcommand{\komplement}[1]{\overline{#1}}
\newcommand{\mathquote}[1]{\mbox{``}\mathtt{#1}\mbox{''}}

\begin{document}

\noindent
{\large Aufgaben zur Vorlesung  ``{\sl Grundlagen der Informatik}''}
\vspace{0.3cm}

\exercise
\begin{enumerate}
\item[(a)] Es seien $M$ und $N$ Mengen und $R$ ist eine bin�re Relation auf $M \times N$,
           es gilt also $R \subseteq M \times N$.
           Beweisen oder widerlegen Sie, dass f�r beliebige Mengen $A$ und $B$ gilt: \\[0.1cm]
           \hspace*{1.3cm} $R(A \cup B) = R(A) \cup R(B)$.
           \punkte{8} 
\item[(b)] Es seien $M$ und $N$ Mengen und es gelte $R \subseteq M \times N$.
           Beweisen oder widerlegen Sie, dass f�r beliebige Mengen $A$ und $B$ gilt: \\[0.1cm]
           \hspace*{1.3cm} $R(A \cap B) = R(A) \cap R(B)$.
           \punkte{6} 
\end{enumerate}
\textbf{Hinweis}: F�r eine bin�re Relation $R \subseteq M \times N$ und eine Menge $C \subseteq M$ ist
die Menge $R(C)$ wie folgt definiert: 
      \\[0.1cm]
      \hspace*{1.3cm}      
      $R(C) := \bigl\{ y \el N \mid \exists x : x \el C \wedge \langle x, y \rangle \in R \bigr\}$.

\exercise
\begin{enumerate}
\item[(a)] Es seien $l_1$ und $l_2$ Listen von ganzen Zahlen.
           Geben Sie eine \textsc{Setl2}-Prozedur $\texttt{both}()$ an, so dass der Aufruf 
           $\mathtt{both}(l_1, l_2)$ die Menge aller der Zahlen berechnet,
           die sowohl in $l_1$ als auch in $l_2$ auftreten.
           \punkte{4}
\item[(b)] Eine Tripel $\langle x, y, z \rangle \in \mathbb{N}^3$ 
           ist ein \emph{pythagor�isches Tripel} falls 
           \\[0.1cm]
           \hspace*{1.3cm}      
                 $x^2 + y^2 = z^2$
           \\[0.1cm]
           gilt. Geben Sie eine \textsc{Setl2}-Prozedur $\texttt{pythagoras}()$ an,
           so dass der Aufruf $\texttt{pythagoras}(n)$ die Menge aller der pythagor�ischen
           Tripel $\langle x, y, z \rangle$ berechnet, f�r $x$, $y$ und $z$ alle
           kleiner als $n$ ist.
           \punkte{6}
\end{enumerate}

\exercise
Wir definieren eine Abstands-Relation $A$ als eine Menge von
Paaren der Form $\bigl\langle \langle x, y \rangle, d \bigr\rangle$ wobei gilt: 
\begin{enumerate}
\item $\langle x, y \rangle \el P \times P$.
\item $d \in \mathbb{N}$ und es gilt $d \geq 1$.
\end{enumerate}
Wir interpretieren eine Abstands-Relation $A$ wie folgt: Ist 
$\bigl\langle \langle x, y\rangle, d \bigr\rangle \in A$, so gibt es eine 
direkte Verbindung zwischen den Punkten $x$ und $y$ und diese Verbindung hat die L�nge
$d$.  Die Komposition $A_1 \circ A_2$ zweier Abstands-Relationen sei so definiert,
dass das Paar $\bigl\langle \langle x, z\rangle, d \bigr\rangle \in A$ genau dann ein
Element der Komposition $A_1 \circ A_2$ ist, wenn es einen Punkt $y\el P$ gibt,
so dass es einerseits in $A_1$ eine direkte Verbindung von $x$ nach $y$ einer L�nge $d_1$
und andererseits in $A_2$ eine direkte Verbindung von $y$ nach $z$ der L�nge $d_2$ gibt,
so dass $d = d_1 + d_2$ ist.
\begin{enumerate}
\item[(a)] Implementieren Sie eine \textsc{Setl2}-Prozedur $\texttt{compose}()$,
           so dass der Aufruf $\texttt{compose}(A_1, A_2)$ f�r zwei 
           Abstands-Relationen $A_1$ und $A_2$ die Komposition $A_1 \circ A_2$ 
           berechnet.            
           \punkte{8}
\item[(b)] Implementieren Sie eine Prozedur \texttt{closure}, so dass der Aufruf
           $\texttt{closure}(A)$ f�r eine Abstands-Relation $A$
           die Menge aller Paare  $\bigl\langle \langle x_1, x_n \rangle, d \bigr\rangle$
           berechnet, so dass es eine endliche Folge $x_1, x_2, x_3, \cdots, x_n$ von
           Punkten aus $P$ gibt, so dass einerseits 
           \\[0.1cm]
           \hspace*{1.3cm}      
            $\bigl\langle \langle x_i, x_{i+1} \rangle, d_i \bigr\rangle \in A$      
           \\[0.1cm]
           und andererseits $d = d_1 + d_2 + \cdots + d_{n-1}$ gilt.
           \punkte{10}

           \noindent
           \textbf{Hinweis}: Sie d�rfen voraussetzen, dass es nur endlich viele Pfade gibt.
\end{enumerate}
\pagebreak

\exercise
Implementieren Sie eine \textsc{Setl2}-Prozedur $\texttt{toLists}()$, so dass der Aufruf
$\texttt{toLists}(M)$ f�r eine Menge $M$ die Menge aller Listen berechnet, die jedes
Element der Menge $M$ genau einmal enthalten.  Beispielsweise soll der Aufruf
\[ \texttt{toLists}(\{1,2,3\}) \]
als Ergebnis die Menge
\[ \bigl\{\; [1,2,3],\; [1,3,2],\; [2,1,3],\; [2,3,1],\; [3,1,2],\; [3,2,1]\; \bigr\} \]
berechnen.
\punkte{8}
\vspace{0.1cm}

\noindent
\textbf{Hinweis}: Implementieren Sie die Prozedur rekursiv.

\exercise
\begin{enumerate}
\item[(a)] Formalisieren Sie die Schlu�-Regel, die in dem folgenden Argument verwendet wird. 
           \begin{center}
           \begin{minipage}[c]{12cm}
           {\em  
                Wenn der linke Motor ausf�llt oder wenn der rechte Motor ausf�llt, st�rzt
                das Flugzeug ab.  Das Flugzeug  st�rzt ab. Der linke Motor f�llt nicht
                aus. Also ist der rechte Motor ausgefallen.
           } 
           \end{minipage}  
           \end{center}
           \textbf{Hinweis}: Die verwendete Schlu�-Regel hat drei Pr�missen.
          \punkte{4}
\item[(b)] �berpr�fen  Sie, ob diese Schlu�-Regel korrekt ist.
           \punkte{6}
\end{enumerate}

\exercise
\begin{enumerate}
\item[(a)] Betrachten Sie die folgende Schlu�--Regel: \\[0.1cm]
           \hspace*{1.3cm} 
           $\schluss{q \wedge p \quad\quad r \vee \neg q}{r \wedge p}$ \\[0.1cm]
           Geben Sie  eine aussagenlogische
           Formel $f$ an, die genau dann eine Tautologie ist, wenn die obige Schlu�--Regel korrekt ist.
           \punkte{2}
\item[(b)] Berechnen Sie die konjunktive Normalform der in (a) berechneten Formel.
           \punkte{8} 
\end{enumerate}
\pagebreak

\exercise
Der Begriff eines \emph{arithmetischen \textsc{Setl2}-Ausdrucks} werde wie folgt induktiv 
definiert:
\begin{enumerate}
\item Jede String ist ein \emph{arithmetischer \textsc{Setl2}-Ausdruck}.
      Ein String stellt eine \emph{Variable} dar.
\item Sind $s$ und $t$ \emph{arithmetische \textsc{Setl2}-Ausdr�cke}, so sind auch
      \begin{enumerate}
      \item $[ s, \textrm{``}\texttt{+}\textrm{''}, t ]$,
      \item $[ s, \textrm{``}\texttt{-}\textrm{''}, t ]$,
      \item $[ s, \textrm{``}\texttt{*}\textrm{''}, t ]$,
      \item $[ s, \textrm{``}\texttt{/}\textrm{''}, t ]$
      \end{enumerate}
      \emph{aritmetische \textsc{Setl2}-Ausdr�cke}.
\end{enumerate}

\begin{enumerate}
\item[(a)] Schreiben sie eine Prozedur $\texttt{allVars}(s)$, die f�r einen gegebenen
           arithmetischen \textsc{Setl2}-Ausdruck die Menge aller Variablen berechnet, 
           die in $s$ auftreten. \\
           \punkte{8} 
\item[(b)] Schreiben sie eine Prozedur $\texttt{countVars}(s,x)$, die f�r einen gegebenen
           arithmetischen \textsc{Setl2}-Ausdruck $s$ und eine Variable $x$
           z�hlt, wie oft die Variable $x$ in $s$ auftritt. \\
           \punkte{8} 
\item[(c)] Schreiben sie eine Prozedur $\texttt{singleVars}(s)$, die f�r einen gegebenen
           arithmetischen \textsc{Setl2}-Ausdruck die Menge aller Variablen berechnet, 
           die in $s$ genau einmal auftreten. \\
           \punkte{4} 
\end{enumerate}


\exercise
Die Menge $M$ sei wie folgt definiert: \\[0.1cm] 
           \hspace*{1.3cm} $M := \bigl\{ 
           \{ t, q, r \}, 
           \{ t, p, \neg  r \}, 
           \{  \neg t, p, r \},
           \{ t,  \neg q  \},
           \{  \neg t, p \},
           \{ q,  \neg p  \},
           \{  \neg t,  \neg p \}
           \bigr\}$.
            \\[0.1cm]
           Zeigen Sie, dass $M \vdash \{\}$ gilt, \underline{ohne} dabei das Verfahren von Davis-Putnam
           zu benutzen. \\
           \punkte{8}

\exercise
Die Menge $M$ sei wie folgt definiert: \\[0.1cm]
\hspace*{1.3cm} $M := \{ \; \{p, q\},
                         \{ \neg p, r,  \neg t\},
                         \{r, s, p \},
                         \{ \neg r, q,  \neg p\},
                         \{r,  \neg s, p\},$ \\[0.1cm]
\hspace*{2.6cm} $        \{ \neg p,  \neg q, s\},
                         \{p,  \neg q, s\},
                         \{ \neg r,  \neg s\},
                         \{ \neg p,  \neg s\},
                         \{q, \neg p, t\} \bigr\}$ \\[0.1cm]
Zeigen Sie mit dem Verfahren von Davis-Putnam, dass $M$ nicht l�sbar ist.
\punkte{12}

\exercise
Geben Sie eine pr�dikatenlogische Struktur $\mathcal{S}$ an, in der die 
Formel
\[ (\forall x\colon \exists y\colon p(x,y)) \rightarrow \exists y\colon \forall x\colon p(x,y) \]
nicht gilt.  Die pr�dikatenlogische Struktur soll wie im Skript mit Hilfe
von \textsc{Setl2} beschrieben werden.  \punkte{10}

\exercise
Wandeln Sie die folgende Formel in pr�dikatenlogische Klausel-Normalform in
Mengen-Schreibweise um. \\[0.1cm]
\hspace*{1.3cm} 
$\Big(\forall x\colon \neg \exists y\colon \big(q(x) \rightarrow p(x,y)\big) \Big) \rightarrow 
 \big(\forall z\colon q(z)\big)
$ 
\punkte{8}
\pagebreak

\exercise
Betrachten Sie die folgende Aussage:
\begin{center}
\textsl{Die Russel-Menge enth�lt genau die Mengen, die sich nicht selbst enthalten.}  
\end{center}
\begin{enumerate}
\item[(a)] Formalisieren Sie diese Aussage als pr�dikatenlogische Formel.
           Benutzen Sie dabei das zweistellige Pr�dikats-Zeichen ``$\in$''.
           \punkte{2}

\item[(b)] Wandeln Sie die unter (a) gefundene Formel in pr�dikatenlogische
           Klausel-Normalform um.
           \punkte{6}

\item[(c)] Zeigen Sie, dass die unter (b) gefundene Formel-Menge widerspr�chlich
           ist.
           \punkte{4}
\end{enumerate}

\exercise
Die Menge der Variablen sei $\mathcal{V} = \{ x_1, x_2, x_3\}$ und
die Menge der Funktions-Zeichen sei $\mathcal{F} = \{ f,h,c,d \}$. Die Stelligkeit der
Funktions-Zeichen werde durch die folgende Funktion gegeben: 
\[ \mathtt{arity} = \bigl\{ \pair(f, 3), \pair(h, 1), \pair(c, 0) \bigr\}. \]
L�sen sie das folgende syntaktische Gleichungs-System mit Hilfe
der in der Vorlesung beschrie\-ben\-en Vereinfachungs-Regeln von Martelli und Montanari.\\[0.1cm]
\hspace*{1.3cm} 
 $\big\{ f(x_1, x_2, h(x_1)) \doteq f(h(c), h(x_3), x_3) \bigr\}$
      \punkte{8}

\exercise
Implementieren Sie ein \textsl{Prolog}-Pr�dikat ``$\mathtt{is\_contained}$'' mit der Typ-Spezifikation
\\[0.1cm]
\hspace*{1.3cm} $\mathtt{is\_contained}( +\textsl{List}(\textsl{Number}), +\textsl{List}(\textsl{Number}) )$. \\[0.1cm]
Ein Aufruf von $\mathtt{is\_contained}$ soll die Form \\[0.1cm]
\hspace*{1.3cm} $\mathtt{is\_contained}(l_1, l_2)$ \\[0.1cm]
haben. Dabei sind $l_1$ und $l_2$ Listen von Zahlen.  Der Aufruf soll genau dann
erfolgreich sein,  wenn alle Elemente von $l_1$ in der Liste $l_2$ enthalten sind.
\punkte{8}
\vspace{0.1cm}

\noindent
\textbf{Hinweis}: Sie d�rfen das eingebaute Pr�dikat $\mathtt{member}$ benutzen.


\end{document}

%%% Local Variables: 
%%% mode: latex
%%% TeX-master: t
%%% End: 
